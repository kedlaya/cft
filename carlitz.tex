% NOTE: THIS IS THE "GOOD," FINAL VERSION OF THIS PAPER.

\documentclass[12pt]{article}
\usepackage{amsmath,amssymb}

\renewcommand{\baselinestretch}{1.25}

% SET MARGINS

\setlength{\textwidth}{6.5in}
\setlength{\oddsidemargin}{0in}
\setlength{\textheight}{9in}
\setlength{\topmargin}{-.5in}

%\textwidth 16.5cm
%\textheight 23cm
%\topmargin -.35in
%\oddsidemargin 0cm
%\evensidemargin 0cm

% ENTRYWISE GREATER THAN OR EQUAL TO, BUT NOT EQUAL TO (& L.T.)

%\newcommand{\gneq}{\geq\not=}
%\newcommand{\lneq}{\leq\not=}

% TYPES OF THEOREMS KTL.

\newcommand{\laba}[1]{\label{alg:#1}}                    %ALGORITHM
\newcommand{\refa}[1]{\ref{alg:#1}}
\newcommand{\labeq}[1]{\label{eq:#1}}                    %EQUATION
\newcommand{\refeq}[1]{\ref{eq:#1}}
\newcommand{\labd}[1]{\label{def:#1}}
\newcommand{\refd}[1]{\ref{def:#1}}
\newcommand{\labf}[1]{\label{fig:#1}}
\newcommand{\reff}[1]{Figure~\ref{fig:#1}}
\newcommand{\labn}[1]{\label{notation:#1}}
\newcommand{\refn}[1]{\ref{notation:#1}}
\newcommand{\labt}[1]{\label{thm:#1}}                    %THEOREM
\newcommand{\reft}[1]{Theorem~\ref{thm:#1}}
\newcommand{\labl}[1]{\label{lemma:#1}}                  %LEMMA
\newcommand{\refl}[1]{Lemma~\ref{lemma:#1}}
\newcommand{\labc}[1]{\label{coro:#1}}                   %COROLLARY
\newcommand{\refc}[1]{Corollary~\ref{coro:#1}}
\newcommand{\labp}[1]{\label{proposition:#1}}            %PROPOSITION
\newcommand{\refp}[1]{Proposition~\ref{proposition:#1}}
\newcommand{\refcj}[1]{\label{conject:#1}}               %CONJECTURE
\newcommand{\labcj}[1]{Conjecture~\ref{conject:#1}}

% HUH?

%\newcommand{\mathbb}{\cal}
%\newcommand{\text}{\mbox}
%\newcommand{\operatorname}{\mbox}

% MISCELLANEOUS STUFF

\newcommand{\hamlos}{\begin{flushright}$\square$ \\[.25in] \end{flushright}}
\newcommand{\Kappa}{{\cal K}}
\newcommand{\Ker}{\operatorname{Ker}}
\newcommand{\by}{\times} %e.g. mxn
\newcommand{\scal}[1]{\langle#1\rangle}
\newcommand{\adjoint}{\star}
\newcommand{\Tal}{\operatorname{Tal}}
\newcommand{\spn}{\operatorname{span}}
\newcommand{\rnk}{\operatorname{rank}}
\newcommand{\lnul}{\operatorname{lnul}}
\newcommand{\rnul}{\operatorname{rnul}}
\newcommand{\Gal}{\operatorname{Gal}}
\newcommand{\Aut}{\operatorname{Aut}}
\newcommand{\ord}{\operatorname{ord}}
\newcommand{\Drin}{\operatorname{Drin}}

% TYPES

\newtheorem{theorem}{Theorem}[section]
\newtheorem{definition}[theorem]{Definition}
\newtheorem{notation}[theorem]{Notation}
\newtheorem{algorithm}[theorem]{Algorithm}
\newtheorem{lemma}[theorem]{Lemma}
\newtheorem{corollary}[theorem]{Corollary}
\newtheorem{proposition}[theorem]{Proposition}
\newtheorem{conjecture}[theorem]{Conjecture}

% SHORTHAND

\newcommand{\nno}[1]{{\R^{#1}_{\geq 0}}}          %non-neg. orthant in R^m
\newcommand{\po}[1]{{\R^{#1}_+}}                  %pos. orthant in R^m
%\renewcommand{\emptyset}{{\rm \O}}            %better emptyset sign
\newcommand{\upto}[1]{1,~\ldots,#1}
\newcommand{\uptoset}[1]{\{\upto{#1}\}}
\newcommand{\uptob}[2]{\uptoset{#1}\setminus\{#2\}}
\newcommand{\pf}{\noindent \emph{Proof} \vskip .10in} %BEGIN STATEMENT
						      %OF PROOF
\newcommand{\rpf}{\noindent ($\implies$)\ }         % (==>)
\newcommand{\lpf}{\noindent ($\impliedby$)\ }       % (<==)
\newcommand{\cd}{\Rightarrow\Leftarrow} %CONTRADICTION
\newcommand{\bd}{{\bf d}}
\newcommand{\bM}{{\bf M}}
\newcommand{\C}{{\mathbb C}}
\newcommand{\D}{{\mathcal D}}
\newcommand{\F}{{\mathbb F}}
\newcommand{\T}{{\mathbb T}}
\newcommand{\N}{{\mathbb N}}
\newcommand{\K}{{\mathbb K}}
\newcommand{\M}{{\mathbb M}}
\newcommand{\oo}{{\mathcal O}}
\newcommand{\R}{{\mathbb R}}   %REAL NUMBERS
\newcommand{\Z}{{\mathbb Z}}
\newcommand{\Q}{{\mathbb Q}}
\newcommand{\X}{{\mathbb X}}
\newcommand{\Y}{{\mathbb Y}}
\newcommand{\diag}{\operatorname{diag}}
\renewcommand{\Im}{\operatorname{Im}}

%BEGIN AND END MATRICES

\newcommand{\mb}{\left[ \begin{matrix}}
\newcommand{\me}{\end{matrix} \right]}

%BIBLIOGRAPHY

\bibliographystyle{amsplain}


\title{Carlitz Modules and the Function Field Analogue of Kronecker-Weber}
\author{Li-Chung Chen}
\date{May 6, 2002}

\begin{document}
\maketitle

\begin{section}{Introduction}

Hilbert's 9th problem asks for an explicit class field theory for all
number fields.  This has proved difficult, and the only known cases are
for $\Q$ (cyclotomic theory) and for imaginary quadratic number fields
(the theory of complex multiplication).  However, the function field
analogue has been solved, as independent work by L. Carlitz, D. Hayes, and
V. Drinfeld has provided explicit construction of all abelian extensions
of any global function field of positive characteristic.

This paper focuses on Carlitz's function field analogues to cyclotomic
number fields.  Using Carlitz modules, we will construct explicit abelian
extensions of $\F(T)$, where $\F$ is a finite field, and we will state
Hayes' result on the maximal abelian extensions of $\F(T)$.  Then we will
briefly discuss Drinfeld's generalization for general global function
fields.

\end{section} %Introduction



\begin{section}{Drinfeld Modules Over $\F(T)$}

Let $\F$ be a finite field with $q = p^s$ elements, and set $A = \F[T]$
and $k = \F(T)$.  For a commutative $k$-algebra $B$ (which we will later
take to be the algebraic closure of $k$, $\bar{k}$), let
$\tau: B \rightarrow B$ be given by $\tau(x) = x^q$.  Because $k$ has
characeristic $p$, $(x+y)^p = x^p + y^p$ holds in $B$.  Repeated
application of this equality shows that $\tau$ is additive.  In fact, a
similar argument shows that any polynomial of the form
$a_0x + a_1x^q + \ldots + a_rx^{q^r}$ acts additively on $B$.  The set of
these polynomials forms a ring (under composition) that is identified with
the ``twisted'' polynomial ring $k\langle\tau\rangle$, in which we have
$\tau a = a^q\tau$ for all $a \in k$.  In other words, we will consider
elements of $k\langle\tau\rangle$ to be polynomials that act on $B$, with
$\tau \leftrightarrow x^q$ and $[1 \in k] \leftrightarrow x$.

Now we can define a Drinfeld module over $k = \F(T)$:
\begin{definition}
A Drinfeld module for $A$ defined over $k$ is an $\F$-algebra homomorphism
$\rho: A \rightarrow k\langle\tau\rangle$ such that
\begin{enumerate}
\item
the constant term of $\rho_a$ is $a$ for all $a \in A$, and
\item
$\rho_a \notin k$ for at least one $a \in A$.
\end{enumerate}
\end{definition}
This is called a module because any commutative $k$-algebra $B$ can be
made into an $A$-module $B_\rho$ by defining
$$a\cdot u = \rho_a(u) \text{ for all }a \in A, u \in B.$$

In this special case where $A = \F[T]$, giving $\rho$ is the same as
specifying $\rho_T$.  The first condition implies that the constant term
of $\rho_T$ is $T$, and the converse clearly holds.  Now if $\rho_T \in k$,
then $\rho$ would map into $k$.  Hence the second condition is equivalent
to $\rho_T \notin k$.  Therefore, a Drinfeld module for $A$ over $k$ is
given by specifying a nonconstant $\rho_T$ with constant term $T$.  We say
that the $\emph{rank}$ of the Drinfeld module $\rho$ is $\deg(\rho_T)$.

Consider the Drinfeld module given by $\rho_T = T + c_1\tau + c_2\tau^2 +
\ldots + c_r\tau^r$, where $c_i \in k$ and $c_r \neq 0$.  This induces an
$A$-module structure $\bar{k}_\rho$ on $\bar{k}$ as described above.  For
nonzero $a \in A$, define the torsion submodule
$$\Lambda_\rho[a] = \left\{\lambda \in \bar{k}:
\rho_a(\lambda) = 0 \right\}.$$

It is clear by induction that $\rho_{T^n}$ is a degree-$rn$ polynomial in
$\tau$ with constant term $T^n$.  Hence for $a \in A$, $\rho_a$ has degree
$r\deg(a)$ in $\tau$ (and degree $q^{r\deg(a)}$ in $x$) with constant term
$a$.  But $\F\langle\tau\rangle$ corresponds to polynomials in $x$ whose
monomials have degrees that are powers of $q$.  Thus the derivative of
$\rho_a(x)$ is $a$ and $\rho_a(x)$ is separable.  Since $\Lambda_\rho[a]$
is just the set of roots of $\rho_a(x)$, it has size $q^{r\deg(a)}$.  This
information is sufficient to determine $\Lambda_\rho[a]$:

\begin{theorem}\labt{rank}
Let $\rho$ be a Drinfeld module of rank $r$.  Then for any nonzero
$a \in A$, $\Lambda_\rho[a] \cong (A/aA)^r$ as $A$-modules (and hence
as $(A/aA)$-modules because both sides are annihilated by $a$).
\end{theorem}

\pf
First consider the case that $a = P^e$ is a prime power.  Since
$\Lambda_\rho[a]$ is finite, by the structure theorem of modules over a
PID, we obtain an $A$-module isomorphism
$$\Lambda_\rho[P^e] \cong A/{Q_1}^{f_1}A \oplus A/{Q_2}^{f_2}A
\oplus \ldots \oplus A/{Q_r}^{f_r}A$$ for some irreducible polynomials
$Q_i$.  But $P^e$ annihilates $\Lambda_\rho[P^e]$, so it annihilates each
$A/{Q_i}^{f_i}A$ and forces $Q_i/P \in \F^*$ (so we may assume $Q_i = P$)
and $f_i \leq e$ for all $i$.

Now note that $$\Lambda_\rho[P] = \left\{\lambda \in \Lambda_\rho[P^e]:
\rho_P(\lambda) = 0 \right\}$$
because any element annihilated by $P$ is also annihilated by $P^e$.  The
left hand side has $q^{r\deg(P)}$ elements as determined earlier, and the
right hand side has
$|(A/PA)^t| = q^{t\deg(P)}$ elements.  Hence $t = r$.  This implies that
$\Lambda_\rho[P^e]$ has $q^{f_1 + \ldots + f_r}$ elements.  But we already
know that $\Lambda_\rho[P^e]$ has $q^{r\deg(P^e)} = q^{re\deg(P)}$ elements
and that each $f_i \leq r$.  Hence $f_i = r$ for all $i$ and we are done
with the prime power case.

For the general case, let the prime decomposition of $a$ be $a =
\alpha{P_1}^{e_1}{P_2}^{e_2}\ldots{P_t}^{e_t}$ (where $\alpha \in \F^*$).
We claim that $\Lambda_\rho[a] = \oplus^t_{i=1} \Lambda_\rho[{P_i}^{e_i}]$.
Clearly $\Lambda_\rho[{P_i}^{e_i}] \subset \Lambda_\rho[a]$ for all $i$.
Let $a_i = \frac{a}{\alpha{P_i}^{e_i}}$.  Because the $a_i$'s have greatest
common divisor 1,
there exist $b_i \in A$ such that $\sum^t_{i=1} b_ia_i = 1$.
Then for any $m \in \Lambda_\rho[a]$, $m = \sum^t_{i=1} b_i(a_im)$.  But
$a_im \in \Lambda_\rho[{P_i}^{e_i}]$ because $P_i^{a_i}(a_im) =
\alpha^{-1}am = 0$.  Thus $\Lambda_\rho[a] =
\sum^t_{i=1} \Lambda_\rho[{P_i}^{e_i}]$.  Now if we have
$c_i \in \Lambda_\rho[{P_i}^{e_i}]$ such that $\sum^t_{i=1} c_i = 0$, then
$0 = a_j\sum^t_{i=1} c_i = a_jc_j$ because ${P_i}^{e_i}$ divides $a_j$
whenever $i \neq j$.  But $a_j$ is a unit in $A/{P_j}^{e_j}A$ and
$\Lambda_\rho[{p_j}^{e_j}]$ is an $(A/{P_j}^{e_j}A)$-module.  Thus
$c_j = 0$ for all $j$ and $\Lambda_\rho[a] =
\oplus^t_{i=1} \Lambda_\rho[{P_i}^{e_i}]$.

Hence by the prime power case, we have
$$\Lambda_\rho[a] = \oplus^t_{i=1} \Lambda_\rho[{P_i}^{e_i}]
\cong \oplus^t_{i=1} (A/{P_i}^{e_i}A)^r \cong (A/aA)^r$$
(using the Chinese remainder theorem).
\hamlos

Let $K_{\rho,a} = k(\Lambda_\rho[a])$.  It was noted above that
$\Lambda_\rho[a]$ is the set of roots of a separable polynomial.  Hence
$K_{\rho,a}/k$ is a Galois extension.  Because $\rho_a(x) \in k[x]$,
$\rho_a(\lambda) = 0$ iff $\rho_a(\sigma\lambda) = 0$ for all
$\sigma \in \Gal(K_{\rho,a}/k)$.  Thus $\rho$ maps $\Lambda_\rho[a]$ into
itself bijectively.  But further, for any $b \in A$,
$\lambda \in \Lambda_\rho[a]$, and $\sigma \in \Gal(K_{\rho,a}/k)$, we
have $\sigma(\rho_b\lambda) = \rho_b(\sigma\lambda)$ because
$\rho_b(x) \in k[x]$.  Thus $\sigma$ acts as an $A$-module automorphism on
$\Lambda_\rho[a]$ and hence as an $A/aA$-module automorphism because $a$
annihilates $\Lambda_\rho[a]$.  Therefore, we get a homomorphism
$$\Gal(K_{\rho,a}/k) \rightarrow \Aut_{A/aA}(\Lambda_\rho[a]).$$

Now if $\sigma \in \Gal(K_{\rho,a}/k)$ maps to the identity, then it fixes
every element in $\Lambda_\rho[a]$.  But these elements generate
$K_{\rho,a}$, so $\sigma$ must be the identity.  Therefore the above map
is injective, and we have established this important fact:
\begin{proposition}
If $K_{\rho,a} = k(\Lambda_\rho[a])$, then $K_{\rho,a}/k$ is Galois and
there is an injection
$$\Gal(K_{\rho,a}/k) \rightarrow GL_r(A/aA).$$
\end{proposition}

\begin{corollary}\labc{abext}
If $\rho$ has rank $1$, then $K_{\rho,a}/k$ is an abelian extension.
\end{corollary}

\pf
This follow immediately because $GL_1(A/aA) = (A/aA)^*$ is abelian and
$\Gal(K_{\rho,a}/k)$ is a subgroup of it.
\hamlos

Therefore, a Drinfeld module (on $\F[t]$ over $\F(t)$) given by $\rho_T =
T + c\tau$ (where $c \in k^*$) produces many examples of abelian
extensions of $\F(t)$.  We will be interested in the special case where
$\rho$ is given by $\rho_T = T + \tau$.  In this case we denote $\rho$ as
$C$ and call it a \emph{Carlitz module}.

\end{section} %Drinfeld Modules Over $\F(T)$



\begin{section}{Carlitz Modules and Their Induced Extensions}

In this section, we will work solely with Carlitz modules and hence will
abbreviate $\Lambda$ for $\Lambda_C = \Lambda_\rho$, $\Lambda_a$ for
$\Lambda_C[a]$, and $K_a$ for $K_{C,a}$.  \refc{abext} shows that
$\Gal(K_a/k)$ is a subgroup of $(A/aA)^*$ for $a \neq 0$.  We will show
that $\Gal(K_a/k) \cong (A/aA)^*$, and along the way we will establish
some properties of Carlitz modules.  We will see that $K_a$ is analogous
to cyclotomic extensions of $\Q$.

First some important remarks about the polynomial $C_a(x)$:
\begin{lemma}\labl{polycoeffs}
For nonzero $a \in A$ of degree $d$, $C_a(x)$ is of the form
$$[a,0]x + [a,1]x^q + \ldots + [a,d]x^{q^d}$$
with $[a,0] = a$, $[a,d] \neq 0$, and each $[a,i] \in A$.  If $a$ is
monic, then $[a,d] = 1$.
\end{lemma}

\pf
Reduce to the case $a = T^n$ and induct on $n$.  The base case works
because $C_T = T + \tau$ is monic in $A\langle\tau\rangle$.  Note that
this lemma fails for any other rank-1 Drinfeld module.
\hamlos

Recall that $\Lambda_a \cong A/aA$ as $A$-modules.  Hence it has
$\Phi(a) = |(A/aA)^*|$ generators.  If $\lambda_a$ is a generator, then
every $\lambda \in \Lambda_a$ is of the form $\phi_b(\lambda_a)$ for some
$b \in A$.  But $\phi_b(x)$ is just a polynomial in $k[x]$, so
$\lambda \in k(\lambda_a)$.  Therefore, $K_a = k(\Lambda_a) =
k(\lambda_a)$.

Let $\oo_a$ denote the integral closure of $A$ in $K_a$.
The integral closure of $A$ in $k$ is $A$ because $A$ is factorial and
hence integrally closed.  Thus $k$ and $A$ are analogous to $\Q$ and
$\Z$, and $(K_a, \oo_a)$ is analogous to a rational cyclotomic field and
its ring of integers.

\begin{lemma}\labl{units}
Suppose $\lambda_a \in \Lambda_a$ is a generator and $b \in A$ is
relatively prime to $a$.  Then $C_b(\lambda_a)/\lambda_a$ is a unit in
$\oo_a$.
\end{lemma}

\pf
It is clear that if $a = \beta a'$ for $a'$ monic and $\beta \in \F^*$,
then $\Lambda_{a'} = \Lambda_a$ and the other corresponding items are
equal.  Thus we may assume that $a$ is monic.

By \refl{polycoeffs}, $C_a(\lambda_a) = 0$ implies $\lambda_a \in \oo_a$.
Further, $C_b(\lambda_a)/\lambda_a$ is a polynomial in $\lambda_a$ with
coefficients in $A$, so it is also in $\oo_a$.  Thus it suffices to check
that $\lambda_a/C_b(\lambda_a)$ is in $\oo_a$.

Because $a$ and $b$ are relatively prime, there exist $f, g \in A$ such
that $fb = 1 + ga$.  Then $C_fC_b = C_1 + C_gC_a$ and $C_f(C_b(\lambda_a))
= \lambda_a + C_g(C_a(\lambda_a)) = \lambda_a$.  Hence
$$\lambda_a/C_b(\lambda_a) = C_f(C_b(\lambda_a))/C_b(\lambda_a).$$
Now $C_b(\lambda_a)$ is a root of $C_a(x)$ and is hence in $\oo_a$.  Thus
similar to above, $C_f(C_b(\lambda_a))/C_b(\lambda_a)$ is a polynomial in
$C_b(\lambda_a)$ with coefficients in $A$; thus $\lambda_a/C_b(\lambda_a)
\in \oo_a$.
\hamlos

With the lemma, we can do the prime power case:
\begin{proposition}\labp{primepower}
Suppose $P \in A$ is monic irreducible and $e \geq 1$.  Then $K_{P^e}$ is
totally ramified with ramification index $\Phi(P^e)$ at the prime $PA$,
but is unramified at all other primes.  Hence
$$[K_{P^e}:k] = \Phi(P^e) \text{ and } \Gal(K_{P^e}/k) \cong (A/P^eA)^*.$$
Finally, the prime ideal above $PA$ is $(\lambda) = \lambda\oo_{P^e}$,
where $\lambda$ is any generator of $\Lambda_{P^e}$.
\end{proposition}

\pf
Let $\lambda$ be a generator of $\Lambda_{P^e}$.  It satisfies the monic
polynomial $C_{P^e}(x) \in A[x]$ and is thus in $\oo_{P^e}$.  Because
$A = \F[T]$ is a factorial integral domain, by Gauss' lemma the minimal
polynomial $g(x)$ of $\lambda$ is a monic polynomial in $A[x]$.  Hence
$C_{P^e}(x) = f(x)g(x)$ where $f(x) \in A[x]$.

By \refl{polycoeffs}, the derivative of $C_{P^e}(x)$ is $P^e$.  Thus
$P^e = C'_{P^e}(\lambda) = f(\lambda)g'(\lambda) + f'(\lambda)g(\lambda) =
f(\lambda)g'(\lambda)$.  Because $\lambda$ is a generator, we showed above
that $K_{P^e} = k(\lambda)$.  Hence $g'(\lambda)$ is contained in the
different $\D$ of $\oo_{P^e}/A$.  Now because $f(x) \in A[x]$ and
$\lambda \in \oo_{P^e}$, we see that $f(\lambda) \in \oo_{P^e}$.  Hence
for any prime ideal $I \subset \oo_{P^e}$,
\begin{eqnarray*}
I \text{ ramifies over }k & \Longrightarrow & I|\D \\
& \Longrightarrow & P^e = f(\lambda)g'(\lambda) \in g'(\lambda)\oo_{P^e}
\subset \D \subset I \\
& \Longrightarrow & P \subset I\cap A \\
& \Longrightarrow & I \text{ is over } PA.
\end{eqnarray*}
Thus $PA$ is the only possible prime ideal in $A$ ramified in $\oo_{P^e}$.

Let $d = \deg(P)$.  Because $\Lambda_{P^e} \cong A/P^eA$, we see that
$\lambda$ is a generator of $\Lambda_{P^e}$ iff $C_{P^e}(\lambda) = 0$
and $C_{P^{e-1}}(\lambda) \neq 0$.  Thus the $\Phi(P^e)$ generators are
all the roots of
\begin{eqnarray*}
\frac{C_{P^e}(x)}{C_{P^{e-1}}(x)} & = &
\frac{C_P(C_{P^{e-1}}(x))}{C_{P^{e-1}}(x)} \\
& = & P + [P,1]C_{P^{e-1}}(x)^{q-1} + \ldots +
[P,d]C_{P^{e-1}}(x)^{q^d-1}
\end{eqnarray*}
(using \refl{polycoeffs}).  Note that this polynomial has degree
$q^{\deg(P^{e-1})}(q^d-1) = q^{d(e-1)}(q^d-1) = \Phi(P^e)$.  This
polynomial and \refl{units} imply that
$$P = \pm \prod_{\text{generator }\lambda} \lambda 
= \lambda^{\Phi(P^e)} \times \text{ unit}.$$
Hence $P\oo_{P^e} = (\lambda)^{\Phi(P^e)}$.

Thus if $I \subset \oo_{P^e}$ is a prime above $PA$, then $I\mid\lambda$
and $e(I/PA)$ is a multiple of $\Phi(P^e)$.  In particular, $[K_{p^e}:k]
\geq e(I/PA) \geq \Phi(P^e)$.  However, since $K_{p^e} = k(\lambda)$ and
$\lambda$ is a root of the degree-$\Phi(P^e)$ polynomial above, we must
have $[K_{p^e}:k] \leq \Phi(P^e)$.  Therefore, we must have equalities and
$[K_{p^e}:k] = e(I/PA) = \Phi(P^e)$.  Since $K_{p^e}/k$ is Galois, we
have $efr = [K_{p^e}:k]$ and hence $f = r = 1$, i.e. $PA$ is totally
ramified and there is only one prime above $PA$.  If $(\lambda)$ factors,
then there would either be more than one prime above $PA$ or a larger
ramification index.  Hence $(\lambda)$ is prime and is the only prime
above $PA$.

Lastly, we already know that $\Gal(K_{P^e}/K)$ is a subgroup of the group
$(A/P^eA)^*$ of order $\Phi(P^e)$.  Because $\Gal(K_{P^e}/K)$ has order
$\Phi(P^e)$, it must be the entire group.
\hamlos

The general case now follows fairly directly:
\begin{theorem}
Suppose $a = \alpha{P_1}^{e_1}{P_2}^{e_2}\ldots{P_t}^{e_t}$ is the prime
decomposition of $a$, where each $P_i$ is monic.  Then $K_a$ is the
compositum of the fields $K_{{P_i}^{e_i}}$.  The only ideals in $A$ that
ramify in $\oo_a$ are the $P_iA$'s.  Furthermore, $[K_a:k] = \Phi(a)$,
from which it follows that $$\Gal(K_a/k) \cong (A/aA)^*.$$
\end{theorem}

\pf
Set $a_i = \frac{m}{\alpha{P_i}^{e_i}}$.  Let $\lambda_a$ be a generator
of $\Lambda_a \cong A/aA$.  Then $\lambda_{{P_i}^{e_i}} =
C_{a_i}(\lambda_a)$ is a generator of $\Lambda_{{P_i}^{e_i}} \cong
a_i(A/aA)$.  Thus $K_{{P_i}^{e_i}} = k(\lambda_{{P_i}^{e_i}}) =
k(C_{a_i}(\lambda_a)) \subset k(\lambda_a) = K_a$, and we have
$\prod^t_{i=1} K_{{P_i}^{e_i}} \subset K_a$.

Conversely, because the $a_i$'s have greatest common divisor 1, there
exist polynomials $b_i \in A$ such that $\sum^t_{i=1}b_ia_i = 1$.  Thus
$$\lambda_a = \sum^t_{i=1}C_{b_i}(C_{a_i}(\lambda_a)) =
\sum^t_{i=1}C_{b_i}(\lambda_{{P_i}^{e_i}}) \in
\prod^t_{i=1} K_{{P_i}^{e_i}}.$$
Since $\lambda_a$ generates $K_a$, we have
$\prod^t_{i=1} K_{{P_i}^{e_i}} = K_a$.

Next, if $PA \subset A$ is a prime different from each $P_iA$, then by
\refp{primepower} it is unramified in every $K_{{P_i}^{e_i}}$.  Hence it
is also unramified in the compositum $K_a$.  Conversely, each $P_iA$ is
ramified in $K_a$ because it is already ramified in $K_{{P_i}^{e_i}}$.

Now we prove that $[K_a:k] = \Phi(a)$ by inducting on $t$.  The base case
$t = 1$ is exactly \refp{primepower}.  If the result holds for $t-1$,
then $[K_{a_t}:k] = \Phi(a_t)$.  But by the above and by
\refp{primepower}, $P_tA$ is unramified in $K_{a_t}$ and totally ramified
in $K_{{P_t}^{e_t}}$.  Hence $K_{a_t}$ and $K_{{P_t}^{e_t}}$ are linearly
disjoint over $K$, and $$[K_a:k] = [K_{a_t}:k][K_{{P_t}^{e_t}}:k]
= \Phi(a_t)\Phi({P_t}^{e_t}) = \Phi(a).$$
Induction is complete.  So once again, because $\Gal(K_a/k)$ is a
subgroup of $(A/aA)^*$ of the correct size, it must be the entire group.
\hamlos

\end{section} %Carlitz Modules and Their Induced Extensions



\begin{section}{The Maximal Abelian Extension of $\F(t)$}

From the preceding discussion, the compositum $k(\Lambda)$ of all the
fields $K_a$ (for nonzero monic $a \in A$) is an abelian extension of $k$.
Further, it is clear that $k(\Lambda) =
\mathop{\varinjlim}_{a\neq 0} K_a$ and that $\Gal(k(\Lambda)/k) \cong
\mathop{\varprojlim}_{a\neq 0} (A/aA)^*$, which is analogous to
$\Gal(\Q(\lambda_{\infty})/\Q) \cong \mathop{\varprojlim}_m (\Z/m\Z)^*
= {\widehat{\Z}}^*$ for number fields.

To construct more abelian extensions, we substitute $1/T$ for $T$
everywhere above and obtain new but isomorphic abelian extensions.  In
particular, $K_{T^{n+1}}$ becomes what we call $k(\Lambda_{T^{-(n+1)}})$,
whose Galois group over $k$ is still of order $\Phi(T^{n+1}) = q^n(q-1)$.
Since $(q^n, q-1) = 1$, the Galois group decomposes uniquely into the
product of groups of order $q^n$ and $q-1$.  Let $L_n$ be the fixed field
of the order-$(q-1)$ subgroup.  Then $L_n/k$ is still an abelian
extension.  Let $L_\infty = \prod^\infty_{n=1} L_n$.

However, we have not yet obtained all the abelian extensions of $k$.  It
can be shown that the constant fields (i.e. the intersection with the
algebraic closure $\bar{\F}$) of $k(\Lambda)$ and $L_{\infty}$ are both
$\F$.  In other words, $k(\Lambda)$ and $L_{\infty}$ are what are called
\emph{geometric} extensions of $k$.  By extending the constant field, we
obtain another abelian extension:

\begin{proposition}
$\Gal(\bar{\F}(T)/\F(T)) \cong \widehat{\Z}$.
\end{proposition}

\pf
We claim that $\Gal(\bar{\F}(T)/\F(T)) \cong \Gal(\bar{\F}/\F)$.  Any
element $\sigma \in \Gal(\bar{\F}/\F)$ extends to an element in
$\Gal(\bar{\F}(T)/\F(T))$ by setting $\sigma(T) = T$.  Conversely,
if $\sigma \in \Gal(\bar{\F}(T)/\F(T))$ and $x \in \bar{\F}$, then
$x$ is algebraic over $\F$ (which is fixed by $\sigma$), so
$\sigma(x)$ is also algebraic over $\F$ and $\sigma(x) \in \bar{\F}$.
Thus $\sigma$ sends $\bar{\F}$ to itself, and $\sigma$ is the
extension of an element of $\Gal(\bar{\F}/\F)$.

Now $\F$ is a finite field of size $q$.  Hence
$\bar{\F} \cong \varinjlim_d \F_{q^d}$ and
$\Gal(\bar{\F}(T)/\F(T)) \cong \Gal(\bar{\F}/\F) \cong
\varprojlim_d \Z/d\Z = \widehat{\Z}$.
\hamlos

Now we can state the function-field analogue of the Kronecker-Weber
theorem:
\begin{theorem}
(Hayes) $k(\Lambda)$, $L_\infty$, and $\bar{\F}(T)$ are linearly
disjoint over $k$ and generate the maximal abelian extension of $k$.
\end{theorem}
The proof uses class field theory and can be seen in [2] or [3].

\end{section} %The Maximal Abelian Extension of $\F(t)$




\begin{section}{Drinfeld Modules}

Finally, we examine Drinfeld modules, the generalization of Carlitz
modules over function fields other than $\F(t)$, and suggest how they
relate to abelian extensions.  We begin with some preliminary setup and
definitions.

\begin{definition}
A function field $k$ over $\F$ is an extension of $\F$ of transcendence
degree $1$.
\end{definition}
\begin{definition}
A prime of a function field $k/\F$ is a discrete valuation ring $R$
with maximal ideal $P$ such that $\F \subset R \subset k$ and $R$ has
quotient field $k$.  This prime is referred to as $P$, and the associated
valuation is denoted $\ord_P(\cdot)$.
\end{definition}

Now let $\F = \F_q$ and let $k/\F$ be a function field with constant field
$\F$.  Let $\infty$ be a fixed prime of $k$ and let $A$ be the ring of
elements of $k$ whose only poles are at $\infty$ (i.e. have nonnegative
valuations at all other primes).  Let $L$ be a field containing $\F$.  As
before, regard $\tau$ as the map raising an element to the $q$th power,
and let $L\langle\tau\rangle$ be the twisted polynomial ring with relation
$\tau a = a^q\tau$ for $a \in L$.  Fix an $\F$-algebra homomorphism
$\delta: A \rightarrow L$.  Then
\begin{definition}
A Drinfeld $A$-module over $L$ is an $\F$-algebra homomorphism
$\rho: A \rightarrow L\langle\tau\rangle$ such that for all $a \in A$,
the constant coefficient of $\rho_a$ is $\delta(a)$.  Also, we require
that the image of $\rho$ not be contained in $L$.  The set of all such
$\rho$ is denoted $\Drin_A(L)$.
\end{definition}

Similar to before, if $\Omega$ is an $L$-algebra, then a Drinfeld module
makes $\Omega$ into an $A$-module $\Omega_\rho$ by setting
$a\cdot u = \rho_a(u)$ for $a \in A$, $u \in \Omega$.  This construction
applies to $M$, a fixed algebraically closed field extension of $L$.  Thus
we can define the torsion submodule $M_\rho[a] = \left\{u \in M_\rho:
\rho_a(u) = 0 \right\}$ as before.

Note that the above definition is consistent with our prior definition of
Drinfeld modules over $\F(T)$.  There $k = \F(T)$, $\infty$ is the prime at
infinity, $A = \F[T]$, $L = k$, and $\delta$ is the inclusion.  There is
something to check here, namely that $\F[T]$ is the set of elements whose
only poles are at $\infty$.  Now by $\infty$ we mean the valuation where
for polynomials $f,g\in \F[T]$, $\ord_\infty(f/g)$ is the highest exponent
in $g$ minus the highest exponent in $f$.  It can be shown that the other
primes of $k$ correspond to nonzero prime ideals $P \subset \F[T]$, with
valuation $\ord_P(z)$ being the exponent of $P$ in the prime factorization
of $z$.  Hence $\F[T]$ is indeed the set of elements regular at all the
non-$\infty$ primes.

In the case over $\F(T)$, giving a Drinfeld module is the same as setting
$\rho_T$ to be a nonconstant polynomial with constant $T$.  The existence
of Drinfeld modules in general is less clear, but it can be shown using
analytic methods.  The construction is similar to the construction of
elliptic curves over the complex numbers $\C$ by means of two-dimensional
$\Z$-lattices and the associated Weierstrass ${\mathcal P}$-functions.

Now in general $\delta$ need not be injective.  Let $Q = \ker(\delta)$.
Then $Q$ is a prime ideal of $A$ because $A/Q$ is contained in the
integral domain $L$.  We say that $Q$ is the \emph{$A$-characteristic of
$L$}.  Then the following theorem defines the rank of a Drinfeld module:
\begin{theorem}
Let $\rho \in \Drin_A(L)$, let $Q$ be the $A$-characteristic of $L$, and
let $M$ be an algebraically closed field containing $L$.  If $P \neq Q$
is a nonzero prime ideal of $A$ and $e \geq 1$, then there exist a
positive integer $r$ independent of $P$ and $e$ such that
$$M_\rho[P^e] \cong (A/P^e)^r.$$
If $Q \neq (0)$ and $e \geq 1$, then there exist an integer $h$,
independent of $e$, such that
$$M_\rho[Q^e] \cong (A/Q^e)^{r-h}.$$
We say that $r$ and $h$ are the rank and height of $\rho$, respectively.
\end{theorem}

\begin{corollary}
This definition of rank coincides with the previous definition over
$\F(T)$.  
\end{corollary}

\pf
Use \reft{rank}.
\hamlos

\begin{proposition}
Let $r = \rnk(\rho)$.  Then for each $a \in A\setminus Q$, there is an
injection $$\Gal(L(M_\rho[a])/L) \rightarrow GL_r(A/aA).$$
\end{proposition}

\pf
This is analogous to the proof in the case of $\F(T)$.  Because
$a \not\in \ker(\delta)$, the constant term of $\rho_a$ is nonzero.
Hence for some polynomial $f(x) = c_0x + c_1x^q + \ldots + c_tx^{q^t}
\in L[x]$ where $c_0 \neq 0$, we have $\rho_a(u) = f(u)$ for all
$u \in M$.  Because $M$ is algebraically closed, $M_\rho[a]$ is just the
set of roots of $f$, and $L(M_\rho[a])$ is $f$'s splitting field.  Now
since $\F \subset L$, the characteristic of $L$ divides $q$ and
$f'(x) = c_0 \in L^*$.  Thus $f$ is separable and $L(M_\rho[a])/L$ is
Galois.

By arguments similar to before, each $\rho \in \Gal(L(M_\rho[a])/L)$ acts
on $M_\rho[a] \cong (A/aA)^r$ as an $A$-module (hence $(A/aA)$-module)
isomorphism.  This defines a homomorphism \\
$\Gal(L(M_\rho[a])/L) \rightarrow GL_r(A/aA)$ that is injective because
$M_\rho[a]$ generates $L(M_\rho[a])$.
\hamlos

\begin{corollary}
If $\rho$ is a rank-$1$ Drinfeld $A$-module over $L$ and $a \in A$ is not
in the $A$-characteristic of $L$, then $L(M_\rho[a])/L$ is an abelian
extension.
\end{corollary}

Thus we see that adjoining torsion points of a Drinfeld module creates
abelian extensions.  This is the start of an explicit construction of
the maximal abelian extension of a function field.  For more details,
see [1] and [4].

\end{section} %Drinfeld Modules



\begin{thebibliography}{1000}

\bibitem{drinfeld} V.G. Drinfeld.  Elliptic Modules (Russian), \textit{
Math. Sbornik} {\bf 94} (1974), 597-627.  English translation: \textit{
Math. USSR, Sbornik} {\bf 23} (1977), 159-170.

\bibitem{hayes} D.R. Hayes.  A brief introduction to Drinfeld modules,
in The Arithmetic of Function Fields (eds. D. Goss et al), Walter de
Gruyter \& Co., New York-Berlin, 1992.

\bibitem{hayes2} D.R. Hayes.  Explicit class field theory for rational
function fields, \textit{Trans. Amer. Math. Soc.} {\bf 189} (1974), 77-91.

\bibitem{hayes3} D.R. Hayes.  Explicit class field theory in global
function fields, Studies in Algebra and Number Theory, \textit{Adv. Math.
Supplementary Studies} {\bf 6} (1979), 173-217.

\bibitem{rosen} M. Rosen.  Number Theory in Function Fields.
Springer-Verlag, New York, 2002.

\end{thebibliography}

\end{document}
