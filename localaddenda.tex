\documentclass[12pt]{article}
\usepackage{amsfonts, amsthm, amsmath}
\usepackage[all]{xy}

\setlength{\textwidth}{6.5in}
\setlength{\oddsidemargin}{0in}
\setlength{\textheight}{8.5in}
\setlength{\topmargin}{0in}
\setlength{\headheight}{0in}
\setlength{\headsep}{0in}
\setlength{\parskip}{0pt}
\setlength{\parindent}{20pt}

\def\kbar{\overline{k}}
\def\AA{\mathbb{A}}
\def\CC{\mathbb{C}}
\def\FF{\mathbb{F}}
\def\NN{\mathbb{N}}
\def\PP{\mathbb{P}}
\def\QQ{\mathbb{Q}}
\def\RR{\mathbb{R}}
\def\ZZ{\mathbb{Z}}
\def\gotha{\mathfrak{a}}
\def\gothb{\mathfrak{b}}
\def\gothm{\mathfrak{m}}
\def\gotho{\mathfrak{o}}
\def\gothp{\mathfrak{p}}
\def\gothq{\mathfrak{q}}
\def\gothr{\mathfrak{r}}
\DeclareMathOperator{\ab}{ab}
\DeclareMathOperator{\coker}{coker}
\DeclareMathOperator{\cyc}{cyc}
\DeclareMathOperator{\disc}{Disc}
\DeclareMathOperator{\Frob}{Frob}
\DeclareMathOperator{\Gal}{Gal}
\DeclareMathOperator{\GL}{GL}
\DeclareMathOperator{\Hom}{Hom}
\DeclareMathOperator{\im}{im}
\DeclareMathOperator{\Ind}{Ind}
\DeclareMathOperator{\Inf}{Inf}
\DeclareMathOperator{\inv}{inv}
\DeclareMathOperator{\Norm}{Norm}
\DeclareMathOperator{\Res}{Res}
\DeclareMathOperator{\Trace}{Trace}
\DeclareMathOperator{\unr}{unr}
\DeclareMathOperator{\Ver}{Ver}
\DeclareMathOperator{\Cl}{Cl}

\def\head#1{\medskip \noindent \textbf{#1}.}

\newtheorem{theorem}{Theorem}
\newtheorem{lemma}[theorem]{Lemma}
\newtheorem{prop}[theorem]{Proposition}
\newtheorem{cor}[theorem]{Corollary}

\begin{document}

\begin{center}
\bf
Math 254B, UC Berkeley, Spring 2002 (Kedlaya) \\
Local Class Field Theory: Addenda
\end{center}

\head{Reference} Neukirch, IV.4-IV.6; also V.1 for the first section. 

\head{The local existence theorem}
We've seen two ways to derive the local reciprocity law, but I somehow
managed not to prove the local existence theorem. Fortunately, it's easy.
(Again, this argument only works in characteristic 0.)
\begin{theorem}
Let $K$ be a finite extension of $\QQ_p$. Then every open subgroup of $K^*$
of finite index is of the form $\Norm_{L/K} L^*$ for some finite (abelian)
extension $L$ of $K$.
\end{theorem}
Recall that $\Norm_{L/K} L^* = \Norm_{M/K} M^*$ for $M = L \cap K^{\ab}$
by the local reciprocity law, so we don't have to make sure $L$ is abelian.
\begin{proof}
Let $N$ be an open subgroup of $K^*$ of index $n$. Then $(K^*)^n \subseteq N$,
and it suffices to prove that $(K^*)^n$ contains $\Norm_{L/K} L^*$ for
some $L$. Let $\zeta_n$ be a primitive $n$-th root of unity, let $\mu_n$
be the group generated by $\zeta_n$, and put
$K_1 = K(\zeta_n)$. By Kummer theory, there is a canonical isomorphism
between $\Hom(\Gal(K_1^{\ab}/K_1), \mu_n)$ and $K_1^*/(K_1^*)^n$. But the latter
is finite for any local field $K_1$ (exercise), so if we let $L$ be the
compositum of all $\ZZ/n\ZZ$-extensions of $K_1$, then $L/K_1$ is a finite
extension. Now $\Gal(L/K_1) \cong K_1^*/\Norm_{L/K_1} L^*$ is a group of
exponent $n$, so $(K_1^*)^n \subseteq \Norm_{L/K_1} L^*$. On the other hand,
\[
\#K_1^*/(K_1^*)^n = \#\Gal(L/K_1) = \#K_1^*/\Norm_{L/K_1} L^*,
\]
so $(K_1^*)^n = \Norm_{L/K_1} L^*$. In particular, $(K^*)^n$ contains
$\Norm_{K_1/K} (K_1^*)^n = \Norm_{L/K} L^*$, as desired.
\end{proof}

\head{More on abstract class field theory}

First of all, the claim that $\Gal(\QQ^{\ab}/\QQ)$ is isomorphic to
$\widehat{\ZZ}$ is false; that group is of course $\widehat{\ZZ}^*$. But
it has a natural quotient isomorphic to $\widehat{\ZZ}$; it's easiest to
construct this factor by factor, by exhibiting a surjection $\ZZ_p^* \to
\ZZ_p$. (Otherwise, put,
$\QQ(\zeta_{p^\infty})/\QQ$ has a subextension with Galois group $\ZZ_p$.)
For $p \neq 2$, $\ZZ_p^*$ can be written as the product of the group of
$(p-1)$-st roots of unity with the subgroup $1+p\ZZ_p$, and the latter
is isomorphic to $p\ZZ_p$ via the logarithm map. For $p=2$,
$\ZZ_p^*$ can be written as the product of $\{\pm 1\}$ with the subgroup
$1+4\ZZ_2$, which again is isomorphic to $4\ZZ_2$ via the logarithm map.

Second, the proof of the multiplicativity of the reciprocity map was
seriously broken. Having finally deciphered the proof in Neukirch, I'll
give a version of that instead.

Recall notation: $L/K$ is a finite extension of finite extensions of $k$,
$H$ is the semigroup of $g \in \Gal(L^{\unr}/K)$ such that $d_K(g) \in \NN$
(that's positive integers, not nonnegative), and $r': H \to A_K/\Norm_{L/K} A_L$
is defined as follows. For $g \in \Gal(L^{\unr}/K)$, let $M$ be the fixed
field of $g$, let $\pi_M$ be a uniformizer of $M$, and set
$r'(g) = \Norm_{M/K}(\pi_M)$.

Given $g_1, g_2 \in H$, set $g_3 = g_1g_2$; we want to show that
$r'(g_1)r'(g_2) = r'(g_3)$. Let $M_i$ be the fixed field of $g_i$, let
$\pi_i$ be a uniformizer of $\pi_i$, and put $\rho_i = r(g_i)
= \Norm_{M_i/K}(\pi_i)$; since $v_K(\rho_i) = d_K(g_i)$, the ratio
$\rho_1 \rho_2/\rho_3$ is a unit; we want to show
that it is in $\Norm_{L/K}(U_L)$.

Again, the trouble with $\rho_1\rho_2/\rho_3$ is that it is made up of
norms from different fields, and to make progress we need to rewrite it
more uniformly. Choose $\phi \in \Gal(L^{\unr}/K)$ with $d_K(\phi)=1$,
and put $d_i = d_K(g_i)$; then we can write $g_i = \phi^{d_i} h_i^{-1}$
for some $h_i$ with $d_K(h_i) = 0$, that is, $h_i \in \Gal(L^{\unr}/K^{\unr})$.

\begin{prop}
Let $M$ be the fixed field of some $h \in \Gal(L^{\unr}/K)$ with
$d_K(h)=n$ a nonnegative integer, and suppose $\phi \in H$ satisfies
$d_K(\phi)=1$. Then for any $x \in A_M$,
\[
\Norm_{M/K}(x) = \Norm_{L^{\unr}/K^{\unr}}(x x^{\phi}\cdots x^{\phi^{n-1}}).
\]
\end{prop}
Put
\[
\sigma_i = \pi_i \pi_i^{\phi} \cdots \phi_i^{\phi^{d_i-1}};
\]
then if $u = \sigma_1 \sigma_2/\sigma_3$, we have $u \in U_{L^{\unr}}$
and $\rho_1\rho_2/\rho_3 = \Norm_{L^{\unr}/K^{\unr}}(u)$. Now
\[
  \frac{u^\phi}{u} = \frac{\sigma_1^\phi \sigma_2^\phi \sigma_3}{\sigma_1
\sigma_2 \sigma_3^\phi}
= \frac{\pi_1^{\phi^{d_1}} \pi_2^{\phi^{d_2}} \pi_3}{\pi_1 \pi_2
\pi_3^{\phi^{d_3}}}
= \frac{\pi_1^{h_1} \pi_2^{h_2} \pi_3}{\pi_1 \pi_2 \pi_3^{h_3}}
\]
since $\pi_i$ is fixed by $g_i = \phi^{d_i} h_i^{-1}$.
The relationship among the $h_i$ is $\phi^{d_1} h_1^{-1} \phi^{d_2}
h_2^{-1} = \phi^{d_3} h_3^{-1}$, that is,
$h_3 = h_2 \phi^{-d_2} h_1 \phi^{d_2}$.

Pick a single uniformizer $\pi_L$ of $L$; then we can write $\pi_i = v_i
\pi_L$ for some unit $v_i \in L^{\unr}$, and
\[
\frac{u^\phi}{u} = \frac{v_1^{h_1} v_2^{h_2} v_3}{v_1 v_2 v_3^{h_3}}
\frac{\pi_L^{h_1} \pi_L^{h_2}}{\pi_L \pi_L^{h_3}}.
\]
The first term is clearly a product of expressions of the form
$y_i^{g_i}/y_i$ for $y_i \in U_{L^{\unr}}$ and $g_i \in
\Gal(L^{\unr}/K^{\unr})$. On the other hand, since
$\pi_L$ and $\pi_L^{h_1}$ are invariant under $\phi$, we get
\begin{align*}
\frac{\pi_L^{h_1} \pi_L^{h_2}}{\pi_L \pi_L^{h_3}}
&= \frac{\pi_L^{h_1} \pi_L^{h_2}}{\pi_L^{\phi^{d_2}}
\pi_L^{h_2 \phi^{-d_2} h_1 \phi^{d_2}}} \\
&= \frac{\pi_L^{h_2}}{\pi_L^{d_2}}
\left( \frac{\pi_L^{d_2}}{\pi_L^{h_2}} \right)^{\phi^{-d_2} h_1 \phi^{d_2}}.
\end{align*}
Thus $\frac{u^{\phi}}{u}$ is the product of expressions of the form
$y_i^{g_i}/y_i$ for $y_i \in U_{L^{\unr}}$ and $g_i \in
\Gal(L^{\unr}/K^{\unr})$. What we need to show is that
$\Norm_{L^{\unr}/K^{\unr}}(U_L) \in \Norm_{L/K}(U_L)$; that will follow
from the following lemma.

\begin{lemma}
If $x \in U_{L^{\unr}}$ has the property that $x^\phi/x
= \prod_i y_i^{g_i}/y_i$ for some $y_i \in U_{L^{\unr}}$ and $g_i \in
\Gal(L^{\unr}/K^{\unr})$, then $\Norm_{L^{\unr}/K^{\unr}}(x)$ belongs to
$\Norm_{M/K} U_M$ for every finite subextension $M$ of $L^{\unr}/K$.
\end{lemma}
\begin{proof}
(This is Neukirch, Lemma IV.5.4.)
It suffices to consider $M$ containing $x$ and each $y_i$,
and also containing $L$.
Let $n = [M:K]$, and let $F$ be the fixed field of $\sigma = \phi^n$.
Let $G$ be the degree $n$ unramified extension of $G$, i.e., the fixed
field of $\sigma^n$. By the vanishing of $H^0_T(\Gal(G/F), U_G)$, we can
write $x = \Norm_{G/F}(\tilde{x})$ and $y_i = \Norm_{G/F}(\tilde{y}_i)$
for some $\tilde{x}, \tilde{y_i} \in U_G$. Put
\[
\alpha = \frac{\tilde{x}^{\phi}}{\tilde{x}} \prod_i \frac{\tilde{y}_i}{\tilde{y}_i}^{g_i};
\]
then $\Norm_{G/F}(\alpha) = 1$. By the vanishing of $H^{-1}_T(\Gal(G/F), U_G)$,
that means $\alpha = w^\sigma/w$ for some $w \in G$. That means we can write
\[
\frac{\tilde{x}^\phi}{\tilde{x}}
= \frac{(w w^\phi\cdots w^{\phi^{n-1}})^\phi}{w w^\phi \cdots w^{\phi^{n-1}}}
\prod_i \frac{\tilde{y_i}^{g_i}}{\tilde{y_i}}.
\]
Apply $\Norm_{L^{\unr}/K^{\unr}}$ to both sides. This kills off
$\tilde{y_i}^{g_i}/\tilde{y_i}$ for each $i$, leaving
\[
\Norm_{L^{\unr}/K^{\unr}}(\tilde{x})^{\phi-1}
= \Norm_{L^{\unr}/K^{\unr}}(w w^\phi\cdots w^{\phi^{n-1}})^{\phi-1}.
\]
Put $z = \Norm_{L^{\unr}/K^{\unr}}(\tilde{x} / (w w^{\phi} \cdots w^{\phi^{n-1}}))$; then the previous equation reads $z^\phi = z$, which means 
$z \in U_K$. We then have
\begin{align*}
  \Norm_{L^{\unr}/K^{\unr}}(x) &= \Norm_{L^{\unr}/K^{\unr}}(
\Norm_{G/F}(\tilde{x})) \\
&= \Norm_{L^{\unr}/K^{\unr}}(\tilde{x} \tilde{x}^{\phi^n} \cdots
\tilde{x}^{\phi^{n(n-1)}}) \\
&= \Norm_{L^{\unr}/K^{\unr}}(z z^{\phi^n} \cdots z^{\phi^{n(n-1)}})
\Norm_{L^{\unr}/K^{\unr}}(w w^{\phi} \cdots w^{\phi^{n^2-1}}) \\
&= z^n \Norm_{L^{\unr}/K^{\unr}}(t t^\phi \cdots t^{\phi^{n-1}}),
\end{align*}
if we put $t = w w^{\phi^n} \cdots w^{\phi^{n(n-1)}}$. Now applying the
above proposition, and noting that trivially $z^n = \Norm_{F/K}(z)$,
we get $\Norm_{L^{\unr}/K^{\unr}}(x) = \Norm_{M/K}(z) \Norm_{F/K}(t)
\in \Norm_{M/K} U_M$, as desired.
\end{proof}


\end{document}



