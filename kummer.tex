%\documentclass[12pt]{article}
%\usepackage{amsfonts, amsthm, amsmath}
%
%\setlength{\textwidth}{6.5in}
%\setlength{\oddsidemargin}{0in}
%\setlength{\textheight}{8.5in}
%\setlength{\topmargin}{0in}
%\setlength{\headheight}{0in}
%\setlength{\headsep}{0in}
%\setlength{\parskip}{0pt}
%\setlength{\parindent}{20pt}
%
%\def\CC{\mathbb{C}}
%\def\FF{\mathbb{F}}
%\def\PP{\mathbb{P}}
%\def\QQ{\mathbb{Q}}
%\def\RR{\mathbb{R}}
%\def\ZZ{\mathbb{Z}}
%\def\gotha{\mathfrak{a}}
%\def\gothb{\mathfrak{b}}
%\def\gothm{\mathfrak{m}}
%\def\gotho{\mathfrak{o}}
%\def\gothp{\mathfrak{p}}
%\def\gothq{\mathfrak{q}}
%\DeclareMathOperator{\disc}{Disc}
%\DeclareMathOperator{\Gal}{Gal}
%\DeclareMathOperator{\GL}{GL}
%\DeclareMathOperator{\Hom}{Hom}
%\DeclareMathOperator{\Norm}{Norm}
%\DeclareMathOperator{\Trace}{Trace}
%\DeclareMathOperator{\Cl}{Cl}
%
%\def\head#1{\medskip \noindent \textbf{#1}.}
%
%\newtheorem{theorem}{Theorem}
%\newtheorem{lemma}[theorem]{Lemma}
%
%\begin{document}
%
%\begin{center}
%\bf
%Math 254B, UC Berkeley, Spring 2002 (Kedlaya) \\
%Kummer Theory
%\end{center}

\head{Reference} Serre, \emph{Local Fields}, Chapter X;
Neukirch section IV.3; or just about
any advanced algebra text (e.g., Lang's \emph{Algebra}). The last lemma is 
from Washington, \emph{Introduction to Cyclotomic Fields}, Chapter 14.

\head{Jargon watch} If $G$ is a group, a \emph{$G$-extension} of a field
$K$ is a Galois extension of $K$ with Galois group $G$.

\medskip
Before attempting to classify all abelian extensions of $\QQ_p$,
we recall an older classification result. This result will
continue to be useful as we proceed to class field theory in general, and 
the technique in its proof prefigures the role to be played
by group cohomology down the line. So watch carefully!

A historical note (due to Franz Lemmermeyer): while the idea of studying field extensions generated by radicals was used extensively by Kummer in his work on Fermat's Last Theorem,
the name \emph{Kummer theory} for the body of results described in this chapter was first applied somewhat later by Hilbert in his \textit{Zahlbericht}, a summary of algebraic number theory as of the end of the 19th century.

\label{T:local Kronecker-Weber}

\begin{theorem} \label{T:Kummer}
If $\zeta_n \in K$, then every $\ZZ/n\ZZ$-extension of $K$ is of the form
$K(\alpha^{1/n})$ for some $\alpha \in K^*$ with the property that
$\alpha^{1/d} \notin K$ for any proper divisor $d$ of $n$, and vice versa.
\end{theorem}

Before describing the proof of Theorem~\ref{T:Kummer}, let me introduce some terminology which marks the tip of the iceberg of group cohomology, which we will see more of later.

If $G$ is a group and $M$ is an abelian group on which $G$ acts
(written multiplicatively),
one defines the group $H^1(G,M)$ as the set of functions $f:
G \to M$ such that $f(gh) = f(g)^h f(h)$, modulo the set of such
functions of the form $f(g) = x (x^g)^{-1}$ for some $x \in M$.

\begin{lemma}[``Theorem 90''] \label{L:theorem 90}
Let $L/K$ be a finite Galois extension with Galois group $G$.
Then $H^1(G, L^*) = 0$.
\end{lemma}
The somewhat unusual common name for this result exists because in the special case where $G$ is cyclic, this statement occurs as Theorem (Satz) 90 in Hilbert's \textit{Zahlbericht}. The general case first appears in Emmy Noether's 1933 paper on the principal ideal theorem (Theorem~\ref{T:principal ideal theorem}), where Noether attributes it to Andreas Speiser.

\begin{proof}
Let $f$ be a function of the form described above.
By the linear independence of automorphisms (see exercises),
there exists $x \in L$ such that $t = \sum_{g \in G} x^g f(g)$
is nonzero. But now
\[
t^h = \sum_{g \in G} x^{gh} f(g)^h =
\sum_{g \in G} x^{gh} f(gh) f(h)^{-1}
= f(h)^{-1} t.
\]
Thus $f$ is zero in $H^1(G,L^*)$.
\end{proof}

\begin{proof}[Proof of Kummer's Theorem]
On one hand, if $\alpha \in K^*$ is such that $\alpha^{1/d} \notin K$
for any proper divisor $d$ of $n$, then the polynomial $x^n - \alpha$
is irreducible over $K$, and every automorphism must have the form
$\alpha \mapsto \alpha \zeta_n^r$ for some $r \in \ZZ/n\ZZ$. Thus
$\Gal(K(\alpha^{1/n})/K) \cong \ZZ/n\ZZ$.

On the other hand,
let $L$ be an arbitrary
$\ZZ/n\ZZ$-extension of $K$. Choose a generator $g \in \Gal(L/K)$,
and let $f: \Gal(L/K) \to L^*$ be the map that sends $rg$ to $\zeta_n^r$
for $r \in \ZZ$.
Then $f \in H^1(\Gal(L/K), L^*)$, so there exists $t \in L$ such that
$t^{rg}/t = f(rg) = \zeta_n^r$ for $r \in \ZZ$. In particular,
$t^n$ is invariant under $\Gal(L/K)$, so $t^n = \alpha$ for some
$\alpha \in K$ and $L = K(t) = K(\alpha^{1/n})$, as desired.
\end{proof}

Another way to state Kummer's theorem is as a bijection
\[
\mbox{$(\ZZ/n\ZZ)^r$-extensions of $K$} \longleftrightarrow
\mbox{$(\ZZ/n\ZZ)^r$-subgroups of $K^*/(K^*)^n$},
\]
where $(K^*)^n$ is the group of $n$-th powers in $K^*$.
(What we proved above was the case $r=1$, but the general case follows easily.)
Another way is in terms of the absolute Galois group of $K$.
Define the \emph{Kummer pairing}
\[
\langle \cdot, \cdot \rangle:
\Gal(\overline{K}/K) \times K^* \to \{1, \zeta_n, \dots, \zeta_n^{n-1} \}
\]
as follows: given $\sigma \in \Gal(\overline{K}/K)$
and $z \in K^*$, choose $y \in \overline{K}^*$ such that $y^n = z$,
and put $\langle \sigma, z \rangle = y^\sigma/y$. Note that this does not
depend on the choice of $y$: the other possibilities are $y \zeta_n^k$
for $k=0, \dots, n-1$, and $\zeta_n^\sigma = \zeta_n$ by the assumption
on $K$, so it drops out.

\begin{theorem}[Kummer reformulated] \label{T:Kummer reformulated}
The Kummer pairing induces an isomorphism
\[
K^*/(K^*)^n \to \Hom(\Gal(\overline{K}/K), \ZZ/n\ZZ).
\]
\end{theorem}
\begin{proof}
The map comes from the pairing; we have to check that it is injective and
surjective. If $y \in K^* \setminus (K^*)^n$, then $K(y^{1/n})$ is a nontrivial
Galois extension of $K$, so there exists some element of
$\Gal(K(y^{1/n})/K)$ that doesn't preserve $y^{1/n}$. Any lift of
that element to $\Gal(\overline{K}/K)$ pairs with $y$ to give something
other than 1; that is, $y$ induces a nonzero homomorphism of
$\Gal(\overline{K}/K)$ to $\ZZ/n\ZZ$. Thus injectivity follows.

On the other hand, suppose $f: \Gal(\overline{K}/K) \to \ZZ/n\ZZ$ is a
homomorphism whose image is the cyclic subgroup of $\ZZ/n\ZZ$ of order $d$.
Let $H$ be the kernel of $f$; then the fixed field $L$ of $H$ is
a $\ZZ/d\ZZ$-extension of $K$ with Galois group $\Gal(\overline{K}/K)/H$.
By Kummer theory, $L = K(y^{1/d})$ for some $y$. But now the homomorphisms
induced by $y^{mn/d}$, as $m$ runs over all integers coprime to $d$,
give all possible homomorphisms of $\Gal(\overline{K}/K)/H$ to $\ZZ/d\ZZ$,
so one of them must equal $f$. Thus surjectivity follows.
\end{proof}

But what about $\ZZ/n\ZZ$-extensions of a field that does not contains
$\zeta_{n}$? These are harder to describe, and indeed describing such 
extensions of $\QQ$ is the heart of this course. There is one thing
one can say: if $L/K$ is a $\ZZ/n\ZZ$-extension, then $L(\zeta_n)/K(\zeta_n)$ is a $\ZZ/d\ZZ$ extension for some divisor $d$ of $n$, and
the latter is a Kummer extension.
\begin{lemma} \label{L:Kummer Galois criterion}
Let $n$ be a prime (or an odd prime power),
let $K$ be a field of characteristic coprime to $n$, let $L = K(\zeta_n)$,
and let $M = L(a^{1/n})$ for some $a \in L^*$. Define the homomorphism
$\omega: \Gal(L/K) \to (\ZZ/n\ZZ)^*$ by the relation
$\zeta_n^{\omega(g)} = \zeta_n^g$. Then $M/K$ is Galois and
abelian if and only if
\begin{equation} \label{eq}
a^g / a^{\omega(g)} \in (L^*)^n \qquad \forall g \in \Gal(L/K).
\end{equation}
\end{lemma}
Note that $\omega(g)$ is only defined up to adding a multiple of $n$,
so $a^{\omega(g)}$ is only defined up to an $n$-th power, i.e., modulo
$(L^*)^n$. (In fact, we will only use one of the implications: if $M/K$ is Galois and abelian, then \eqref{eq} holds. However, we include both implications for completeness.)

\begin{proof}
If $a^g/a^{\omega(g)} \in (L^*)^n$ for all $g \in \Gal(L/K)$,
then $a$, $a^{\omega(g)}$ and $a^g$ all generate the same subgroup
of $(L^*)/(L^*)^n$. Thus $L(a^{1/n}) = L((a^g)^{1/n})$ for all $g \in
\Gal(L/K)$, so $M/K$ is Galois. Thus it suffices to assume $M/K$ is
Galois, then prove that $M/K$ is abelian if and only if (\ref{eq}) holds.
In this case, we must have $a^g/a^{\rho(g)} \in (M^*)^n$ for some map
$\rho: \Gal(L/K) \to (\ZZ/n\ZZ)^*$, whose codomain is cyclic by our
assumption on $n$.

Note that $\Gal(M/K)$ admits a homomorphism $\omega$ to a cyclic group whose
kernel $\Gal(M/L) \subseteq \ZZ/n\ZZ$ is also abelian. Thus $\Gal(M/K)$
is abelian if and only if $g$ and $h$ commute for any $g \in \Gal(M/K)$
and $h \in \Gal(M/L)$, i.e., if $h = g^{-1}hg$.
(Since $g$ commutes with powers of itself, $g$ then
commutes with everything.)

Let $A \subseteq L^*/(L^*)^n$ be the subgroup generated by $a$. Then
the Kummer pairing gives rise to a pairing
\[
\Gal(M/L) \times A \to \{1, \zeta_{n}, \dots, \zeta_n^{n-1}\}
\]
which is bilinear and nondegenerate, so $h = g^{-1}hg$ if and only if
$\langle h, s^g \rangle = \langle ghg^{-1}, s^g \rangle$ for all
$s \in A$. But the Kummer pairing is \emph{equivariant} with respect
to $\Gal(L/K)$ as follows:
\[
\langle h,s \rangle^g = \langle g^{-1}hg, s^g \rangle,
\]
because
\[
\left( \frac{(s^{1/n})^h}{s^{1/n}} \right)^g
= \frac{((s^g)^{1/n})^{g^{-1}hg}}{(s^g)^{1/n}}.
\]
(Here by $s^{1/n}$ I mean an arbitrary $n$-th root of $s$ in $M$,
and by $(s^g)^{1/n}$ I mean $(s^{1/n})^g$. Remember that the value of the
Kummer pairing doesn't depend on which $n$-th root you choose.)
Thus $h = ghg^{-1}$ if and only if $\langle h,s^g \rangle =
\langle h,s \rangle^g$ for all $s \in A$, or equivalently,
just for $s=a$.
But
\[
\langle h,a \rangle^g = \langle h,a \rangle^{\omega(g)}
= \langle h, a^{\omega(g)} \rangle.
\]
Thus $g$ and $h$ commute if and only if $\langle h, a^g \rangle
= \langle h, a^{\omega(g)}\rangle$, if and only if (by nondegeneracy)
$a^g/a^{\omega(g)} \in (L^*)^n$, as desired.
\end{proof}

\head{Exercises}

\begin{enumerate}
\item
Prove the linear independence of automorphisms: if $g_1, \dots, g_n$
are distinct automorphisms of $L$ over $K$, then there do not exist
$x_1, \dots, x_n \in L$ such that $x_1 y^{g_1} + \cdots + x_n y^{g_n} = 0$
for all $y \in L$. (Hint: suppose the contrary, choose a counterexample
with $n$ as small as possible, then make an even smaller counterexample.)
\item
Prove the additive analogue of Theorem 90: if $L/K$ is a finite
Galois extension with Galois group $G$, then $H^1(G, L) = 0$, where the
abelian group is now the additive group of $L$.
(Hint: by the normal basis theorem
(see for example Lang, \emph{Algebra}),
there exists $\alpha \in L$ whose conjugates form a basis
of $L$ as a $K$-vector space.)
\item
Prove the following extension of Theorem 90 (also due to Speiser).
Let $L/K$ be a finite Galois extension with Galois group $G$.
Despite the fact that $H^1(G, \GL(n,L))$ does not make sense as a group (because $\GL(n,L)$ is not abelian), show nonetheless that ``$H^1(G, \GL(n,L))$ is trivial'' in the sense that every function $f: G \to \GL(n,L)$ for which $f(gh) = f(g)^h f(h)$ for all $g,h \in G$ can be written as $x (x^g)^{-1}$ for some $x \in \GL(n,L)$. (Hint: to imitate the proof in the case $n=1$,
one must find an $n \times n$ matrix $x$ over $L$ such that $t = \sum_{g \in G} x^g f(g)$
is not only nonzero but \emph{invertible}. To establish this, note that the set of possible values of $t$ on one hand is an $L$-vector space, and on the other hand satisfies no nontrivial $L$-linear relation.)
\end{enumerate}

%\end{document}


