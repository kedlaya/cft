%\documentclass[12pt]{article}
%\usepackage{amsfonts, amsthm, amsmath}
%
%\setlength{\textwidth}{6.5in}
%\setlength{\oddsidemargin}{0in}
%\setlength{\textheight}{8.5in}
%\setlength{\topmargin}{0in}
%\setlength{\headheight}{0in}
%\setlength{\headsep}{0in}
%\setlength{\parskip}{0pt}
%\setlength{\parindent}{20pt}
%
%\def\AA{\mathbb{A}}
%\def\CC{\mathbb{C}}
%\def\FF{\mathbb{F}}
%\def\PP{\mathbb{P}}
%\def\QQ{\mathbb{Q}}
%\def\RR{\mathbb{R}}
%\def\ZZ{\mathbb{Z}}
%\def\gotha{\mathfrak{a}}
%\def\gothb{\mathfrak{b}}
%\def\gothm{\mathfrak{m}}
%\def\gotho{\mathfrak{o}}
%\def\gothp{\mathfrak{p}}
%\def\gothq{\mathfrak{q}}
%\DeclareMathOperator{\disc}{Disc}
%\DeclareMathOperator{\Gal}{Gal}
%\DeclareMathOperator{\GL}{GL}
%\DeclareMathOperator{\Hom}{Hom}
%\DeclareMathOperator{\Norm}{Norm}
%\DeclareMathOperator{\Real}{Re}
%\DeclareMathOperator{\Trace}{Trace}
%\DeclareMathOperator{\Cl}{Cl}
%
%\def\head#1{\medskip \noindent \textbf{#1}.}
%
%\newtheorem{theorem}{Theorem}
%\newtheorem{lemma}[theorem]{Lemma}
%
%\begin{document}
%
%\begin{center}
%\bf
%Math 254B, UC Berkeley, Spring 2002 (Kedlaya) \\
%Zeta Functions and the Chebotarev Density Theorem
%\end{center}

\head{Reference} Lang, \textit{Algebraic Number Theory}, Chapter VIII for starters; see also Milne, Chapter VI and
Neukirch, Chapter VII. For advanced reading, see
Tate's thesis (last chapter of Cassels-Frohlich), but wait until
we introduce the adeles.

\medskip
Although this is supposed to be a course on algebraic number theory, the
following analytic discussion is so fundamental that we must at least allude
to it here.

Let $K$ be a number field. The \emph{Dedekind zeta function} $\zeta_K(s)$
is a function on the complex plane given, for $\Real(s) > 1$,
by the absolutely convergent product and sum
\[
\zeta_K(s) = \prod_\gothp (1 - \Norm(\gothp)^{-s})^{-1}
=
\zeta_K(s) = \sum_{\gotha} \Norm(\gotha)^{-s},
\]
where in the sum $\gotha$ runs over the nonzero ideals of $\gotho_K$.

A fundamental fact about the zeta function is the following. We omit the proof.
\begin{theorem} \label{T:meromorphic continuation}
The function $\zeta_K(s)$ extends to a meromorphic function on $\CC$
whose only pole
is a simple pole at $s=1$ of residue $1$.
\end{theorem}
The case $K=\QQ$ is of course the famous Riemann zeta function.
There is also a functional equation relating the values of $\zeta_K$ at
$s$ and $1-s$, and an extended Riemann hypothesis: aside from ``trivial''
zeros along the negative real axis, the zeroes of $\zeta_K$ all have
real part $1/2$.

More generally, let $\gothm$ be a formal product of places of $K$,
and let $\chi_\gothm: \Cl^\gothm(K) \to \CC^*$ be a character of the
ray class group of conductor $\gothm$. Extend $\chi_\gothm$ to a function
on all ideals of $K$ by declaring its value to be 0 on ideals not coprime
to $\gothm$. Then we define the $L$-function
\[
L(s, \chi_\gothm) = \prod_{\gothp \not| \gothm} (1 - \chi(\gothp) \Norm(\gothp)^{-s})^{-1}
= \sum_{(\gotha, \gothm) = 1} \chi(\gotha) \Norm(\gotha)^{-s}.
\]
Then we have another basic fact whose proof we also omit.
\begin{theorem} \label{T:analytic continuation}
If $\chi_\gothm$ is not trivial, then
$L(s, \chi_\gothm)$ extends to an analytic function on $\CC$.
\end{theorem}
If $\chi_\gothm$ is trivial, then $L(s, \chi_\gothm)$ is just the Dedekind
zeta function with the Euler factors for primes dividing $\gothm$ removed,
so it still has a pole at $s=1$. 

\begin{theorem} \label{T:nonvanishing of L}
If $\chi_\gothm$ is not the trivial character, then
$L(1, \chi_\gothm) \neq 0$.
\end{theorem}
This is already a nontrivial, but important result over $\QQ$. It implies
Dirichlet's famous theorem that there are infinitely many primes in
arithmetic progression, by implying that for any nontrivial $\chi_\gothm$,
$\sum_{\gothp} \chi(\gothp) \Norm(\gothp)^{-s}$ remains bounded as
$s \to \infty$. In fact, we say that a set of primes $S$ in a number field
$K$ has Dirichlet density $d$ if
\[
\lim_{s \to 1^+} \frac{\sum_{\gothp \in S} \Norm(\gothp)^{-s}}{\log \frac{1}{s-1}} = d.
\]
Then the fact implies that the Dirichlet density of the
set of primes congruent to $a$ modulo $m$ (assuming $a$ is coprime to $m$)
is $1/\phi(m)$.

The fact also implies that for any number field $K$ and any formal
product of places $\gothm$,
there are infinitely many primes in each
class of the ray class group of conductor $\gothm$, the set of such 
primes having Dirichlet density
$1/\#\Cl^\gothm(K)$. (Proof: see exercises.)

Finally, we point out a result of class field theory that also applies
to nonabelian extensions. Recall that if $L/K$ is any Galois extension
of number fields with Galois group $G$,
$\gothp$ is a prime of $K$, and $\gothq$ is a prime
above $\gothp$ which is unramified, then there is a well-defined Frobenius
associated to $\gothq$ (it's the element $g$ of the decomposition group
$G_{\gothq}$ such that $x^g \equiv x^{\#(\gotho_K/\gothp)} \pmod{\gothq}$);
but as a function of $\gothp$, this Frobenius is only well-defined up
to conjugation in $G$.
\begin{theorem}[Chebotarev Density Theorem]
Let $L/K$ be an arbitrary Galois extension of number fields,
with Galois group $G$. Then for any $g \in G$, there exist infinitely many
primes $\gothp$ of $K$ such that there is a prime $\gothq$ of $L$ above
$\gothp$ with Frobenius $g$. In fact, the Dirichlet density of such 
primes $\gothp$ is the order of the conjugacy class of $G$ divided by
$\#G$.
\end{theorem}
\begin{proof}
This follows from everything we have said so far, plus Artin reciprocity,
in case $L/K$ is abelian. In the general case, let $f$ be the order of
$g$, and let $K'$ be the fixed
field of $g$; then we know that the set of primes of $K'$ with Frobenius
$g \in \Gal(L/K') \subset G$ has Dirichlet density $1/f$. The same is true
if we restrict to primes of absolute degree 1 (see exercises).

Let $Z$ be the centralizer of $g$ in $G$; that is, $Z = \{z \in G:
zg = gz\}$. Then for each prime of $K$ (of absolute degree 1)
with Frobenius in the conjugacy
class of $g$, there are $\#Z/f$ primes of $K'$ above it (also
of absolute degree 1) with Frobenius $g$.
(Say $\gothp$ is such a prime and $\gothq$ is a prime of $L$ above $\gothp$
with Frobenius $g$. Then for $h \in G$, the Frobenius of $\gothq^h$
is $hgh^{-1}$, so the number of primes $\gothq$ with Frobenius $g$ is
$\#Z$. But each prime of $L'$ below one of these is actually below
$f$ of them.)
Thus the density of primes of $K$ with Frobenius in the conjugacy class
of $g$ is $(1/f)(1/(\#Z/f)) = 1/\#Z$. To conclude, note that the order
of the conjugacy class of $G$ is $\#G/\#Z$.
\end{proof}

\head{Exercises}

\begin{enumerate}
\item
Show that the Dirichlet density of the set of all primes of a
number field is 1.
\item
Show that in any number field,
the Dirichlet density of the set of primes $\gothp$ of absolute
degree greater than 1 is zero. 
\item
Let $\gothm$ be a formal product of places of the number field $K$.
Using Theorems~\ref{T:meromorphic continuation}, \ref{T:analytic continuation},
 and~\ref{T:nonvanishing of L}, prove that the set of primes of $K$
lying in any specified class of the ray class group of conductor
$\gothm$ is $1/\#\Cl^\gothm(K)$. (Hint: combine the quantities 
$\sum_{\gothp} \chi(\gothp) \Norm(\gothp)^{-s}$ to cancel
out all but one class.)
\end{enumerate}

%\end{document}


