%\documentclass[12pt]{article}
%\usepackage{amsfonts, amsthm, amsmath}
%\usepackage[all]{xy}
%
%\setlength{\textwidth}{6.5in}
%\setlength{\oddsidemargin}{0in}
%\setlength{\textheight}{8.5in}
%\setlength{\topmargin}{0in}
%\setlength{\headheight}{0in}
%\setlength{\headsep}{0in}
%\setlength{\parskip}{0pt}
%\setlength{\parindent}{20pt}
%
%\def\kbar{\overline{k}}
%\def\AA{\mathbb{A}}
%\def\CC{\mathbb{C}}
%\def\FF{\mathbb{F}}
%\def\PP{\mathbb{P}}
%\def\QQ{\mathbb{Q}}
%\def\RR{\mathbb{R}}
%\def\ZZ{\mathbb{Z}}
%\def\gotha{\mathfrak{a}}
%\def\gothb{\mathfrak{b}}
%\def\gothm{\mathfrak{m}}
%\def\gotho{\mathfrak{o}}
%\def\gothp{\mathfrak{p}}
%\def\gothq{\mathfrak{q}}
%\def\gothr{\mathfrak{r}}
%\DeclareMathOperator{\ab}{ab}
%\DeclareMathOperator{\coker}{coker}
%\DeclareMathOperator{\disc}{Disc}
%\DeclareMathOperator{\Frob}{Frob}
%\DeclareMathOperator{\Gal}{Gal}
%\DeclareMathOperator{\GL}{GL}
%\DeclareMathOperator{\Hom}{Hom}
%\DeclareMathOperator{\im}{im}
%\DeclareMathOperator{\Ind}{Ind}
%\DeclareMathOperator{\Inf}{Inf}
%\DeclareMathOperator{\inv}{inv}
%\DeclareMathOperator{\Norm}{Norm}
%\DeclareMathOperator{\Res}{Res}
%\DeclareMathOperator{\Trace}{Trace}
%\DeclareMathOperator{\unr}{unr}
%\DeclareMathOperator{\Cl}{Cl}
%
%\def\head#1{\medskip \noindent \textbf{#1}.}
%
%\newtheorem{theorem}{Theorem}
%\newtheorem{lemma}[theorem]{Lemma}
%\newtheorem{cor}[theorem]{Corollary}
%\newtheorem{prop}[theorem]{Proposition}
%
%\begin{document}
%
%\begin{center}
%\bf
%Math 254B, UC Berkeley, Spring 2002 (Kedlaya) \\
%Cohomology of local fields: some computations
%\end{center}

\head{Reference} Milne, III.2 and III.3; Neukirch, V.1.

\head{Notation convention} If you catch me writing $H^i(L/K)$
for $L/K$ a Galois extension of fields,
that's shorthand for $H^i(\Gal(L/K), L^*)$. Likewise for $H_i$ or
$H^i_T$.

\medskip
We now make some computations of $H^i_T(L/K)$ for $L/K$
a finite Galois
extension of local fields.
To begin with, recall that by ``Theorem 90'' (Lemma~\ref{L:theorem 90}), $H^1(L/K) = 0$.
Our goal in this chapter will be to supplement this fact with a computation of $H^2(L/K)$.
\begin{prop} \label{P:local h2}
For any finite Galois extension $L/K$ of local fields, $H^2(L/K)$
is cyclic of order $[L:K]$. Moreover, this group can be canonically
identified with $\frac{1}{[L:K]}\ZZ/\ZZ$ in such a way that
if $M/L$ is another finite extension such that
$M/K$ is also Galois, the inflation homomorphism
$H^2(L/K) \to H^2(M/K)$ corresponds to the inclusion
$\frac{1}{[L:K]}\ZZ/\ZZ \subseteq \frac{1}{[M:K]}\ZZ/\ZZ$.
\end{prop}

Before continuing, it is worth keeping in a safe place the exact sequence
\[
1 \to \gotho_L^* \to L^* \to L^*/\gotho_L^* = \pi_L^\ZZ \to 1.
\]
In this exact sequence of $G
= \Gal(L/K)$-modules, the action on $\pi_L^\ZZ$ is always trivial
(since the valuation on $L$ is Galois-invariant). For convenience, we
write $U_L$ for the unit group $\gotho_L^*$.

\head{The unramified case}
Recall that unramified extensions are cyclic, since their Galois groups are
also the Galois groups of extensions of finite fields. 

\begin{prop}
For any finite extension $L/K$ of \emph{finite} fields, the map
$\Norm_{L/K}: L^* \to K^*$ is surjective.
\end{prop}
\begin{proof}
One can certainly give an elementary proof of this using the fact that
$L^*$ is cyclic (exercise).
But one can also see it using the machinery we have at hand.
Because $L^*$ is a finite module, its Herbrand quotient is 1. Also,
we know $H^1_T(L/K)$ is trivial by Lemma~\ref{L:theorem 90}.
Thus $H^0_T(L/K)$ is trivial too, that is,
$\Norm_{L/K}: L^* \to K^*$ is surjective.
\end{proof}

\begin{prop}
For any finite unramified extension $L/K$ of local fields, the map
$\Norm_{L/K}: U_L \to U_K$ is surjective.
\end{prop}
\begin{proof}
Say $u \in U_K$ is a unit. Pick $v_0 \in U_L$ such that in the residue fields,
the norm of $v_0$ coincides with $u$. Thus $u/\Norm(v_0) \equiv 1 \pmod{\pi}$,
where $\pi$ is a uniformizer of $K$. Now we construct units
$v_i \equiv 1 \pmod{\pi^i}$ such that $u_i = u/\Norm(v_0\cdots v_i) \equiv 1 
\pmod{\pi^{i+1}}$: simply take $v_i$ so that 
$\Trace((1-v_i)/\pi^i) \equiv (1-u_{i-1})/\pi^i \pmod{\pi}$. (That's possible
because the trace map on residue fields is surjective by the normal basis theorem.)
Then the product $v_0v_1\cdots$ converges to a unit $v$ with norm $u$.
\end{proof}
\begin{cor}
For any finite unramified extensions $L/K$ of local fields, then
$H^i_T(\Gal(L/K), U_L) = 1$ for all $i \in \ZZ$.
\end{cor}
\begin{proof}
Again, $\Gal(L/K)$ is cyclic, so by Theorem~\ref{T:cyclic group periodicity}
we need only check this for $i=0,1$. For $i=1$, the desired equality is Lemma~\ref{L:theorem 90}; for $i=0$, it is the previous
proposition.  
\end{proof}

Using the Herbrand quotient, we get
$h(L^*) = h(U_L) h(L^*/U_L)$. The previous corollary says that $h(U_L) = 1$,
and 
\begin{align*}
h(L^*/U_L) &= h(\ZZ) \\
&= \#H^0_T(\Gal(L/K), \ZZ)/\#H^1_T(\Gal(L/K), \ZZ) \\
&= \#\Gal(L/K)^{\ab} / \#\Hom(\Gal(L/K), \ZZ) \\
&= [L:K].
\end{align*}
Since $H^1_T(\Gal(L/K), L^*)$ is trivial, we conclude $H^0_T(\Gal(L/K), L^*)$
has order $[L:K]$. In fact, it is cyclic: the long exact sequence of Tate
groups gives
\[
1 \to
H^0_T(\Gal(L/K), L^*) \to H^0_T(\Gal(L/K), \ZZ) = \Gal(L/K) \to 1.
\]

Consider the short exact sequence
\[
0 \to \ZZ \to \QQ \to \QQ/\ZZ \to 0
\]
of modules with trivial Galois action. Since $\QQ$ is injective as an abelian
group, it is also injective as a $G$-module for any group $G$ (exercise).
Thus we get an isomorphism $H^0_T(\Gal(L/K), \ZZ) \to H^{-1}_T(\Gal(L/K),
\QQ/\ZZ)$. But the latter is
\[
H^1(\Gal(L/K), \QQ/\ZZ) = \Hom(\Gal(L/K),
\QQ/\ZZ);
\]
since $\Gal(L/K)$ has a canonical generator (Frobenius), we
can evaluate there and get a canonical map $\Hom(\Gal(L/K), \QQ/\ZZ)
\to \ZZ/[L:K]\ZZ \subset \QQ/\ZZ$. Putting it all together, we get a 
canonical map
\[
H^2(\Gal(L/K), L^*) \cong H^0_T(\Gal(L/K), L^*)
\cong H^1(\Gal(L/K), \QQ/\ZZ) \hookrightarrow \QQ/\ZZ.
\]
In this special
case, this is none other than the local invariant map! In fact, by
taking direct limits, we get a canonical isomorphism
\[
H^2(K^{\unr}/K) \cong \QQ/\ZZ.
\]

What's really going on here is that $H^0_T(\Gal(L/K), L^*)$ is a cyclic
group generated by a uniformizer $\pi$ (since every unit is a norm).
Under the map $H^0_T(\Gal(L/K), L^*) \to \QQ/\ZZ$, that uniformizer
is being mapped to $1/[L:K]$.

\head{The cyclic case}

Let $L/K$ be a cyclic but possibly ramified extension of local fields.
Again, $H^1_T(L/K)$ is trivial by Lemma~\ref{L:theorem 90}, so all there is
to compute is $H^0_T(L/K)$. We are going to show again that it
has order $[L:K]$. (It's actually cyclic again, but we won't prove this just
yet.)

\begin{lemma}
Let $L/K$ be a finite Galois extension of local fields. Then there
is an open, Galois-stable subgroup $V$
of $\gotho_L$ such that $H^i(\Gal(L/K), V) = 0$ for all $i>0$
(i.e., $V$ is acyclic for cohomology).
\end{lemma}
\begin{proof}
By the normal basis theorem, there exists $\alpha \in L$ such that
$\{\alpha^g: g \in \Gal(L/K)\}$ is a basis for $L$ over $K$. Without loss
of generality, we may rescale to get $\alpha \in \gotho_L$; then put
$V = \sum \gotho_K \alpha^g$. As in the proof of Theorem~\ref{T:additive theorem 90},
$V$ is induced: $V = \Ind^G_{1} \gotho_K$, so is acyclic.
\end{proof}

The following proof uses that we are in characteristic 0, but it can be modified to work also in the function field case.
\begin{lemma}
Let $L/K$ be a finite Galois extension of local fields. Then there
is an open, Galois-stable subgroup $W$
of $U_L = \gotho_L^*$ such that $H^i(\Gal(L/K), W) = 0$ for all $i>0$.
\end{lemma}
\begin{proof}
Take $V$ as in the previous lemma. If we choose $\alpha$ sufficiently divisible,
then $V$ lies in the radius of convergence of the exponential series
\[
\exp(x) = \sum_{i=0}^\infty \frac{x^i}{i!}
\]
(you need $v_p(x) > 1/(p-1)$, to be precise), and we may take $W = \exp(V)$.
\end{proof}

Since the quotient $U_L/W$ is finite, its Herbrand quotient is 1, so
$h(U_L) = h(V) = 1$. So again we may conclude that
$h(L^*) = h(U_L) h(\ZZ) = [L:K]$, and so $H^0_T(\Gal(L/K), L^*) = [L:K]$.
However, we cannot yet check that $H^0_T(\Gal(L/K), L^*)$ is cyclic because the groups
$H^1_T(\Gal(L/K), U_L)$ are not guaranteed to vanish; see the exercises.

\head{Note} This is all that we need for ``abstract''
local class field theory. We'll revisit this point later.

\head{The general case}

For those in the know, there is a spectral sequence underlying this next
result; see Milne, Remark II.1.35.
\begin{prop}[Inflation-Restriction Exact Sequence] \label{P:inflation restriction}
Let $G$ be a finite group, $H$ a normal subgroup, and $M$ a $G$-module.
If $H^i(H, M) = 0$ for $i=1, \dots, r-1$, then 
\[
0 \to H^r(G/H, M^H) \stackrel{\Inf}{\to} H^r(G,M) \stackrel{\Res}{\to}
H^r(H,M)
\]
is exact.
\end{prop}
\begin{proof}
For $r=1$, the condition on $H^i$ is empty. In this case, $H^1(G,M)$
classifies crossed homomorphisms $\phi:G \to M$. If one of these
factors through $G/H$, it becomes a constant map when restricted to $H$;
if that constant value itself belongs to $M^H$, then it must be zero
and so the restriction to $H$ is trivial.
Conversely, if there exists some $m \in M$ such that 
$\phi(h) = m^h - m$ for all $h \in H$, then
$\phi'(g) = \phi(g) - m^g + m$ is another crossed homomorphism representing the same class in $H^1(G,M)$, but taking the value 0 on each $h \in H$. For $g \in G, h \in H$, we have
\[
\phi'(hg) = \phi'(h)^g + \phi'(g) = \phi'(g),
\]
so $\phi'$ is constant on cosets of $H$ and so may be viewed as a crossed homomorphism from $G/H$ to $M$. On the other hand,
\[
\phi'(g) = \phi'(gh) = \phi'(g)^h + \phi(h) = \phi'(g)^h
\]
so $\phi'$ takes values in $M^H$.
 Thus the sequence is exact at $H^1(G,M)$; exactness at
$H^i(G/H,M^H)$ is similar but easier.

If $r>1$, we induct on $r$ by dimension shifting. 
Recall (from Proposition~\ref{P:adjoint property}) that there is an injective homomorphism $M \to \Ind^G_1 M$ of $G$-modules.
Let $N$ be the $G$-module
which makes the sequence
\[
0 \to M \to \Ind^G_{1} M \to N \to 0
\]
exact. We construct a commutative diagram
\[
\xymatrix{
0 \ar[r] & H^{r-1}(G/H, N^H) \ar^{\Inf}[r] \ar[d] & H^{r-1}(G, N) \ar^{\Res}[r] \ar[d] & 
H^{r-1}(H,N) \ar[d] \\
0 \ar[r] & H^{r}(G/H, M^H) \ar^{\Inf}[r] & H^{r}(G, M) \ar^{\Res}[r] & 
H^{r}(H,M).
}
\]
The second vertical arrow arises from the long exact sequence for $G$-cohomology;
since $\Ind^G_{1} M$ is an induced $G$-module, this arrow is an isomorphism.
Similarly, the third vertical arrow arises from the long exact sequence for $H$-cohomology,
and it is an isomorphism because $\Ind^G_1 M$ is also an induced $H$-module; moreover, $H^i(H, N) = 0$ for $i=1, \dots, r-2$. 
Finally, taking $H$-invariants yields another exact sequence
\[
0 \to M^H \to (\Ind^G_1 M)^H \to N^H \to H^1(H, M) = 0,
\]
so we may take the long exact sequence for $G/H$-cohomology to obtain the first vertical arrow; it is an isomorphism because $(\Ind^G_1 M)^H$ is an induced $G/H$-module. The induction hypothesis implies that the top row is exact, so the bottom row is also exact.
\end{proof}
By Lemma~\ref{L:theorem 90}, we have the following.
\begin{cor} \label{C:inflation restriction h2}
If $M/L/K$ is a tower of fields with $M/K$ and $L/K$ finite and Galois,
the sequence
\[
0 \to H^2(L/K) \stackrel{\Inf}{\to} H^2(M/K) \stackrel{\Res}{\to} H^2(M/L)
\]
is exact.
\end{cor}

We now prove the following.
\begin{prop}
For any finite Galois extension $L/K$ of local fields, the group $H^2(\Gal(L/K), L^*)$
has order at most $[L:K]$.
\end{prop}
A key fact we need to recall is that any finite Galois extension of local fields
is \emph{solvable}: the maximal unramified extension is cyclic, the
maximal tamely ramified extension is cyclic over that, and the rest is
an extension of order a power of $p$, so its Galois group is automatically
solvable. This lets us induct on $[L:K]$.
\begin{proof}
We've checked the case of $L/K$ cyclic, so we may use it as the basis for an
induction. If $L/K$ is not cyclic, since it is solvable, we can find a
Galois subextension $M/K$. Now the exact sequence
\[
0 \to H^2(M/K) \to H^2(L/K) \to H^2(L/M)
\]
implies that $\#H^2(L/K) \leq \#H^2(M/K)
\#H^2(L/M) = [M:K][L:M] = [L:K]$.
\end{proof}

To complete the proof that $H^2(L/K)$ is cyclic of order
$[L:K]$, it now suffices to produce a cyclic subgroup of order $[L:K]$.
Let $M/K$ be an unramified extension of degree $[L:K]$. Then we have a diagram
\[
\xymatrix{
& & H^2(M/K) \ar[d]^{\Inf} \ar[rd]
&  \\
0 \ar[r] & H^2(L/K) \ar^{\Inf}[r] & H^2(ML/K) \ar^{\Res}[r]
& H^2(ML/L)
}
\]
in which the bottom row is exact and the vertical arrows are injective,
both by Corollary~\ref{C:inflation restriction h2}. It suffices to show that
the diagonal arrow
$H^2(M/K) \to H^2(ML/L)$ is the zero map; then we can push a generator
of $H^2(M/K)$ down to $H^2(ML/K)$, then pull it back to $H^2(L/K)$ by exactness
to get an element of order $[L:K]$.

Let $e = e(L/K)$ and $f = f(L/K)$ be the ramification index and residue field degree, so that
$[ML:L] = e$. Let $U$ be the maximal unramified subextension of $L/K$;
then we have a canonical isomorphism $\Gal(ML/L) \cong \Gal(M/U)$ of cyclic groups.
By using the same generators in both groups, we can make a commutative diagram
\[
\xymatrix{
H^0_T(M/K) \ar^{\Res}[r] \ar[d] & H^0_T(M/U) \ar[r]  \ar[d]
& H^0_T(ML/L) \ar[d]  \\
H^2(M/K) \ar^{\Res}[r] & H^2(M/U) \ar[r]
 & H^2(ML/L)
}
\]
in which the vertical arrows are isomorphisms. 
(Remember that extended functoriality for Tate groups starts in degree 0, yielding 
the first horizontal arrow.)
The composition in the bottom row is the map $H^2(M/K) \to H^2(ML/L)$ which we want to be zero; it thus suffices to check that the top row composes to zero. This composition is none other than the canonical map
$K^*/\Norm_{M/K} M^* \to L^*/\Norm_{ML/L} (ML)^*$.
Now $K^*/\Norm_{M/K} M^*$ is a cyclic group of order $ef$
generated by $\pi_K$, a uniformizer
of $K$, and $L^*/\Norm_{ML/L} (ML)^*$ is a cyclic group of order
$e$ generated by $\pi_L$, a uniformizer of $L$. But
$\pi_K$ is a unit of $\gotho_L$ times $\pi_L^e$, so the map in question is indeed zero.

\head{Note}
If $L/K$ is a finite extension of degree $n$, then the map
$\Res: H^2(K^{\unr}/K) \to H^2(L^{\unr}/L)$ translates, via the local
reciprocity map, into a map from $\QQ/\ZZ$ to itself. This map turns out
to be multiplication by $n$ (see Milne, Proposition II.2.7).

\head{The local invariant map}

By staring again at the above argument, we can in fact prove that
$H^2(\overline{K}/K) \cong \QQ/\ZZ$.
First of all, we have
an injection $H^2(K^{\unr}/K) \to H^2(\overline{K}/K)$ by
Corollary~\ref{C:inflation restriction h2}, and the former is canonically
isomorphic to $\QQ/\ZZ$; so we have to prove that this injection is
actually also surjective. Remember that $H^2(\overline{K}/K)$ is the
direct limit of $H^2(M/K)$ running over all finite extensions $M$ of $K$.
What we just showed above is that if $[M:K] = n$ and $L$ is the unramified
extension of $K$ of degree $n$, then the images of $H^2(M/K)$ and
$H^2(L/K)$ in $H^2(ML/K)$ are the same. In particular, that means that
$H^2(M/K)$ is in the image of the map $H^2(K^{\unr}/K) \to 
H^2(\overline{K}/K)$. Since that's true for any $M$, we get that the
map is indeed surjective, hence an isomorphism.

Next time, we'll use this map to obtain the local reciprocity map.

\head{Exercises}

\begin{enumerate}
\item
Give an elementary proof (without cohomology)
that the norm map from one finite field to another
is always surjective.
\item
Give an example of a cyclic ramified extension $L/K$ of local fields
in which the groups $H^i_T(\Gal(L/K), U_L)$ are nontrivial.
\end{enumerate}

%\end{document}
