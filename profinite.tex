%\documentclass[12pt]{article}
%\usepackage{amsfonts, amsthm, amsmath}
%
%\setlength{\textwidth}{6.5in}
%\setlength{\oddsidemargin}{0in}
%\setlength{\textheight}{8.5in}
%\setlength{\topmargin}{0in}
%\setlength{\headheight}{0in}
%\setlength{\headsep}{0in}
%\setlength{\parskip}{0pt}
%\setlength{\parindent}{20pt}
%
%\def\AA{\mathbb{A}}
%\def\CC{\mathbb{C}}
%\def\FF{\mathbb{F}}
%\def\NN{\mathbb{N}}
%\def\PP{\mathbb{P}}
%\def\QQ{\mathbb{Q}}
%\def\RR{\mathbb{R}}
%\def\ZZ{\mathbb{Z}}
%\def\gotha{\mathfrak{a}}
%\def\gothb{\mathfrak{b}}
%\def\gothm{\mathfrak{m}}
%\def\gotho{\mathfrak{o}}
%\def\gothp{\mathfrak{p}}
%\def\gothq{\mathfrak{q}}
%\DeclareMathOperator{\disc}{Disc}
%\DeclareMathOperator{\Fix}{Fix}
%\DeclareMathOperator{\Gal}{Gal}
%\DeclareMathOperator{\GL}{GL}
%\DeclareMathOperator{\Hom}{Hom}
%\DeclareMathOperator{\Inf}{Inf}
%\DeclareMathOperator{\Norm}{Norm}
%\DeclareMathOperator{\Trace}{Trace}
%\DeclareMathOperator{\Cl}{Cl}
%
%\def\head#1{\medskip \noindent \textbf{#1}.}
%
%\newtheorem{theorem}{Theorem}
%\newtheorem{lemma}[theorem]{Lemma}
%\newtheorem{prop}[theorem]{Proposition}
%
%\begin{document}
%
%\begin{center}
%\bf
%Math 254B, UC Berkeley, Spring 2002 (Kedlaya) \\
%Profinite Groups and Infinite Galois Theory
%\end{center}

\head{Reference} 
Neukirch, Sections IV.1 and IV.2.

\medskip

We've mostly spoken so far about finite extensions of fields and
the corresponding finite Galois groups. However, Galois theory can be made
to work perfectly well for infinite extensions, and it's convenient to do so;
it will be more convenient at times to work with the absolute Galois group
of field instead of with the Galois groups of individual extensions.

Recall the Galois correspondence for a finite extension: if $L/K$ is Galois
and $G = \Gal(L/K)$, then the (normal) subgroups $H$ of $G$ correspond to the
(Galois) subextensions $M$ of $L$, the correspondence in each direction
being given by
\[
H \mapsto \Fix{H},
\qquad
M \mapsto \Gal(L/M).
\]
To see what we have to be careful about, here's one example. Let $\FF_q$
be a finite field; recall that $\FF_q$ has exactly one finite extension of
any degree. Moreover, for each $n$, $\Gal(\FF_{q^n}/\FF_q)$ is cyclic of
degree $n$, generated by the Frobenius map $\sigma$ which sends $x$ to
$x^q$. In particular, $\sigma$ generates a cyclic subgroup of
$\Gal(\overline{\FF_q}/\FF_q)$. But this Galois group is much bigger than
that! Namely,
let $\{s_n\}_{n=1}^\infty$ be a sequence with $s_n \in \ZZ/n\ZZ$, such that
if $m | n$, then $s_m \equiv s_n \pmod{m}$. 
The set of such sequences forms
a group $\widehat{\ZZ}$ by componentwise addition.
This group is much bigger
than $\ZZ$, and any element gives an automorphism of $\overline{\FF_q}$:
namely, the automorphism acts on $\FF_{q^n}$ as $\sigma^{s_n}$. In fact,
$\Gal(\overline{\FF_q}/\FF_q) \cong \widehat{\ZZ}$, and it is not true that
every subgroup of $\widehat{\ZZ}$ corresponds to a subfield of
$\overline{\FF_q}$: the subgroup generated by $\sigma$ has fixed field
$\FF_q$, and you don't recover the subgroup generated by $\sigma$ by taking
automorphisms over the fixed field.

In order to recover the Galois correspondence, we need to impose a little
extra structure on Galois groups; namely, we give them a topology.

A \emph{profinite group} is a topological group which is Hausdorff and
compact, and which admits a basis of neighborhoods of the identity consisting
of normal subgroups. More explicitly, a profinite group is a group $G$
plus a collection of subgroups of $G$ of finite index designated as
\emph{open subgroups}, such that the intersection of two open subgroups is
open, but the intersection of all of the open subgroups is trivial.
Profinite groups act a lot like finite groups; some of the ways in which
this is true are reflected in the exercises.

Examples of profinite groups include the group $\widehat{\ZZ}$ in which
the subgroups $n\widehat{\ZZ}$ are open, and the $p$-adic integers $\ZZ_p$
in which the subgroups $p^n \ZZ_p$ are open. More generally, for any local
field $K$, the additive group $\gotho_K$ and the multiplicative group
$\gotho_K^*$ are profinite. (The additive and multiplicative groups of $K$
are not profinite, because they're only locally compact, not compact.)
For a nonabelian example, see the exercises.

\head{Warning}
A profinite group may have subgroups of finite index that are
not open. For example, let $G = 1 + t \FF_p [[ t ]]$
(under multiplication). Then $G$ is profinite with the subgroups
$1 + t^n \FF_p [[ t ]]$ forming a basis of open
subgroups; in particular, it has countably many open subgroups.
But $G$ is isomorphic to a countable direct product of
copies of $\ZZ_p$, with generators $1 + t^{i}$ for $i$ not divisible by $p$.
Thus it has \emph{uncountably} many subgroups of finite index, most
of which are not open!

If $L/K$ is a Galois extension, but not necessarily finite, we make
$G = \Gal(L/K)$ into a profinite group by declaring that the open subgroups
of $G$ are precisely $\Gal(L/M)$ for all finite subextensions $M$ of $L$.

\begin{theorem}[The Galois correspondence]
  Let $L/K$ be a Galois extension (not necessarily finite). Then
there is a correspondence between (Galois) subextensions $M$ of $L$ and
(normal) \emph{closed} subgroups $H$ of $\Gal(L/K)$, given by
\[
H \mapsto \Fix H, \qquad M \mapsto \Gal(L/M).
\]
\end{theorem}
For example, the Galois correspondence works for $\overline{\FF_q}/\FF_q$
because the open subgroups of $\widehat{\ZZ}$ are precisely
$n\widehat{\ZZ}$ for all positive integers $n$.

Another way to construct profinite groups uses inverse limits. Suppose we
are given a partially ordered set $I$, a
family $\{G_i\}_{i \in I}$ of finite groups and a map $f_{ij}: G_i \to G_j$
for each pair $(i,j) \in I \times I$ such that $i > j$. For simplicity,
let's assume the $f_{ij}$ are all surjective (this is slightly more restrictive than absolutely necessary, but is always true for Galois groups). Then there is
a profinite group $G$ with open subgroups $H_i$ for $i \in I$ such that
$G/H_i \cong G_i$ and some other obvious compatibilities hold:
let $G$ be the set of families $\{g_i\}_{i \in I}$, where each $g_i$
is in $G_i$ and $f_{ij}(g_i) = g_j$.

For example, the group $\ZZ_p$ either as the completion of $\ZZ$ for the $p$-adic absolute value or as the inverse limit of the groups $\ZZ/p^n\ZZ$.
Similarly, the group $\widehat{\ZZ}$ can be viewed as the inverse limit
of the groups $\ZZ/n\ZZ$, with the usual surjections from $\ZZ/m\ZZ$
to $\ZZ/n\ZZ$ if $m$ is a multiple of $n$ (that is, the ones sending 1 to 1).
In fact, \emph{any} profinite group can be reconstructed as the inverse
limit of its quotients by open subgroups. (And it's enough to use just a
set of open subgroups which form a basis for the topology, i.e.,
for $\ZZ_p$, you can use $p^{2n}\ZZ_p$ as the subgroups.)

\head{Rule of thumb} If profinite groups make your head hurt, you can
always think instead of inverse systems of finite groups. But that might
make your head hurt more!

\head{Cohomology of profinite groups}

One can do group cohomology for groups which are profinite, not just finite,
but one has to be a bit careful: these groups only make sense when you
carry along the profinite topology. Thus if $G$ is profinite, by a \emph{$G$-module}
we mean a topological abelian group $M$ with a \emph{continuous} $G$-action
$M \times G \to M$. In particular, we say $M$ is \emph{discrete} if it
has the discrete topology; that implies that the stabilizer of any element of
$M$ is open, and that $M$ is the union of $M^H$ over all open subgroups
$H$ of $G$. Canonical example: $G = \Gal(L/K)$ acting on $L^*$, even if $L$
is not finite.

The category of discrete $G$-modules has enough injectives, so you can
define cohomology groups for any discrete $G$-module, and all the usual
abstract nonsense will still work. The main point is that you can compute them
from their finite quotients.
\begin{prop}
The group $H^i(G, M)$ is the direct limit of 
$H^i(G/H, M^H)$ using the inflation homomorphisms.
\end{prop}
That is, if $H_1 \subseteq H_2 \subseteq G$, you have the inflation
homomorphism
\[
\Inf: H^i(G/H_2, M^{H_2}) \to H^i(G/H_1, M^{H_1}),
\]
so the groups $H^i(G/H, M^H)$ form a direct system,
and $H^i(G,M)$ is the direct limit of these. (That is, you take the union
of all of the $H^i(G/H, M^H)$, then you identify pairs that become the
same somewhere down the line.)

Or if you prefer, you can compute these groups using continuous cochains:
use continuous maps $G^{i+1} \to M$ that satisfy the same algebraic
conditions as do the usual cochains. For example, $H^1(G,M)$ classifies
continuous crossed homomorphisms modulo principal ones, et cetera.

\head{Warning} The passage from finite to profinite groups is only
well-behaved for cohomology. In particular, we will not attempt to define
either homology or the Tate groups. (Remember that the formation of the
Tate groups involves the norm map, i.e., summing over elements of the group.)

\head{Exercises}

\begin{enumerate}
\item
Prove that every open subgroup of a profinite group contains an open
normal subgroup.
\item
For any ring $R$, we denote by $\GL_n(R)$ the group of $n \times n$ matrices
over $R$ which are invertible (equivalently, whose determinant is a unit).
Prove that $\GL_n(\widehat{\ZZ})$ is a profinite group, and say as much as
you can about its open subgroups.
\item
Let $A$ be an abelian torsion group. Show that $\Hom(A, \QQ/\ZZ)$ is
a profinite group, if we take the open subgroups to be all subgroups
of finite index. This group is called the \emph{Pontryagin dual} of $A$.
\item
Neukirch exercise IV.2.4:
a closed subgroup $H$ of a profinite group $G$ is called
a \emph{$p$-Sylow subgroup} of $G$ if, for every open normal subgroup $N$
of $G$, $HN/N$ is a $p$-Sylow subgroup of $G/N$. Prove that:
\begin{enumerate}
\item[(a)] For every prime $p$, there exists a $p$-Sylow subgroup of $G$.
\item[(b)] Every subgroup of $G$, the quotient of which by any open 
normal subgroup is a $p$-group, is contained in a $p$-Sylow subgroup.
\item[(b)] Every two $p$-Sylow subgroups of $G$ are conjugate.
\end{enumerate}
You may use Sylow's theorem (that (a)-(c) hold for finite groups)
without further comment. Warning: Sylow subgroups are usually not
open.
\item
Neukirch exercise IV.2.4:
Compute the $p$-Sylow subgroups of $\widehat{\ZZ}$, of $\ZZ_p^*$, and
of $\GL_2(\ZZ_p)$.
\end{enumerate}

%\end{document}


