%\documentclass[12pt]{article}
%\usepackage{amsfonts, amsthm, amsmath}
%\usepackage[all]{xy}
%
%\setlength{\textwidth}{6.5in}
%\setlength{\oddsidemargin}{0in}
%\setlength{\textheight}{8.5in}
%\setlength{\topmargin}{0in}
%\setlength{\headheight}{0in}
%\setlength{\headsep}{0in}
%\setlength{\parskip}{0pt}
%\setlength{\parindent}{20pt}
%
%\def\AA{\mathbb{A}}
%\def\CC{\mathbb{C}}
%\def\FF{\mathbb{F}}
%\def\PP{\mathbb{P}}
%\def\QQ{\mathbb{Q}}
%\def\RR{\mathbb{R}}
%\def\ZZ{\mathbb{Z}}
%\def\gotha{\mathfrak{a}}
%\def\gothb{\mathfrak{b}}
%\def\gothm{\mathfrak{m}}
%\def\gotho{\mathfrak{o}}
%\def\gothp{\mathfrak{p}}
%\def\gothq{\mathfrak{q}}
%\DeclareMathOperator{\Cor}{Cor}
%\DeclareMathOperator{\disc}{Disc}
%\DeclareMathOperator{\Gal}{Gal}
%\DeclareMathOperator{\GL}{GL}
%\DeclareMathOperator{\Hom}{Hom}
%\DeclareMathOperator{\Ind}{Ind}
%\DeclareMathOperator{\Norm}{Norm}
%\DeclareMathOperator{\Res}{Res}
%\DeclareMathOperator{\Trace}{Trace}
%\DeclareMathOperator{\Cl}{Cl}
%
%\def\head#1{\medskip \noindent \textbf{#1}.}
%
%\newtheorem{theorem}{Theorem}
%\newtheorem{lemma}[theorem]{Lemma}
%\newtheorem{prop}[theorem]{Proposition}
%
%\begin{document}
%
%\begin{center}
%\bf
%Math 254B, UC Berkeley, Spring 2002 (Kedlaya) \\
%Cohomology of the id\`eles 2: The ``Second Inequality''
%\end{center}

\head{Reference} 
Milne VII.5; Neukirch VI.4.

\medskip
In the previous chapter, we proved that for $L/K$ a cyclic extension of number fields,
the Herbrand quotient $h(C_L)$ of the id\`ele class group of $L$ is equal
to $[L:K]$. This time we'll prove the following.
\begin{theorem} \label{T:first and second inequality}
Let $L/K$ be a Galois extension of number fields, with Galois group $G$.
Then:
\begin{enumerate}
\item[(a)] the group $I_K/(K^* \Norm_{L/K} I_L)$ is finite of order at most
$[L:K]$;
\item[(b)] the group $H^1(G, C_L)$ is trivial;
\item[(c)] the group $H^2(G, C_L)$ is finite of order at most $[L:K]$.
\end{enumerate}
\end{theorem}
By the first inequality, for $L/K$ cyclic, these three are equivalent and
all imply that $H^2(G, C_L)$ has order exactly $[L:K]$. That would suffice
to prove the class field axiom in Neukirch's abstract class field theory.

There are two basic ways to prove this result: an analytic proof and
an algebraic proof. Although the analytic proof is somewhat afield of
what we have been doing (it requires some properties of zeta functions
that we haven't discussed previously), it's somewhat simpler overall than the
algebraic proof. So we'll sketch it first before proceeding to the algebraic
proof.

\head{The analytic proof}

For the analytic proof, we need to recast the Second Inequality back into
classical, ideal-theoretic language. Let $L/K$ be a finite Galois extension
and $\gothm$ a formal product of places of $K$. Back when we defined
generalized ideal class groups, we defined the group $I_{\gothm}$ of
fractional ideals of $K$ coprime to $\gothm$ and $P_{\gothm}$ the group
of principal ideals admitting a generator $\alpha$ such that 
$\alpha \equiv 1 \pmod{\gothp^e}$ if the prime power $\gothp^e$ occurs
in $\gothm$ for a finite prime $\gothp$, and $\tau(\alpha) > 0$ if 
$\tau$ is the real embedding corresponding to a real place in $\gothm$.
Also, let $J_{\gothm}$ be the group of fractional ideals of $L$
coprime to $\gothm$. Then the Second Inequality states that
\[
\# I_{\gothm} / P_{\gothm} \Norm_{L/K} J_{\gothm} \leq [L:K].
\]
Note that we don't have to assume $\gothm$ is divisible by the ramified
primes of $L/K$.

We'll need the following special case of the Chebotarev density theorem,
which fortunately we can prove without already having all of class field
theory.
\begin{prop}
Let $L$ be a finite extension of $K$ and let $M/K$ be its Galois closure.
Then the set $S$ of prime ideals of $K$ that split completely in $L$ has
Dirichlet density $1/[M:K]$.
\end{prop}
\begin{proof}
A prime of $K$ splits completely in $L$ if and only if it splits completely
in $M$, so we may assume $L=M$ is Galois. Recall that the set $T$ of unramfied
primes
$\gothq$ of $L$ of absolute degree 1 has Dirichlet density 1;
each such prime lies over an unramified prime $\gothp$ of $K$ of absolute
degree 1 
which splits completely in $L$.

Now recall how the Dirichlet density works: the set $T$ having
Dirichlet density 1 means that
\[
\sum_{\gothq \in T} \frac{1}{\Norm(\gothq)^s} \sim \frac{1}{s-1}
\qquad s \searrow  1
\]
($s$ approaching 1 from above, that is). If we group the primes in $T$ by
which prime of $S$ they lie over, then we get
\[
[L:K] \sum_{\gothp \in T} \frac{1}{\Norm(\gothp)^s} \sim \frac{1}{s-1}.
\]
That is, the Dirichlet density of $S$ is $1/[L:K]$.
\end{proof}

Now for the inequality. For $\chi: I_{\gothm}/P_{\gothm} \to \CC^*$ a
character, we defined the L-function
\[
L(s, \chi) = \prod_{\gothp \not| \gothm} \frac{1}{1 - \chi(\gothp) 
\Norm(\gothp)^{-s}}.
\]
We'll use some basic properties of this function which can be found in any standard algebraic number theory text.
For starters,
\[
\log L(s, 1) \sim \log \zeta_K(s) \sim \log \frac{1}{s-1} \qquad
s \searrow 1,
\]
while if $\chi$ is not the trivial
character, $L(s, \chi)$ is holomorphic at $s=1$. If $L(s, \chi) =
(s-1)^{m(\chi)} g(s)$ where $g$ is holomorphic and nonvanishing at
$s=1$, then $m(\chi) \geq 0$, and
\[
\log L(s, \chi) \sim m(\chi) \log(s-1) = - m(\chi) \log \frac{1}{s-1}.
\]
Let $H$ be a subgroup of $I_{\gothm}$ containing $P_{\gothm}$.
By finite Fourier analysis, or orthogonality of
characters,
\[
\sum_{\chi: I_{\gothm}/H \to \CC^*} \log L(s, \chi) \sim
\#(I_{\gothm}/H) \sum_{\gothp \in H} \frac{1}{\Norm(\gothp)^{-s}}.
\]
We conclude that the set of primes in $H$ has Dirichlet density
\[
\frac{1 - \sum_{\chi \neq 1} m(\chi)}{\#(I_{\gothm}/H)};
\]
this is $1/\#(I_{\gothm}/H)$ if the $m(\chi)$ are all zero, and 0 otherwise.

We apply this with $H = P_{\gothm} \Norm_{L/K} J_{\gothm}$. This 
in particular includes
every prime of $K$ that splits completely, since such a prime is the norm
of any prime of $L$ lying over it. Thus the set of primes in $H$ has
Dirichlet density, on one hand, is at least $1/[L:K]$. On the other hand,
this set has density either zero or $1/\#(I_{\gothm}/H)$. We conclude
$\#I_{\gothm}/H \leq [L:K]$, as desired.

\head{The algebraic proof}

We now proceed to the algebraic proof of the Second Inequality.
To prove the theorem in general, we can very quickly reduce to the
case of $L/K$ not just cyclic, but cyclic of prime order. The reductions
are similar to those we used to compute $H^2(L^*)$ in the local case.
First of all, if we have the theorem for all solvable groups, then if
$G$ is general and $H$ is a Sylow $p$-subgroup of $G$, then for any 
$G$-module $M$,
\[
\Res: H^i_T(G, M) \to H^i_T(H, M)
\]
is injective on $p$-primary components (because $\Cor \circ \Res$
is multiplication by $[G:H]$), so we can deduce the desired result.
Thus it suffices to consider $L/K$ solvable. In that case, starting
from the cyclic-of-prime-order case we can induct using the 
inflation-restriction exact sequence (Corollary~\ref{C:inflation restriction h2}): if $K'/K$ is a subextension
and $H = \Gal(L/K')$, then for $i=1,2$, 
\[
0 \to H^i(G/H, C_{K'}) \to H^i(G, C_L) \to H^i(H, C_L)
\]
(using the fact that $H^1(H, C_L) = 0$ by the induction hypothesis).
Upshot: we need only consider $L/K$ cyclic of prime order $p$.

One more reduction to make things simpler: we reduce to the case where
$K$ contains a $p$-th root of unity. Let $K' = K(\zeta_p)$ and $L' =
L(\zeta_p)$; then $K'$ and $L$ are linearly disjoint over $K$ (since
their degrees are coprime), so $[L':K'] = [L:K] = p$ and the Galois groups
of $L/K$ and $L'/K'$ are canonically isomorphic. To complete the reduction, it suffices to check
that the homomorphism
\[
H^0_T(\Gal(L/K), C_L) \to H^0_T(\Gal(L'/K'), C_{L'})
\]
induced by the inclusion $C_L \to C_{L'}$ is injective. These groups
are both killed by multiplication by $p$, since for $x \in C_K$,
$\Norm_{L/K} (x) = x^p$. Thus multiplication by $d = [K':K]$, which divides
$p-1$, is an isomorphism on these groups. If $x \in C_K$ maps to the
identity in $H^0_T(\Gal(L'/K'), C_{L'})$, we can choose a representative
of the same class as $x$ in $H^0_T(\Gal(L/K), C_L)$ of the form $y^d$;
then $y$ also maps to the identity in $H^0_T(\Gal(L'/K'), C_{L'})$.
That is, $y = \Norm_{L'/K'}(z')$ for some $z' \in C_{L'}$, and
\[
y^d = \Norm_{K'/K}(y) = \Norm_{L'/K}(z') \in \Norm_{L/K} C_L.
\]
Thus $x \in \Norm_{L/K} C_L$, so the homomorphism is injective.

\head{The key case}

To sum up: it suffices to prove the theorem for $K$ containing a $p$-th
root of unity $\zeta_p$ and $L/K$ cyclic of order $p$. We now address
this case.

As in the proof of the First Inequality, we will use a set $S$ of places
of $K$ containing the infinite places, the primes that ramify in $L$,
and enough additional primes so that $I_K = I_{K,S} K^*$; we also include all places above $(p)$. Again, we
put $K_S = I_{K,S} \cap K^*$. Also we write $s = \# S$.

The plan is to explicitly produce a subgroup of $C_K$ of index $[L:K]$
consisting of norms from $C_L$. We do this by using an auxiliary set of 
places $T$ disjoint from $S$. For such $T$, we define
\[
J = \prod_{v \in S} (K_v^*)^p \times \prod_{v \in T} K_v^*
\times \prod_{v \notin S \cup T} \gotho_{K_v}^*.
\]
Let $\Delta = (L^*)^p \cap K_S$. We will show that:
\begin{enumerate}
\item[(a)] $L = K(\Delta^{1/p})$;
\item[(b)] we can choose a set $T$ of $s-1$ primes such that $\Delta$
is the kernel of the map $K_S \to \prod_{v \in T} K_v^*/(K_v^*)^p$;
\item[(c)] for such a set $T$, if we put $C_{K,S,T} = J K^*/K^*$,
then
\[
\#C_K/C_{K,S,T} = [L:K] = p;
\]
\item[(d)]
with the same notation, $C_{K,S,T} \subseteq \Norm_{L/K} C_L$.
\end{enumerate}
That will imply $\#C_K/\Norm_{L/K} C_L \leq p$, as desired.

We first concentrate on (a).
By Kummer theory, since $K$ contains a
primitive $p$-th root of unity, we can write $L = K(D^{1/p})$ for
$D = (L^*)^p \cap K^*$. Thus $K(\Delta^{1/p}) \subseteq L$ and
since there is no room between $K$ and $L$ for an intermediate extension ($[L:K]$
being prime),
all that we have to check is that 
$K(\Delta^{1/p}) \neq K$.
Choose a single $x \in D$ such that $L = K(x^{1/p})$.
For each $v \notin S$, the extension $K_v(x^{1/p})/K_v$ is unramified,
so we can write $x$ as a unit times a $p$-th power, say
$x = u_v y_v^p$. If we put $y_v = 1$ for $v \in S$, we can assemble the
$y_v$ into an id\`ele $y$, which by $I_K = K^* I_{K,S}$ we can rewrite as
$zw$ for $z \in K^*$ and $w \in I_{K,S}$. Now for $v \notin S$,
$(x/z^p)_v = u_v/w_v^p \in \gotho_{K_v}^*$. Thus $x/z^p \in (L^*)^p \cap
K_S \in \Delta$ but $x \notin (K^*)^p$. We conclude $L
= K(\Delta^{1/p})$.

Now we move to (b). Put $N = K(K_S^{1/p})$. By Kummer theory,
\[
\Gal(N/K) \cong \Hom(K_S/K_S^p, \ZZ/p\ZZ).
\]
By the generalization of Dirichlet's units theorem to $S$-units,
$K_S$ modulo torsion is a free abelian group of rank $s-1$,
and the torsion subgroup consists of roots of unity, so is cyclic
of order divisible by $p$. Thus $K_S/K_S^p \cong (\ZZ/p\ZZ)^s$.
Choose generators $g_1, \dots, g_{s-1}$ of 
$\Gal(N/L)$; these correspond in $\Hom(K_S/K_S^p, \ZZ/p\ZZ)$
to a set of homomorphisms whose common kernel is precisely
$\Delta/K_S^p$. 

So to establish (b), we need to find for each $g_i$
a place $v_i$ such that the kernel of $g_i$ is the same as the kernel of
$K_S \to K_{v_i}^*/(K_{v_i}^*)^p$. Let $N_i$ be the fixed field of $g_i$; 
by the First Inequality (see Corollary~\ref{C:split completely}), there are infinitely
many primes of $N_i$ that do not split in $N$. So we can choose
a place $w_i$ of each $N_i$ such that their restrictions $v_i$ to $K$ are
distinct, not contained in $S$, and don't divide $p$. 

We claim $N_i$ is
the maximal subextension of $N/K$ in which $v_i$ splits completely (a/k/a
the \emph{decomposition field} of $v_i$). On one hand, $v_i$ does not split
completely in $N$, so the decomposition field is no larger than $N_i$.
On the other hand, the decomposition field is the fixed field of
the decomposition group, which has exponent $p$ and is cyclic (since
$v_i$ does not ramify in $N$). Thus it must have index $p$ in $N$,
so must be $N_i$ itself.

Thus $L = \cap N_i$ is the maximal subextension of $N$ in which all of
the $v_i$ split completely. We conclude that for $x \in K_S$,
$x$ belongs to $\Delta$ iff $K_{v_i}(x^{1/p}) = K_{v_i}$ for all $i$,
which occurs iff $x \in K_{v_i}^p$. That is, $\Delta$ is precisely the
kernel of the map $K_S \to \prod_i K_{v_i}^*/(K_{v_i}^*)^p$.
In fact, under this map, $K_S$ actually maps to the units in
$K_{v_i}^*$ for each $i$. This proves (b).

Next, we verify (c), using the following lemma.
\begin{lemma} \label{L:adelic intersection}
$J \cap K^* = (K_{S \cup T})^p$.
\end{lemma}
\begin{proof}
Clearly $K_{S \cup T}^p \subseteq J \cap K^*$; we have to work
to show the other inclusion. Take $y \in J \cap K^*$ and
$M = K(y^{1/p})$. We'll show that $\Norm_{M/K} C_M = C_K$; by the First
Inequality, this will imply $M = K$, so $y \in (K^*)^p \cap J
= (K_{S \cup T})^p$.

Since $I_K = I_{K,S}K^*$, it is enough to choose $\alpha \in I_{K,S}$
and show that $\alpha/x \in \Norm_{M/K} I_M$ for some $x \in K^*$.
As noted above, the map
\[
K_S \to \prod_{v \in T} \gotho_{K_v}^*/(\gotho_{K_v}^*)^p
\]
is surjective, and $\#K_S/\Delta = p^{s-1}$. That's also the order of
the product, so the map is actually an isomorphism. Thus we can find
$x \in K_S$ so that $\alpha/x$ has component the $p$-th power of a unit
of $K_v$
at each $v \in T$. In particular, such a component is the norm of its
$p$-th root, so $\alpha/x$ is a norm at each $v \in T$. For $v \in S$,
we don't have anything to check: because $y$ is a $p$-th power at $v$,
$M_w = K_v$. Finally, for $v \notin S \cup T$, $M_w/K_v$ is unramified,
so any unit is a norm. Thus $\alpha/x$ is indeed a norm. We conclude
$\Norm_{M/K} C_M = C_K$, so $M=K$ and $y \in K_{S\cup T}^p$,
as desired.
\end{proof}

Given the lemma, we now have an exact sequence
\[
1 \to (I_{K,S\cup T} \cap K^*)/(J \cap K^*) 
\to I_{K,S\cup T}/J \to
I_{K,S \cup T}K^* / JK^* \to 1.
\]
We can rewrite $I_{K,S \cup T}K^*$ as simply $I_K$, so the group on the
right is precisely $C_K/C_{K,S,T}$. By the lemma, the group on the left is
$K_{S\cup T}^*/(K_{S \cup T}^*)^n$, which has order $p^{2s-1}$ because
$K_{S\cup T}$ is free of rank $2s-2$ plus a cyclic group of order a multiple
of $p$. The group in the middle is the product of $K_v^*/(K_v^*)^p$
over all $v \in S$, and each of those has order $p^2$ (generated by
$\zeta_p$ and a uniformizer of $K_v$). Adding it all up,
we get $\#C_K/C_{K,S,T} = p$, proving (c).

Finally, to check (d), it suffices to check that $J
\subseteq \Norm_{L/K} I_L$, which we may check component by component.
It's automatic for the places $v \notin S \cup T$, since those places
are unramified, so every unit is a norm. For places $v \in S$, any element
of $(K_v^*)^p$ is a norm from $K_v(K_v^{1/p})$ by local reciprocity, so also from $L_w$.
Finally, for places $v \in T$, from the construction of $T$, we see
that $\Delta \subseteq (K_v^*)^p$, so $L_w = K_v$, and so
$K_v^*$ consists entirely of norms.

\head{Aside} We get from this calculation that $H^{-1}_T(G, C_L) = 1$, so
$H^0_T(G, L^*) \to H^0_T(G, C_L)$ is injective. That is,
\[
K^*/\Norm_{L/K} L^* \to \bigoplus_v K_v^*/\Norm_{L_w/K_v} L_w^*
\]
is injective. In other words, we have an interesting ``local-to-global''
statement, namely Hasse's Norm Theorem: if $L/K$
is cyclic, $x \in K^*$ is a norm if and only if it is locally a norm.

%\end{document}
