%\documentclass[12pt]{article}
%\usepackage{amsfonts, amsthm, amsmath}
%\usepackage[all]{xy}
%
%\setlength{\textwidth}{6.5in}
%\setlength{\oddsidemargin}{0in}
%\setlength{\textheight}{8.5in}
%\setlength{\topmargin}{0in}
%\setlength{\headheight}{0in}
%\setlength{\headsep}{0in}
%\setlength{\parskip}{0pt}
%\setlength{\parindent}{20pt}
%
%\def\kbar{\overline{k}}
%\def\AA{\mathbb{A}}
%\def\CC{\mathbb{C}}
%\def\FF{\mathbb{F}}
%\def\PP{\mathbb{P}}
%\def\QQ{\mathbb{Q}}
%\def\RR{\mathbb{R}}
%\def\ZZ{\mathbb{Z}}
%\def\gotha{\mathfrak{a}}
%\def\gothb{\mathfrak{b}}
%\def\gothm{\mathfrak{m}}
%\def\gotho{\mathfrak{o}}
%\def\gothp{\mathfrak{p}}
%\def\gothq{\mathfrak{q}}
%\def\gothr{\mathfrak{r}}
%\DeclareMathOperator{\ab}{ab}
%\DeclareMathOperator{\coker}{coker}
%\DeclareMathOperator{\Cor}{Cor}
%\DeclareMathOperator{\disc}{Disc}
%\DeclareMathOperator{\Frob}{Frob}
%\DeclareMathOperator{\Gal}{Gal}
%\DeclareMathOperator{\GL}{GL}
%\DeclareMathOperator{\Hom}{Hom}
%\DeclareMathOperator{\im}{im}
%\DeclareMathOperator{\Ind}{Ind}
%\DeclareMathOperator{\Inf}{Inf}
%\DeclareMathOperator{\inv}{inv}
%\DeclareMathOperator{\Norm}{Norm}
%\DeclareMathOperator{\Res}{Res}
%\DeclareMathOperator{\Trace}{Trace}
%\DeclareMathOperator{\unr}{unr}
%\DeclareMathOperator{\Cl}{Cl}
%
%\def\head#1{\medskip \noindent \textbf{#1}.}
%
%\newtheorem{theorem}{Theorem}
%\newtheorem{lemma}[theorem]{Lemma}
%\newtheorem{cor}[theorem]{Corollary}
%\newtheorem{prop}[theorem]{Proposition}
%
%\begin{document}
%
%\begin{center}
%\bf
%Math 254B, UC Berkeley, Spring 2002 (Kedlaya) \\
%Local Class Field Theory, via Tate's Theorem
%\end{center}

\head{Reference} Milne II.3, III.5.

\medskip

For $L/K$ a finite extension of local fields, we have now computed that
$H^1(L/K) = 0$ (Lemma~\ref{L:theorem 90})
and $H^2(L/K)$ is cyclic of order $[L:K]$ (Proposition~\ref{P:local h2}).
In this chapter, we use these ingredients to establish all of the statements of local class field theory.

\head{Tate's theorem}

We first prove the theorem of Tate stated earlier (Theorem~\ref{T:tate thm1}).
\begin{theorem}[Tate] \label{T:tate thm2}
Let $G$ be a finite (solvable) group and let $M$ be a $G$-module. Suppose
that for all subgroups $H$ of $G$ (including $G$ itself), $H^1(H,M)=0$
and $H^2(H,M)$ is cyclic of order $\#H$.
Then there are isomorphisms
$H^i_T(G, \ZZ) \to H^{i+2}_T(G, M)$ which are canonical up to a choice of
generator of $H^2(G, M)$.
\end{theorem}
\begin{proof}
Let $\gamma$ be a generator of $H^2(G, M)$. Since $\Cor \circ \Res
= [G:H]$, $\Res(\gamma)$ generates $H^2(H,M)$ for any $H$. We start
out by explicitly constructing a $G$-module containing $M$ in which
$\gamma$ becomes a coboundary.

Choose a 2-cocycle $\phi: G^3 \to M$ representing $\gamma$; by the definition
of a cocycle, 
\begin{gather*}
\phi(g_0 g, g_1 g, g_2 g) = \phi(g_0, g_1, g_2)^g, \\
\phi(g_1, g_2, g_3) - \phi(g_0, g_2, g_3) + \phi(g_0, g_1, g_3)
- \phi(g_0, g_1, g_2) = 0.
\end{gather*}
Moreover, $\phi$ is a coboundary if and only if it's of the form
$d(\rho)$, that is, $\phi(g_0, g_1, g_2) = \rho(g_1, g_2) -
\rho(g_0, g_2) + \rho(g_0, g_1)$. This $\rho$ must itself be $G$-invariant:
$\rho(g_0, g_1)^g = \rho(g_0g, g_1g)$. Thus $\phi$ is a coboundary if
$\phi(e, g, hg) = \rho(e,h)^g - \rho(e,hg) + \rho(e,g)$.

Let $M[\phi]$ be the direct sum of $M$ with the free abelian group
with one generator $x_g$ for each element $g$
of $G - \{e\}$, with the $G$-action
\[
x_h^g = x_{hg} - x_g + \phi(e, g, hg).
\]
(The symbol $x_e$ should be interpreted as $\phi(e,e,e)$.)
Using the cocycle property of $\phi$,
one may verify that this is indeed a $G$-action; by construction,
the cocycle $\phi$ becomes zero in $H^2(G, M[\phi])$ by setting
$\rho(e,g) = x_g$. (Milne calls $M[\phi]$ the \emph{splitting module} of
$\phi$.)

The map $\alpha: M[\phi] \to \ZZ[G]$ sending $M$ to zero and $x_g$
to $[g]-1$ is a homomorphism of $G$-modules. Actually it maps into the
augmentation ideal $I_G$, and the sequence
\[
0 \to M \to M[\phi] \to I_G \to 0
\]
is exact. (Note that this is backwards from the usual exact sequence featuring $I_G$ as a submodule, which will appear again momentarily.)
For any subgroup $H$ of $G$, we can restrict to $H$-modules, then
take the long exact sequence:
\[
0 = H^1(H,M) \to H^1(H, M[\phi]) \to H^1(H, I_G)
\to H^2(H, M) \to H^2(H, M[\phi]) \to H^2(H, I_G).
\]
To make headway with this, view $0 \to I_G \to \ZZ[G] \to \ZZ \to 0$
as an exact sequence of $H$-modules. 
 Since $\ZZ[G]$ is induced, its Tate
groups all vanish. So we get a dimension shift:
\[
H^1(H, I_G) \cong H^0_T(H, \ZZ) = \ZZ/(\#H)\ZZ.
\]
Similarly, $H^2(H, I_G) \cong H^1(H, \ZZ) = 0$.
Also, the map $H^2(H, M) \to H^2(H, M[\phi])$ is zero because
we constructed this map so as to kill off the generator $\phi$. 
Thus $H^2(H, M[\phi]) = 0$ and $H^1(H, I_G) \to H^2(H,M)$ is surjective. But these groups are both finite
of the same order! So the map is also injective, and $H^1(H, M[\phi])$
is also zero.

Now apply Lemma~\ref{L:tate thm lemma} below to conclude that $H^i_T(G, M) = 0$ for all $i$.
This allows us to use the four-term exact sequence
\[
0 \to M \to M[\phi] \to \ZZ[G] \to \ZZ \to 0
\]
(as in the proof of Theorem~\ref{T:cyclic group periodicity})
to conclude that $H^i_T(G, \ZZ) \cong H^{i+2}_T(G, M)$.
\end{proof}

Note: we only need the results of this section for $G$ solvable, because
in our desired application $G$ is the Galois group of a finite extension
of local fields. But one can remove this restriction: see the note after
this lemma.
\begin{lemma} \label{L:tate thm lemma}
Let $G$ be a finite (solvable) group and $M$ a $G$-module. Suppose that
$H^i(H,M) =0$ for $i=1,2$ and $H$ any subgroup of $G$ (including $G$
itself). Then $H^i_T(G,M) = 0$ for all $i \in \ZZ$.
\end{lemma}
\begin{proof}
For $G$ cyclic, this follows by periodicity. We prove the general result
by induction on $\#G$. Since $G$ is solvable, it has a proper subgroup $H$ for which
$G/H$ is cyclic. By the induction hypothesis, $H^i_T(H,M) = 0$ for all
$i$. Thus by the inflation-restriction exact sequence (Proposition~\ref{P:inflation restriction}),
\[
0 \to H^i(G/H, M^H) \to H^i(G, M) \to H^i(H, M)
\]
is exact for all $i>0$. The term on the end being zero, we have
$H^i(G/H, M^H) \cong H^i(G,M) = 0$ for $i=1, 2$. By periodicity (Theorem~\ref{T:cyclic group periodicity}),
$H^i_T(G/H, M^H) = 0$ for all $i$, so $H^i(G/H, M^H) = 0$ for all
$i>0$, and $H^i(G,M) = 0$ for $i>0$. As for $H^0_T(G,M)$, we have that
$H^0_T(G/H, M^H) = 0$, so for any $x \in M^G$, there exists $y \in M^H$
such that $\Norm_{G/H}(y) = x$. Since $H^0_T(H,M) = 0$, there exists
$z \in M$ such that $\Norm_{H}(z) = x$. Now
$\Norm_G(z) = \Norm_{G/H} \circ \Norm_H(z) = x$. Thus
$H^0_T(G,M) = 0$, as desired.

So far so good, but we want to kill off the Tate groups with negative indices
too, so we do a dimension shift. 
Make the exact sequence
\[
0 \to N \to \Ind^G_1 M \to M \to 0
\]
in which $m \otimes [g]$ maps to $m^g$. The term in the middle is acyclic, so
$H^{i+1}_T(H', N) \cong H^{i}_T(H', M)$ for any subgroup $H'$
of $G$. In particular, $H^1(H', N) = 
H^2(H', N) = 0$, so the above argument gives $H^i_T(G, N) = 0$ for $i\geq 0$.
Now from $H^0_T(G, N) = 0$
we get $H^{-1}_T(G, M) = 0$; since the same argument applies to $N$,
we also get $H^{-2}_T(G, M) = 0$ and so on.
\end{proof}
To go from the solvable case to the general case, one shows that
the $p$-primary component of $H^i(G,M)$ injects into $H^i(G_p, M)$,
where $G_p$ is the $p$-Sylow subgroup. (Apply $\Cor \circ \Res$ from
$G$ to $G_p$; the result is multiplication by $[G:H]$ which is prime to
$p$.)

\head{The results of local class field theory}

Let $L/K$ be a finite Galois extension of local fields.
For any intermediate extension $M/K$, we know that
$H^1(L/M) = 0$ and $H^2(L/M)$ is cyclic of order $[L:M]$. We may thus apply Theorem~\ref{T:tate thm2} with for $G = \Gal(L/K)$, $M = L^*$
to obtain isomorphisms $H^i_T(G, \ZZ) \to H^{i+2}_T(G,M)$,
thus proving Theorem~\ref{T:cup product isomorphism}. This yields
 a canonical isomorphism 
\[
K^*/\Norm_{L/K} L^* = H^0_T(L/K)
\to  H^{-2}_T(\Gal(L/K), \ZZ) = \Gal(L/K)^{\ab}.
\]
This establishes the existence of the local reciprocity map (Theorem~\ref{T:local reciprocity};
note that part (a) follows from the explicit computations in Chapter~\ref{chap:localcomp})
and the norm limitation theorem (Theorem~\ref{T:norm limitation}), modulo one subtlety:
if $M/K$ is another finite Galois extension containing $L$, we need to know that the diagram
\[
\xymatrix{
K^*/\Norm_{M/K} M^* \ar[r] \ar[d]  & \Gal(M/K)^{\ab} \ar[d] \\
K^*/\Norm_{L/K} L^* \ar[r] & \Gal(L/K)^{\ab}
}
\]
commutes,
so the maps $K^* \to \Gal(L/K)^{\ab}$ fit together to give a map $K^* \to \Gal(K^{\sep}/K)^{\ab}$. 
In other words, we need a commuting diagram
\[
\xymatrix{
H^0_T(\Gal(M/K), M^*)  \ar[r] \ar[d] & H^{-1}_T(\Gal(M/K), I_{\Gal(M/K)}) \ar[d] \\
 H^0_T(\Gal(L/K), L^*) \ar[r]& H^{-1}_T(\Gal(L/K), I_{\Gal(L/K)})
}
\]
This appears to be a gap in Milne's presentation.
To fix it, choose a 2-cocycle $\phi_M: \Gal(M/K)^3 \to M^*$ representing the preferred generator of $H^2(M/K)$; then the upper horizontal arrow is a connecting homomorphism
for the exact sequence
\[
1 \to M^* \to M^*[\phi_M] \to I_{\Gal(M/K)} \to 1.
\]
The lower horizontal arrow arises similarly from the exact sequence
\[
1 \to L^* \to L^*[\phi_L] \to I_{\Gal(L/K)} \to 1,
\]
where $\phi_L$ represents a class whose inflation is $[G:H]$ times the class represented by $\phi_M$.
Further details omitted.

In any case, it remains to prove the local existence theorem (Theorem~\ref{T:local existence}). We begin with a lemma, in which we take advantage of Kummer theory to establish an easy case of the existence theorem.
\begin{lemma} \label{L:hilbert symbol}
Let $\ell$ be a prime number. Let $K$ be a local field containing a primitive $\ell$-th root of unity. Then $x \in K^*$ is an $\ell$-th power in $K$ if and only if belongs to $\Norm_{L/K} L^*$ for every cyclic extension $L$ of $K$ of degree $\ell$.
\end{lemma}
The same statement holds even if $\ell$ is not prime (exercise)
and can be interpreted in terms of the \emph{Hilbert symbol}, whose properties generalize quadratic reciprocity to higher powers; see Milne, III.4.
\begin{proof}
Let $M$ be the compositum of all cyclic $\ell$-extensions of $K$.
The group $K^*/(K^*)^{\ell}$ is finite (exercise),
and hence is isomorphic to $(\ZZ/\ell \ZZ)^n$ for some positive integer $n$.
By Kummer theory (Theorem~\ref{T:Kummer reformulated}),
we also have $\Gal(M/K) \cong (\ZZ/\ell \ZZ)^n$. By the local reciprocity law,
$K^*/\Norm_{M/K}M^* \cong (\ZZ/\ell \ZZ)^n$; consequently, on one hand
$(K^*)^{\ell} \subseteq \Norm_{M/K}M^*$, and on other hand these subgroups of $K^*$ have the same index $\ell^n$. They are thus equal, proving the claim.
\end{proof}

This allows to deduce a corollary of the existence theorem which is needed in its proof.
\begin{cor} \label{C:universal norms divisible}
Let $K$ be a local field.
Then the intersection of the groups $\Norm_{L/K} L^*$ for all finite extensions $L$ of $K$ is the trivial group.
\end{cor}
\begin{proof}
Let $D_K$ be the intersection in question; note that $D_K \subseteq U_K$ by considering unramified extensions of $K$, so $D_K$ is in particular a compact topological group.
By Lemma~\ref{L:hilbert symbol}, every element of $D_K$ is an $\ell$-th power in $K$ for every prime $\ell$; it remains to check that one can find an $\ell$-th root which is also in $D_K$. This would then imply that $D_K$ is a divisible subgroup of $U_K$, and hence the trivial group (see exercises).

For $L/K$ a finite extension, it is true but  not immediately clear that 
\[
\Norm_{L/K} D_L = D_K;
\]
that is, for $x \in D_K$, for each finite extension $M$ of $K$, $x = \Norm_{M/K}(z)$ for some $z \in M$, but may not be apparent that the elements $y = \Norm_{M/L}(z)$ can be chosen to be equal. However, for a given $M$, the set of such $y$ is a nonempty compact subset of $U_L$, and any finite intersection of these sets is nonempty (since it contains the set corresponding to the compositum of the corresponding fields), so the whole intersection is nonempty.

Let $\ell$ be a prime number and choose $x \in D_K$. For each finite extension $L$ of $K$ containing a primitive $\ell$-th root of unity, let $E(L)$ be the set of $\ell$-th roots of $x$ in $K$ which belong to $\Norm_{L/K} L^*$.
This set is finite (it can contain at most $\ell$ elements) and nonempty: we have $x = \Norm_{L/K}(y)$ for some $y \in D_L$, so $y$ has an $\ell$-th root $z$ in $L$ 
and $\Norm_{L/K}(z) \in E(L)$. Again by the finite intersection property,
we find an $\ell$-th root of $x$ in $K$ belonging to $D_K$, completing the proof.
\end{proof}

Returning to the local existence theorem,
let $U$ be an open subgroup of $K^*$ of finite index; we wish to find a finite abelian extension $L$ of $K$ such that $U = \Norm_{L/K}L^*$. 
We note first that by the local reciprocity law, it is enough to construct $L$ so that
$U$ contains $\Norm_{L/K}L^*$: in this case, we will have $\Gal(L/K) \cong K^*/\Norm_{L/K}L^*$, and then $U/\Norm_{L/K}L^*$ will corresponding to $\Gal(L/M)$ for some intermediate extension $M/K$ having the desired effect. We note next that by the norm limitation theorem, it suffices to produce \emph{any} finite extension $L/K$, not necessarily abelian, such that $U$ contains $\Norm_{L/K}L^*$.

Let $m\ZZ \subseteq \ZZ$ be the image of $U$ in $K^*/U_K \cong \ZZ$;
by choosing $L$ to contain the unramified extension of $K$ of degree $m$,
we may ensure that the image of $\Norm_{L/K} L^*$ in $K^*/U_K$ is also contained in $m\ZZ$.
It thus remains to further ensure that 
\[
(\Norm_{L/K} L^*) \cap U_K \subseteq U \cap U_K.
\]
Since $U_K$ is compact, each open subgroup $(\Norm_{L/K}L^*) \cap U_K$ is also closed and hence compact. By Corollary~\ref{C:universal norms divisible}, as $L/K$ runs over all finite extensions of $K$,
the intersection of the groups $(\Norm_{L/K} L^*) \cap U_K$ is trivial; in particular, the intersection of the compact subsets 
\[
((\Norm_{L/K} L^*) \cap U_K) \cap (U_K \setminus U)
\]
of $U_K$ is empty. By the finite intersection property (and taking a compositum), there exists a single $L/K$ for which $(\Norm_{L/K} L^*) \cap U_K \subseteq U \cap U_K$;
this completes the proof of Theorem~\ref{T:local existence}.


\head{Making things explicit}

It is natural to ask whether the local reciprocity map can be described more explicitly. In fact, given an explicit cocycle $\phi$ generating $H^2(L/K)$, we can trace 
through the arguments to get the local reciprocity map. However, the argument is somewhat messy, so I won't torture you with all of the details; the point is
simply to observe that everything we've done can be used for explicit
computations. (This observation is apparently due to Dwork.) If you find
this indigestible, you may hold out until we hit abstract class field theory; that
point of view will give a different (though of course related) mechanism
for computing the reciprocity map.

Put $G = \Gal(L/K)$.
First recall
that $G^{\ab} = H^{-2}_T(G, \ZZ)$ is isomorphic to $H^{-1}_T(G, I_G)
= I_G/I_G^2$, with $g \mapsto [g]-1$. Next, use the exact sequence
\[
0 \to M \to M[\phi] \to I_G \to 0
\]
and apply the ``snaking'' construction: pull $[g]-1$ back to
$x_g \in M[\phi]$, take the norm to get $\prod_h x_g^h = \prod_h
(x_{gh} x_h^{-1} \phi(e,h,gh))$ (switching to multiplicative notation).
The $x_{gh}$ and $x_h$ term cancel out when you take the product, so we
get $\prod_h \phi(e, h, gh) \in L^*$ as the inverse image of $g \in \Gal(L/K)$.

As noted above, one needs $\phi$ to make this truly explicit; one can get
$\phi$ using explicit generators of $L/K$ if you have them. For $K = \QQ_p$,
one can use roots of unity; for general $K$, one can use the Lubin-Tate
construction. In general, one can at least do the following, imitating
our proof that $H^2(L/K)$ is cyclic of order $n$. Let $M/K$
be unramified of degree $n$; then $H^2(M/K) \to H^2(ML/K)$ is injective,
and its image lies in the image of $H^2(L/K) \to H^2(ML/K)$.

Now $H^2(M/K)$ is isomorphic to $H^0_T(M/K) = K^*/\Norm_{M/K}M^*$, which
is generated by a uniformizer $\pi \in K$. To explicate that isomorphism,
we recall generally how to construct the isomorphism
$H^0_T(G,M) \to H^2_T(G,M)$ for $G$ cyclic with a distinguished generator $g$.
Recall the exact sequence we used to produce the isomorphism
in Theorem~\ref{T:cyclic group periodicity}:
\[
0 \to M \to M \otimes_{\ZZ} \ZZ[G] \to M \otimes_{\ZZ} \ZZ[G] \to M \to 0.
\]
(Remember, $G$ acts on both factors in $M \otimes_{\ZZ}
\ZZ[G]$. The first map is $m \mapsto \sum_{h \in G} m \otimes [h]$,
the second is $m \otimes [h] \mapsto m \otimes ([gh] - [h])$, and
the third is $[h] \mapsto 1$.)
Let $A = M \otimes_{\ZZ} I_G$ be the kernel of the third arrow, so
$0 \to M \to M \otimes_{\ZZ} \ZZ[G] \to A \to 0$
and $0 \to A \to M \otimes_{\ZZ} \ZZ[G] \to M \to 0$ are exact.

Given $x \in H^0_T(M/K) = M^G/\Norm_G(M)$, lift it to $x \otimes [1]$.
Now view this as a 0-cochain $\phi_0: G \to M \otimes_{\ZZ} \ZZ[G]$ given by
$\phi_0(h) = x \otimes [h]$. Apply $d$ to get a 1-cocycle:
\[
\phi_1(h_0, h_1) = \phi_0(h_1) - \phi_0(h_0) = x \otimes ([h_1]- [h_0])
\]
which actually takes values in $A$. Now snake again: pull this back to
a 1-cochain $\psi_1: G^2 \to M \otimes_{\ZZ} \ZZ[G]$ given by
\[
\psi_1(g^i, g^{i+j}) = x \otimes ([g^i] + [g^{i+1}] + \cdots + [g^{j-1}])
\]
for $i,j=0, \dots, \#G-1$.
Apply $d$ again: now we have a 2-cocycle $\psi_2: G^3 \to M \otimes_{\ZZ}
\ZZ[G]$ given by (again for $i,j=0, \dots, \#G-1$)
\begin{align*}
\psi_2(e, g^i, g^{i+j}) &= \psi_1(g^i, g^{i+j}) - \psi_1(e, g^{i+j})
+ \psi_1(e, g^i) \\
&= x \otimes ([e] + \cdots + [g^{i-1}] + [g^i] + \cdots + [g^{i+j-1}]
- [e] - \cdots - [g^{i+j-1}]) \\
&= \begin{cases} 0 & i+j < \#G \\
-x \otimes ([e] + \cdots + [g^{\#G-1}]) & i+j \geq \#G.
\end{cases}
\end{align*}
This pulls back to a 2-cocycle $\phi_2: G^3 \to M$ given by
\[
\phi_2(e, g^i, g^{i+j}) = \begin{cases} 0 & i+j < \#G \\
-x & i+j \geq \#G.
\end{cases}
\]
If you prefer, you can shift by a coboundary to get $x$ if $i+j < \#G$
and 0 if $i+j \geq \#G$.

Back to the desired computation. Applying this to $\Gal(M/K)$ acting on
$M^*$, with the canonical generator $g$ equal to the Frobenius,
we get that $H^2(M/K)$ is generated by a cocycle $\phi$ with
$\phi(e, g^i, g^{i+j}) = \pi$ if $i+j < \#G$ and 1 otherwise. Now push
this into $H^2(ML/K)$; the general theory says the image comes from 
$H^2(L/K)$. That is, for $h \in \Gal(ML/K)$, let $f(h)$ be the integer $i$
such that $h$ restricted to $\Gal(M/K)$ equals $g^i$. Then
there exists a 1-cochain $\rho: \Gal(ML/K)^2 \to (ML)^*$ such that
$\phi(e, h_1, h_2h_1) /(\rho(h_1, h_2h_1) \rho(e, h_2h_1)^{-1} \rho(e, h_1))$
belongs to $L^*$ and depends only on the images of $h_1, h_2$ in $\Gal(M/K)$.
Putting $\sigma(h) = \rho(e, h)$, we thus have
\[
\frac{\phi(e, h_1, h_2h_1) \sigma(h_2h_1)}{\sigma(h_2)^{h_1} \sigma(h_1)}
\]
depends only on $h_1, h_2$ modulo $\Gal(ML/L)$.

The upshot: once you compute such a $\sigma$ (which I won't describe how to
do, since it requires an explicit description of $L/K$), to
find the inverse image of $g \in \Gal(L/K)$ under the Artin map, 
choose a lift $g_1$ of $g$ into $\Gal(ML/K)$, then compute
\[
\prod_h \frac{\phi(e, h, gh) \sigma(gh)}{\sigma(g)^h \sigma(h)}
\]
for $h$ running over a set of lifts of the elements of $\Gal(L/K)$ into
$\Gal(ML/K)$.

\head{Exercises}

\begin{enumerate}
\item
Prove that for any local field $K$ and any positive integer $n$ not divisible by the characteristic of $K$, the group
$K^*/(K^*)^{n}$ is finite.
\item
Prove that for any local field $K$ of characteristic $0$, the intersection of the groups
$(K^*)^n$ over all positive integers $n$ is the trivial group. (Hint: first get the intersection into $\gotho_K^*$, then use prime-to-$p$ exponents to get it into the 1-units, then use powers of $p$ to finish. The last step is the only one which fails in characteristic $p$.)
\item
Extend Lemma~\ref{L:hilbert symbol} to the case where $\ell$ is an arbitrary positive integer, not necessarily prime. (Hint: it may help to use the structure theorem for finite abelian groups.)
\end{enumerate}

%\end{document}
