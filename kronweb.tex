%\documentclass[12pt]{article}
%\usepackage{amsfonts, amsthm, amsmath}
%
%\setlength{\textwidth}{6.5in}
%\setlength{\oddsidemargin}{0in}
%\setlength{\textheight}{8.5in}
%\setlength{\topmargin}{0in}
%\setlength{\headheight}{0in}
%\setlength{\headsep}{0in}
%\setlength{\parskip}{0pt}
%\setlength{\parindent}{20pt}
%
%\def\CC{\mathbb{C}}
%\def\FF{\mathbb{F}}
%\def\PP{\mathbb{P}}
%\def\QQ{\mathbb{Q}}
%\def\RR{\mathbb{R}}
%\def\ZZ{\mathbb{Z}}
%\def\gotha{\mathfrak{a}}
%\def\gothb{\mathfrak{b}}
%\def\gothm{\mathfrak{m}}
%\def\gotho{\mathfrak{o}}
%\def\gothp{\mathfrak{p}}
%\def\gothq{\mathfrak{q}}
%\DeclareMathOperator{\disc}{Disc}
%\DeclareMathOperator{\Gal}{Gal}
%\DeclareMathOperator{\Norm}{Norm}
%\DeclareMathOperator{\Trace}{Trace}
%\DeclareMathOperator{\Cl}{Cl}
%
%\def\head#1{\medskip \noindent \textbf{#1}.}
%\def\fixme#1{\textbf{FIXME! #1}}
%
%\newtheorem{theorem}{Theorem}

%\begin{document}
%
%\begin{center}
%\bf
%Math 254B, UC Berkeley, Spring 2002 (Kedlaya) \\
%The Kronecker-Weber Theorem
%\end{center}
%
\head{Reference} Our approach follows
Washington, \textit{Introduction to Cyclotomic Fields}, Chapter 14.
A variety of other methods can be found in other texts.

\head{Abelian extensions of $\QQ$}

Though class field theory has its origins in the law of quadratic 
reciprocity discovered by Gauss, its proper beginning is indicated by the
Kronecker-Weber theorem, first stated by Kronecker in 1853 and proved by
Weber in 1886. Although one could skip this theorem and deduce it as a
consequence of more general results later on, I prefer to work through
it explicitly. It will provide a ``trailer'' for the rest of the course,
giving us a preview of a number of key elements:
\begin{itemize}
\item reciprocity laws;
\item passage between local and global fields, using Galois theory;
\item group cohomology, and applications to classifying field extensions;
\item computations in local fields.
\end{itemize}

An \emph{abelian extension} of a field is a Galois extension with abelian 
Galois group. An example of an abelian extension of $\QQ$ is the cyclotomic
field $\QQ(\zeta_n)$ (where $n$ is a positive integer and $\zeta_n$ is a primitive $n$-th root of 
unity), whose Galois group is $(\ZZ/n\ZZ)^*$, or any
subfield thereof. Amazingly, there are no other examples!
\begin{theorem}[Kronecker-Weber] \label{T:Kronecker-Weber}
If $K/\QQ$ is a finite abelian extension, then
$K \subseteq \QQ(\zeta_n)$ for some positive integer $n$.
\end{theorem}
For example, every quadratic extension of $\QQ$ is contained in a cyclotomic
field, a fact known to Gauss.

The smallest $n$ such that $K \subseteq \QQ(\zeta_n)$ is called the
\emph{conductor} of $K/\QQ$. It plays an important role in the splitting
behavior of primes of $\QQ$ in $K$, as we will see a bit later.

We will prove this theorem in the next few lectures. Our approach will be to
deduce it from a local analogue (see Theorem~\ref{T:local Kronecker-Weber2}).
\begin{theorem}[Local Kronecker-Weber] \label{T:local Kronecker-Weber1}
If $K/\QQ_p$ is a finite abelian extension, then
$K \subseteq \QQ_p(\zeta_n)$ for some $n$, where $\zeta_n$
is a primitive $n$-th root of unity.
\end{theorem}

Before proceeding, it is worth noting explicitly a nice property of
abelian extensions that we will exploit below. Let $L/K$ be a Galois
extension with Galois group $G$, let $\gothp$ be a prime of $K$,
let $\gothq$ be a prime of $L$ over $\gothp$, and let $G_{\gothq}$ and
$I_{\gothq}$ be the decomposition and inertia groups of $\gothq$,
respectively. Then any other prime $\gothq'$ over $\gothp$ can be written
as $\gothq^g$ for some $g \in G$, and the decomposition and inertia groups
of $\gothq'$ are the conjugates $g^{-1} G_{\gothq} g$ and $g^{-1} I_{\gothq}
g$, respectively. (Note: my Galois actions will always be right actions,
denoted by superscripts.) If $L/K$ is \emph{abelian}, though, these
conjugations have no effect. So it makes sense to talk about \emph{the}
decomposition and inertia groups of $\gothp$ itself!

\head{A reciprocity law}

Assuming the Kronecker-Weber theorem, we can deduce strong results
about the way primes of $\QQ$ split in an abelian extension. Suppose
$K/\QQ$ is abelian, with conductor $m$. Then we get a surjective homomorphism
\[
(\ZZ/m\ZZ)^* \cong \Gal(\QQ(\zeta_m)/\QQ) \to \Gal(K/\QQ).
\]
On the other hand, suppose $p$ is a prime not dividing $m$, so that
$K/\QQ$ is unramified above $p$. As noted above, there is a well-defined
decomposition group $G_p \subseteq \Gal(K/\QQ)$. Since there is no ramification
above $p$, the corresponding inertia group is trivial, so $G_p$ is generated
by a Frobenius element $F_p$, which modulo any prime above $p$,
acts as $x \mapsto x^p$. We can formally extend the map $p \mapsto F_p$
to a homomorphism from
$S_m$, the subgroup of $\QQ$ generated by all primes not dividing $m$,
to $\Gal(K/\QQ)$. This is called the \emph{Artin map} of $K/\QQ$.

The punchline is that the Artin map factors through the map $(\ZZ/m\ZZ)^*
\to \Gal(K/\QQ)$ we wrote down above! Namely, note that the image of $r$
under the latter map takes $\zeta_m$ to $\zeta_m^r$. For this image to
be equal to $F_p$, we must have $\zeta_m^r \equiv \zeta_m^p \pmod{\gothp}$
for some prime $\gothp$ of $K$ above $p$. But $\zeta_m^r (1 - \zeta_m^{r-p})$
is only divisible by primes above $m$ (see exercises) unless $r-p \equiv 0
\pmod{m}$. Thus $F_p$ must be equal to the image of $p$ under the map
$(\ZZ/m\ZZ)^* \to \Gal(K/\QQ)$.

The \emph{Artin reciprocity law} states that a similar phenomenon arises
for any abelian extension of any number field; that is, the Frobenius elements
corresponding to various primes are governed by the way the primes ``reduce''
modulo some other quantity. There are several complicating factors in the
general case, though.
\begin{itemize}
\item Prime ideals in a general number field are not always principal, so we
can't always take a generator and reduce it modulo something.
\item There can be lots of units in a general number field, so even when
a prime ideal is principal, it is unclear which generator to choose.
\item It is not known in general how to explicitly construct generators
for all of the abelian extensions of a general number field.
\end{itemize}
Thus our approach will have to be a bit more indirect.

\head{Reduction to the local case}

Our reduction of Kronecker-Weber to local Kronecker-Weber relies on
a key result typically seen in a first course on algebraic number theory. (See for instance Neukirch III.2.)
\begin{theorem}[Minkowski] \label{T:Minkowski}
There are no nontrivial extensions of $\QQ$ which are unramified everywhere.
\end{theorem}

Using Minkowski's theorem, let us deduce the Kronecker-Weber theorem from the local Kronecker-Weber theorem.
\begin{proof}[Proof of Theorem~\ref{T:Kronecker-Weber}]
For each prime $p$ over which $K$ ramifies,
pick a prime $\gothp$ of $K$ over $p$; by local Kronecker-Weber
(Theorem~\ref{T:local Kronecker-Weber1}),
$K_{\gothp} \subseteq \QQ_p(\zeta_{n_p})$ for some positive integer $n_p$.
Let $p^{e_p}$ be the largest power of $p$ dividing $n_p$, and put 
$n = \prod_p p^{e_p}$. (This is a finite product since only finitely many
primes ramify in $K$.) 

We will prove that $K \subseteq \QQ(\zeta_n)$, by proving that
$K(\zeta_n) = \QQ(\zeta_n)$. Write $L = K(\zeta_n)$
and let $I_p$ be the inertia group of $p$ in $L$. If we let
$U$ be the maximal unramified subextension of $L_\gothq$ over $\QQ_p$
for some prime $\gothq$ over $p$, then $L_\gothq = U(\zeta_{p^{e_p}})$
and $I_p \cong \Gal(L_\gothq/U) \cong (\ZZ/p^{e_p}\ZZ)^*$.
Let $I$ be the group generated by all of the $I_p$; then
\[
|I| \leq \prod |I_p| = \prod \phi(p^{e_p}) = \phi(n) = [\QQ(\zeta_n):\QQ].
\]
On the other hand, the fixed field of $I$ is an everywhere unramified extension of 
$\QQ$, which can only be $\QQ$ itself by Minkowski's theorem. That is,
$I = \Gal(L/\QQ)$. But then
\[
[L:\QQ] = |I| \leq [\QQ(\zeta_n):\QQ],
\]
and $\QQ(\zeta_n) \subseteq L$, so we must have $\QQ(\zeta_n) = L$
and $K \subseteq \QQ(\zeta_n)$, as desired.
\end{proof}

\head{Exercises}

\begin{enumerate}
\item
For $m \in \ZZ$
not a perfect square, determine the conductor of $\QQ(\sqrt{m})$.
(Hint: first consider the case where $|m|$ is
prime.)
\item
Recover the law of quadratic reciprocity from the Artin reciprocity law,
using the fact that $\QQ(\sqrt{(-1)^{(p-1)/2} p})$ has conductor $p$.
\item
Prove that if $m,n$ are coprime integers in $\ZZ$, then
$1 - \zeta_m$ and $n$ are coprime in $\ZZ[\zeta_m]$.
(Hint: look at the polynomial $(1-xx)^m-1$ modulo
a prime divisor of $n$.)
\item
Prove that if $m$ is not a prime power, $1-\zeta_m$ is
a unit in $\ZZ[\zeta_m]$.
\end{enumerate}

%\end{document}


