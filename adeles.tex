%\documentclass[12pt]{article}
%\usepackage{amsfonts, amsthm, amsmath}
%
%\setlength{\textwidth}{6.5in}
%\setlength{\oddsidemargin}{0in}
%\setlength{\textheight}{8.5in}
%\setlength{\topmargin}{0in}
%\setlength{\headheight}{0in}
%\setlength{\headsep}{0in}
%\setlength{\parskip}{0pt}
%\setlength{\parindent}{20pt}
%
%\def\AA{\mathbb{A}}
%\def\CC{\mathbb{C}}
%\def\FF{\mathbb{F}}
%\def\PP{\mathbb{P}}
%\def\QQ{\mathbb{Q}}
%\def\RR{\mathbb{R}}
%\def\ZZ{\mathbb{Z}}
%\def\gotha{\mathfrak{a}}
%\def\gothb{\mathfrak{b}}
%\def\gothm{\mathfrak{m}}
%\def\gotho{\mathfrak{o}}
%\def\gothp{\mathfrak{p}}
%\def\gothq{\mathfrak{q}}
%\DeclareMathOperator{\disc}{Disc}
%\DeclareMathOperator{\fin}{fin}
%\DeclareMathOperator{\Gal}{Gal}
%\DeclareMathOperator{\GL}{GL}
%\DeclareMathOperator{\Hom}{Hom}
%\DeclareMathOperator{\Norm}{Norm}
%\DeclareMathOperator{\Trace}{Trace}
%\DeclareMathOperator{\Cl}{Cl}
%
%\def\head#1{\medskip \noindent \textbf{#1}.}
%
%\newtheorem{theorem}{Theorem}
%\newtheorem{lemma}[theorem]{Lemma}
%\newtheorem{prop}[theorem]{Proposition}
%\newtheorem{cor}[theorem]{Corollary}
%
%\begin{document}
%
%\begin{center}
%\bf
%Math 254B, UC Berkeley, Spring 2002 (Kedlaya) \\
%Adeles and Ideles
%\end{center}

\head{Reference} 
Milne, Section V.4; Neukirch, Section VI.1 and VI.2; Lang, \textit{Algebraic Number Theory}, Chapter VII.

\medskip
The $p$-adic numbers, and more general local
fields, were introduced into number theory as a way to translate local facts about number fields (i.e., facts concerning a single prime ideal) into
statements of a topological flavor. To prove the statements of class field theory, we need an analogous global construction.
To this end, we construct a topological object that includes all of the
completions of a number field, including both the archimedean and nonarchimedean
ones. This object will be the ring of ad\`eles, and it will lead us to
the right target group for use in the abstract class field theory we
have just set up.

\head{Spelling note} 
There is a lack of consensus regarding the presence or absence of accents in  the words \emph{ad\`ele} and \emph{id\`ele}. The term \emph{id\`ele} is thought to be a contraction of ``ideal element''; it makes its first appearance, with the accent, in Chevalley's 1940 paper ``La th\'eorie du corps de classes.'' The term \emph{ad\`ele} appeared in the 1950s, possibly as a contraction of ``additive id\`ele''; it appears to have been suggested by Weil as a replacement for Tate's term ``valuation vector'' and Chevalley's term ``repartition''.
Based on this history, we have opted for the accented spellings here.

\head{Jargon watch}
By a \emph{place} of a number field $K$, we mean either an archimedean
completion $K \hookrightarrow \RR$ or $K \hookrightarrow \CC$
(an \emph{infinite place}), or a
$\gothp$-adic completion $K \hookrightarrow K_\gothp$ for some nonzero
prime ideal $\gothp$ of $\gotho_K$ (a \emph{finite place}). (Note:
there is only one place for each pair of complex embeddings of $K$.)
Each place corresponds to an equivalence class of absolute values on
$K$; if $v$ is a place, we write $K_v$ for the corresponding completion,
which is either $\RR$, $\CC$, or $K_\gothp$ for some prime $\gothp$.

\head{The ad\`eles}

The basic idea is that we want some sort of ``global completion'' of a
number field $K$. In fact, we already know one way to complete $\ZZ$,
namely its profinite completion $\widehat{\ZZ} = \prod_p \ZZ_p$. But we
really want something containing $\QQ$. We define the \emph{ring of
finite ad\`eles} $\AA^{\fin}_\QQ$ as any of the following isomorphic objects:
\begin{itemize}
\item the tensor product $\widehat{\ZZ} \otimes_{\ZZ} \QQ$;
\item the direct limit of $\frac{1}{n} \widehat{\ZZ}$ over all nonzero
integers $n$;
\item the \emph{restricted direct product} $\sideset{}{'_p}\prod \QQ_p$,
where we only allow tuples $(\alpha_p)$ for which $\alpha_p \in \ZZ_p$
for almost all $p$.
\end{itemize}
For symmetry, we really should allow \emph{all} places, not just the finite
places. So we also define the \emph{ring of ad\`eles} over $\QQ$ as
$\AA_{\QQ} = \RR \times \AA^{\fin}_{\QQ}$. Then $\AA_{\QQ}$ is a locally
compact topological ring with a canonical embedding $\QQ \hookrightarrow
\AA_{\QQ}$.

Now for a general number field $K$.
The profinite completion
$\widehat{\gotho_K}$ is canonically isomorphic to $\prod_{\gothp}
\gotho_{K_{\gothp}}$, so we define the \emph{ring of finite ad\`eles}
$\AA^{\fin}_{K}$ as any of the following isomorphic objects:
\begin{itemize}
\item the tensor product $\widehat{\gotho_K} \otimes_{\gotho_K} K$;
\item the direct limit of $\frac{1}{\alpha} \widehat{\gotho_K}$ over all
nonzero $\alpha \in \gotho_K$;
\item the \emph{restricted direct product} $\sideset{}{'_p}\prod K_{\gothp}$,
where we only allow tuples $(\alpha_p)$ for which $\alpha_p \in \gotho_{K_{\gothp}}$
for almost all $\gothp$.
\end{itemize}
The ring of ad\`eles $\AA_K$ is the product of $\AA^{\fin}_K$ with each
archimedean completion. (That's one copy of $\RR$ for each real embedding and
one copy of $\CC$ for each conjugate pair of complex embeddings.)

One has a natural norm on the ring of ad\`eles, because one has a natural norm
on each completion:
\[
|(\alpha_v)_v| = \prod_v |\alpha_v|_v.
\]
One should normalize these in the following way: for $v$ real, take
$|\cdot|_v$ to be the usual absolute value. For $v$ complex, take
$|\cdot|_v$ to be the \emph{square} of the usual absolute value. (That means
the result is not an absolute value, in that it doesn't satisfy the triangle
inequality. Sorry.) For $v$ nonarchimedean corresponding to a prime above $p$,
normalize so that $|p|_v = p^{-1}$.

Again, there is a natural embedding of $K$ into $\AA_K$ because there
is such an embedding for each completion.
With the normalization as above, one has the product formula:
\begin{prop}
If $\alpha \in K$, then $|\alpha| = 1$.
\end{prop}
In particular, $K$ is \emph{discrete} in $\AA_K$ (the difference between
two elements of $K$ cannot be simultaneously small in all embeddings).
This is a generalization/analogue of the fact that $\gotho_K$ is
discrete in Minkowski space (the product of the archimedean completions).

For any finite set $S$ of places, let $\AA_S$ (resp. $\AA^{\fin}_S$)
be the subring of $\AA_K$ (resp. $\AA^{\fin}_K$) consisting of
those ad\`eles which are integral at all finite places not contained in $S$.
Then we have the following result, which is essentially the Chinese remainder
theorem.
\begin{prop} \label{P:adelic CRT}
For any finite set $S$ of places, $K + \AA_S^{\fin} = \AA_K^{\fin}$ and
$K + \AA_S = \AA_K$.
\end{prop}
\begin{cor}
The quotient group $\AA_K/K$ is compact.
\end{cor}
\begin{proof}
Choose a compact subset $T$ of the Minkowski space $M$ containing a fundamental domain for the lattice $\gotho_K$.
Then every element of $M \times \AA^{\fin}_K$ is congruent modulo $\gotho_K$
to an element of $T \times \AA^{\fin}_K$. By the proposition,
the compact set $T \times \AA^{\fin}_K$ surjects onto $\AA_K/K$, so the
latter is also compact.
\end{proof}

\head{Alternate description: restricted products of topological groups}

Let $G_1, G_2, \dots$ be a sequence of locally compact topological groups
and let $H_i$ be a compact subgroup of $G_i$. The \emph{restricted product} $G$
of the pairs $(G_i, H_i)$ is the set of tuples $(g_i)_{i=1}^\infty$ such
that $g_i \in H_i$ for all but finitely many indices $i$. For each set $S$,
this product contains the subgroup $G_S$ of tuples $(g_i)$ such that $g_i
\in H_i$ for $i \notin S$, and indeed $G$ is the direct limit of the $G_S$.
We make $G$ into a topological group by giving each $G_S$ the product topology
and saying that $U \subset G$ is open if its intersection with each $G_S$
is open there.

In this language, the additive group of ad\`eles over $\QQ$
is simply the restricted
product of the pairs $(\RR, \RR)$ and $(\QQ_p, \ZZ_p)$ for each $p$,
and likewise over a number field.

\head{Id\`eles and the id\`ele class group}

An \emph{id\`ele} is a unit in the ring $\AA_K$. In other words, it is a
tuple $(\alpha_v)$, one element of $K_v^*$ for each place $v$ of $K$, such that
$\alpha_v \in \gotho_{K_v}^*$ for all but finitely many finite places $v$.
Let $I_K$ denote the group of id\`eles of $K$ (sometimes thought
of as $\GL_1(\AA_K)$). This group is the restricted product of the pairs
$(\RR^*, \RR^*)$, $(\CC^*, \CC^*)$, and $(K_{\gothp}^*, \gotho_{\gothp}^*)$.

For example, for each element $\beta \in K$, we get an ad\`ele
in which $\alpha_v = \beta$ for all $v$; this ad\`ele is an id\`ele if $\beta \neq 0$. We
call these the \emph{principal ad\`eles} and \emph{principal id\`eles},
and define the \emph{id\`ele class group} of $K$ as the quotient
$C_K = I_K/K^*$ of the id\`eles by the principal id\`eles.

\head{Warning}  While the embedding of the id\'eles into the ad\'eles is continuous,
the restricted product topology on id\'eles does not coincide with the subspace topology for the embedding!
For
example, the set of id\`eles whose component at each finite prime $\gothp$ is in $\gotho_{\gothp}^*$ 
is open, but not an
intersection of the id\`ele group with an open subset of the ad\`eles. See the exercises for one way to fix this, and the last part of this section for another.

\medskip
There is a homomorphism from $I_K$ to the group of fractional ideals
of $K$:
\[
(\alpha_\nu)_\nu \mapsto \prod_{\gothp} \gothp^{v_{\gothp}(\alpha_\gothp)},
\]
which is continuous for the discrete topology on the group of fractional
ideals.
The principal id\`ele corresponding to $\alpha \in K$ maps to the principal
ideal generated by $\alpha$. Thus we have a surjection $C_K \to \Cl(K)$.

Since the norm is trivial on $K^*$, we get a well-defined norm map
$|\cdot|: C_K \to \RR^*_+$. Let $C_K^0$ be the kernel of the norm map;
then $C_K^0$ also surjects onto $\Cl(K)$. (The surjection onto $\Cl(K)$ ignores
the infinite places, so you can adjust there to force norm 1.)
\begin{prop} \label{P:idele group compact}
The group $C_K^0$ is compact.
\end{prop}
This innocuous-looking fact actually implies two key theorems of algebraic number theory:
\begin{enumerate}
\item[(a)]
The class group of $K$ is finite.
\item[(b)]
The group of units of $K$ has rank $r+s-1$, where $r$ and $s$ are the number
of real and complex places, respectively. More generally, if $S$ is a finite
set of places containing the archimedean places,
the group of $S$-units of $K$ (elements of $K$ which
have valuation 0 at each finite place not contained in $S$) has rank
$\#(S)-1$.
\end{enumerate}
In fact, (a) is immediate: $C_K^0$ is compact and it surjects onto 
$\Cl(K)$, so the latter must also be compact for the discrete topology,
i.e., it must be finite. (In fact, $\Cl(K)$ is isomorphic to the group
of connected components of $C_K^0$.)
To see (b), let $I_S$ be the group of id\`eles which are units outside $S$,
and define the map $\log: I_S \to \RR^{\#(S)}$ by taking log of the absolute
value of the norm of each component in $S$. By the product formula, this maps
into the sum-of-coefficients-zero hyperplane $H$ in $\RR^{\#(S)}$, and the
image of the group $K_S^*$ of $S$-units is discrete therein. (Restricting 
an element of $K_S^*$ to a bounded subset of $H$ bounds all of its
absolute values, so this follows from the discreteness of $K$ in $\AA_K$.)
Let $W$ be the span in $H$ of the image of $K_S^*$; then
we get a continuous homomorphism $C_K^0 \to H/W$ whose image generates
$H/W$. But its image is compact; this is a contradiction unless $H/W$
is the zero vector space. Thus $K_S^*$ must be a lattice in $H$,
so it has rank $\dim H = \#S - 1$.

\begin{proof}[Proof of Proposition~\ref{P:idele group compact}]
The inverse images of any two positive real numbers under the norm map are
homeomorphic. So rather than prove that the inverse image of 0 is compact,
we'll prove that the inverse image of some $\rho > 0$ is compact.
Namely, we choose $\rho>c$, where $c$ has the property that any id\`ele of
norm $\rho > c$ is congruent modulo $K^*$ to an id\`ele whose components
all have norms in $[1, \rho]$. (The existence of $c$ is left as an exercise.)

The set
of id\`eles with each component having norm in $[1, \rho]$ is the product
of ``annuli'' in the archimedean places and finitely many of the
nonarchimedean places, and the group of units in the rest. (Most of the
nonarchimedean places don't have any valuations between 1 and $\rho$.)
This is a compact set, the set of id\`eles therein of norm $\rho$ is a closed
subset and so is also compact, and the latter set surjects onto
$C_K^0$, so that's compact too.
\end{proof}

One more comment worth making: what are the open subgroups of $I_K$?
In fact, for each formal product $\gothm$ of places, one gets an open
subgroup of id\`eles $(\alpha_v)_v$ such that:
\begin{enumerate}
\item[(a)] if $v$ is a real place occurring in $\gothm$, then
$\alpha_v > 0$;
\item[(b)] if $v$ is a finite place corresponding to the prime $\gothp$,
occurring to the power $e$, then $\alpha_v \equiv 1 \pmod{\gothp^e}$.
\end{enumerate}
Moreover, every open subgroup contains one of these. Thus
using the surjection $C_K \mapsto \Cl(K)$, we get a bijection between open
subgroups of $C_K$ and generalized ideal class groups!

\head{A presentation of $\AA_\QQ$}
In the special case $K = \QQ$, the id\`ele class group has a nice presentation.
Namely, given an arbitrary id\`ele in $I_\QQ$, there is a unique positive
rational with the same norms at the finite places. Thus
\[
I_\QQ \cong \RR^* \times \prod_p \ZZ_p^*.
\]
This definitely does not generalize: as noted above, the id\`ele class group
has multiple connected components when the class number is bigger than 1.

\head{Aside: beyond class field theory}
You can think of the id\`ele group as $\GL_1(\AA_K)$. In that case, class field
theory will become a correspondence between one-dimensional representations
of $\Gal(\overline{K}/K)$ and certain representations of $\GL_1(\AA_K)$. This
is the form in which class field theory generalizes to the nonabelian case:
the Langlands program predicts a correspondence between $n$-dimensional
representations of $\Gal(\overline{K}/K)$ and certain representations
of $\GL_n(\AA_K)$. In fact, with $K$ replaced by the function field of a
curve over a finite field, this prediction is a deep theorem of L. Lafforgue
(based on work of Drinfeld).

\head{Exercises}

\begin{enumerate}
\item
Prove Proposition~\ref{P:adelic CRT}.
\item
Show that the restricted direct product topology on $I_K$ is the subspace topology for the embedding into
$\AA_K \times \AA_K$ given by the map $x \mapsto (x,x^{-1})$.
\item
Complete the proof of Proposition~\ref{P:idele group compact} by establishing the existence of the constant $c$. (Hint: see Lang, Section V.1, Theorem 0.)
\end{enumerate}

%\end{document}
