%\documentclass[12pt]{article}
%\usepackage{amsfonts, amsthm, amsmath}
%\usepackage[all]{xy}
%
%\setlength{\textwidth}{6.5in}
%\setlength{\oddsidemargin}{0in}
%\setlength{\textheight}{8.5in}
%\setlength{\topmargin}{0in}
%\setlength{\headheight}{0in}
%\setlength{\headsep}{0in}
%\setlength{\parskip}{0pt}
%\setlength{\parindent}{20pt}
%
%\def\AA{\mathbb{A}}
%\def\CC{\mathbb{C}}
%\def\FF{\mathbb{F}}
%\def\PP{\mathbb{P}}
%\def\QQ{\mathbb{Q}}
%\def\RR{\mathbb{R}}
%\def\ZZ{\mathbb{Z}}
%\def\gotha{\mathfrak{a}}
%\def\gothb{\mathfrak{b}}
%\def\gothm{\mathfrak{m}}
%\def\gotho{\mathfrak{o}}
%\def\gothp{\mathfrak{p}}
%\def\gothq{\mathfrak{q}}
%\DeclareMathOperator{\disc}{Disc}
%\DeclareMathOperator{\Gal}{Gal}
%\DeclareMathOperator{\GL}{GL}
%\DeclareMathOperator{\Hom}{Hom}
%\DeclareMathOperator{\Ind}{Ind}
%\DeclareMathOperator{\Norm}{Norm}
%\DeclareMathOperator{\Trace}{Trace}
%\DeclareMathOperator{\Cl}{Cl}
%
%\def\head#1{\medskip \noindent \textbf{#1}.}
%
%\newtheorem{theorem}{Theorem}
%\newtheorem{lemma}[theorem]{Lemma}
%\newtheorem{prop}[theorem]{Proposition}
%
%\begin{document}
%
%\begin{center}
%\bf
%Math 254B, UC Berkeley, Spring 2002 (Kedlaya) \\
%Cohomology of the id\`eles 1: The ``First Inequality''
%\end{center}

\head{Reference} 
Milne VII.2-VII.4; Neukirch VI.3; but see below about Neukirch.

\medskip

By analogy with local class field theory, we want to prove that for $K,L$
number fields and $C_K, C_L$ their id\`ele class groups,
\[
H^1(\Gal(L/K), C_L) = 1, \qquad H^2(\Gal(L/K), C_L) = \ZZ/[L:K]\ZZ.
\]
In this chapter, we'll look at the special case $L/K$ cyclic, and prove
that
\[
\#H^0_T(\Gal(L/K), C_L)/\#H^1_T(\Gal(L/K), C_L) = [L:K].
\]
That is,
the Herbrand quotient of $C_L$ is $[L:K]$. As we'll see, this
will end up reducing to looking at units in a real vector space, much
as in the proof of Dirichlet's units theorem.

This will imply the ``First Inequality''.
\begin{theorem} \label{T:first inequality}
For $L/K$ a cyclic extension of number fields,
\[
\#H^0_T(\Gal(L/K), C_L) \geq [L:K].
\]
\end{theorem}
The ``Second Inequality'' will be the reverse, which will be a bit
more subtle (see Theorem~\ref{T:first and second inequality}).

\head{Some basic observations}
But first, some general observations. Put $G = \Gal(L/K)$.
\begin{prop}
  For each $i>0$, $H^i(G, I_L) = \oplus_v H^i(G_v, L_v^*)$. For each
$i$, $H^i_T(G, I_L) = \oplus_v H^i_T(G_v, L_v^*)$.
\end{prop}
\begin{proof}
For any finite set $S$ of places of $K$ containing all infinite places
and all ramified primes, let $I_{L,S}$ be the set of id\`eles
with a unit at each component other than at the places dividing 
any places in $S$. Note that $I_{L,S}$ is stable under $G$ (because we
defined it in terms of places of $K$, not $L$). By definition,
$I_L$ is the direct limit of the $I_{L,S}$ over all $S$, so $H^i(G,I_L)$
is the direct limit of the $H^i(G, I_{L,S})$. The latter is the product of
$H^i(G, \prod_{w|v} L_w^*)$ over all $v \in S$ and 
$H^i(G, \prod_{w|v} \gotho_{L_w}^*)$ over all $v \notin S$, but the latter
is trivial because $v \notin S$ cannot ramify. By Shapiro's lemma (Lemma~\ref{L:Shapiro}),
$H^i(G, \prod_{w|v} L_w^*) = H^i(G_v, L_w^*)$, so we have what we want.
The argument for Tate groups is analogous.
\end{proof}
Notice what this says for $i=0$ on the Tate groups: an id\`ele is a norm
if and only if each component is a norm. Obvious, perhaps, but useful.

In particular,
\[
H^1(G, I_L) = 0, \qquad H^2(G, I_L) = \bigoplus_v \frac{1}{[L_w:K_v]} \ZZ/\ZZ.
\]

One other observation: if $S$ contains all 
infinite places and all ramified places, then
\[
\Norm_{L/K} I_{L,S} = \prod_{v \in S} U_v \times \prod_{v \notin S} \gotho_{K_v}^*
\]
where $U_v$ is open in $K_v^*$. The group on the right is open in $I_K$, so
$\Norm_{L/K} I_K$ is open.

By quotienting down to $C_K$, we see that $\Norm_{L/K} C_K$ is open. In fact,
the snake lemma on the diagram
\[
\xymatrix{
0 \ar[r] & L^* \ar[r] \ar^{\Norm_{L/K}}[d] & I_L \ar[r] \ar^{\Norm_{L/K}}[d]
& C_L \ar[r] \ar^{\Norm_{L/K}}[d] & 0 \\
0 \ar[r] & K^* \ar[r] & I_K \ar[r] & C_K \ar[r] & 0
}
\]
implies that the quotient $I_K/(K^* \times \Norm_{L/K} I_L)$ is isomorphic
to $C_K$.

\head{Cohomology of the units}

Remember, we're going to be assuming $G = \Gal(L/K)$ is cyclic until further notice,
so that we may use periodicity of the Tate groups, and the Herbrand quotient.

First of all, working with $I_L$ all at once is a bit unwieldy; we'd rather
work with $I_{L,S}$ for some finite set $S$. In fact, we can choose $S$ to
make our lives easier: we choose $S$ containing all infinite places,
and all ramified primes, and perhaps some extra primes so that
\[
I_L = I_{L,S} L^*.
\]
This is possible because the ideal class group of $L$ is finite, so 
it is generated by some finite set of primes, which we introduce into $S$; then I can
move a generator of any other prime to some stuff in $S$ times units.
(This argument can also be used to prove that $C^0_L$ is compact, but then one doesn't recover the finiteness of the ideal class group as a corollary.)

%The reason I can do this is the same as my ``cheating''
%proof that $C^0_L$ is compact: it's enough to make sure $S$ contains 
%a set of primes that generate the ideal class group of $L$, so then I can
%move a generator of any other prime to some stuff in $S$ times units.
%
Put $L_S = L^* \cap I_{L,S}$; that is, $L_S$ is the group of $S$-units
in $L$. From the exact sequence
\[
1 \to L_S \to I_{L,S} \to I_{L,S}/L_S = C_L \to 1
\]
we have, in case $L/K$ is cyclic, an equality of Herbrand quotients
\[
h(C_L) = h(I_{L,S})/h(L_S).
\]
From the computation of $H^i(G, I_{L,S})$, it's easy to read off its
Herbrand quotient:
\[
h(I_{L,S}) = \prod_{v \in S} \#H^0_T(G_v, L_w^*)= \prod_{v \in S}
[L_w:K_v].
\]
So to get $h(C_L) = [L:K]$, we need
\[
h(L_S) = \frac{1}{[L:K]} \prod_{v \in S} [L_w:K_v].
\]
This will in fact be true even if we only assume $S$ contains all infinite
places, as we now check.

Let $T$ be the set of places of $L$ dividing the places of $S$.
Let $V$ be the real vector space consisting of one copy of $\RR$ for
each place in $T$. Define the map $L_S \to V$ by sending
\[
\alpha \to \prod_w \log |\alpha|_w,
\]
with the caveat that the norm at a complex place is the \emph{square} of
the usual absolute value; the kernel of this map consists solely of roots
of unity (by Kronecker's theorem: any algebraic integer whose conjugates in $\CC$ all have norm 1 is a root of unity). Let $M$ be the quotient of $L_S$ by the group of roots of unity;
since the latter is finite, $h(M) = h(L_S)$.
Let $H \subset V$ be the hyperplane of vectors
with sum of coordinates 0; by the product formula, $M$ maps into $H$.
As noted earlier, in fact $M$ is a discrete subgroup of $H$ of rank equal
to the dimension of $H$; that is, $M$ is a lattice in $H$. Moreover,
we have an action of $G$ on $V$ compatible with the embedding of $M$;
namely, $G$ acts on the places in $T$, so acts on $V$ by permuting the
coordinates.

\head{Caveat} There seems to be an error in Neukirch's derivation at this
point. Namely, his Lemma VI.3.4 is only proved assuming that $G$ acts
transitively on the coordinates of $V$; but in the above situation,
this is not the case: $G$ permutes the places above any given place $v$
of $K$ but those are separate orbits. So we'll follow Milne instead.

We can write down two natural lattices in $V$. One of them is the lattice
generated by $M$ together with the all-ones vector, on which $G$ acts
trivially. As a $G$-module, the Herbrand quotient of that lattice is
$h(M) h(\ZZ) = [L:K] h(M)$. The other is the lattice $M'$ in which, in the
given coordinate system, each element has integral coordinates. To
compute its Herbrand quotient, notice that the projection of this 
lattice onto the coordinates corresponding to the places $w$ above some
$v$ form a copy of $\Ind^G_{G_v} \ZZ$. Thus
\[
h(G, M') = \prod_v h(G, \Ind^G_{G_v} \ZZ) = \prod_v h(G_v, \ZZ)
= \prod_v \#G_v = \prod_v [L_w:K_v].
\]

So all that remains is to prove the following.
\begin{lemma}
Let $V$ be a real vector space on which a finite group $G$ acts linearly,
and let $L_1$ and $L_2$ be $G$-stable lattices in $V$ for which
$h(L_1)$ and $h(L_2)$ are both defined. Then $h(L_1) = h(L_2)$.
\end{lemma}
In fact, one can show that if one of the Herbrand quotients is defined, so
is the other.
\begin{proof}
We first show that $L_1 \otimes_{\ZZ} \QQ$ and $L_2 \otimes_{\ZZ} \QQ$
are isomorphic as $\QQ[G]$-modules. We are given that $L_1 \otimes_{\ZZ} \RR$
and $L_2 \otimes_{\ZZ} \RR$ are isomorphic as $\RR[G]$-modules.
That is, the real vector space $W = \Hom_{\RR}(L_1 \otimes_{\ZZ} \RR,
L_2 \otimes_{\ZZ}
\RR)$, on which $G$ acts by the formula $T^g(x) = T(x^{g^{-1}})^g$, contains an invariant vector which, as a linear
transformation, is invertible. Now $W$ can also be written as
\[
\Hom_{\ZZ}(L_1, L_2) \otimes_{\ZZ} \RR;
\]
that is, $\Hom_{\ZZ}(L_1, L_2)$ sits inside as a sublattice. The fact that
$W$ has an invariant vector says that a certain set of linear equations
has a nonzero solution over $\RR$, namely the equations that express
the fact that the action of $G$ leaves the vector invariant. But those
equations have coefficients in $\QQ$, so there must already be invariant
vectors over $\QQ$. Moreover, if we fix an isomorphism (not $G$-equivariant)
between $L_2 \otimes_{\ZZ} \RR$ and $L_1 \otimes_{\ZZ} \RR$, we can
compose this with any element of $W$ to get a map from $L_1$ to itself,
which has a determinant; and by hypothesis, there is some invariant
vector of $W$ whose determinant is nonzero. Thus the determinant
doesn't vanish identically on the set of invariant vectors in $W$,
so it also doesn't vanish identically on the set of invariant vectors in
$\Hom_{\ZZ}(L_1, L_2) \otimes_{\ZZ} \QQ$.

Thus there is a $G$-equivariant isomorphism between $L_1 \otimes_{\ZZ} \QQ$
and $L_2 \otimes_{\ZZ} \QQ$; that is, $L_1$ is isomorphic to a sublattice of
$L_2$. Since a lattice has the same Herbrand quotient as any sublattice
(the quotient is finite, so its Herbrand quotient is 1), that means
$h(L_1) = h(L_2)$.
\end{proof}

\head{Aside: splitting of primes}

As a consequence of the First Inequality, we record the following fact which is \emph{a posteriori} an immediate consequence of the adelic reciprocity law, but which will be needed in the course of the proofs.
(See Neukirch, Corollary VI.3.8 for more details).
\begin{cor} \label{C:split completely}
For any nontrivial extension $L/K$ of number fields, there are infinitely many primes of $K$ which do not split completely in $L$.
\end{cor}
\begin{proof}
Suppose first that $L/K$ is of prime order. Then if all but finitely many
primes split completely, we can put the remaining primes into $S$ and
deduce that $C_K = \Norm_{L/K} C_L$, whereas the above calculation
forces $H^0_T(\Gal(L/K), C_L) \geq [L:K]$, contradiction.

In the general case, let $M$ be the Galois closure of $L/K$; then a prime of $K$ splits completely in $L$ if and only if it splits completely in $M$. Since $\Gal(M/K)$ is a nontrivial finite group, it contains a cyclic subgroup of prime order; let $N$ be the fixed field of this subgroup. By the previous paragraph, there are infinitely many prime ideals of $N$ which do not split completely in $M$, proving the original result.
\end{proof}

%\end{document}
