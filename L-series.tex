\documentclass{amsart}

\usepackage{amssymb}

\def\ZZ{{\mathbb Z}}
\def\QQ{{\mathbb Q}}
\def\RR{{\mathbb R}}
\def\CC{{\mathbb C}}
\def\PP{{\mathbb P}}

\def\mf{\mathfrak}

\def\oalpha{\overline{\alpha}}
\def\obeta{\overline{\beta}}

\def\rhotilde{{\tilde{\rho}}}
\def\rhohat{{\hat{\rho}}}

\def\un{\mathrm{un}}

\def\CC{\mathbb{C}}   %% complex number
\def\RR{\mathbb{R}}   %% real numbers
\def\ZZ{\mathbb{Z}}
\def\NN{\mathbb{N}}
\def\GG{\mathbb{G}}
\def\GLnC{\textbf{GL}_n \left( \mathbb{C} \right)}  %% general linear group
\def\GL1C{\textbf{GL}_1 \left( \mathbb{C} \right)}  %% general linear group
\def\GL{\textbf{GL}}
\def\TnC{\textbf{B}_n \left( \mathbb{C} \right)}    %% upper triangular invertible matrices
\def\tnC{\textbf{b}_n \left( \mathbb{C} \right)}    %% upper triangular matrices
\def\UnC{\textbf{U}_n \left( \mathbb{C} \right)}    %% unipotent matrices
\def\unC{\textbf{u}_n \left( \mathbb{C} \right)}    %% Lie alg. of unipotent matrices
\def\DnC{\textbf{D}_n \left( \mathbb{C} \right)}    %% diagonal matrices
\def\dnC{\textbf{d}_n \left( \mathbb{C} \right)}    %% Lie alg. of diagonal matrices
\def\MnC{\textbf{M}_n \left( \mathbb{C} \right)}    %% matrices
\def\p1Mx{\pi_1 \left( M, x \right)}
\def\ker{\textbf{ker }}
\def\exp{\textrm{exp}}
\def\Cl{\textnormal{Cl}}
\def\bI{\textbf{I}}
\def\bP{\textbf{P}}
\def\bN{\textbf{N}}

\newcommand\im{\operatorname{im}}
\newcommand\Id{\operatorname{Id}}
\newcommand\ad{\operatorname{ad}}
\newcommand\Ad{\operatorname{Ad}}
\newcommand\Tor{\operatorname{Tor}}
\newcommand\coker{\operatorname{coker}}
\newcommand\rank{\operatorname{rank}}
\newcommand\Spec{\operatorname{Spec}}
\newcommand\Hom{\operatorname{Hom}}
\newcommand\Homcts{\operatorname{Hom_{\mathrm{cts}}}}
\newcommand\End{\operatorname{End}}
\newcommand\Ext{\operatorname{Ext}}
\newcommand\Aut{\operatorname{Aut}}
\newcommand\Res{\operatorname{Res}}
\newcommand\Vol{\operatorname{Vol}}

\def\Re{\textnormal{Re}}
\def\Im{\textnormal{Im}}


%%%%%%%%%%%%%%%%% environments %%%%%%%%%%%%%%%%%%

\newtheorem{theorem}{Theorem}[section]
\newtheorem{lemma}[theorem]{Lemma}
\newtheorem{proposition}[theorem]{Proposition}
\newtheorem{corollary}[theorem]{Corollary}
%\newtheorem{bigtheorem}{Theorem}

\theoremstyle{definition}
\newtheorem{definition}[theorem]{Definition}
\newtheorem{example}[theorem]{Example}
\newtheorem{conjecture}[theorem]{Conjecture}
\newtheorem{postulate}[theorem]{Postulate}
%\newtheorem{para}[theorem]{}

\theoremstyle{remark}
\newtheorem{remark}[theorem]{Remark}
%\newtheorem{notation}[theorem]{Notation}
%\newtheorem{question}[theorem]{Question}
%\newtheorem{variant}[theorem]{Variant}

\begin{document}

\title{Zeta Functions and $L$-series for a Number Field}

\author{Carl Miller}

\maketitle

\section{Introduction}

Every number field $K$ has an associated zeta function
$\zeta_K(z)$, whose analytic properties are closely tied to number
theoretic information about $K$. Zeta functions are also a subset
of a larger class of analytic functions called $L$-series, which
come from one-dimensional representations of subgroups of the
ideal class group.  These functions are useful for proving many
results in class field theory.

We are concerned specifically with generalized Dirichlet
$L$-series, which are associated to the ray class group of $K$
for a chosen modulus $\mf{m}$. We will define an $L$-series and
prove convergence for $\Re(z) > 1$.  Since we are primarily
interested in the values of the $L$-series near $z=1$, we will
need to show that the $L$-series has an analytic continuation to a
neighborhood of $1$.  Our proof of the continuation relies on an
important theorem about the distribution of norms in a
generalized ideal class.  In Section \ref{distribution} we will
give a summary proof of this theorem, referring the reader to
Lang for some technical lemmas. Then, as an example application,
we use $L$-series to prove a result in class field theory known
as the Universal Norm Index Inequality.

\section{Notation and Definitions}

Throughout, let $K/\QQ$ be a number field, let $A$ denote its
ring of integers, and let $\bI$ and $\bP$ denote the group of
fractional ideals of $K$ and principal fractional ideals of $K$,
respectively.  Let $\mf{m} = \mf{m}_0 \mf{m}_\infty$ be a formal
product of places of $K$, with $\mf{m}_0$ the finite part and
$\mf{m}_\infty$ the infinite part.  If $v \mid \mf{m}_\infty$ is
a valuation, let $\sigma_v : K \to k_v$ denote the corresponding
embedding, with $k_v = \RR$ or $\CC$.  For convenience we will
write $x \equiv 1 \mod \mf{m}$ to mean that
\[
x \equiv 1 \mod \mf{p}^m
\]
for all $\mf{p}^m \mid \mf{m}_0$, and
\[
\sigma_v(x) > 0
\]
for all real $v \mid \mf{m}_\infty$.  Let $\bI^\mf{m}$ denote the
group of fractional ideals that are coprime to $\mf{m}_0$, let
$\bP^\mf{m} \subseteq \bI^\mf{m}$ denote the subgroup generated
by principal ideals $(x)$ with $x \equiv 1 \mod \mf{m}$, and let
$\Cl_\mf{m}(K) = \bI^\mf{m} / \bP^\mf{m}$ denote the ray class
group of $\mf{m}$. If $J \subseteq A$ is an ideal, let $\bN (J)$
denote the norm of $J$.

\section{Dirichlet Characters}

Let $\Xi_\mf{m}$ be the group of complex characters $\chi :
\Cl_\mf{m}(K) \to \CC^*$.  We note a few properties of this
group.  Recall that the ray class group is always finite (see
\cite{ANF}, pp. 112).

\begin{proposition}
The order of $\Xi_\mf{m}$ is the same as the order of the group
$\Cl_\mf{m}(K)$.
\end{proposition}
\begin{proof}
Write $\Cl_\mf{m}(K)$ as a direct product of finite cyclic groups,
\[
\Cl_\mf{m}(K) \cong \ZZ/n_1 \ZZ \times \ldots \times \ZZ / n_r \ZZ
\]
and let $g_i \in \Cl_\mf{m}(K)$ be their respective generators.  A
character $\chi$ is determined by the independent choice of
mappings of each $g_i$ to an $n_i$-th root of unity.  There are
$n_1n_2\cdots n_r = \left| \Cl_\mf{m}(K) \right|$ possible
choices.
\end{proof}

\begin{proposition}
\label{nontriv} Let $\chi_0 \in \Xi_\mf{m}$ denote the trivial
character. Any character $\chi : \Cl_\mf{m}(K) \to \CC^*$
satisfies
\[
\sum_{g \in \Cl_\mf{m}(K)} \chi(g) = \left\{ \begin{array}{cl}
\left| \Cl_\mf{m}(K) \right| & \textnormal{ if } \chi = \chi_0, \\
0 & \textnormal{ if } \chi \neq \chi_0.
\end{array} \right.
\]
\end{proposition}
\begin{proof}
Note that for any $h \in \Cl_\mf{m}(K)$,
\[
\sum_{g \in \Cl_\mf{m}(K)} \chi(g) = \sum_{gh \in \Cl_\mf{m}(K)}
\chi(gh) = \chi(h) \sum_{g \in \Cl_\mf{m}(K)} \chi(g).
\]
So either $\sum_{g \in \Cl_\mf{m}(K)} \chi(g) = 0$, or $\chi(h) =
1$ for all $h \in \Cl_\mf{m}(K)$, and hence $\sum_{g \in
\Cl_\mf{m}(K)} = \left| \Cl_\mf{m}(K) \right|$, as desired.
\end{proof}

By a similar trick, we have

\begin{proposition}
\label{sumchar} For any ideal class $g \in \Cl_\mf{m}(K)$,
\[
\sum_{\chi \in \Xi_\mf{m}} \chi(g) = \left\{
\begin{array}{cl} \left| \Xi_\mf{m} \right| =
\left| \Cl_\mf{m}(K) \right| & \textnormal{ if } g = \bP^\mf{m}, \\
0 & \textnormal{ otherwise.}
\end{array} \right.  \qed
\]
\end{proposition}


\section{Zeta Functions and L-series}
\label{defs}

For a generalized Dirichlet character $\chi : \Cl_\mf{m}(K) \to
\CC^*$, the $L$-series $L_\mf{m}(z, \chi)$ is given by
\[
L_\mf{m}(z, \chi) = \sum_{(J, \mf{m}) = 1}
\frac{\chi(J)}{\bN(J)^z}
\]
where the sum is over integral ideals $J$ that are coprime to
$\mf{m}$. Note that if $\mf{m} = 1$ and $\chi: \Cl(K) \to \CC^*$
is trivial, we obtain the zeta function of the number field $K$:
\[
\zeta_K(z) = \sum_{J \subseteq A} \frac{1}{\bN(J)^z}.
\]
It will be useful when working with $L$-series to refer to the
zeta function of an ideal class $\mf{J} \in \Cl_\mf{m} (K)$,
given by
\[
\zeta(z, \mf{J}) = \sum_{J \in \mf{J}} \frac{1}{\bN(J)^z}.
\]
Note that
\[
L_\mf{m}(z, \chi) = \sum_{\mf{J} \in \Cl_\mf{m} (K)} \chi(\mf{J})
\zeta(z, \mf{J}).
\]
We show first that these series converge when $\Re(z)
> 1$. Note that for $p$ prime, the number of ideals $\mf{p}$
satisfying $\bN(\mf{p}) = p$ is less than or equal to $d = [K :
\QQ]$, and secondly, the number of prime factors (counted with
multiplicity) for a given $n$ is less than or equal to $\log_2
n$.  Combining these facts shows that the number of ideals $J$
satisfying $\bN (J) = n$ is less than or equal to $(\log_2 n)^d$.
Thus all the series above are dominated (in absolute value) by
\[
\sum_{n=1}^\infty \frac{(\log_2 n)^d}{n^{\Re(z)}},
\]
which converges provided that $\Re(z) > 1$.

The following product form of the $L$-series will be useful in
later proofs.
\begin{proposition} For $z \in \CC$ with $\Re(z) > 1$,
\[
L_\mf{m} (z, \chi) = \prod_{\mf{p} \nmid \mf{m}} \frac{1}{1 -
\frac{\chi( \mf{p} )}{\bN (\mf{p})^z}}.
\]
\end{proposition}
\begin{proof}
Taking the formal logarithm of the series gives
\[
\sum_{\mf{p} \nmid \mf{m}} \sum_{n=1}^\infty
\frac{\chi(\mf{p})^n}{n \bN(\mf{p})^{nz}}
\]
which is dominated by
\[
\sum_{p \in \NN} \sum_{n=1}^\infty \frac{d}{n p^{nz}} < \sum_{m
\in \NN} \frac{d}{m^z},
\]
and this converges for $\Re(z) > 1$.  Thus we may safely expand
and multiply out the series:
\begin{eqnarray*}
\prod_{\mf{p} \nmid \mf{m}} \frac{1}{1 - \frac{\chi( \mf{p}
)}{\bN (\mf{p})^z}} & = & \prod_{\mf{p} \nmid \mf{m}} \left(1 +
\frac{\chi( \mf{p} )}{\bN (\mf{p})^z} + \frac{\chi( \mf{p}
)^2}{\bN (\mf{p})^{2z}} + \cdots \right) \\
\end{eqnarray*}
The function $\chi( \cdot) / \bN(\cdot)^z$ is multiplicative, thus
\begin{eqnarray*}
\prod_{\mf{p} \nmid \mf{m}} \left(1 + \frac{\chi( \mf{p} )}{\bN
(\mf{p})^z} + \frac{\chi( \mf{p} )^2}{\bN (\mf{p})^{2z}} + \cdots
\right) = \sum_{(J,\mf{m}) = 1} \frac{\chi(J)}{\bN(J)^z}
\end{eqnarray*}
as desired.
\end{proof}

Already we have seen a connection between the $L$-series and the
number theoretic properties of $K$: the multiplicative expression
for $L_\mf{m} (z, \chi)$ is immediately equivalent to unique
factorization for ideals in $A$.

\section{The Riemann Zeta Function}

\label{riemann}

We have seen that the series for $L_\mf{m} (z, \chi)$ converges
to an analytic function for $\Re(z) > 1$.  However since we are
interested in the values of $L_\mf{m} (z, \chi)$ near $z = 1$, we
need to prove the existence of a larger analytic continuation. We
begin by considering the Riemann zeta function,
\[
\zeta_\QQ(z) = \sum_{p \in \NN} \frac{1}{p^z}.
\]
The convergence theorem for the Riemann zeta function will be
useful as an intermediate step in proving a similar theorem for
general $L$-series.
\begin{theorem}
There exists an analytic continuation of $\zeta_\QQ(z)$ to the
domain $\{ z : \Re(z) > 0, z \neq 1 \}$.
\end{theorem}
\begin{proof}
Consider the alternating series
\[
\zeta_2(z) = 1 - \frac{1}{2^z} + \frac{1}{3^z} - \frac{1}{4^z} +
\ldots .
\]
Let $\sigma = \Re(z)$; suppose $\sigma > 0$.  Then
\begin{eqnarray*}
\left| \zeta_2(z) \right| & = & \left | z \int_1^2
\frac{ds}{s^{z+1}} + z \int_3^4 \frac{ds}{s^{z+1}} + z \int_5^6
\frac{ds}{s^{z+1}} + \ldots \right| \\
& \leq & \frac{|z|}{1^{\sigma + 1}} + \frac{|z|}{3^{\sigma + 1}} +
\frac{|z|}{5^{\sigma + 1}} + \ldots
\end{eqnarray*}
which is convergent.  Thus $\zeta_2(z)$ is analytic on $\{ z :
\Re(z) > 0 \}$.  Now note that
\[
\zeta_\QQ(z) - \left( \frac{2}{2^z} \right) \zeta_\QQ(z) =
\zeta_2(z),
\]
for $\Re(z) > 1$, hence the formula
\[
\zeta_\QQ(z) = \left( 1 - \frac{1}{2^{z-1}} \right)^{-1}
\zeta_2(z)
\]
gives an analytic continuation of $\zeta_\QQ(z)$ to the region
$\{ z : \Re(z) > 0 \}$ minus the zeroes of the function $\left( 1
- 1/2^{z-1} \right)$.  We can take care of all but one of these
exceptional points by considering another alternating zeta series,
\[
\zeta_3(z) = \sum_{n=1}^\infty \left( \frac{1}{(3n + 1)^z} +
\frac{1}{(3n + 2)^z} - \frac{2}{(3n + 3)^z} \right).
\]
By similar reasoning as above, $\zeta_3(z)$ converges for $\Re(z)
> 0$, and
\[
\zeta_\QQ(z) = \left( 1 - \frac{1}{3^{z-1}} \right)^{-1}
\zeta_3(z)
\]
Thus we have an analytic continuation of $\zeta_\QQ(z)$ to $\{ z
: \Re(z) > 0 \}$ excluding the zeroes of $\left( 1 - 1/3^{z-1}
\right)$. Now suppose $z$ is a zero of both $\left( 1 - 1/2^{z-1}
\right)$ and $\left(1 - 1/3^{z-1} \right)$.  Then $2^{z-1} =
3^{z-1} = 1$, hence $z = 1 + \theta i$ and
\[
\theta \log 2 \equiv \theta \log 3 \equiv 0 \mod 2 \pi
\]
But $\log 2 / \log 3$ is irrational, so $\theta = 0$.  Thus $z =
1$ is the only excluded point.
\end{proof}
\begin{proposition}
The zeta function $\zeta_\QQ(z)$ has a simple pole at $z = 1$
with residue $1$.
\end{proposition}
\begin{proof}
Using the expression from the last proof,
\[
\zeta_\QQ(z) = \left(1 - \frac{1}{2^{z-1}} \right)^{-1}
\zeta_2(z).
\]
Now $\zeta_2(1)$ is the power series expression for $\log 2$,
while the residue of $\left(1 - 1/2^{z-1} \right)^{-1}$ at $1$ is
$1 / \log 2$.  Thus $\Res_{z = 1} \zeta_\QQ(z) = 1$.
\end{proof}
In fact any zeta function may be extended to $\CC - \{ 1 \}$, but
it is not necessary for us to prove that here (see
\cite{neukirch}, Ch. VII).  All ``nontrivial'' zeroes of the
Riemann zeta function lie in the strip $\{ z: 0 \leq z \leq 1 \}
\subseteq \CC$, so what we have proved above is sufficient to
state the famous Riemann hypothesis.
\begin{conjecture}
All nontrivial zeroes of $\zeta_\QQ (z)$ lie on the line $\{ z :
\Re(z) = 1/2 \}$.
\end{conjecture}

\section{The Distribution of Norms in an Ideal Class}

\label{distribution}

Let $\mf{J} \in \Cl_\mf{m}(K)$ be an ideal class. In order to
evaluate the zeta function $\zeta(z, \mf{J})$, we need an
asymptotic estimate for the function
\[
j(\mf{J}, t) = \left| \left\{ J \in \mf{J} : J \subseteq A, \bN(J)
\leq t \right\} \right|.
\]
Choose an (integral) ideal $B$ from the inverse class
$\mf{J}^{-1}$. The one-to-one map $J \mapsto JB$ takes an integral
ideal in $\mf{J}$ to a principal ideal; the image of this map is
precisely the set of ideals in $P^\mf{m}$ that are generated by
elements of $B$.  Also, $\bN(J) \leq t$ if and only if $\bN(JB)
\leq t\bN(B)$.  So it suffices to count the number of ideals $(x)
\in P^\mf{m}$ such that $x \in B$ and $\bN(x) \leq t \bN(B)$.

To put this another way:
\begin{lemma}
\label{anotherway} The function $j(\mf{J}, t)$ is equal to the
number of elements $x \in B$ satisfying
\[
x \equiv 1 \mod \mf{m} \quad \quad \textnormal{and} \quad \quad
\bN(x) \leq t\bN(B)
\]
modulo the unit group $U^\mf{m}$.  $\qed$
\end{lemma}

We will describe a geometric interpretation for the set of such
elements. Let $S_\infty$ denote the set of infinite primes of $K$.
The embeddings $\left\{ \sigma_v \right\}_{v \in S_\infty}$ give
an embedding
\[
K \to \prod_{v \in S_\infty} k_v \cong \RR^n.
\]
The vector space $\prod_{v \in S_\infty} k_v$ has a norm
\[
\bN(\mathbf{x}) = \prod_{v \in S_\infty} v(x_v)^{N_v}
\]
(here $N_v$ denotes the multiplicity of $v$ in $\mf{m}_\infty$).
Thus we define the unit ball $B^1 \subseteq \prod_{v \in S_\infty}
k_v$. Note that the image of an ideal under this map is a lattice.

We wish to make use of the following idea: if a region $D
\subseteq \RR^n$ is sufficiently nice, then we can estimate its
volume to an arbitrary degree of accuracy by counting lattice
points inside $D$ for sufficiently fine lattices.  This is made
precise in the theorem below.
\begin{theorem}
\label{lattice} Suppose $D \subseteq \RR^n$ is a subset whose
boundary is $(n-1)$-Lipshitz parametrizable, and suppose $L
\subseteq \RR^n$ is a lattice with fundamental parallelotope
$P$.  Let $\lambda(t, D, L)$ be the number of lattice points in
$tD$.  Then
\[
\lambda(t, D, L) = \frac{\Vol (D)}{\Vol (P)} t^n + O(t^{n-1}).
\]
\end{theorem}
\begin{proof}
See \cite{lang}, pp. 128.
\end{proof}
We will show that $j(\mf{J}, t)$ is equal to $\lambda(t, D, L)$
for appropriately chosen $D$ and $L$.  First note that the group
of units $U^\mf{m}$ acts on $\prod_{v \in S_\infty} k_v$ via
$u(\mathbf{x}) = (q(u) \cdot \mathbf{x})$.  This action becomes
fixed-point free if we restrict to the subset
\[
Q(S_\infty) = \left\{ (q_v)_{v \in S_\infty} : q_v \neq 0
\textnormal{ for all } v \textnormal{, and } q_v > 0 \textnormal{
for real } v \mid \mf{m}_\infty \right\}.
\]
Let $V \subseteq U^\mf{m}$ be a maximal free subgroup.
\begin{lemma}
There exists a fundamental domain $F \subseteq \RR^n$ for the
action of $V$ on $Q(S_\infty)$ such that $tF = F$ for all $t
> 0$.  Also, $F \cap B^1$ has $(N-1)$-parametrizable boundary.
\end{lemma}
\begin{proof}
See \cite{lang}, pp. 131.
\end{proof}

Now let $D = F \cap B^1$ and let $L \subseteq \RR^n$ be the image
of the $B \mf{m}_0$.

Choose (by the Chinese Remainder Theorem) an element $y \in B$
such that $y \equiv 1 \mod \mf{m}_0$. Consider the translated
lattice $y + L$. Every element of $(y + L) \cap F$ is the image of
an element $x \in B$ such that $(x) \in \bP^\mf{m}$ (recall Lemma
\ref{anotherway}).  Conversely, every such principal ideal $(x)$
has exactly $\left| U^\mf{m} / V \right|$ representatives in $(y +
L) \cap F$, since $F$ is a fundamental domain for the action $V
\subseteq U^\mf{m}$. Let $w_\mf{m} = \left| U^\mf{m} / V \right|$
(this is the number of roots of unity of $K$, which is finite).
We have
\begin{eqnarray*}
w_{\mf{m}} j( \mf{J}, t^n ) & = & \left| \left\{ \mathbf{x} \in
\RR^n : x \in F, \bN(x) \leq \bN(B) t^n \right\} \right| \\
& = & \lambda(t \bN(B), D, y + L).
\end{eqnarray*}
The translation by $y$ is asymptotically irrelevant; by Theorem
\ref{lattice},
\[
w_\mf{m} j( \mf{J}, t^n ) = \frac{\bN(B)\Vol (D)}{\Vol (P)} t^n +
O(t^{n-1}).
\]
where $P$ denotes the fundamental parallelotope of $L$.
Therefore,
\[
j( \mf{J}, t) = \frac{\bN(B)\Vol (D)}{w_\mf{m} \Vol (P)} t +
O(t^{1-1/n}).
\]
Now, the volume of the fundamental parallelotope corresponding to
$B$ is linearly related to the norm of $B$ (see \cite{lang}, pp.
115):
\[
\frac{\bN(B)}{\Vol (P)} = \frac{2^r}{\sqrt{\left| D_K \right|}}
\]
where $r$ is the number of complex primes of $K$ and $D_K$ is the
discriminant of $K$.  Thus let
\[
\rho_\mf{m} = \frac{\bN(B)\Vol (D)}{w_\mf{m} \Vol (P)} = \frac{2^r
\Vol(D)}{w_\mf{m} \sqrt{\left| D_K \right|}}.
\]
We have proven
\begin{theorem}
\label{asymp} There exist constants $\rho_\mf{m}$ and $n > 1$ such
that for any $\mf{J} \in \Cl_\mf{m} (K)$, the function $j(\mf{J},
t) = \left| \left\{ J \in \mf{J} : \bN(J) \leq t \right\}
\right|$ satisfies
\[
j(\mf{J}, t) = \rho_m t + O(t^{1 - 1/n}).
\]
\end{theorem}

Thus we see that the distribution of the norm function on an
ideal class is asymptotically linear, and moreover, the linear
coefficient is the same for any ideal class in $\Cl_m(K)$.

We note that there is a nice formula for $\Vol(D)$ in terms of the
invariants of the number field $K$ (but we won't need to prove
that here). This leads to a formula for the residue of
$\zeta_K(z)$ at $1$ called the Class Number Formula (see
\cite{neukirch}, pp. 467).

\section{L-series Convergence}

Consider the ideal class zeta function,
\[
\zeta( z, \mf{J}) = \sum_{J \in \mf{J}} \frac{1}{\bN(J)^z}.
\]
Let $a_m = \left| \left\{ J \in \mf{J} : \bN(J) = m \right\}
\right|$.  We have
\[
\zeta(z, \mf{J}) = \sum_{m=1}^\infty \frac{a_m}{m^z},
\]
and by Theorem \ref{asymp},
\[
\sum_{m = 1}^t a_m = \rho_\mf{m}t + O(t^{1 - 1/n}).
\]
We will demonstrate an analytic continuation of $\zeta(z,
\mf{J})$ via comparison with the Riemann zeta function.  Let $s_t
= \sum_{m=1}^t (a_m  - \rho_\mf{m})$. Observe:
\begin{eqnarray*}
\left| \sum_{m=1}^t \frac{a_m - \rho_\mf{m}}{m^z} \right| & = &
\left| \sum_{m=1}^t \frac{s_m}{m^z} - \sum_{m=0}^{t-1}
\frac{s_m}{(m+1)^z} \right| \\
& = & \left| \frac{s_t}{t^z} + \sum_{m=1}^{t-1} s_m \left(
\frac{1}{m^z} - \frac{1}{(m+1)^z} \right) \right| \\
& \leq & \left| \frac{s_t}{t^z} \right| + \left| \sum_{m=1}^{t-1}
s_mz \int_m^{m+1}
\frac{ds}{s^{z+1}} \right| \\
& \leq & \left| \frac{s_t}{t^z} \right| + \sum_{m=1}^{t-1}
\frac{\left| s_mz \right| }{m^{1 - 1/n}} \int_m^{m+1}
\frac{ds}{\left| s^{z + 1/n} \right| }  \\
\end{eqnarray*}
Let $\sigma = \Re(z)$ and suppose $\sigma > 1 - 1/n$.  $s_t =
O(t^{1 - 1/n})$, so let $C$ be an upper bound for $ \left| s_t /
t^{1 - 1/n} \right|$.  We have
\begin{eqnarray*}
\left| \sum_{m=1}^t \frac{a_m - \rho_\mf{m}}{m^z} \right| & = & C+
C |z|  \sum_{m=1}^{t-1} \int_m^{m+1}
\frac{ds}{s^{\delta + 1/n}} \\
& = & C + C |z| \int_1^t \frac{ds}{s^{\delta + 1/n}}
\end{eqnarray*}
which is bounded.  Thus,
\[
\sum_{m=1}^t \frac{a_m - \rho_\mf{m}}{m^z}
\]
converges to an analytic function for $\Re(z) > 1 - 1/n$.  Using
the results of Section \ref{riemann},
\[
\zeta(z, \mf{J}) = \left( \sum_{m=1}^t \frac{a_m -
\rho_\mf{m}}{m^z} \right) + \rho_\mf{m} \zeta_\QQ(z)
\]
is analytic on $\{ z: \Re(z) > 1 - 1/n \}$, except at $z=1$ where
it has a pole with residue $\rho_\mf{m}$.  The zeta function for
the number field $K$ satisfies
\[
\zeta_K(z) = \sum_{\mf{J} \in \Cl(K)} \zeta(z, \mf{J}),
\]
thus $\zeta_K(z)$ is analytic on $\{ z: \Re(z) > 1 - 1/n \}$
except for a pole at $1$ with residue $\rho_1 \left| \Cl(K)
\right|$.  Finally, if $\chi: \Cl_\mf{m} \to \CC^*$ is a
nontrivial character, then
\begin{eqnarray*}
L_\mf{m}(z, \chi) & = & \sum_{\mf{J} \in \Cl(K)} \chi( \mf{J} )
\zeta(z, \mf{J}) \\
& = & \sum_{\mf{J} \in \Cl(K)} \chi(\mf{J}) \left( \sum_{m=1}^t
\frac{a_m(\mf{J}) - \rho_\mf{m}}{m^z} \right) + \left(
\sum_{\mf{J} \in \Cl(K)}
 \chi(\mf{J}) \right) \rho_\mf{m} \zeta_\QQ(z) \\
& = & \sum_{\mf{J} \in \Cl(K)} \chi(\mf{J}) \left( \sum_{m=1}^t
\frac{a_m(\mf{J}) - \rho_\mf{m}}{m^z} \right)
\end{eqnarray*}
using Proposition \ref{nontriv}.  Thus $L_\mf{m}(z, \chi)$ is
analytic on all of the region $\{ z: \Re(z) > 1 - 1/n \}$.

\section{The Universal Norm Index Inequality}

An inequality that is useful in class field theory can be proven
from the information about $\zeta_K(z)$ and $L_\mf{m}(z, \chi)$
near $z=1$ that we developed above. We will need some asymptotic
notation: for analytic functions $f$ and $g$, write $f \sim g
\quad (z \to 1)$ to mean that $f/g$ can be continued analytically
to $1$. Write $f \precsim g \quad (z \to 1)$ and $f \succsim g
\quad (z \to 1)$ to mean that $f/g$ is bounded above near $1$,
and bounded below near $1$, respectively.  Recall the series for
the logarithm of $L_\mf{m}$:
\[
L_\mf{m}(z, \chi) = \exp \left( \sum_{\mf{p} \nmid \mf{m}} \sum_m
\frac{\chi(\mf{p})^m}{m \bN(\mf{p})^{mz}} \right).
\]
The sum of the second- and higher-order terms in the series
converges for $z$ near $1$, thus
\[
L_\mf{m}(z, \chi) \sim \exp \left( \sum_{\mf{p} \nmid \mf{m}}
\frac{\chi(\mf{p})}{\bN(\mf{p})^z} \right) \quad (z \to 1).
\]

Now, suppose $L/K$ is a Galois extension, and $B \subseteq L$ the
ring of integers. Let $\mf{N}_L^\mf{m} \subseteq K$ be the group
of norms of fractional ideals in $L$ that are coprime to $\mf{m}$.
We are interested in the subgroup of the ray class group,
\[
\bP^\mf{m} \subseteq \mf{N}^\mf{m}_L \bP^\mf{m} \subseteq
\bI^\mf{m}.
\]
\begin{theorem}
If $\mf{m}$ is divisible by all the ramified primes of $L/K$, then
\[
(\bI^\mf{m} : \bP^\mf{m} \mf{N}^\mf{m}_L ) \leq [L : K].
\]
(The degree $(\bI^\mf{m} : \bP^\mf{m} \mf{N}^\mf{m}_L )$ is called
the norm index of $K$.)
\end{theorem}

\begin{proof}
Write $H = \bP^\mf{m} \mf{N}^\mf{m}_L$ for convenience.  Consider
the product
\[
\prod_{\chi: \bI^\mf{m} / H \to \CC^*} L_\mf{m}(z, \chi).
\]
Recall that $L_\mf{m}(z, \chi)$ is analytic at $1$ unless $\chi$
is trivial, in which case $L_\mf{m}(z, \chi)$ has a simple pole
at $1$.  Thus,
\[
\frac{1}{z-1} \succsim \prod_{\chi} L_\mf{m}(z, \chi)  \quad (z
\to 1).
\]
Replacing $L_\mf{m}(z,\chi)$ with the approximation given above,
\begin{eqnarray*}
\frac{1}{z-1} & \succsim &  \exp\left( \sum_\chi \sum_{\mf{p}
\nmid \mf{m}} \frac{\chi(\mf{p})}{\bN(\mf{p})^z} \right) \\
& = & \exp\left[ \sum_{\mf{J} \in \bI^\mf{m} / H} \left( \sum_\chi
\chi(\mf{J}) \right) \sum_{\mf{p} \in \mf{J}}
\frac{1}{\bN(\mf{p})^z} \right] \\
& = & \exp\left[ (\bI^\mf{m} : H) \sum_{\mf{p} \in H}
\frac{1}{\bN(\mf{p})^z} \right]
\end{eqnarray*}
by Proposition \ref{sumchar}.


%Let $S(L/K)$ be the set of primes of $K$ that split completely in
%$L$.  Note that a prime $\mf{p}$ in $K$ splits completely in $L$
%if and only if it is a norm of a prime in $L$ and it is
%unramified.  Thus $S(L/K) \subseteq H$. So,
%\begin{eqnarray*}
%\frac{1}{z-1} & \succsim & \exp\left( (I^\mf{m} : H) \sum_{\mf{p}
%\in H} \frac{1}{\bN(\mf{p})^z} \right) \\
%& \succsim & \exp\left( (I^\mf{m} : H) \sum_{\mf{p} \in S(L/K)}
%\frac{1}{\bN(\mf{p})^z} \right) \\
%\end{eqnarray*}
Now, if $\mf{p}$ is a prime in $K$, and $\left\{ \mf{q}_i
\right\}_{i=1}^k$ are the primes lying above $\mf{p}$ with
residue field extension degrees $\deg \mf{q}_i$, then $\bN(\mf{q})
= \bN(\mf{p})^{\deg \mf{q}_i}$, so
\begin{eqnarray*}
\sum_{j=1}^k \frac{1}{\bN(\mf{q}_j)^z} & = & \sum_{\deg \mf{q_j} =
1} \frac{1}{\bN(\mf{p})^z} + \left\{ \textnormal{terms with exponent} \geq 2z \right\} \\
& \leq & \frac{[L:K]}{\bN(\mf{p})^z} + \left\{ \textnormal{terms with exponent} \geq 2z \right\} \\
\end{eqnarray*}
And every prime $\mf{q}$ in $B$ lies over a prime in $H$ by
definition.  Thus, by ignoring terms with exponent $\geq 2z$
(which are asymptotically irrelevant near $z=1$) we find
\begin{eqnarray*}
\frac{1}{z-1} & \succsim & \exp\left( (\bI^\mf{m} : H)
\sum_{\mf{p} \in H}
\frac{1}{\bN(\mf{p})^z} \right) \\
& \succsim & \exp\left( \frac{(\bI^\mf{m} : H)}{[L : K]}
\sum_{\mf{q} \subseteq B}
\frac{1}{\bN(\mf{q})^z} \right) \\
& = & \zeta_L(z)^{\frac{(\bI^m : H)}{[L:K]}} \\
& \succsim & \left(\frac{1}{z-1} \right)^{\frac{(\bI^m :
H)}{[L:K]}}
\end{eqnarray*}
Thus $(\bI^m : H) \leq [L : K]$, as desired.
\end{proof}

\begin{thebibliography}{00}

\bibitem{lang}
S.~Lang: Algebraic Number Theory, Addison-Wesley, 1970.

\bibitem{ANF}
G.~Janusz: Algebraic Number Fields, Acadmic Press, Inc., 1973.

\bibitem{neukirch}
J.~Neukirch: Algebraic Number Theory, Springer-Verlag, 1999.

\end{thebibliography}

\end{document}
