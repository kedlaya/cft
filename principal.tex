%\documentclass[12pt]{article}
%\usepackage{amsfonts, amsthm, amsmath}
%\usepackage[all]{xy}
%
%\setlength{\textwidth}{6.5in}
%\setlength{\oddsidemargin}{0in}
%\setlength{\textheight}{8.5in}
%\setlength{\topmargin}{0in}
%\setlength{\headheight}{0in}
%\setlength{\headsep}{0in}
%\setlength{\parskip}{0pt}
%\setlength{\parindent}{20pt}
%
%\def\AA{\mathbb{A}}
%\def\CC{\mathbb{C}}
%\def\FF{\mathbb{F}}
%\def\PP{\mathbb{P}}
%\def\QQ{\mathbb{Q}}
%\def\RR{\mathbb{R}}
%\def\ZZ{\mathbb{Z}}
%\def\gotha{\mathfrak{a}}
%\def\gothb{\mathfrak{b}}
%\def\gothm{\mathfrak{m}}
%\def\gotho{\mathfrak{o}}
%\def\gothp{\mathfrak{p}}
%\def\gothq{\mathfrak{q}}
%\def\gothr{\mathfrak{r}}
%\DeclareMathOperator{\ab}{ab}
%\DeclareMathOperator{\disc}{Disc}
%\DeclareMathOperator{\Frob}{Frob}
%\DeclareMathOperator{\Gal}{Gal}
%\DeclareMathOperator{\GL}{GL}
%\DeclareMathOperator{\Hom}{Hom}
%\DeclareMathOperator{\Norm}{Norm}
%\DeclareMathOperator{\Trace}{Trace}
%\DeclareMathOperator{\Cl}{Cl}
%
%\def\head#1{\medskip \noindent \textbf{#1}.}
%
%\newtheorem{theorem}{Theorem}
%\newtheorem{lemma}[theorem]{Lemma}
%
%\begin{document}
%
%\begin{center}
%\bf
%Math 254B, UC Berkeley, Spring 2002 (Kedlaya) \\
%The Principal Ideal Theorem
%\end{center}

\head{Reference} 
Milne, section V.3 (but you won't find the proofs I've omitted there either);
Neukirch, section VI.7 (see also IV.5); 
Lang, \textit{Algebraic Number Theory}, section XI.5.

\medskip
For a change, we're going to prove something, if only assuming the
Artin reciprocity law which we haven't proved. Or rather, we're going to sketch
a proof that you will fill in by doing the exercises. (Why should I have all
the fun?)

The following theorem
is due to Furtw\"angler, a student of Hilbert. (It's also called the
``capitulation'' theorem, because in the old days the word ``capitulate''
meant ``to become principal''. Etymology left to the reader.)
\begin{theorem}[Principal ideal theorem] \label{T:principal ideal theorem}
Let $L$ be the Hilbert class field of the number field $K$. Then
every ideal of $K$ becomes principal in $L$.
\end{theorem}
(Warning: this does not mean that $L$ has class number 1!)
Example: if $K = \QQ(\sqrt{-5})$, then $L = \QQ(\sqrt{-5}, \sqrt{-1})$,
and the nonprincipal ideal class of $K$ is represented by $(2, 1+\sqrt{-5})$,
which is generated by $1+\sqrt{-1}$ in $L$.

The idea is that given Artin reciprocity, this reduces to a question
in group theory. Namely, let $M$ be the Hilbert class field of $L$; 
then an ideal of $L$ is principal if and only if its image under the Artin
map $I_L \to \Gal(M/L)$ is trivial. So what we need is to find a
map $V$ such that
\[
\xymatrix{
\Cl(K) \ar[r] \ar[d] & \Gal(L/K) \ar^{V}[d] \\
\Cl(L) \ar[r] & \Gal(M/L)
}
\]
commutes, then show that $V$ is the zero map. (The horizontal arrows are the
Artin maps.)

Regarding the relationship between $K$, $L$ and $M$:
\begin{enumerate}
\item[(a)] $M$ is Galois over $K$ (because its image under any element of
$\Gal(\overline{K}/K)$ is still an abelian extension of $L$ unramified
at all finite and infinite places,
and so is contained in $M$) and unramified everywhere
(since $M/L$ and $L/K$ are unramified);
\item[(b)] $L$ is the maximal subextension of $M/K$ which is abelian over $K$
(since any abelian subextension is unramified over $K$, and so is contained in $L$).
\end{enumerate}
Given a finite group $G$, let $G^{\ab}$ denote the maximal abelian quotient
of $G$; that is, $G^{\ab}$ is the quotient of $G$ by its commutator subgroup
$G'$. Then (b) implies that $\Gal(M/L)$ is the commutator subgroup
of $\Gal(M/K)$ and $\Gal(M/K)^{\ab} = \Gal(L/K)$.

Before returning to the principal ideal theorem, we need to do a bit of
group theory.
Let $G$ be a finite group and $H$ a subgroup (but not necessarily
normal!).
Let $g_1, \dots, g_n$ be
left
coset representatives of $H$ in $G$: that is, $G = g_1H \cup \cdots \cup g_nH$.
For $g \in G$, put $\phi(g) = g_i$ if $g \in g_iH$ (i.e.,
$g_i^{-1}g \in H$). Put
\[
V(g) = \prod_{i=1}^n \phi(gg_i)^{-1}(gg_i);
\]
then $V(g)$ always lands in $H$. In particular, it induces a map
$V: G \to H^{\ab}$.
\begin{theorem} \label{T:transfer homomorphism}
The map $V: G \to H^{\ab}$ is a homomorphism, does not depend on the choice
of the $g_i$, and induces a homomorphism $G^{\ab} \to H^{\ab}$ (i.e.,
kills commutators in $G$).
\end{theorem}
The map $G^{\ab} \to H^{\ab}$ is called the \emph{transfer map} (in German,
``Verlagerung'', hence the $V$).

Now let's return to that diagram:
\[
\xymatrix{
\Cl(K) \ar[r] \ar[d] & \Gal(L/K) = \Gal(M/K)^{\ab} \ar^{V}[d] \\
\Cl(L) \ar[r] & \Gal(M/L) = \Gal(M/L)^{\ab}
}
\]
and show that the transfer map $V$ does actually make this diagram commute;
it's enough to check this when we stick a prime $\gothp$ of $K$ in at the
top left. For consistency of notation, put $G = \Gal(M/K)$ and $H
= \Gal(M/L)$, so that $G/H = \Gal(L/K)$.
Choose a prime $\gothq$ of $L$ over $\gothp$ and a prime $\gothr$ of $M$
over $\gothq$, let $G_{\gothr} \subseteq G$
be the decomposition group of $\gothr$
over $K$ (i.e., the set of automorphisms mapping $\gothr$ to itself)
and let $g \in G_{\gothr}$ be the Frobenius of $\gothr$.
(Note: since $G$ is not abelian, $g$ depends on the choice of $\gothr$,
not just on $\gothq$. That is, there's no Artin map into $G$.)

Let $\gothq_1, \dots, \gothq_r$ be the primes
of $L$ above $\gothp$; then the image of $\gothp$ in $L$ is
$\prod_i \gothq_i$, and the image of that product under the Artin map
is $\prod_i \Frob_{M/L}(\gothq_i)$. To show that this equals $V(g)$, we make a careful choice of the coset representatives $g_i$ in the
definition of $V$. Namely, decompose $G$ as a union of double cosets
$G_{\gothr} \tau_i H$. Then the primes of $L$ above $\gothp$ correspond
to these double cosets, where the double coset $G_{\gothr} \tau_i H$
corresponds to $L \cap \gothr^{\tau_i}$.
Let $m$ be the order of $\Frob_{L/K}(\gothp)$
and write $G_{\gothr} \tau_i H = \tau_iH
\cup g\tau_i H \cup \cdots \cup g^{m-1}\tau_i H$
for each $i$; we then use the elements $g_{ij} = g^j \tau_i$ as the left
coset representatives to define $\phi$
and $V$.
Thus the equality $V(g) = \prod_i \Frob_{M/L}(\gothq_i)$ follows
from the following lemma.
\begin{lemma} \label{L:transfer Frobenius}
If $L \cap \gothr^{\tau_i} = \gothq_i$, then
$\Frob_{M/L}(\gothq_i) = \prod_{j=0}^{m-1} \phi(g g_{ij})^{-1} g g_{ij} $.
\end{lemma}

Thus the principal ideal theorem now follows from the following fact.
\begin{theorem} \label{T:transfer vanishes}
Let $G$ be a finite group and $H$ its commutator subgroup. Then the transfer
map $V: G^{\ab} \to H^{\ab}$ is zero.
\end{theorem}

\head{Exercises}

\begin{enumerate}
\item
Prove Theorem~\ref{T:transfer homomorphism}. (Hint: one approach to proving independence from choices
is to change one $g_i$ at a time. Also, notice that $\phi(gg_1), \dots,
\phi(gg_n)$ are a permutation of $g_1, \dots, g_n$.)
\item
Prove Lemma~\ref{L:transfer Frobenius}.
(Hint: see Neukirch, Proposition IV.5.9.)
\item
With notation as in Theorem~\ref{T:transfer homomorphism},
let $\ZZ[G]$ be the group algebra of $G$ (formal linear combinations
$\sum_{g \in G} n_g [g]$ with $n_g \in \ZZ$, multiplied by putting
$[g][h] = [gh]$) and let $I_G$ be the ideal of sums $\sum n_g[g]$ with
$\sum n_g = 0$ (called the \emph{augmentation ideal}; see
Chapter~\ref{chap:homology}). Let 
\[
\delta: H/H' \to (I_{H}+I_GI_{H})/I_GI_{H}
\]
be the homomorphism taking the class of $h$ to the class of $[h]-1$.
Prove that $\delta$ is an isomorphism. (Hint:
show that the elements 
\[
[g]([h]-1) \qquad \mbox{for $g \in \{g_1,\dots,g_n\}, h \in H$}
\]
form a basis of $I_H + I_G I_H$ as a $\ZZ$-module. For more clues, see Neukirch, Lemma VI.7.7.)
\item
With notation as in the previous exercise, prove that the following diagram commutes:
\[
\xymatrix{
G/G' \ar^{V}[r] \ar^{\delta}[d] & H/H' \ar^{\delta}[d] \\
I_G/I_G^2 \ar^(.3){S}[r] & (I_{H}+I_GI_{H})/I_GI_{H},
}
\]
where $S$
is given by $S(x) = x ([g_1] + \cdots + [g_n])$.
\item
Prove Theorem~\ref{T:transfer vanishes}.
(Hint: quotient by the commutator subgroup of $H$ to reduce
to the case where $H$ is abelian.
Apply the classification of finite abelian groups
to write $G/H$ as a product of cyclic groups $\ZZ/e_1 \ZZ \times \cdots \times \ZZ/e_m \ZZ$.
Let $f_i$ be an element of $G$ lifting a generator of $\ZZ/e_i \ZZ$
and put $h_i = f_i^{-e_i} \in H$; then $0 = \delta(f_i^{e_i} h_i)$,
which can be rewritten as $\delta(f_i) \mu_i$ for some $\mu_i \in \ZZ[G]$
congruent to $e_i$ modulo $I_G$.
Now check that 
\[n
\mu_1 \cdots \mu_m \equiv [g_1] + \cdots + [g_n] \pmod{I_H \ZZ[G]}.
\]
For more details, see
Neukirch, Theorem VI.7.6.)
\end{enumerate}

%\end{document}


