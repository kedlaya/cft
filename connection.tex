%\documentclass[12pt]{article}
%\usepackage{amsfonts, amsthm, amsmath}
%\usepackage[all]{xy}
%
%\setlength{\textwidth}{6.5in}
%\setlength{\oddsidemargin}{0in}
%\setlength{\textheight}{8.5in}
%\setlength{\topmargin}{0in}
%\setlength{\headheight}{0in}
%\setlength{\headsep}{0in}
%\setlength{\parskip}{0pt}
%\setlength{\parindent}{20pt}
%
%\def\AA{\mathbb{A}}
%\def\CC{\mathbb{C}}
%\def\FF{\mathbb{F}}
%\def\NN{\mathbb{N}}
%\def\PP{\mathbb{P}}
%\def\QQ{\mathbb{Q}}
%\def\RR{\mathbb{R}}
%\def\ZZ{\mathbb{Z}}
%\def\gotha{\mathfrak{a}}
%\def\gothb{\mathfrak{b}}
%\def\gothm{\mathfrak{m}}
%\def\gotho{\mathfrak{o}}
%\def\gothp{\mathfrak{p}}
%\def\gothq{\mathfrak{q}}
%\DeclareMathOperator{\ab}{ab}
%\DeclareMathOperator{\Cor}{Cor}
%\DeclareMathOperator{\disc}{Disc}
%\DeclareMathOperator{\Gal}{Gal}
%\DeclareMathOperator{\GL}{GL}
%\DeclareMathOperator{\Hom}{Hom}
%\DeclareMathOperator{\Ind}{Ind}
%\DeclareMathOperator{\Norm}{Norm}
%\DeclareMathOperator{\Res}{Res}
%\DeclareMathOperator{\sign}{sign}
%\DeclareMathOperator{\smcy}{smcy}
%\DeclareMathOperator{\Trace}{Trace}
%\DeclareMathOperator{\unr}{unr}
%\DeclareMathOperator{\Cl}{Cl}
%
%\def\head#1{\medskip \noindent \textbf{#1}.}
%
%\newtheorem{theorem}{Theorem}
%\newtheorem{lemma}[theorem]{Lemma}
%\newtheorem{prop}[theorem]{Proposition}
%
%\begin{document}
%
%\begin{center}
%\bf
%Math 254B, UC Berkeley, Spring 2002 (Kedlaya) \\
%The Connection With Local Reciprocity
%\end{center}

\head{Reference} 
Milne VII.5; Neukirch VI.4.

\medskip
So far, we've used abstract class field theory to construct
reciprocity isomorphisms
\[
r'_{L/K}: C_K/\Norm_{L/K} C_L \to \Gal(L/K)^{\ab}
\]
and to establish the adelic form of the existence theorem. We also
know that if $L/K$ is a small cyclotomic extension, then this
map induces the usual Artin map.

This time, we'll verify that this map coincides with the product of the
local reciprocity maps. As noted earlier, this is enough to recover the
classical Artin reciprocity law and existence theorem.

I've also included a sketch of a Galois-cohomological approach to the reciprocity isomorphism (as found in Milne),
using $H^2$ and an explicit computation in local class field theory.
One of the sketchy points is that this computation 
requires a little of the Lubin-Tate construction, which makes the local existence theorem rather explicit but will not be discussed herein.

\head{The relationship with local reciprocity}

\head{Caveat} This still does not follow Milne or Neukirch.

\medskip
For any extension $L/K$ of number fields,
we currently have the map $r_{L/K}: I_K \to \Gal(L/K)^{\ab}$
formed as the product of the local reciprocity maps,
and the abstract reciprocity map $r'_{L/K}: I_K \to \Gal(L/K)^{\ab}$,
which actually factors through $C_K$ and even through $C_K/\Norm_{L/K} C_L$.
We want to show that these agree. Before doing so, let's observe some
consequences of that which we'll then use in the proof that they agree.

If $L/K$ is abelian, $v$ is a place of $K$ and $w$ is a place of $L$
above $v$, then we have an injection $K_v^* \to I_K$, which we then
funnel through $r'_{L/K}$ to get a map into $\Gal(L/K)$.
The following properties would follow from knowing that $r = r'$, but must be checked independently as part of the proof.
\begin{lemma} \label{L:reciprocity corollaries}
The following statements hold:
\begin{enumerate}
\item[(i)] the composite map $K_v^* \to \Gal(L/K)$ actually maps into
the decomposition group of $w$;
\item[(ii)] the subgroup $\Norm_{L_w/K_v} L_w^*$ is contained in the kernel
of $K_v^* \to \Gal(L/K)$.
\end{enumerate}
\end{lemma}
In (ii), we would also know that ``contained in'' can be replaced by 
``equal to'', but we won't try to check that independently.
\begin{proof}
For (i), let $M$ be the
fixed field of the decomposition group of $w$; then we have the compatibility
\[
\xymatrix{
\Gal(L/K) \ar[r] \ar[d] & C_K/\Norm_{L/K} C_L \ar[d] \\
\Gal(M/K) \ar[r] & C_K/\Norm_{M/K} C_M
}
\]
and the image of $K_v^* \to I_K$ lands in $\Norm_{M/K} I_M$ because
$v$ splits completely in $M$. So this image lies in the kernel of
$\Gal(L/K) \to \Gal(M/K)$, which is to say $\Gal(L/M)$, the decomposition
group of $w$.

For (ii), we need only check that 
$\Norm_{L_w/K_v} L_w^*$ is contained in the kernel
of $K_v^* \to C_K/\Norm_{L/K} C_L$.
But $\Norm_{L_w/K_v} L_w^*$ is already in the kernel of
$K_v^* \to I_K/\Norm_{L/K} I_L$, so we're all set.
\end{proof}

Our plan now is to attempt to recover the local reciprocity map
from the maps $r'_{L/K}$. To do this, we need some auxiliary global extensions,
provided by the Existence Theorem.
\begin{lemma}
Let $K$ be a number field, $v$ a place of $K$ and $M$ a finite
abelian extension
of $K_v$. Then there exists a finite abelian extension $L$ of $K$ such that
for any place $w$ of $L$ above $v$, $L_w$ contains $M$.
\end{lemma}
\begin{proof}
This is easy if $v$ is infinite: if $v$ is complex there is nothing to
prove, and if $v$ is real then we may take $L = K(\sqrt{-1})$. So
assume hereafter that $v$ is finite.

By the Existence Theorem (Theorem~\ref{T:adelic existence theorem2}) and Lemma~\ref{L:reciprocity corollaries}(ii),
it suffices to produce an open subgroup $U$ of $C_K$ of finite index
such that the preimage of $U$ under $K_v^* \to C_K$ is contained in
$N = \Norm_{M/K_v} M^*$. Let $S$ be the set of infinite places
and $T = S \cup \{v\}$,
and let $G = K_T \cap N$. Then one can choose an additional place
$u$ (finite and distinct from $v$) and an open subgroup $V$ of $\gotho_{K_u}^*$
such that $V \cap K_T \subseteq G$. Now put
\[
W = N \times V \times \prod_{w \in S} K_w^* \times \prod_{w \notin S \cup \{u,v\}} \gotho_K^*
\]
and $U = W K^*/K^*$. If $\alpha_v \in K_v^*$ maps into $U$, then there
exists $\beta \in K^*$ such that $\alpha_v \beta \in W$. That means
first of all that $\beta \in K_T$ and then that $\beta \in V$, so that
$\beta \in G$ and so also $\beta \in N$. It also means that $\alpha_v \beta
\in N$. Thus $\alpha_v \in N$, as desired.
\end{proof}

For each place $v$ of $K$ and each abelian extension
$M$ of $K_v$, we can now write down
a map $r'_{K,v}: K_v^* \to \Gal(M/K_v)$ 
by choosing an abelian extension $L$ such that $M \subseteq L_w$ for
any place $w$ of $L$ above $v$, letting $N$ be the fixed field of the
decomposition group of $w$,
and setting $r'_{K,v}$ equal to the composition
\[
K_v^* \stackrel{r'_{L/K}}{\to} \Gal(L/N) = \Gal(L_w/K_v) \to \Gal(M/K_v).
\]
By the same compatibility as above, this doesn't change if we enlarge $L$.
Thus it doesn't depend on the choice of $L$ at all! (Any two choices of $L$
sit inside an abelian extension of $K$; compare both with that bigger field.)

Again by the usual compatibilities,
these maps fit together to give a single map
$r'_{K,v}: K_v^* \to \Gal(K_v^{\ab}/K_v)$. This map has the following
properties:
\begin{enumerate}
\item[(a)]
For $M/K$ unramified,
the induced map $K_v^* \to \Gal(M/K_v^*)$ kills units
and maps a uniformizer of $K_v$ to the Frobenius automorphism. Since that
extension is generated by roots of unity, we can check this using a 
suitable small cyclotomic extension of $K$, on which $r'$ may be computed explicitly.
We leave further details to the reader.

\item[(b)]
For any finite extension $M/K_v^*$, $r'_{K,v}$ induces an isomorphism
\[
K_v^*/\Norm_{M/K_v} M^* \to \Gal(M/K_v^*).
\]
Note that \emph{a priori} we only know that this map is injective, but by the local reciprocity law the
two groups have the same order, so it's actually an isomorphism. (For this and other reasons, we do not get an independent proof of
local class field theory by this process.)
\end{enumerate}
But these properties \emph{uniquely} characterize the local reciprocity map!
We conclude that $r'_{K,v}$ is the local reciprocity map for $K_v$,
and so $r_{L/K} = r'_{L/K}$ and at long last Artin reciprocity (and the
classical existence theorem, and the whole lot) follows.
Hooray!

It's worth repeating that only now do we know that the product
$r_{L/K}$ of the local reciprocity maps kills principal id\`eles. That fact,
which relates local behavior for different primes in a highly global
fashion, is the basis of various \emph{higher reciprocity laws}. See Milne,
Chapter VIII for details.

\head{An explicit computation in local CFT}

We sketch an alternate approach for comparing the ``abstract'' reciprocity
map $r'_{L/K}$ with the product $r_{L/K}$ of the local reciprocity maps,
following Milne (and Neukirch V.2).

We first verify that $r = r'$ for cyclotomic extensions of $\QQ$,
using an explicit computation in local class field theory. Namely,
we compute that if we identify $\Gal(\QQ(\zeta_{p^m})/\QQ)$ with 
$(\ZZ/p^m\ZZ)^*$, then
the local reciprocity maps are given by
\[
r_{\QQ_{\ell}(\zeta_{p^m})/\QQ_{\ell}}(a)
= \begin{cases} \sign(a) & \ell = \infty \\
\ell^{v_{\ell}}(a) & \ell \neq \infty, p \\
u^{-1} & \ell = p.
\end{cases}
\]
This is straightforward for $\ell = \infty$. For $\ell \neq \infty, p$,
we have an unramified extension of local fields, where we know the 
local reciprocity map takes a uniformizer to a Frobenius. In this case
the latter is simply $\ell$.

The hard work is in the case $\ell=p$. For that computation one uses what
amounts to a very special case of the Lubin-Tate construction of
explicit class field theory for local fields, using formal groups.
Put $K = \QQ_p$, $\zeta = \zeta_{p^m}$ and $L = \QQ_p(\zeta)$.

Suppose without loss of generality that $u$ is a positive integer, and let $\sigma
\in \Gal(L/K)$ be the automorphism corresponding to $u^{-1}$.
Since $L/K$ is totally ramified at $p$, we have
$\Gal(L/K) \cong \Gal(L^{\unr}/K^{\unr})$, and we can view $\sigma$
as an element of $\Gal(L^{\unr}/K)$. Let $\phi_L \in \Gal(L^{\unr}/L)$
denote the Frobenius, and put $\tau = \sigma \phi_L$. Then
$\tau$ restricts to the Frobenius in $\Gal(K^{\unr}/K)$
and to $\sigma$ in $\Gal(L/K)$. By Neukirch's definition of the reciprocity
map, we may compute $r^{-1}_{L/K}(\sigma)$ as
$\Norm_{M/K} \pi_M$, where $M$ is the fixed field of
$\tau$ and $\pi_M$ is a uniformizer. We want that norm to be $u$
times a norm from $L$ to $K$, i.e.,
\[
r^{-1}_{L/K}(\sigma) \in u \Norm_{L/K} L^*.
\]

Define the polynomial
\[
e(x) = x^p + upx
\]
and put
\[
P(x) = e^{(n-1)}(x)^{p-1} + pu,
\]
where $e^{(k+1)}(x) = e(e^{(k)}(u))$. Then $P(x)$ satisfies Eisenstein's
criterion, so its splitting field over $\QQ_p$ is totally ramified,
any root of $P$ is a uniformizer, and the norm of said uniformizer is
$(-1)^{[L:K]} pu \in \Norm_{L/K} L^*$, since
$\Norm_{L/K}(\zeta-1) = (-1)^{[L:K]}(p)$.

The punch line is that the splitting field of $P(x)$ is precisely
$M$! Here is where the Lubin-Tate construction comes to the rescue...
and where I will stop this sketch. See Neukirch V.2 and V.4 and/or Milne
I.3.

\head{A bit about Brauer groups}

For background about Brauer groups, see Milne, IV. We'll be following Milne VII.8 for now,
and omitting many details.

\begin{prop}
Put $L = K(\zeta_n)$. Then
$r_{L/K}: I_K \to \Gal(L/K)$ maps all principal id\`eles to the
identity.
\end{prop}
\begin{proof}
For $K = \QQ$, this follows from the previous section (factor $n$ into
prime powers and apply the previous argument to each factor).
In general, we have a compatibility
\[
\xymatrix{
I_L \ar[r] \ar^{\Norm_{L_w/\QQ_p}}[d] & \Gal(L_w(\zeta_n)/L_w) \ar[d]\\
I_{\QQ} \ar[r] & \Gal(\QQ_p(\zeta_n)/\QQ_p)
}
\]
and we know the bottom row kills principal id\`eles and the
right column is injective. Thus the top row kills principal id\`eles too.
\end{proof}

To make more progress, we need to bring in $H^2$, as we did in local
reciprocity. (Unfortunately, trying to compute $H^2$ of the id\`ele class
group is a headache, so we can't imitate the argument perfectly.)
Recall there that we saw that every element of $H^2(L/K)$
could be ``brought in'' from a suitable unramified extension of $K$.
We have a similar situation here with ``unramified'' replaced by
``cyclotomic''.
\begin{prop}
Let $L/K$ be any finite Galois extension of number fields. Then for any
element $x$ of $H^2(\Gal(L/K), L^*)$, there exists a cyclic, cyclotomic
extension $M$ of $K$ and an element $y$ of $H^2(\Gal(M/K), M^*)$ such that
$x$ and $y$ map to the same element of $H^2(\Gal(ML/K), ML^*)$.
\end{prop}
\begin{proof}
Omitted. See above references.
\end{proof}

Hereafter $L/K$ is abelian.
From the exact sequence
\[
0 \to L^* \to I_L \to C_L \to 0
\]
we get a fragment
\[
1 = H^1(\Gal(L/K), C_L)
\to H^2(\Gal(L/K), L^*) \to H^2(\Gal(L/K), I_L)
\]
so the map $H^2(\Gal(L/K), L^*) \to H^2(\Gal(L/K), I_L)
= \oplus H^2(\Gal(L/K), I_L)$ is injective. Each factor in the direct
sum is canonically a subgroup of $\QQ/\ZZ$, so we get a sum map
$H^2(\Gal(L/K), I_L) \to \QQ/\ZZ$.

It turns out (see Milne, Lemma VII.8.5)
that for any map $\Gal(L/K) \to \QQ/\ZZ$,
there is a commuting diagram
\[
\xymatrix{
K^* \ar[r] \ar[d] & I_K \ar^{r_{L/K}}[r] \ar[d] &\Gal(L/K) \ar[d] \\
H^2(L^*) \ar[r] & H^2(I_L) \ar[r] & \QQ/\ZZ
}
\]
If $L/K$ is cyclic, we may choose the map $\Gal(L/K) \to \QQ/\ZZ$
to be injective, and then the first vertical arrow will be surjective.
(In fact, it's $K^* \to K^*/\Norm_{L/K} L^* = H^0_T(L^*)$ plus the
periodicity isomorphism $H^0_T(L^*) \to H^2_T(L^*)$.)
Then the fact that $r_{L/K}$ kills principal id\`eles implies that
the composite $H^2(L^*) \to \QQ/\ZZ$ is the zero map.

Now if we know $H^2(\Gal(L/K), L^*) \to \QQ/\ZZ$ vanishes for all
cyclic extensions, we know it in particular for cyclic cyclotomic extensions.
But then the previous proposition tells us that it also vanishes for
any finite Galois extension!
Now we can use the diagram in reverse: it tells us that 
for $a \in K^*$, $r_{L/K}(a)$ is killed by any homomorphism
$\Gal(L/K) \to \QQ/\ZZ$. Since $\Gal(L/K)$ is an abelian group, that 
implies $r_{L/K}(a)$ is trivial.

To conclude, we now have that $r_{L/K}$ kills principal id\`eles in general.
By construction, it also kills norms (since it does so locally),
so it induces a surjection $C_K/\Norm_{L/K} C_L \to \Gal(L/K)$.
(Remember, the fact that it's surjective follows from the First
Inequality.)
But the order of the first group is less than or equal to the order of the
second by the Second Inequality. So it's an isomorphism, and 
the reciprocity law is established. Hooray again!

%\end{document}
