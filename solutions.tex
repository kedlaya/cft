\documentclass[12pt]{article}
\usepackage{amsfonts, amsthm, amsmath}
\usepackage[all]{xy}

\setlength{\textwidth}{6.5in}
\setlength{\oddsidemargin}{0in}
\setlength{\textheight}{8.5in}
\setlength{\topmargin}{0in}
\setlength{\headheight}{0in}
\setlength{\headsep}{0in}
\setlength{\parskip}{0pt}
\setlength{\parindent}{20pt}

\def\AA{\mathbb{A}}
\def\CC{\mathbb{C}}
\def\FF{\mathbb{F}}
\def\PP{\mathbb{P}}
\def\QQ{\mathbb{Q}}
\def\RR{\mathbb{R}}
\def\ZZ{\mathbb{Z}}
\def\gotha{\mathfrak{a}}
\def\gothb{\mathfrak{b}}
\def\gothm{\mathfrak{m}}
\def\gotho{\mathfrak{o}}
\def\gothp{\mathfrak{p}}
\def\gothq{\mathfrak{q}}
\def\gothr{\mathfrak{r}}
\DeclareMathOperator{\ab}{ab}
\DeclareMathOperator{\coker}{coker}
\DeclareMathOperator{\disc}{Disc}
\DeclareMathOperator{\Frob}{Frob}
\DeclareMathOperator{\Gal}{Gal}
\DeclareMathOperator{\GL}{GL}
\DeclareMathOperator{\Hom}{Hom}
\DeclareMathOperator{\id}{id}
\DeclareMathOperator{\im}{im}
\DeclareMathOperator{\Ind}{Ind}
\DeclareMathOperator{\Norm}{Norm}
\DeclareMathOperator{\Trace}{Trace}
\DeclareMathOperator{\Cl}{Cl}

\def\head#1{\medskip \noindent \textbf{#1}.}

\newtheorem{theorem}{Theorem}
\newtheorem{lemma}[theorem]{Lemma}
\newtheorem{cor}[theorem]{Corollary}

\begin{document}

\begin{center}
\bf
Math 254B, UC Berkeley, Spring 2002 (Kedlaya) \\
Solutions to Selected Exercises (updated 21 Feb)
\end{center}

\head{Local Kronecker-Weber}

3. From the previous exercise we have constructed a cyclotomic extension $F$
of $\QQ_2$ with $\Gal(F/\QQ_2) = \ZZ/2\ZZ \times (\ZZ/2^r\ZZ)^2$. If
$K \not\subset F$, then let $KF$ be their compositum and $G =
\Gal(KF/\QQ_2)$. Now:
\begin{enumerate}
\item[(a)] we have surjective but not injective maps
$f: G \to \Gal(K/\QQ_2) = (\ZZ/2^r\ZZ)H$ and
$h: G \to \Gal(F/\QQ_2) = \ZZ/2\ZZ \times (\ZZ/2^r \ZZ)^2 = J$;
\item[(b)] the map $f \times h: G \to H \times J$
is injective;
\item[(c)] the map $h$ is \emph{not} injective (or else $KF=F$).
\end{enumerate}
If we write $G$ as a product 
$\prod_i \ZZ/2^{e_i}\ZZ$, then there must be at least three factors 
(since $G \to J$ is surjective) but at most four factors (since $G \to
H \times J$ is injective), and each $e_i$ is at most $r$. In fact,
at least two of the $e_i$ must equal $r$ (lift generators in $J$ of a
subgroup isomorphic to $(\ZZ/2^r\ZZ)^2$ into $G$), and the other two
cannot equal 0 and 1 (otherwise $h$ can't fail to be injective). So
either the other two are both at least 1, in which case $G$ has a quotient
isomorphic to $(\ZZ/2\ZZ)^4$, or one of them is at least 2, in which case
$G$ has a quotient isomorphic to $(\ZZ/4\ZZ)^3$.

5. If $\Gal(K/\QQ_2) = (\ZZ/4\ZZ)^3$, then $K$ must contain
$L = \QQ_2(\sqrt{-1})$ or else $KL/\QQ_2$ contains a subextension isomorphic
to $(\ZZ/2\ZZ)^4$, which we ruled out in the previous exercise. Now
we have $\Gal(K/L) = (\ZZ/2\ZZ) \times (\ZZ/4\ZZ)^2$ sitting inside
$\Gal(K/\QQ_2) = (\ZZ/4\ZZ)^3$; choose a subgroup of $\Gal(K/L)$ isomorphic
to $(\ZZ/4\ZZ)^2$. Its fixed field $M$ satisfies $L \subset M$ and
$\Gal(M/\QQ_2) = \ZZ/4\ZZ$.

Suppose $M = \QQ_2(\sqrt{\alpha})$ for $\alpha = a + b \sqrt{-1} \in L$;
then $\beta = a - b \sqrt{-1}$ must also
have a square root in $M$ since $M/\QQ_2$ is Galois. Thus
$\alpha \beta = a^2+b^2 \in \QQ_2^*$ has a square in $M$.

Choose $r \in \ZZ$ so that $c = a2^r$ and $d = b2^r$
are both in $\ZZ_2$ but not
both in $2\ZZ_2$. If $a^2+b^2$ is not a square in $\QQ_2$, then
$\QQ_2(\sqrt{a^2+b^2}) = \QQ_2(\sqrt{c^2+d^2})$
is a quadratic subextension of $M$. But $M$ has only one quadratic
subextension, namely $L$. Thus $\QQ_2(\sqrt{c^2+d^2}) = \QQ_2(\sqrt{-1})$;
but this cannot be, because $c^2+d^2$ is either even, or congruent to 1
mod 4. 

Thus $a^2+b^2$ is a square in $\QQ_2$.
Now let $\sigma$ be a generator of $\Gal(M/\QQ_2)$. If $\gamma \in M$
is a square root of $\alpha$, then $\gamma^\sigma$ is a square root of
$\alpha^\sigma = \beta$, so $\gamma \gamma^\sigma$ is a square root of
$\alpha \beta$, and so must lie in $\QQ_2$. But $\gamma^{\sigma^2}
= -\gamma$, so $(\gamma \gamma^\sigma)^\sigma = -\gamma \gamma^\sigma$,
which is impossible if $\gamma \gamma^\sigma$ is in $\QQ_2$.
Thus we have a contradiction in all cases.

Alternate argument: write
\[
\sqrt{a \pm b \sqrt{-1}} =  \left(
\sqrt{(a + \sqrt{a^2+b^2})/2} \pm 
\sqrt{(a - \sqrt{a^2+b^2})/2} \right)^2.
\]
If $a^2+b^2$ is the square of $c \in \QQ_2$, that means $L$ contains 
the additional quadratic subfield $\sqrt{(a+c)/2}$, contradiction.
Or argue that $L$ must be totally ramified, otherwise the fixed field of
the inertia group would be an additional quadratic subfield (since
it must be an unramified extension, it can't be $\QQ_2(\sqrt{-1})$),
so $\alpha$ has valuation 1/4 and $\alpha \beta$ has valuation 1/2,
so it's not in $\QQ_2$.

\head{The Hilbert Class Field}

1. There is a subtle point that many of you may not have noticed: after
determining which extensions $K(\sqrt{d})$ for $d \in \ZZ$ are unramified
over $K$, you
have to check that that there are no additional unramified extensions of $K$
of the form $K(\sqrt{\alpha})$ for $\alpha \in \gotho_K \setminus \ZZ$; the
argument is in some ways similar to that given for Local Kronecker-Weber
5 above. If
$K(\sqrt{\alpha})/K$ is unramified, then $(\alpha)$ is the square of a
fractional ideal in $K$. (This follows from Artin reciprocity, but you can
also see it directly: if $(\alpha)$ is divisible by an odd power of some
prime, that prime must ramify in $K(\sqrt{\alpha})$. Or note that
the discriminant of $K(\sqrt{\alpha})/K$ is equal to $(4\alpha)$ divided
by the square of an ideal, by using the basis $1, \sqrt{\alpha}$.)
Let $\beta = \overline{\alpha}$; then $\Norm(\alpha) = \alpha \beta$
is the square of an ideal in $\QQ$, so it is a perfect square.

In particular, $K(\sqrt{\alpha})$ is Galois over $K$, so its Galois
group is either $\ZZ/4\ZZ$ or $(\ZZ/2\ZZ)^2$. If it were the former, we
could choose a generator $\sigma$, apply it to a square root $\gamma$
of $\alpha$ to get a square root of $\beta$, and then
$\gamma \gamma^\sigma$ would be a square root of $\alpha \beta$, thus
a rational number. But $\gamma^{\sigma^2} = -\gamma$ otherwise $\gamma$
only has order 2, so $(\gamma \gamma^\sigma)^\sigma = -\gamma \gamma^\sigma$,
contradiction.

Thus $\Gal(K(\sqrt{\alpha}/\QQ) = \ZZ/2\ZZ \times \ZZ/2\ZZ$, and 
$\Gal(K/\QQ)$ acts on this group. There must be a subgroup which is fixed
by this action, corresponding to a subfield $L$ which is in fact Galois
over $\QQ$, namely $L = \QQ(\sqrt{d})$ for some $d \in \QQ$. But now
$K(\sqrt{\alpha}) = K(\sqrt{d})$, so indeed we don't get any extra
unramified extension by adjoining square roots of non-rationals.

Alternate argument: if $L/K$ is unramified,
pick any prime $p$ of $\QQ$ which ramifies in $K$. Then $p$ is not totally
ramified in $L$, so its inertia group is isomorphic to $\ZZ/2\ZZ$. But
$p$ does not ramify in the fixed field of the inertia group, so that field
is a quadratic field in $L$ other than $K$, forcing $\Gal(L/K) = \ZZ/2\ZZ
\times \ZZ/2\ZZ$, and continue as above.

\end{document}


