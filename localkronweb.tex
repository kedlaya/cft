%\documentclass[12pt]{article}
%\usepackage{amsfonts, amsthm, amsmath}
%
%\setlength{\textwidth}{6.5in}
%\setlength{\oddsidemargin}{0in}
%\setlength{\textheight}{8.5in}
%\setlength{\topmargin}{0in}
%\setlength{\headheight}{0in}
%\setlength{\headsep}{0in}
%\setlength{\parskip}{0pt}
%\setlength{\parindent}{20pt}
%
%\def\CC{\mathbb{C}}
%\def\FF{\mathbb{F}}
%\def\NN{\mathbb{N}}
%\def\PP{\mathbb{P}}
%\def\QQ{\mathbb{Q}}
%\def\RR{\mathbb{R}}
%\def\ZZ{\mathbb{Z}}
%\def\gotha{\mathfrak{a}}
%\def\gothb{\mathfrak{b}}
%\def\gothm{\mathfrak{m}}
%\def\gotho{\mathfrak{o}}
%\def\gothp{\mathfrak{p}}
%\def\gothq{\mathfrak{q}}
%\DeclareMathOperator{\disc}{Disc}
%\DeclareMathOperator{\Gal}{Gal}
%\DeclareMathOperator{\Norm}{Norm}
%\DeclareMathOperator{\Trace}{Trace}
%\DeclareMathOperator{\Cl}{Cl}
%
%\def\head#1{\medskip \noindent \textbf{#1}.}
%\def\fixme#1{\textbf{FIXME! #1}}
%
%\newtheorem{theorem}{Theorem}
%\newtheorem{lemma}[theorem]{Lemma}
%
%\begin{document}
%
%\begin{center}
%\bf
%Math 254B, UC Berkeley, Spring 2002 (Kedlaya) \\
%The Local Kronecker-Weber Theorem
%\end{center}

\head{Reference} Washington, \textit{Introduction to Cyclotomic Fields},
Chapter 14.

\medskip

We now prove the local Kronecker-Weber theorem
(Theorem~\ref{T:local Kronecker-Weber1}), modulo some steps
which will be left as exercises. As shown previously, this will imply
the original Kronecker-Weber theorem.
\begin{theorem}[Local Kronecker-Weber] \label{T:local Kronecker-Weber2}
If $K/\QQ_p$ is a finite abelian extension, then
$K \subseteq \QQ_p(\zeta_n)$ for some positive integer $n$.
\end{theorem}

Since $\Gal(K/\QQ_p)$ decomposes into a product of cyclic groups of 
prime-power order, by the structure theorem for finite abelian groups
we may write $K$ as the compositum of extensions of $\QQ_p$ whose Galois
groups are cyclic of prime-power order. In other words, it suffices to prove
local Kronecker-Weber under the assumption that $\Gal(K/\QQ_p) \cong
\ZZ/q^r \ZZ$ for some prime $q$ and some positive integer $r$.

We first recall the following facts from the theory of local fields
(e.g., see Neukirch II.7).
\begin{lemma} \label{lem:unram}
Let $L/K$ be an unramified extension of finite extensions of $\QQ_p$. 
Then $L = K(\zeta_{q-1})$, where $q$ is the cardinality of the residue
field of $L$.
\end{lemma}
\begin{lemma} \label{lem:tame}
Let $L/K$ be a totally and 
tamely ramified extension of finite extensions of $\QQ_p$ of degree $e$.
(Recall that tamely ramified means that $p$ does not divide $e$.) Then
there exists a generator $\pi$ of the maximal ideal of the valuation ring
of $K$ such that $L = K(\pi^{1/e})$.
\end{lemma}

We also recall one similarly easy but possible less familiar fact, whose proof we leave as an exercise.
\begin{lemma} \label{lem:zetap}
The fields $\QQ_p((-p)^{1/(p-1)})$ and $\QQ_p(\zeta_p)$ are equal.
\end{lemma}

We now proceed to the proof of the local Kronecker-Weber theorem.

\noindent
\textbf{Case 1: $q \neq p$.}

Let $L$ be the maximal unramified subextension of $K$. By
Lemma~\ref{lem:unram}, $L = \QQ_p(\zeta_n)$ for some $n$.
Let $e = [K:L]$. Since $e$ is a power of $q$, $e$ is not divisible by $p$,
so $K$ is totally and tamely ramified over $L$. Thus by Lemma~\ref{lem:tame},
there exists $\pi \in L$ generating the maximal ideal of $\gotho_L$ such 
that $K = L(\pi^{1/e})$.
Since $L/\QQ_p$ is unramified, $p$ also generates the maximal ideal of
$\gotho_L$, so we can write $\pi = -pu$ for some unit $u \in \gotho_L^*$.
Now $L(u^{1/e})/L$ is unramified since $e$ is prime to $p$ and $u$ is a unit.
In particular, $L(u^{1/e})/\QQ_p$ is unramified, hence abelian. Then
$K(u^{1/e})/\QQ_p$ is the compositum of the two abelian extensions
$K/\QQ_p$ and $L(u^{1/e})/\QQ_p$, so it's also abelian. Hence any subextension
is abelian, in particular $\QQ_p((-p)^{1/e})/\QQ_p$.

For $\QQ_p((-p)^{1/e})/\QQ_p$ to be Galois, it must contain the $e$-th roots
of unity (since it must contain all of the $e$-th roots of $-p$, and we
can divide one by another to get an $e$-th root of unity). But
$\QQ_p((-p)^{1/e})/\QQ_p$ is totally ramified, whereas $\QQ_p(\zeta_e)/\QQ_p$
is unramified. This is a contradiction unless $\QQ_p(\zeta_e)$ is actually
equal to $\QQ_p$, which only happens if $e|(p-1)$ (since the residue field
$\FF_p$ of $\QQ_p$ contains only $(p-1)$-st roots of unity).

Now $K \subseteq L((-p)^{1/e}, u^{1/e})$ as noted above. But on one hand, $L(u^{1/e})$
is unramified over $L$, so $L(u^{1/e}) = L(\zeta_m)$ for some $m$; on the
other hand, because $e|(p-1)$, we have
$\QQ_p((-p)^{1/e}) \subseteq \QQ_p((-p)^{1/(p-1)}) =
\QQ_p(\zeta_p)$ by Lemma~\ref{lem:zetap}.
Putting it all together,
\[
K \subseteq L((-p)^{1/e}, u^{1/e}) \subseteq \QQ_p(\zeta_n, \zeta_p, \zeta_m)
\subseteq \QQ_p(\zeta_{mnp}).
\]

\noindent
\textbf{Case 2: $q = p \neq 2$.}

Suppose $\Gal(K/\QQ_p) \cong \ZZ/p^r\ZZ$. We can use roots of unity to
construct two other extensions of $\QQ_p$ with this Galois group. Namely,
$\QQ_{p}(\zeta_{p^{p^r}-1})/\QQ_p$ is unramified of degree $p^r$, and 
automatically has cyclic Galois group; meanwhile, the index $p-1$ subfield
of $\QQ_p(\zeta_{p^{r+1}})$ is totally ramified with Galois group $\ZZ/p^r\ZZ$.
By assumption, $K$ is not contained in the compositum of these two fields,
so for some $s>0$,
\[
\Gal(K(\zeta_{p^{p^r}-1}, \zeta_{p^{r+1}})/\QQ_p) \cong
(\ZZ/p^r\ZZ)^2 \times \ZZ/p^s \ZZ \times \ZZ/(p-1)\ZZ.
\]
This group admits
$(\ZZ/p\ZZ)^3$ as a quotient, so we have an extension of $\QQ_p$ with
Galois group $(\ZZ/p\ZZ)^3$. It thus suffices to prove the following lemma.

\begin{lemma} \label{lem:three}
For $p \neq 2$, there is no extension of $\QQ_p$ with 
Galois group $(\ZZ/p\ZZ)^3$.
\end{lemma}
\begin{proof}
For convenience, put $\pi = \zeta_p - 1$. Then $\pi$ is a uniformizer of
$\QQ_p(\zeta_p)$.

If $\Gal(K/\QQ_p) \cong (\ZZ/p\ZZ)^3$, then
$\Gal(K(\zeta_p)/\QQ_p(\zeta_p)) \cong (\ZZ/p\ZZ)^3$ as well,
and $K(\zeta_p)$ is abelian over $\QQ_p$ with Galois group
$(\ZZ/p\ZZ)^* \times (\ZZ/p\ZZ)^3$. Applying
Kummer theory to $K(\zeta_p)/\QQ_p(\zeta_p)$ produces a subgroup $B
\subseteq \QQ_p(\zeta_p)^*/(\QQ_p(\zeta_p)^*)^p$ isomorphic to
$(\ZZ/p\ZZ)^3$ such that $K(\zeta_p) = \QQ_p(\zeta_p, B^{1/p})$.
Let $\omega: \Gal(\QQ_p(\zeta_p)/\QQ_p) \to (\ZZ/p\ZZ)^*$ be the
canonical map; since $\QQ_p(\zeta_p, b^{1/p}) \subseteq K(\zeta_p)$ is also
abelian over $\QQ_p$, by Lemma~\ref{L:Kummer Galois criterion},
\[
b^g/b^{\omega(g)} \in (\QQ_p(\zeta_p)^*)^p \qquad
(\forall b \in B, g \in \Gal(\QQ_p(\zeta_p)/\QQ_p)).
\]
Recall the structure of $\QQ_p(\zeta_p)^*$: the maximal ideal of
$\ZZ_p[\zeta_p]$ is generated by $\pi$, while
each unit of $\ZZ_p[\zeta_p]$ is congruent to a $(p-1)$-st root of unity modulo
$\pi$, and so
\[
\QQ_p(\zeta_p)^* = \pi^\ZZ \times (\zeta_{p-1})^\ZZ \times U_1,
\]
where $U_1$ denotes the set of units of $\ZZ_p[\zeta_p]$ congruent to
1 modulo $\pi$. Correspondingly,
\[
(\QQ_p(\zeta_p)^*)^p = \pi^{p\ZZ} \times (\zeta_{p-1})^{p\ZZ}
\times U_1^p.
\]
Now choose a representative $a \in L^*$ of some nonzero
element of $B$; without
loss of generality, we may assume $a = \pi^m u$ for some
$m \in \ZZ$ and $u \in U_1$. Then
\[
\frac{a^g}{a^{\omega(g)}}
= \frac{(\zeta_p^{\omega(g)}-1)^m}{\pi^{m\omega(g)}} \frac{u^g}{u^{\omega(g)}};
\]
but $v_\pi(\pi) = v_\pi(\zeta_p^{\omega(g)}-1) = 1$. Thus 
the valuation of the right hand side is $m(1-\omega(g))$, which can only
be a multiple of $p$ for all $g$ if $m \equiv 0 \pmod{p}$. (Notice we
just used that $p$ is odd!) That is,
we could have taken $m=0$ and $a = u \in U_1$.

As for $u^g/u^{\omega(g)}$, note that $U_1^p$ is precisely the set of units
congruent to 1 modulo $\pi^{p+1}$ (see exercises).
Since $\zeta_p = 1 + \pi + O(\pi^2)$, we can write
$u = \zeta_p^b(1 + c\pi^d + O(\pi^{d+1}))$, with $c \in \ZZ$
and $d \geq 2$. Since $\pi^g/\pi \equiv \omega(g) \pmod{\pi}$, we get
\[
u^g = \zeta_p^{b\omega(g)} (1 + c \omega(g)^d \pi^d + O(\pi^{d+1})),
\quad
u^{\omega(g)} = \zeta_p^{b\omega(g)} (1 + c \omega(g) \pi^d + O(\pi^{d+1})).
\]
But these two have to be congruent modulo $\pi^{p+1}$. Thus either
$d \geq p+1$ or $d \equiv 1 \pmod{p-1}$, the latter only occurring for
$d=p$.

What this means is that the set of possible $u$ is generated by
$\zeta_p$ and by $1 + \pi^p$. But these only generate a subgroup of
$U_1/U_1^p$ isomorphic to $(\ZZ/p\ZZ)^2$, whereas $B \cong (\ZZ/p\ZZ)^3$.
Contradiction.
\end{proof}

\noindent
\textbf{Case 3: $p=q=2$.}
This case is similar to the previous case, but a bit messier, because
$\QQ_2$ does admit an extension with Galois group $(\ZZ/2\ZZ)^3$.
We defer this case to the exercises.

\head{Exercises}

\begin{enumerate}
\item
Prove Lemma~\ref{lem:zetap}. (Hint: prove that $(\zeta_p-1)^{p-1}/p - 1$
belongs to the maximal ideal of $\ZZ_p[\zeta_p]$.)
\item
Prove that (in the notation of Lemma~\ref{lem:three})
$U_1^p$ is the set of units congruent to 1 modulo $\pi^{p+1}$.
(Hint: in one direction, write $u \in U_1$ as a power of $\zeta_p$ times
a unit congruent to 1 modulo $\pi^2$. In the other direction,
use the binomial series for $(1+x)^{1/p}$.)
\item
Prove that for any $r>0$, there is an extension of $\QQ_2$ with Galois group
$\ZZ/2\ZZ \times (\ZZ/2^r\ZZ)^2$ contained in $\QQ_2(\zeta_n)$ for some
$n>0$.
\item
Suppose that $K/\QQ_2$ is a $\ZZ/2^r\ZZ$-extension not contained in
$\QQ_2(\zeta_n)$ for any $n>0$. Prove that there exists
an extension of $\QQ_2$ with Galois group $(\ZZ/2\ZZ)^4$ or $(\ZZ/4\ZZ)^3$.
\item
Prove that there is no extension of $\QQ_2$ with Galois group $(\ZZ/2\ZZ)^4$.
(Hint: use Kummer theory.)
\item
Prove that there is no extension of $\QQ_2$ with Galois group $(\ZZ/4\ZZ)^3$.
(Hint: reduce to showing that there exists no extension of $\QQ_2$ containing
$\QQ_2(\sqrt{-1})$ with Galois group $\ZZ/4\ZZ$.)
\end{enumerate}

%\end{document}


