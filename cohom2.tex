%\documentclass[12pt]{article}
%\usepackage{amsfonts, amsthm, amsmath}
%\usepackage[all]{xy}
%
%\setlength{\textwidth}{6.5in}
%\setlength{\oddsidemargin}{0in}
%\setlength{\textheight}{8.5in}
%\setlength{\topmargin}{0in}
%\setlength{\headheight}{0in}
%\setlength{\headsep}{0in}
%\setlength{\parskip}{0pt}
%\setlength{\parindent}{20pt}
%
%\def\AA{\mathbb{A}}
%\def\CC{\mathbb{C}}
%\def\FF{\mathbb{F}}
%\def\PP{\mathbb{P}}
%\def\QQ{\mathbb{Q}}
%\def\RR{\mathbb{R}}
%\def\ZZ{\mathbb{Z}}
%\def\gotha{\mathfrak{a}}
%\def\gothb{\mathfrak{b}}
%\def\gothm{\mathfrak{m}}
%\def\gotho{\mathfrak{o}}
%\def\gothp{\mathfrak{p}}
%\def\gothq{\mathfrak{q}}
%\def\gothr{\mathfrak{r}}
%\DeclareMathOperator{\ab}{ab}
%\DeclareMathOperator{\coker}{coker}
%\DeclareMathOperator{\disc}{Disc}
%\DeclareMathOperator{\Frob}{Frob}
%\DeclareMathOperator{\Gal}{Gal}
%\DeclareMathOperator{\GL}{GL}
%\DeclareMathOperator{\Hom}{Hom}
%\DeclareMathOperator{\id}{id}
%\DeclareMathOperator{\im}{im}
%\DeclareMathOperator{\Ind}{Ind}
%\DeclareMathOperator{\Norm}{Norm}
%\DeclareMathOperator{\Trace}{Trace}
%\DeclareMathOperator{\Cl}{Cl}
%
%\def\head#1{\medskip \noindent \textbf{#1}.}
%
%\newtheorem{theorem}{Theorem}
%\newtheorem{lemma}[theorem]{Lemma}
%\newtheorem{cor}[theorem]{Corollary}
%
%\begin{document}
%
%\begin{center}
%\bf
%Math 254B, UC Berkeley, Spring 2002 (Kedlaya) \\
%The Cohomology of Finite Groups II: Concrete Nonsense
%\end{center}

\head{Reference} Milne, II.1.

\medskip
In the previous chapter, we associated to a finite group $G$ and a (right) $G$-module
$M$ a sequence of abelian groups $H^i(G, M)$, called the cohomology
groups of $M$. (They're also called the \emph{Galois cohomology} groups because in
number theory, $G$ will invariably be the Galois group of some extension
of number fields, and $A$ will be some object manufactured from this
extension.) What we didn't do is make the construction at all usable in
practice! This time we will remedy this.

Recall the last point (and the last exercise) from the last chapter: if 
\[
0 \to M \to M_0 \to M_1 \to \cdots
\]
is an acyclic resolution of $M$ (i.e., the sequence is exact, and 
$H^i(G, M_j) = 0$ for $i > 0$ and all $j$), then
\[
H^i(G, M) = \ker(M_i^G \to M_{i+1}^G)/\im(M_{i-1}^G \to M_i^G).
\]
Thus to compute cohomology, we are going to need an ample supply of
acyclic $G$-modules. We will get these using a process known as \emph{induction}.
By way of motivation, we note first that if $G$ is the trivial group, 
\emph{every} $G$-module is acyclic: if $0 \to M \to I_0 \to I_1 \cdots$ is
an injective resolution, taking $G$-invariants has no effect, so
$0 \to I_0 \to I_1 \to \cdots$ is still exact except at $I_0$
(where we omitted $M$).

If $H$ is a subgroup of $G$ and $M$ is an $H$-module, 
we define the \emph{induced} $G$-module associated to $M$ to be 
$\Ind^G_H M = M \otimes_{\ZZ[H]} \ZZ[G]$. We may also identify $\Ind^G_H M$ with the set of functions $\phi: G \to M$ such that $\phi(gh)
= \phi(g)^h$ for $h \in H$, with the $G$-action on the latter being given by 
$\phi^g(g') = \phi(gg')$: namely, the element $m \otimes [g] \in M \otimes_{\ZZ[H]} \ZZ[G]$ corresponds to the function $\phi_{m,g}$ taking $g'$ to $m^{gg'}$ if $gg' \in H$ and to 0 otherwise.

\begin{lemma}[Shapiro's lemma] \label{L:Shapiro}
  If $H$ is a subgroup of $G$ and $N$ is an $H$-module, then
there is a canonical isomorphism $H^i(G, \Ind^G_H N) \to H^i(H, N)$.
In particular, $N$ is an acyclic $H$-module if and only if
$\Ind^G_H(N)$ is an acyclic $G$-module.
\end{lemma}
\begin{proof}
The key points are:
\begin{enumerate}
\item[(a)] $(\Ind^G_H N)^G = N^H$, so there is an isomorphism for $i=0$
(this is clearest from the description using functions);
\item[(b)] the functor $\Ind^G_H$ from $H$-modules to $G$-modules
is exact (that is, $\ZZ[G]$ is flat over $\ZZ[H]$, which is easy to see
because it in fact is free over $\ZZ[H]$);
\item[(c)] if $I$ is an injective $H$-module, then $\Ind^G_H(I)$ is an
injective $G$-module. This follows from the existence of a  canonical
isomorphism $\Hom_G(M, \Ind^G_H I) = \Hom_H(M, I)$, for which see Proposition~\ref{P:adjoint property} below.
\end{enumerate}
Now take an injective resolution of $N$, apply $\Ind^G_H$ to it, and the
result is an injective resolution of $\Ind^G_H N$.
\end{proof}
We say $M$ is an \emph{induced} $G$-module if it has the form $\Ind^G_{1}
N$
for some abelian group $N$, i.e.,
it can be written as $N \otimes_\ZZ \ZZ[G]$. (The subscript 1 stands for the trivial group, since $G$-modules for $G = 1$ are just abelian groups.)


\begin{cor} \label{C:induced acyclic}
If $M$ is an induced $G$-module, then $M$ is acyclic.
\end{cor}

To complete the previous argument, we need an important property of induced modules.
\begin{prop} \label{P:adjoint property}
Let $H$ be a subgroup of $G$, let $M$ be a $G$-module, and let $N$ be an $H$-module. Then there are natural isomorphisms
\begin{align*}
\Hom_G(M, \Ind^G_H N) &\cong \Hom_H(M, N) \\
\Hom_G(\Ind^G_H N, M) &\cong \Hom_H(N,M).
\end{align*}
\end{prop}
In other words, the restriction functor from $G$-modules to $H$-modules and the induction functor from $H$-modules to $G$-modules form a pair of \emph{adjoint functors} in both directions. This is rather unusual; it is far more common to have such a relationship in only one direction.
\begin{proof}
To begin with, note that if we take $N = M$ (or more precisely, $N$ is a copy of $M$ with only the action of $H$ retained), then the identity map between $M$ and $N$ is supposed to correspond both to a homomorphism $M \to \Ind^G_H M$ and to a homomorphism $\Ind^G_H M \to M$. Let us write these maps down first: the map $\Ind^G_H M \to M$ is
\[
\sum_{g \in G} m_g \otimes [g] \mapsto \sum_{g \in G} (m_g)^g,
\]
while the map $M \to \Ind^G_H M$ is
\[
m \mapsto \sum_i m^{g_i} \otimes [g_i^{-1}]
\]
where $g_i$ runs over a set of left coset representatives of $H$ in $G$. Note that this second map doesn't depend on the choice of the representatives; for $g \in G$, we can use the coset representatives $gg_i$ instead, so the equality
\[
m^{g} \mapsto \sum_{i} m^{gg_i} \otimes [g_i^{-1}]
= \left( \sum_{i} m^{gg_i} \otimes [(g g_i)^{-1}] \right)[g]
\]
means that we do in fact get a map compatible with the $G$-actions.
(Note that the composition of these two maps is not the identity! For more on this point, see the discussion of extended functoriality in Chapter~\ref{chap:homology}.)

Now let $N$ be general. Given a homomorphism $M \to N$ of $H$-modules, we get a corresponding homomorphism
$\Ind^G_H M \to \Ind^G_H N$ of $G$-modules, which we can then compose with the above map $M \to \Ind^G_H M$ to get a homomorphism $M \to \Ind^G_H N$ of $G$-modules. We thus get a map
\[
\Hom_H(M,N) \to \Hom_G(M, \Ind^G_H N);
\]
to get the map in the other direction, start with a homomorphism $M \to \Ind^G_H N$, identify the target with functions $\phi: G \to N$, then compose with the map $\Ind^G_H N \to N$ taking $\phi$ to $\phi(e)$.

In the other direction, given a homomorphism $N \to M$ of $H$-modules, we get a corresponding homomorphism
$\Ind^G_H N \to \Ind^G_H M$ of $G$-modules, which we can then compose with the above map $\Ind^G_H M \to M$ to get a homomorphism $\Ind^G_H N \to M$ of $G$-modules. 
We thus get a map
\[
\Hom_H(N,M) \to \Hom_G(\Ind^G_H N,M);
\]
to get the map in the other direction, start with a homomorphism $\Ind^G_H N \to M$ of $G$-modules and evaluate it on $n \otimes [e]$ to get a homomorphism $N \to M$ of $H$-modules.
\end{proof}

The point of all of this is that it is much easier to embed $M$ into an acyclic $G$-module than into an injective $G$-module: use the map $M \to \Ind^G_1 M$ constructed in
Proposition~\ref{P:adjoint property}!
Immediate consequence: if $M$ is finite, it can be embedded into a
finite acyclic $G$-module, and thus $H^i(G,M)$ is finite for all $i$. (But
contrary to what you might expect, for fixed $M$, the groups $H^i(G,M)$
do not necessarily become zero for $i$ large, even if $M$ is finite! 
We'll see explicit examples next time.)

Another consequence is the following result. (The case $i=1$ was an exercise earlier.)
\begin{theorem} \label{T:additive theorem 90}
Let $L/K$ be a finite Galois extension of fields. Then
\[
H^i(\Gal(L/K), L) = 0 \mbox{ for all $i>0$.}
\]
\end{theorem}
\begin{proof}
Put $G = \Gal(L/K)$.
The normal basis theorem (see Lang, \emph{Algebra} or Milne, Lemma
II.1.24)
states that there exists $\alpha \in L$ whose conjugates form a basis of
$L$ as a $K$-vector space. This implies that $L \cong \Ind^{G}_{1}
K$, so $L$ is an induced $G$-module and so is acyclic.
\end{proof}

Now let's see an explicit way to compute group cohomology.
Given a group $G$ and a $G$-module $M$, define the $G$-modules $N_i$
for $i \geq 0$ as the set of functions $\phi: G^{i+1} \to M$, with the $G$-action
\[
(\phi^g)(g_0, \dots, g_i) = \phi(g_0g^{-1}, \dots, g_ig^{-1})^g.
\]
Notice that this module is induced: we have $N_i = \Ind_{1}^G N_{i,0}$
where $N_{i,0}$ is the subset of $N_i$ consisting of functions for which
$\phi(g_0, \dots, g_i) = 0$ when $g_0 \neq e$.

Define the map $d_i: N_i \to N_{i+1}$ by
\[
(d_i \phi)(g_0, \dots, g_{i+1})
= \sum_{j=0}^{i+1} (-1)^j \phi(g_0, \dots, \widehat{g_j}, \dots, g_{i+1}),
\]
where the hat over $g_j$ means you omit it from the list.
Then one checks that the sequence
\[
0 \to M \to N_0 \to N_1 \to \dots
\]
is exact. Since the $N_i$ are induced, this is an acyclic resolution:
thus the cohomology of the complex
\[
0 \to N_0^G \to N_1^G \to \cdots
\]
coincides with the cohomology groups $H^i(G, M)$. And now we have
something we can actually compute! (Terminology: the elements of
$N_i^G$ in the kernel of $d_i$ are called \emph{(homogeneous) $i$-cochains};
the ones in the image of $d_{i-1}$ are called \emph{$i$-coboundaries}.)

\head{Fun with $H^1$}

For example, we can give a very simple description of $H^1(G,M)$.
Namely, a 1-cochain $\phi: G^2 \to M$ is determined by $\rho(g) = \phi(e, g)$,
which by $G$-invariance satisfies the relation
\begin{align*}
0 &= (d_1\phi)(e, h, gh) \\
&= \phi(h, gh) - \phi(e, gh) + \phi(e, h) \\
&= (\phi^h)(h,gh) - \rho(gh) + \rho(h) \\
&= \phi(e, g)^h - \rho(gh) + \rho(h) \\
&= \rho(g)^h + \rho(h) - \rho(gh).
\end{align*}
It is the coboundary of a 0-cochain $\psi: G \to M$ if and only if
\[
\rho(g) = \phi(e,g) = \psi(g) - \psi(e) = \psi(e)^g - \psi(e).
\]
That is, $H^1(G,M)$ consists of crossed homomorphisms modulo principal
crossed homomorphisms, consistent with the definition
we gave in Chapter~\ref{chap:Kummer theory}. 

We may also interpret $H^1(G,M)$ as the set of isomorphism classes of \emph{principal homogeneous spaces} of $M$.
Such objects are sets $A$ with both a $G$-action and an $M$-action, subject to the
following restrictions:
\begin{enumerate}
\item[(a)] for any $a \in A$, the map $M \to A$ given by $m \mapsto m(a)$ is a bijection;
\item[(b)] for $a \in A$, $g \in G$ and $m \in M$, $m(a)^g = m^g(a)$ 
(i.e., the $G$-action and $M$-action commute).
\end{enumerate}
To define the associated class in $H^1(G,M)$, pick any $a \in A$, 
take the map $\rho: G \to M$ given by $\rho(g) = a^g - a$, and
let $\phi$ be the 1-cocycle with $\phi(e, g) = \rho(g)$.
The verification that this defines a bijection is left to the reader.
(For example, the identity in $H^1(G,M)$ corresponds to the trivial 
principal homogeneous space $A=M$, on which $G$ acts as it does on $M$ while $M$ acts by 
translation: $m(a) = m+a$.)

This interpretation of $H^1$ appears prominently in the theory of elliptic curves:
For example, if $L$ is a finite extension of $K$ and $E$ is an elliptic curve over
$E$, then $H^1(\Gal(L/K), E(\overline{K}))$ is the set of $K$-isomorphism
classes of curves whose Jacobians are $K$-isomorphic to $E$
(but which might not themselves be isomorphic to $E$ by virtue of not
having a $K$-rational point). 
For another example,  $H^1(\Gal(L/K), \Aut(E))$ parametrizes
twists of $E$, elliptic curves defined over $K$ which are $L$-isomorphic
to $E$. (E.g., $y^2 = x^3+x+1$ versus $2y^2 = x^3 + x +1$, with $L = \QQ(\sqrt{2})$.) See Silverman,
\textit{The Arithmetic of Elliptic Curves}, especially Chapter X,
for all this and more fun with $H^1$, including the infamous \emph{Selmer group}
and \emph{Tate-Shafarevich group}.

\head{Fun with $H^2$}

We can also give an explicit interpretation of $H^2(G,M)$ (see
Milne, example II.1.18(b)). It classifies
short exact sequences
\[
1 \to M \to E \to G \to 1
\]
of (not necessarily abelian) groups on which $G$ has a fixed action on $M$.
(The action is given as follows: given $g \in G$ and $m \in M$, choose $h
\in E$ lifting $G$; then $h^{-1}mh$ maps to the identity in $G$, so comes
from $M$, and we call it $m^g$ since it depends only on $g$.)
Namely, given the sequence, choose a map $s: G \to E$ (not a homomorphism)
such that $s(g)$ maps to $g$ under the map $E \to G$. Then the map
$\phi: G^3 \to M$ given by
\[
\phi(a,b,c) = s(a)^{-1} s(ba^{-1})^{-1} s(cb^{-1})^{-1} s(ca^{-1}) s(a)
\]
is a homogeneous 2-cocycle, and any two choices of $s$ give maps that differ by
a 2-coboundary.

What ``classifies'' means here is that two sequences give the same
element of $H^2(G,M)$ if and only if one can find an arrow $E \to E'$ making
the following diagram commute:
\[
\xymatrix{
1 \ar[r] & M \ar^{\id}[d] \ar[r] & E \ar[d] \ar[r] & G \ar^{\id}[d] \ar[r] &
1 \\
1 \ar[r] & M \ar[r] & E' \ar[r] & G \ar[r] & 1
}
\]
Note that two sequences may not be isomorphic under this definition even
if $E$ and $E'$ are abstractly isomorphic as groups. For example, if
$G = M = \ZZ/p\ZZ$ and the action is trivial, then $H^2(G, M) = \ZZ/p\ZZ$
even though there are only two possible groups $E$, namely $\ZZ/p^2\ZZ$
and $\ZZ/p\ZZ \times \ZZ/p\ZZ$.

\head{Extended functoriality}

We already saw that if we have a homomorphism of $G$-modules, we get
induced homomorphisms on cohomology groups.
 But what if we want to relate $G$-modules for different groups $G$,
as will happen in our study of class field theory? It turns out that in
a suitable sense, the cohomology groups are also functorial with respect to changing $G$.

Let $M$ be a $G$-module and $M'$ a $G'$-module. Suppose we are given
a homomorphism $\alpha: G' \to G$ of groups and a homomorphism
$\beta: M \to M'$ of abelian groups (note that they go in opposite
directions!). We say these are \emph{compatible} if
$\beta(m^{\alpha(g)}) = \beta(m)^g$ for all $g \in G$ and $m \in M$.
In this case, one gets canonical homomorphisms $H^i(G, M) \to H^i(G', M')$
(construct them on pairs of injective
resolutions, then show that any two choices are homotopic).

The principal examples are as follows.
\begin{enumerate}
\item[(a)]
Note that cohomology groups don't
seem to carry a nontrivial $G$-action, because you compute them by taking
invariants. This can be reinterpreted in terms of
extended functoriality: let $\alpha: G \to G$ be the conjugation by some
fixed $h$: $g \mapsto h^{-1}gh$, and let $\beta: M \to M$ be the map
$m \mapsto m^h$. Then the induced homomorphisms $H^i(G,M) \to H^i(G,M)$ are
all the identity map.

\item[(b)] If $H$ is a subgroup of $G$, $M$ is a $G$-module, and
$M'$ is just $M$ with all but the $H$-action forgotten, we get
the \emph{restriction homomorphisms}
\[
\Res: H^i(G, M) \to H^i(H, M).
\]
Another way to get the same map: 
use the adjunction homomorphism
$M \to \Ind^G_H M$ from Proposition~\ref{P:adjoint property} sending $m$ to $\sum_i m^{g_i} \otimes [g_i^{-1}]$, where
$g_i$ runs over a set of right coset representatives of $H$ in $G$,
then apply Shapiro's Lemma to get
\[
H^i(G, M) \to H^i(G, \Ind^G_H M) \stackrel{\sim}{\to} H^i(H, M).
\]
\item[(c)]
Take notation as in (a), but this time 
consider the map $\Ind^G_H M \to M$ taking $m \otimes [g]$ to $m^g$.
We then get maps
$H^i(G, \Ind^G_H M) \to H^i(G,M)$ which, together with the isomorphisms
of Shapiro's lemma, give what are called the \emph{corestriction
homomorphisms}:
\[
\Cor: H^i(H, M) \stackrel{\sim}{\to} H^i(G, \Ind^G_H M) \to H^i(G, M).
\]
\item[(d)]
The composition $\Cor \circ \Res$ is induced by the homomorphism of
$G$-modules $M \to \Ind^G_H M \to M$ given by
\[
m \mapsto \sum_i m^{g_i} \otimes [g_i^{-1}] \to \sum_i m = [G:H]m.
\]
Thus $\Cor \circ \Res$ acts as multiplication by $[G:H]$ on each 
(co)homology group. Bonus consequence (hereafter excluding the case of $H^0$):
if we take $H$ to be the trivial group,
then the group in the middle is isomorphic to $H^i(H, M) = 0$.
So every cohomology group for $G$ is killed by $\#G$, and in particular
is a torsion group. In fact, if $M$ is finitely generated as an abelian
group, this means
$H^i(G, M)$ is always finite, because each of these will be finitely generated
and torsion. (Of course, this won't happen in many of
our favorite examples, e.g., $H^i(\Gal(L/K), L^*)$ for $L$ and $K$ fields.)
\item[(e)]
If $H$ is a \emph{normal} subgroup of $G$, let $\alpha$ be the surjection
$G \to G/H$, and let $\beta$ be the injection $M^H \hookrightarrow M$.
Note that $G/H$ acts on $M^H$; in this case, we get the \emph{inflation
homomorphisms}
\[
\Inf: H^i(G/H, M^H) \to H^i(G, M).
\]
The inflation and restriction maps will interact in an interesting way; see
Proposition~\ref{P:inflation restriction}.
\end{enumerate}


\head{Exercises}

\begin{enumerate}
\item
Complete the proof of the correspondence between $H^1(G,M)$ and principal homogeneous spaces.
\item
The set $H^2(G,M)$ has the structure of an abelian group. Describe the
corresponding structure on short exact sequences $0 \to M \to E \to G \to 0$.
\item
Let $G = S_3$ (the symmetric group on three letters), let $M = \ZZ^3$
with the natural $G$-action permuting the factors, and let $N = M^G$.
Compute $H^i(G, M/N)$ for $i=1,2$ however you want: you can explicitly
compute cochains, use the alternate interpretations given above,
or use the exact sequence $0 \to N \to M \to M/N \to 0$. Better yet, use
more than one method and make sure that you get the same answer.
\item
(Artin-Schreier)
Let $L/K$ be a $\ZZ/p\ZZ$-extension of fields of characteristic $p>0$.
Prove that $L = K(\alpha)$ for some $\alpha$ such that
$\alpha^p - \alpha \in K$. (Hint: let $K^{\sep}$ be a separable closure of $K$ containing $L$, and consider the short exact sequence
$0 \to \FF_p \to K^{\sep} \to K^{\sep} \to 0$ in which the map $K^{\sep} \to K^{\sep}$
is given by $x \mapsto x^p - x$.)
%\item
%Check the \emph{adjoint property} of induction: for $G$ a finite group, $H$ a subgroup, $M$ a $G$-module, and $N$ an $H$-module, there is a canonical bijection 
%\[
%\Hom_H(N, M) \cong \Hom_G(\Ind^G_H N, M),
%\]
%where a map $f: N \to M$ corresponds to the map $\Ind^G_H N \to M$ taking $n \otimes [g]$ to $f(n)^g$. In particular, by taking $N = M$ we get a map $\Ind^G_H M \to M$; by contrast, the proof of Corollary~\ref{C:induced acyclic} gives a map in the other direction.
%We'll use both directions later; see the discussion of extended functoriality in 
%Chapter~\ref{chap:homology}.
\end{enumerate}

%\end{document}


