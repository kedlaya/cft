%\documentclass[12pt]{article}
%\usepackage{amsfonts, amsthm, amsmath}
%\usepackage[all]{xy}
%
%\setlength{\textwidth}{6.5in}
%\setlength{\oddsidemargin}{0in}
%\setlength{\textheight}{8.5in}
%\setlength{\topmargin}{0in}
%\setlength{\headheight}{0in}
%\setlength{\headsep}{0in}
%\setlength{\parskip}{0pt}
%\setlength{\parindent}{20pt}
%
%\def\AA{\mathbb{A}}
%\def\CC{\mathbb{C}}
%\def\FF{\mathbb{F}}
%\def\PP{\mathbb{P}}
%\def\QQ{\mathbb{Q}}
%\def\RR{\mathbb{R}}
%\def\ZZ{\mathbb{Z}}
%\def\gotha{\mathfrak{a}}
%\def\gothb{\mathfrak{b}}
%\def\gothm{\mathfrak{m}}
%\def\gotho{\mathfrak{o}}
%\def\gothp{\mathfrak{p}}
%\def\gothq{\mathfrak{q}}
%\DeclareMathOperator{\Cor}{Cor}
%\DeclareMathOperator{\disc}{Disc}
%\DeclareMathOperator{\Gal}{Gal}
%\DeclareMathOperator{\GL}{GL}
%\DeclareMathOperator{\Hom}{Hom}
%\DeclareMathOperator{\Ind}{Ind}
%\DeclareMathOperator{\Norm}{Norm}
%\DeclareMathOperator{\Res}{Res}
%\DeclareMathOperator{\smcy}{smcy}
%\DeclareMathOperator{\Trace}{Trace}
%\DeclareMathOperator{\Cl}{Cl}
%
%\def\head#1{\medskip \noindent \textbf{#1}.}
%
%\newtheorem{theorem}{Theorem}
%\newtheorem{lemma}[theorem]{Lemma}
%\newtheorem{prop}[theorem]{Proposition}
%
%\begin{document}
%
%\begin{center}
%\bf
%Math 254B, UC Berkeley, Spring 2002 (Kedlaya) \\
%The Existence Theorem \\
%\end{center}

\head{Reference} 
Milne VII.9, Neukirch VI.6.

\medskip
With the ``abstract'' reciprocity theorem in hand,
we now prove the Existence Theorem, that every generalized
ideal class group of a number field
is identified by Artin reciprocity with the Galois group of a 
suitable abelian extension. Recall that the idelic formulation of this statement is as follows
(see Theorem~\ref{T:adelic existence theorem1}).
\begin{theorem} \label{T:adelic existence theorem2}
For $K$ a number field, the finite abelian extensions $L/K$ correspond
one-to-one with the open subgroups of $C_K$ of finite index, via the
map $L \mapsto \Norm_{L/K} C_L$.
\end{theorem}

The proof of this result is very similar to the proof we gave in 
the local case. For example, the reciprocity law immediately lets us reduce
to the following proposition.
\begin{prop}
Every open subgroup $U$ of $C_K$ of finite index contains 
$\Norm_{L/K} C_L$ for some finite extension $L$ of $K$.
\end{prop}
The proof of this proposition again uses Kummer theory, but more in the spirit of the algebraic proof of the Second Inequality.
\begin{proof}
We first prove this proposition in case $U$ has prime index $p$.
Let $J$ be the preimage of $U$ under the projection
$I_K \to C_K$, so that $J$ is open in $I_K$ of finite index.
Then $J$ contains a subgroup of the form
\[
V = \prod_{v \in S} \{1\} \times \prod_{v \notin S} \gotho_{K_v}^*
\]
for some set $S$ of places of $K$ containing
the infinite places and all places dividing $(p)$, which we may choose
large enough so that $I_{K,S} K^* = I_K$. Let $K_S = K^* \cap I_{K,S}$
be the group of $S$-units of $K$.

The group $J$ must also contain
$I_K^p$, so in particular contains
\[
W_S = \prod_{v \in S} (K_v^*)^p \times \prod_{v \notin S} U_v.
\]
Put $C_S = W_S K^*/K^*$; then $C_S \subseteq U$, so it suffices to show that
$C_S$ contains a norm subgroup.

If $K$ contains a primitive $p$-th root of unity, then an argument
as in the algebraic proof of the Second Inequality gives
$C_{S} = \Norm_{L/K} C_L$ for $L = K(K_S^{1/p})$.
Namely, one first computes that $\#C_K/C_S = p^{\#S} = [L:K]$
as in that proof, by reading orders off of the short exact sequence
\[
1 \to K_S / (W_S \cap K_S) \to I_{K,S}/{W_S} \to C_K /C_S \to 1:
\]
on one hand, we have $W_S \cap K_S = K_S^p$ (as in the proof of Lemma~\ref{L:adelic intersection}),
which gives $\#K_S/(W_S \cap K_S) = p^{\#S}$; on the other hand,
$I_{K,S}/W_S$ is the product of $\#S$ quotients of the form
$K_v^*/(K_v^*)^p$, each of which has order $p^2$ (generated by a uniformizer
and a $p$-th root of unity, since $p$ is prime to the residue characteristic).

One then checks that $W_S \subseteq \Norm_{L/K} I_L$ by checking this
place by place; the places not in $S$ are straightforward (they don't
ramify in $L$, so local units are local norms), and the
ones in $S$ follow from the fact that for any local field $M$
containing a $p$-th root of unity,
if $N = M((M^*)^{1/p})$, then
\[
\Norm_{N/M} N^* = (M^*)^p,
\]
which we proved in the course of proving the local existence theorem
(see Lemma~\ref{L:hilbert symbol}).
Putting this all together, we have
$C' \subseteq \Norm_{L/K} C_L$ and these two groups have the same index
$[L:K]$ in $C_K$ by the First and Second Inequalities
(Theorem~\ref{T:first inequality}, Theorem~\ref{T:first and second inequality}).

We next drop the restriction that $K$ contains a $p$-th root of unity
by reducing to the previous case. Namely, put $K' = K(\zeta_p)$.
For a choice of $S$ as above, let $S'$ be the set of places of $K'$ above $S$;
we can make $S$ large enough so that $I_{K',S'} (K')^* = I_{K'}$.
Then as above, $C_{S'} = \Norm_{L'/K'} C_{L'}$ if $L'$ is the extension of
$K'$ obtained by adjoining all $p$-th roots. Also as above,
$\Norm_{K'/K} W_{S'} \subseteq W_S$, so
\[
\Norm_{L'/K} C_{L'} = \Norm_{K'/K} (\Norm_{L'/K'} C_{L'})
= \Norm_{K'/K} C_{S'} \subseteq C_S \subseteq U.
\]

Finally, we handle the case where $U$ has arbitrary index, by induction
on that index using the above result as the base case. If $\#C_K/U$
is not prime, choose an intermediate subgroup $V$ between $U$ and $C_K$.
By the induction hypothesis, $V$ contains $N = \Norm_{L/K} C_L$ for some 
finite extension $L$ of $K$. Then 
\[
\#N / (U \cap N) =
\#UN / U \leq \#V/U.
\]
Let $W$ be the subgroup of $C_L$ consisting of those
$x$ whose norms lie in $U$. Then
\[
\#C_L/W \leq \# N/(U \cap N) \leq \#V/U,
\]
so by the induction hypothesis, $W$ contains $\Norm_{M/L} C_M$ for some
finite extension $M/L$. Thus $U$ contains $\Norm_{M/K} C_M$, as desired.  
\end{proof}

%\end{document}
