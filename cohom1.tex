%\documentclass[12pt]{article}
%\usepackage{amsfonts, amsthm, amsmath}
%\usepackage[all]{xy}
%
%\setlength{\textwidth}{6.5in}
%\setlength{\oddsidemargin}{0in}
%\setlength{\textheight}{8.5in}
%\setlength{\topmargin}{0in}
%\setlength{\headheight}{0in}
%\setlength{\headsep}{0in}
%\setlength{\parskip}{0pt}
%\setlength{\parindent}{20pt}
%
%\def\AA{\mathbb{A}}
%\def\CC{\mathbb{C}}
%\def\FF{\mathbb{F}}
%\def\PP{\mathbb{P}}
%\def\QQ{\mathbb{Q}}
%\def\RR{\mathbb{R}}
%\def\ZZ{\mathbb{Z}}
%\def\gotha{\mathfrak{a}}
%\def\gothb{\mathfrak{b}}
%\def\gothm{\mathfrak{m}}
%\def\gotho{\mathfrak{o}}
%\def\gothp{\mathfrak{p}}
%\def\gothq{\mathfrak{q}}
%\def\gothr{\mathfrak{r}}
%\DeclareMathOperator{\ab}{ab}
%\DeclareMathOperator{\coker}{coker}
%\DeclareMathOperator{\disc}{Disc}
%\DeclareMathOperator{\Frob}{Frob}
%\DeclareMathOperator{\Gal}{Gal}
%\DeclareMathOperator{\GL}{GL}
%\DeclareMathOperator{\Hom}{Hom}
%\DeclareMathOperator{\im}{im}
%\DeclareMathOperator{\Ind}{Ind}
%\DeclareMathOperator{\Norm}{Norm}
%\DeclareMathOperator{\Trace}{Trace}
%\DeclareMathOperator{\Cl}{Cl}
%
%\def\head#1{\medskip \noindent \textbf{#1}.}
%
%\newtheorem{theorem}{Theorem}
%\newtheorem{lemma}[theorem]{Lemma}
%
%\begin{document}
%
%\begin{center}
%\bf
%Math 254B, UC Berkeley, Spring 2002 (Kedlaya) \\
%The Cohomology of Finite Groups I: Abstract Nonsense
%\end{center}

\head{Reference} Milne, II.1. See Serre, \textit{Galois Cohomology} for
a much more general presentation. (We will generalize ourselves from finite
to profinite groups a bit later on.) Warning: some authors (like Milne, and
Neukirch for the most part) put
group actions on the left and some (like Neukirch in chapter IV, and myself here)
put them on the right. Of course, the theory is the same either way!

\head{Caveat} This material may seem a bit dry. If
so, don't worry; only a small part of the theory will be relevant for
class field theory. However, it doesn't make sense to learn that small
part without knowing what it is a part of!

\medskip
Let $G$ be a finite group and $A$ an abelian group (itself not necessarily
finite) with a right $G$-action,
also known as a $G$-module. I'll write the $G$-action as a superscript,
i.e., the image of the action of $g$ on $m$ is $m^g$.
Alternatively, $A$ can be viewed as a right
module for the group algebra $\ZZ[G]$. A \emph{homomorphism} of $G$-modules
$\phi: M \to N$ is a homomorphism of abelian groups that is compatible
with the $G$-actions: i.e., $\phi(m^g) = \phi(m)^g$. (For those keeping
score, the category of $G$-modules is an abelian category.)

We would like to define some invariants of the pair $(G, A)$
that we can use to get information about $G$ and $A$. We will use
the general methodology of homological algebra to do this.
Before doing so, though,
we need a few lemmas about $G$-modules.

A $G$-module $M$ is \emph{injective} if for every inclusion $A \subset B$
of $G$-modules and every $G$-module homomorphism $\phi: A \to M$, there
is a homomorphism $\psi: B \to M$ that extends $\phi$.

\begin{lemma} \label{L:enough injectives}
Every $G$-module can be embedded into some injective $G$-module. (That is,
the category of $G$-modules has enough injectives.)
\end{lemma}
\begin{proof}
Exercise.
\end{proof}

In particular, any $G$-module $M$ admits an \emph{injective resolution}:
a complex
\[
0 \to M \to I_0 \stackrel{d_0}{\to} I_1 \stackrel{d_1}{\to} I_2
\stackrel{d_2}{\to} \dots
\]
(that is, $d_{i+1} \circ d_i = 0$ for all $i$)
in which each $I_i$ is injective and the complex itself is \emph{exact}:
$\im d_i = \ker d_{i+1}$.
(To wit, embed $M$ into $I_0$, embed $I_0/M$ into $I_1$, et cetera.)

Given a $G$-module $M$, let $M^G$ be the abelian group of $G$-invariant
elements of $M$:
\[
M^G = \{m \in M: m^g = m \quad \forall g \in G\}.
\]
The functor $M \to M^G$ from $G$-modules to abelian groups is left exact but 
not right exact: if $0 \to M' \to M \to M'' \to 0$ is an exact sequence,
then $0 \to (M')^G \to M^G \to (M'')^G$ is exact, but $M^G \to (M'')^G$ may
not be exact. (Example: take the sequence $0 \to \ZZ/p\ZZ \to \ZZ/p^2\ZZ
\to \ZZ/p\ZZ \to 0$ of $G$-modules for $G = \ZZ/p\ZZ$, which acts on the middle
factor by $a^g = a(1+pg)$. Then $M^G \to (M'')^G$ is the zero map but
$(M'')^G$ is nonzero.)

This is the general situation addressed by homological algebra: it provides
a canonical way to extend the truncated exact sequence
$0 \to (M')^G \to M^G \to (M'')^G$. (Or if you prefer, it helps measure the
failure of exactness of the $G$-invariants functor.) To do this, given
$M$ and an injective resolution as above, take $G$-invariants:
the result
\[
0 \to I_0^G \stackrel{d_0}{\to} I_1^G \stackrel{d_1}{\to} I_2^G
\stackrel{d_2}{\to} \dots
\]
is still a complex, but no longer exact. We turn this failure into success
by defining the $i$-th cohomology group as the quotient
\[
H^i(G, M) = \ker(d^i)/\im(d^{i-1}).
\]
By convention, we let $d_{-1}$ be the map $0 \to I_0^G$, so
$H^0(G,M)=M^G$.

Given a homomorphism $f: M \to N$ and a
injective resolution $0 \to N \to J_0 \to J_1 \to \cdots$, there exists a
commutative diagram
\[
\xymatrix{ 0 \ar[r] & M \ar[r] \ar_f[d] & I_0 \ar^{d_0}[r] \ar_{f_0}[d] &
 I_1  \ar_{f_1}[d] \ar^{d_1}[r] & I_2 \ar_{f_2}[d] \ar^{d_2}[r] &
\cdots \\
0 \ar[r] & N \ar[r] & J_0 \ar^{d_0}[r] & J_1 \ar^{d_1}[r] & 
J_2 \ar^{d_2}[r] &
\cdots
}
\]
and likewise after taking $G$-invariants, so we get maps
$H^i(f): H^i(G, M) \to H^i(G, N)$.
\begin{lemma} \label{L:injective resolutions to cohomology maps}
The map $H^i(f)$ does not depend
on the choice of the $f_i$ (given the choices
of injective resolutions).
\end{lemma}
\begin{proof}
This proof is a bit of ``abstract nonsense''. It suffices to check that if
$f=0$, then the $H^i(f)$ are all zero regardless of what the $f_i$ are.
In that case, it turns out one can
construct maps $g_i: I_{i+1} \to J_i$ (and by convention $g_{-1} = 0$)
such that
$f_i = g_i \circ d_i  + d_{i-1} \circ g_{i-1}$. (Such a set of maps is
called a \emph{homotopy}.) Details left as an exercise. (Warning: the diagonal
arrows in the diagram below don't commute!)
\[
\xymatrix{ 0 \ar[r] & M \ar[r] \ar_f[d] & I_0 \ar^{d_0}[r] \ar_{f_0}[d] &
 I_1  \ar_{g_0}[dl] \ar_{f_1}[d] \ar^{d_1}[r] & \ar_{g_1}[dl] I_2 
\ar_{f_2}[d] \ar^{d_2}[r] &
\cdots \\
0 \ar[r] & N \ar[r] & J_0 \ar^{d_0}[r] & J_1 \ar^{d_1}[r] & 
J_2 \ar^{d_2}[r] &
\cdots
}
\]
\end{proof}
In particular, if $M=N$ and $f$ is the identity, we get a canonical map
between $H^i(G,M)$ and $H^i(G,N)$ for each $i$. That is, the groups
$H^i(G,M)$ are well-defined independent of the choice of the injective
resolution. Likewise, the map $H^i(f)$ is also independent of the choice
of resolutions.

If you know any homological algebra, you'll recognize what comes next:
given a short exact sequence $0 \to M' \to M \to M'' \to 0$ of $G$-modules,
there is a canonical long exact sequence
\begin{gather*}
0 \to H^0(G, M') \to \cdots \to H^i(G, M'') \stackrel{\delta_i}{\to}
\\
\stackrel{\delta_i}{\to} H^{i+1}(G, M') \to H^{i+1}(G, M) \to H^{i+1}(G,M'') \to \cdots,
\end{gather*}
where the $\delta_i$ are certain ``connecting homomorphisms'' (or ``snake
maps''). I won't punish you with the proof of this; if you've never seen
it before, deduce it yourself from the Snake Lemma. (For the proof of the
latter, engage in ``diagram chasing'', or see the movie \emph{It's My Turn}.
To define $\delta$: given $x \in \ker(f_2) \subseteq M_2$, lift $x$ to
$M_1$, push it into $N_1$ by $f_1$, then check that the image has
a preimage in $N_0$. Then verify that the result is well-defined, et
cetera.)
\begin{lemma}[Snake Lemma] \label{L:snake lemma}
Given a commuting diagram
\[
\xymatrix{
0 \ar[r] & M_0 \ar[r] \ar^{f_0}[d] & M_1 \ar[r] \ar^{f_1}[d] & M_2 \ar[r] \ar^{f_2}[d]
& 0 \\
0 \ar[r] & N_0 \ar[r] & N_1 \ar[r] & N_2 \ar[r] & 0
}
\]
in which the rows are exact, there is a canonical map $\delta:
\ker(f_2) \to \coker(f_0)$ such that
the sequence
\[
0 \to \ker(f_0) \to \ker(f_1) \to \ker(f_2) \stackrel{\delta}{\to}
\coker(f_0) \to \coker(f_1) \to \coker(f_2) \to 0
\]
is exact.
\end{lemma}
One important consequence of the long exact sequence is that if
$0 \to M' \to M \to M'' \to 0$ is a short exact sequence of $G$-modules
and $H^1(G, M') = 0$, then $0 \to (M')^G \to M^G \to (M'')^G \to 0$ is
also exact.

More abstract nonsense:
\begin{itemize}
\item
If $0 \to M' \to M \to M'' \to 0$ is a short exact sequence of $G$-modules
and $H^i(G, M) = 0$ for all $i>0$, then the connecting homomorphisms
in the long exact sequence induce isomorphisms $H^i(G, M'') \to
H^{i+1}(G, M')$ for all $i > 0$ (and a surjection for $i=0$). This sometimes allows one to prove
general facts by proving them first for $H^0$, where they have a direct
interpretation, then ``dimension shifting''; however, getting from $H^0$ to $H^1$ typically requires some extra attention.
\item If $M$ is an injective $G$-module, then $H^i(G,M) = 0$ for all 
$i>0$. (Use $0 \to M \to M \to 0 \to \cdots$ as an injective resolution.)
This fact has a sort of converse: see next bullet.
\item We say $M$ is \emph{acyclic} if $H^i(G,M) =0$ for all $i>0$;
so in particular, injective $G$-modules are acyclic.
It turns out that we 
can replace the injective resolution in the definition by an acyclic resolution
for the purposes of doing a computation; see exercises.
\end{itemize}

Of course, the abstract nature of the proofs so far gives us almost no
insight into what the objects are that we've just constructed. We'll remedy
that next time by giving more concrete descriptions that one can actually
compute with.

\head{Exercises}

\begin{enumerate}
\item
Let $G$ be the one-element group. Show that a $G$-module 
(i.e., abelian group) is injective if and
only if it is divisible, i.e., the map $x \mapsto nx$ is surjective
for any nonzero integer $n$. (Hint: you'll need Zorn's lemma or equivalent
in one direction.)
\item
Let $A$ be an abelian group, regarded as a $G$-module for $G$ the trivial
group. Prove that $A$ can be embedded in an injective $G$-module.
\item
Prove Lemma~\ref{L:enough injectives}. (Hint: 
for $M$ a $G$-module, the previous exercises show that the underlying abelian group of $M$ embeds into a divisible group $N$. Now map
$M$ into $\Hom_{\ZZ}(\ZZ[G], N)$ and check that the latter is an injective $G$-module.)
\item
Prove Lemma~\ref{L:injective resolutions to cohomology maps}, following the sketch given. (Hint: construct $g_i$ given
$f_{i-1}$ and $g_{i-1}$, using that the $J$'s are injective $G$-modules.)
\item
Prove that if $0 \to M \to M_0 \to M_1 \to \cdots$ is an exact sequence of $G$-modules
and each $M_i$ is acyclic, then the cohomology groups of the complex
$0 \to M_0^G \to M_1^G \to \cdots$ coincide with $H^i(G, M)$. 
(Hint: construct the canonical long exact sequence from the
exact sequence 
\[
0 \to M \to M_0 \to M_0/M \to 0,
\]
then do dimension shifting using the fact that
\[
0 \to M_0/M \to M_1 \to M_2 \to \cdots
\]
is again exact. Don't forget to be careful about $H^1$!)
\end{enumerate}

%\end{document}


