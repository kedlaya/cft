\documentclass{amsbook}
\usepackage{amsmidx}
\usepackage{url}
\usepackage[all]{xy}

\newtheorem{theorem}{Theorem}[chapter]
\newtheorem{lemma}[theorem]{Lemma}
\newtheorem{cor}[theorem]{Corollary}
\newtheorem{prop}[theorem]{Proposition}

\def\AA{\mathbb{A}}
\def\CC{\mathbb{C}}
\def\FF{\mathbb{F}}
%\def\NN{\mathbb{N}}
\def\PP{\mathbb{P}}
\def\QQ{\mathbb{Q}}
\def\RR{\mathbb{R}}
\def\ZZ{\mathbb{Z}}
\def\kbar{\overline{k}}
\def\gotha{\mathfrak{a}}
\def\gothb{\mathfrak{b}}
\def\gothm{\mathfrak{m}}
\def\gotho{\mathfrak{o}}
\def\gothp{\mathfrak{p}}
\def\gothq{\mathfrak{q}}
\def\gothr{\mathfrak{r}}

\DeclareMathOperator{\ab}{ab}
\DeclareMathOperator{\Aut}{Aut}
\DeclareMathOperator{\Cl}{Cl}
\DeclareMathOperator{\coker}{coker}
\DeclareMathOperator{\Cor}{Cor}
\DeclareMathOperator{\cyc}{cyc}
\DeclareMathOperator{\disc}{Disc}
\DeclareMathOperator{\fin}{fin}
\DeclareMathOperator{\Fix}{Fix}
\DeclareMathOperator{\Frob}{Frob}
\DeclareMathOperator{\Gal}{Gal}
\DeclareMathOperator{\GL}{GL}
\DeclareMathOperator{\Hom}{Hom} 
\DeclareMathOperator{\id}{id}
\DeclareMathOperator{\im}{im}
\DeclareMathOperator{\Ind}{Ind}
\DeclareMathOperator{\Inf}{Inf}
\DeclareMathOperator{\inv}{inv}
\DeclareMathOperator{\Norm}{Norm}
\DeclareMathOperator{\Real}{Re}
\DeclareMathOperator{\Res}{Res}
\DeclareMathOperator{\sep}{sep}
\DeclareMathOperator{\sign}{sign}
\DeclareMathOperator{\smcy}{smcy}
\DeclareMathOperator{\Trace}{Trace}
\DeclareMathOperator{\unr}{unr}
\DeclareMathOperator{\Ver}{Ver}

\def\head#1{\medskip \noindent \textbf{#1}.}

\begin{document}

\title{Notes on class field theory (updated 17 Mar 2017)}
\author{Kiran S. Kedlaya}
\address{Department of Mathematics, University of California, San Diego}
\email{kedlaya@ucsd.edu}

\maketitle
\frontmatter

\tableofcontents

\chapter*{Preface}

This text is a lightly edited version of the lecture notes of a course on class field theory (Math 254B)
that I gave at
UC Berkeley in the spring of 2002. To describe the scope of the course, I can do no better than to quote from the original syllabus:

\begin{quote}
Class field theory, the study of abelian extensions of number fields, was a crowning achievement of number theory in the first half of the 20th century. It brings together, in a unified fashion, the quadratic and higher reciprocity laws of Gauss, Legendre et al, and vastly generalizes them. Some of its consequences (e.g., the Chebotarev density theorem) apply even to nonabelian extensions.

Our approach in this course will be to begin with the formulations of the statements of class field theory, omitting the proofs (except for the Kronecker-Weber theorem, which we prove first). We then proceed to study the cohomology of groups, an important technical tool both for class field theory and for many other applications in number theory. From there, we set up a local form of class field theory, then proceed to the main results. 
\end{quote}

The assumed background for the course was a one-semester graduate course in algebraic number theory,
including the following topics: 
number fields and rings of integers; structure of the class and unit groups; splitting, ramification,
and inertia of prime ideals under finite extensions; different and discriminant; basic properties of local fields.
In fact, most of the students in Math 254B had attended such a course that I gave the previous
semester (Math 254A) based on chapters I, II, and III of Neukirch's \textit{Algebraic Number Theory}; for that reason, it was natural to use that book as a primary reference. 
However, no special features of that presentation are assumed, so just about any graduate-level text on algebraic
number theory (e.g., Fr\"ohlich-Taylor, Janusz, Lang) should provide suitable background.

After the course ended, I kept the lecture notes posted on my web site in their originally written, totally uncorrected state.
Despite their roughness, I heard back from many people over the years who had found them useful;
as a result, I decided to prepare a corrected version of the notes. In so doing, I made a conscious decision
to suppress any temptation to modify 
the presentation with the benefit of hindsight, or to fill in additional material to make the text more self-contained.
This decision, while largely dictated by lack of time and energy, was justified by the belief that the
informality of the original notes contributed to their readibility. In other words, this is not intended as a standalone
replacement for a good book on class field theory!

I maintain very few claims of originality concerning the presentation of the material. 
Besides Neukirch, the main source of inspiration was Milne's lecture notes on algebraic number theory (see \url{http://jmilne.org/math/CourseNotes/cft.html}, version 3.10; note that a more recent version is available, but we have not verified that all references remain valid). These two sources are referenced simply as ``Neukirch'' and ``Milne,'' with additional references described more explicitly  as they occur.
The basic approach may be summarized as follows: I follow Milne's treatment of local class field theory using group cohomology, then follow Neukirch to recast local class field theory in the style of Artin-Tate's class formations, then reuse the same framework to obtain global class field theory. 

This document is not yet in a final state. Consequently, corrections and comments are welcome. Thanks to Zonglin Jiang, Justin Lacini, and Zongze Liu for their feedback on previous drafts.

\mainmatter
\part{Trailer: Abelian extensions of the rationals}

\chapter{The Kronecker-Weber theorem}
%\documentclass[12pt]{article}
%\usepackage{amsfonts, amsthm, amsmath}
%
%\setlength{\textwidth}{6.5in}
%\setlength{\oddsidemargin}{0in}
%\setlength{\textheight}{8.5in}
%\setlength{\topmargin}{0in}
%\setlength{\headheight}{0in}
%\setlength{\headsep}{0in}
%\setlength{\parskip}{0pt}
%\setlength{\parindent}{20pt}
%
%\def\CC{\mathbb{C}}
%\def\FF{\mathbb{F}}
%\def\PP{\mathbb{P}}
%\def\QQ{\mathbb{Q}}
%\def\RR{\mathbb{R}}
%\def\ZZ{\mathbb{Z}}
%\def\gotha{\mathfrak{a}}
%\def\gothb{\mathfrak{b}}
%\def\gothm{\mathfrak{m}}
%\def\gotho{\mathfrak{o}}
%\def\gothp{\mathfrak{p}}
%\def\gothq{\mathfrak{q}}
%\DeclareMathOperator{\disc}{Disc}
%\DeclareMathOperator{\Gal}{Gal}
%\DeclareMathOperator{\Norm}{Norm}
%\DeclareMathOperator{\Trace}{Trace}
%\DeclareMathOperator{\Cl}{Cl}
%
%\def\head#1{\medskip \noindent \textbf{#1}.}
%\def\fixme#1{\textbf{FIXME! #1}}
%
%\newtheorem{theorem}{Theorem}

%\begin{document}
%
%\begin{center}
%\bf
%Math 254B, UC Berkeley, Spring 2002 (Kedlaya) \\
%The Kronecker-Weber Theorem
%\end{center}
%
\head{Reference} Our approach follows
Washington, \textit{Introduction to Cyclotomic Fields}, Chapter 14.
A variety of other methods can be found in other texts.

\head{Abelian extensions of $\QQ$}

Though class field theory has its origins in the law of quadratic 
reciprocity discovered by Gauss, its proper beginning is indicated by the
Kronecker-Weber theorem, first stated by Kronecker in 1853 and proved by
Weber in 1886. Although one could skip this theorem and deduce it as a
consequence of more general results later on, I prefer to work through
it explicitly. It will provide a ``trailer'' for the rest of the course,
giving us a preview of a number of key elements:
\begin{itemize}
\item reciprocity laws;
\item passage between local and global fields, using Galois theory;
\item group cohomology, and applications to classifying field extensions;
\item computations in local fields.
\end{itemize}

An \emph{abelian extension} of a field is a Galois extension with abelian 
Galois group. An example of an abelian extension of $\QQ$ is the cyclotomic
field $\QQ(\zeta_n)$ (where $n$ is a positive integer and $\zeta_n$ is a primitive $n$-th root of 
unity), whose Galois group is $(\ZZ/n\ZZ)^*$, or any
subfield thereof. Amazingly, there are no other examples!
\begin{theorem}[Kronecker-Weber] \label{T:Kronecker-Weber}
If $K/\QQ$ is a finite abelian extension, then
$K \subseteq \QQ(\zeta_n)$ for some positive integer $n$.
\end{theorem}
For example, every quadratic extension of $\QQ$ is contained in a cyclotomic
field, a fact known to Gauss.

The smallest $n$ such that $K \subseteq \QQ(\zeta_n)$ is called the
\emph{conductor} of $K/\QQ$. It plays an important role in the splitting
behavior of primes of $\QQ$ in $K$, as we will see a bit later.

We will prove this theorem in the next few lectures. Our approach will be to
deduce it from a local analogue (see Theorem~\ref{T:local Kronecker-Weber2}).
\begin{theorem}[Local Kronecker-Weber] \label{T:local Kronecker-Weber1}
If $K/\QQ_p$ is a finite abelian extension, then
$K \subseteq \QQ_p(\zeta_n)$ for some $n$, where $\zeta_n$
is a primitive $n$-th root of unity.
\end{theorem}

Before proceeding, it is worth noting explicitly a nice property of
abelian extensions that we will exploit below. Let $L/K$ be a Galois
extension with Galois group $G$, let $\gothp$ be a prime of $K$,
let $\gothq$ be a prime of $L$ over $\gothp$, and let $G_{\gothq}$ and
$I_{\gothq}$ be the decomposition and inertia groups of $\gothq$,
respectively. Then any other prime $\gothq'$ over $\gothp$ can be written
as $\gothq^g$ for some $g \in G$, and the decomposition and inertia groups
of $\gothq'$ are the conjugates $g^{-1} G_{\gothq} g$ and $g^{-1} I_{\gothq}
g$, respectively. (Note: my Galois actions will always be right actions,
denoted by superscripts.) If $L/K$ is \emph{abelian}, though, these
conjugations have no effect. So it makes sense to talk about \emph{the}
decomposition and inertia groups of $\gothp$ itself!

\head{A reciprocity law}

Assuming the Kronecker-Weber theorem, we can deduce strong results
about the way primes of $\QQ$ split in an abelian extension. Suppose
$K/\QQ$ is abelian, with conductor $m$. Then we get a surjective homomorphism
\[
(\ZZ/m\ZZ)^* \cong \Gal(\QQ(\zeta_m)/\QQ) \to \Gal(K/\QQ).
\]
On the other hand, suppose $p$ is a prime not dividing $m$, so that
$K/\QQ$ is unramified above $p$. As noted above, there is a well-defined
decomposition group $G_p \subseteq \Gal(K/\QQ)$. Since there is no ramification
above $p$, the corresponding inertia group is trivial, so $G_p$ is generated
by a Frobenius element $F_p$, which modulo any prime above $p$,
acts as $x \mapsto x^p$. We can formally extend the map $p \mapsto F_p$
to a homomorphism from
$S_m$, the subgroup of $\QQ$ generated by all primes not dividing $m$,
to $\Gal(K/\QQ)$. This is called the \emph{Artin map} of $K/\QQ$.

The punchline is that the Artin map factors through the map $(\ZZ/m\ZZ)^*
\to \Gal(K/\QQ)$ we wrote down above! Namely, note that the image of $r$
under the latter map takes $\zeta_m$ to $\zeta_m^r$. For this image to
be equal to $F_p$, we must have $\zeta_m^r \equiv \zeta_m^p \pmod{\gothp}$
for some prime $\gothp$ of $K$ above $p$. But $\zeta_m^r (1 - \zeta_m^{r-p})$
is only divisible by primes above $m$ (see exercises) unless $r-p \equiv 0
\pmod{m}$. Thus $F_p$ must be equal to the image of $p$ under the map
$(\ZZ/m\ZZ)^* \to \Gal(K/\QQ)$.

The \emph{Artin reciprocity law} states that a similar phenomenon arises
for any abelian extension of any number field; that is, the Frobenius elements
corresponding to various primes are governed by the way the primes ``reduce''
modulo some other quantity. There are several complicating factors in the
general case, though.
\begin{itemize}
\item Prime ideals in a general number field are not always principal, so we
can't always take a generator and reduce it modulo something.
\item There can be lots of units in a general number field, so even when
a prime ideal is principal, it is unclear which generator to choose.
\item It is not known in general how to explicitly construct generators
for all of the abelian extensions of a general number field.
\end{itemize}
Thus our approach will have to be a bit more indirect.

\head{Reduction to the local case}

Our reduction of Kronecker-Weber to local Kronecker-Weber relies on
a key result typically seen in a first course on algebraic number theory. (See for instance Neukirch III.2.)
\begin{theorem}[Minkowski] \label{T:Minkowski}
There are no nontrivial extensions of $\QQ$ which are unramified everywhere.
\end{theorem}

Using Minkowski's theorem, let us deduce the Kronecker-Weber theorem from the local Kronecker-Weber theorem.
\begin{proof}[Proof of Theorem~\ref{T:Kronecker-Weber}]
For each prime $p$ over which $K$ ramifies,
pick a prime $\gothp$ of $K$ over $p$; by local Kronecker-Weber
(Theorem~\ref{T:local Kronecker-Weber1}),
$K_{\gothp} \subseteq \QQ_p(\zeta_{n_p})$ for some positive integer $n_p$.
Let $p^{e_p}$ be the largest power of $p$ dividing $n_p$, and put 
$n = \prod_p p^{e_p}$. (This is a finite product since only finitely many
primes ramify in $K$.) 

We will prove that $K \subseteq \QQ(\zeta_n)$, by proving that
$K(\zeta_n) = \QQ(\zeta_n)$. Write $L = K(\zeta_n)$
and let $I_p$ be the inertia group of $p$ in $L$. If we let
$U$ be the maximal unramified subextension of $L_\gothq$ over $\QQ_p$
for some prime $\gothq$ over $p$, then $L_\gothq = U(\zeta_{p^{e_p}})$
and $I_p \cong \Gal(L_\gothq/U) \cong (\ZZ/p^{e_p}\ZZ)^*$.
Let $I$ be the group generated by all of the $I_p$; then
\[
|I| \leq \prod |I_p| = \prod \phi(p^{e_p}) = \phi(n) = [\QQ(\zeta_n):\QQ].
\]
On the other hand, the fixed field of $I$ is an everywhere unramified extension of 
$\QQ$, which can only be $\QQ$ itself by Minkowski's theorem. That is,
$I = \Gal(L/\QQ)$. But then
\[
[L:\QQ] = |I| \leq [\QQ(\zeta_n):\QQ],
\]
and $\QQ(\zeta_n) \subseteq L$, so we must have $\QQ(\zeta_n) = L$
and $K \subseteq \QQ(\zeta_n)$, as desired.
\end{proof}

\head{Exercises}

\begin{enumerate}
\item
For $m \in \ZZ$
not a perfect square, determine the conductor of $\QQ(\sqrt{m})$.
(Hint: first consider the case where $|m|$ is
prime.)
\item
Recover the law of quadratic reciprocity from the Artin reciprocity law,
using the fact that $\QQ(\sqrt{(-1)^{(p-1)/2} p})$ has conductor $p$.
\item
Prove that if $m,n$ are coprime integers in $\ZZ$, then
$1 - \zeta_m$ and $n$ are coprime in $\ZZ[\zeta_m]$.
(Hint: look at the polynomial $(1-xx)^m-1$ modulo
a prime divisor of $n$.)
\item
Prove that if $m$ is not a prime power, $1-\zeta_m$ is
a unit in $\ZZ[\zeta_m]$.
\end{enumerate}

%\end{document}




\chapter{Kummer theory}
\label{chap:Kummer theory}
%\documentclass[12pt]{article}
%\usepackage{amsfonts, amsthm, amsmath}
%
%\setlength{\textwidth}{6.5in}
%\setlength{\oddsidemargin}{0in}
%\setlength{\textheight}{8.5in}
%\setlength{\topmargin}{0in}
%\setlength{\headheight}{0in}
%\setlength{\headsep}{0in}
%\setlength{\parskip}{0pt}
%\setlength{\parindent}{20pt}
%
%\def\CC{\mathbb{C}}
%\def\FF{\mathbb{F}}
%\def\PP{\mathbb{P}}
%\def\QQ{\mathbb{Q}}
%\def\RR{\mathbb{R}}
%\def\ZZ{\mathbb{Z}}
%\def\gotha{\mathfrak{a}}
%\def\gothb{\mathfrak{b}}
%\def\gothm{\mathfrak{m}}
%\def\gotho{\mathfrak{o}}
%\def\gothp{\mathfrak{p}}
%\def\gothq{\mathfrak{q}}
%\DeclareMathOperator{\disc}{Disc}
%\DeclareMathOperator{\Gal}{Gal}
%\DeclareMathOperator{\GL}{GL}
%\DeclareMathOperator{\Hom}{Hom}
%\DeclareMathOperator{\Norm}{Norm}
%\DeclareMathOperator{\Trace}{Trace}
%\DeclareMathOperator{\Cl}{Cl}
%
%\def\head#1{\medskip \noindent \textbf{#1}.}
%
%\newtheorem{theorem}{Theorem}
%\newtheorem{lemma}[theorem]{Lemma}
%
%\begin{document}
%
%\begin{center}
%\bf
%Math 254B, UC Berkeley, Spring 2002 (Kedlaya) \\
%Kummer Theory
%\end{center}

\head{Reference} Serre, \emph{Local Fields}, Chapter X;
Neukirch section IV.3; or just about
any advanced algebra text (e.g., Lang's \emph{Algebra}). The last lemma is 
from Washington, \emph{Introduction to Cyclotomic Fields}, Chapter 14.

\head{Jargon watch} If $G$ is a group, a \emph{$G$-extension} of a field
$K$ is a Galois extension of $K$ with Galois group $G$.

\medskip
Before attempting to classify all abelian extensions of $\QQ_p$,
we recall an older classification result. This result will
continue to be useful as we proceed to class field theory in general, and 
the technique in its proof prefigures the role to be played
by group cohomology down the line. So watch carefully!

A historical note (due to Franz Lemmermeyer): while the idea of studying field extensions generated by radicals was used extensively by Kummer in his work on Fermat's Last Theorem,
the name \emph{Kummer theory} for the body of results described in this chapter was first applied somewhat later by Hilbert in his \textit{Zahlbericht}, a summary of algebraic number theory as of the end of the 19th century.

\label{T:local Kronecker-Weber}

\begin{theorem} \label{T:Kummer}
If $\zeta_n \in K$, then every $\ZZ/n\ZZ$-extension of $K$ is of the form
$K(\alpha^{1/n})$ for some $\alpha \in K^*$ with the property that
$\alpha^{1/d} \notin K$ for any proper divisor $d$ of $n$, and vice versa.
\end{theorem}

Before describing the proof of Theorem~\ref{T:Kummer}, let me introduce some terminology which marks the tip of the iceberg of group cohomology, which we will see more of later.

If $G$ is a group and $M$ is an abelian group on which $G$ acts
(written multiplicatively),
one defines the group $H^1(G,M)$ as the set of functions $f:
G \to M$ such that $f(gh) = f(g)^h f(h)$, modulo the set of such
functions of the form $f(g) = x (x^g)^{-1}$ for some $x \in M$.

\begin{lemma}[``Theorem 90''] \label{L:theorem 90}
Let $L/K$ be a finite Galois extension with Galois group $G$.
Then $H^1(G, L^*) = 0$.
\end{lemma}
The somewhat unusual common name for this result exists because in the special case where $G$ is cyclic, this statement occurs as Theorem (Satz) 90 in Hilbert's \textit{Zahlbericht}. The general case first appears in Emmy Noether's 1933 paper on the principal ideal theorem (Theorem~\ref{T:principal ideal theorem}), where Noether attributes it to Andreas Speiser.

\begin{proof}
Let $f$ be a function of the form described above.
By the linear independence of automorphisms (see exercises),
there exists $x \in L$ such that $t = \sum_{g \in G} x^g f(g)$
is nonzero. But now
\[
t^h = \sum_{g \in G} x^{gh} f(g)^h =
\sum_{g \in G} x^{gh} f(gh) f(h)^{-1}
= f(h)^{-1} t.
\]
Thus $f$ is zero in $H^1(G,L^*)$.
\end{proof}

\begin{proof}[Proof of Kummer's Theorem]
On one hand, if $\alpha \in K^*$ is such that $\alpha^{1/d} \notin K$
for any proper divisor $d$ of $n$, then the polynomial $x^n - \alpha$
is irreducible over $K$, and every automorphism must have the form
$\alpha \mapsto \alpha \zeta_n^r$ for some $r \in \ZZ/n\ZZ$. Thus
$\Gal(K(\alpha^{1/n})/K) \cong \ZZ/n\ZZ$.

On the other hand,
let $L$ be an arbitrary
$\ZZ/n\ZZ$-extension of $K$. Choose a generator $g \in \Gal(L/K)$,
and let $f: \Gal(L/K) \to L^*$ be the map that sends $rg$ to $\zeta_n^r$
for $r \in \ZZ$.
Then $f \in H^1(\Gal(L/K), L^*)$, so there exists $t \in L$ such that
$t^{rg}/t = f(rg) = \zeta_n^r$ for $r \in \ZZ$. In particular,
$t^n$ is invariant under $\Gal(L/K)$, so $t^n = \alpha$ for some
$\alpha \in K$ and $L = K(t) = K(\alpha^{1/n})$, as desired.
\end{proof}

Another way to state Kummer's theorem is as a bijection
\[
\mbox{$(\ZZ/n\ZZ)^r$-extensions of $K$} \longleftrightarrow
\mbox{$(\ZZ/n\ZZ)^r$-subgroups of $K^*/(K^*)^n$},
\]
where $(K^*)^n$ is the group of $n$-th powers in $K^*$.
(What we proved above was the case $r=1$, but the general case follows easily.)
Another way is in terms of the absolute Galois group of $K$.
Define the \emph{Kummer pairing}
\[
\langle \cdot, \cdot \rangle:
\Gal(\overline{K}/K) \times K^* \to \{1, \zeta_n, \dots, \zeta_n^{n-1} \}
\]
as follows: given $\sigma \in \Gal(\overline{K}/K)$
and $z \in K^*$, choose $y \in \overline{K}^*$ such that $y^n = z$,
and put $\langle \sigma, z \rangle = y^\sigma/y$. Note that this does not
depend on the choice of $y$: the other possibilities are $y \zeta_n^k$
for $k=0, \dots, n-1$, and $\zeta_n^\sigma = \zeta_n$ by the assumption
on $K$, so it drops out.

\begin{theorem}[Kummer reformulated] \label{T:Kummer reformulated}
The Kummer pairing induces an isomorphism
\[
K^*/(K^*)^n \to \Hom(\Gal(\overline{K}/K), \ZZ/n\ZZ).
\]
\end{theorem}
\begin{proof}
The map comes from the pairing; we have to check that it is injective and
surjective. If $y \in K^* \setminus (K^*)^n$, then $K(y^{1/n})$ is a nontrivial
Galois extension of $K$, so there exists some element of
$\Gal(K(y^{1/n})/K)$ that doesn't preserve $y^{1/n}$. Any lift of
that element to $\Gal(\overline{K}/K)$ pairs with $y$ to give something
other than 1; that is, $y$ induces a nonzero homomorphism of
$\Gal(\overline{K}/K)$ to $\ZZ/n\ZZ$. Thus injectivity follows.

On the other hand, suppose $f: \Gal(\overline{K}/K) \to \ZZ/n\ZZ$ is a
homomorphism whose image is the cyclic subgroup of $\ZZ/n\ZZ$ of order $d$.
Let $H$ be the kernel of $f$; then the fixed field $L$ of $H$ is
a $\ZZ/d\ZZ$-extension of $K$ with Galois group $\Gal(\overline{K}/K)/H$.
By Kummer theory, $L = K(y^{1/d})$ for some $y$. But now the homomorphisms
induced by $y^{mn/d}$, as $m$ runs over all integers coprime to $d$,
give all possible homomorphisms of $\Gal(\overline{K}/K)/H$ to $\ZZ/d\ZZ$,
so one of them must equal $f$. Thus surjectivity follows.
\end{proof}

But what about $\ZZ/n\ZZ$-extensions of a field that does not contains
$\zeta_{n}$? These are harder to describe, and indeed describing such 
extensions of $\QQ$ is the heart of this course. There is one thing
one can say: if $L/K$ is a $\ZZ/n\ZZ$-extension, then $L(\zeta_n)/K(\zeta_n)$ is a $\ZZ/d\ZZ$ extension for some divisor $d$ of $n$, and
the latter is a Kummer extension.
\begin{lemma} \label{L:Kummer Galois criterion}
Let $n$ be a prime (or an odd prime power),
let $K$ be a field of characteristic coprime to $n$, let $L = K(\zeta_n)$,
and let $M = L(a^{1/n})$ for some $a \in L^*$. Define the homomorphism
$\omega: \Gal(L/K) \to (\ZZ/n\ZZ)^*$ by the relation
$\zeta_n^{\omega(g)} = \zeta_n^g$. Then $M/K$ is Galois and
abelian if and only if
\begin{equation} \label{eq}
a^g / a^{\omega(g)} \in (L^*)^n \qquad \forall g \in \Gal(L/K).
\end{equation}
\end{lemma}
Note that $\omega(g)$ is only defined up to adding a multiple of $n$,
so $a^{\omega(g)}$ is only defined up to an $n$-th power, i.e., modulo
$(L^*)^n$. (In fact, we will only use one of the implications: if $M/K$ is Galois and abelian, then \eqref{eq} holds. However, we include both implications for completeness.)

\begin{proof}
If $a^g/a^{\omega(g)} \in (L^*)^n$ for all $g \in \Gal(L/K)$,
then $a$, $a^{\omega(g)}$ and $a^g$ all generate the same subgroup
of $(L^*)/(L^*)^n$. Thus $L(a^{1/n}) = L((a^g)^{1/n})$ for all $g \in
\Gal(L/K)$, so $M/K$ is Galois. Thus it suffices to assume $M/K$ is
Galois, then prove that $M/K$ is abelian if and only if (\ref{eq}) holds.
In this case, we must have $a^g/a^{\rho(g)} \in (M^*)^n$ for some map
$\rho: \Gal(L/K) \to (\ZZ/n\ZZ)^*$, whose codomain is cyclic by our
assumption on $n$.

Note that $\Gal(M/K)$ admits a homomorphism $\omega$ to a cyclic group whose
kernel $\Gal(M/L) \subseteq \ZZ/n\ZZ$ is also abelian. Thus $\Gal(M/K)$
is abelian if and only if $g$ and $h$ commute for any $g \in \Gal(M/K)$
and $h \in \Gal(M/L)$, i.e., if $h = g^{-1}hg$.
(Since $g$ commutes with powers of itself, $g$ then
commutes with everything.)

Let $A \subseteq L^*/(L^*)^n$ be the subgroup generated by $a$. Then
the Kummer pairing gives rise to a pairing
\[
\Gal(M/L) \times A \to \{1, \zeta_{n}, \dots, \zeta_n^{n-1}\}
\]
which is bilinear and nondegenerate, so $h = g^{-1}hg$ if and only if
$\langle h, s^g \rangle = \langle ghg^{-1}, s^g \rangle$ for all
$s \in A$. But the Kummer pairing is \emph{equivariant} with respect
to $\Gal(L/K)$ as follows:
\[
\langle h,s \rangle^g = \langle g^{-1}hg, s^g \rangle,
\]
because
\[
\left( \frac{(s^{1/n})^h}{s^{1/n}} \right)^g
= \frac{((s^g)^{1/n})^{g^{-1}hg}}{(s^g)^{1/n}}.
\]
(Here by $s^{1/n}$ I mean an arbitrary $n$-th root of $s$ in $M$,
and by $(s^g)^{1/n}$ I mean $(s^{1/n})^g$. Remember that the value of the
Kummer pairing doesn't depend on which $n$-th root you choose.)
Thus $h = ghg^{-1}$ if and only if $\langle h,s^g \rangle =
\langle h,s \rangle^g$ for all $s \in A$, or equivalently,
just for $s=a$.
But
\[
\langle h,a \rangle^g = \langle h,a \rangle^{\omega(g)}
= \langle h, a^{\omega(g)} \rangle.
\]
Thus $g$ and $h$ commute if and only if $\langle h, a^g \rangle
= \langle h, a^{\omega(g)}\rangle$, if and only if (by nondegeneracy)
$a^g/a^{\omega(g)} \in (L^*)^n$, as desired.
\end{proof}

\head{Exercises}

\begin{enumerate}
\item
Prove the linear independence of automorphisms: if $g_1, \dots, g_n$
are distinct automorphisms of $L$ over $K$, then there do not exist
$x_1, \dots, x_n \in L$ such that $x_1 y^{g_1} + \cdots + x_n y^{g_n} = 0$
for all $y \in L$. (Hint: suppose the contrary, choose a counterexample
with $n$ as small as possible, then make an even smaller counterexample.)
\item
Prove the additive analogue of Theorem 90: if $L/K$ is a finite
Galois extension with Galois group $G$, then $H^1(G, L) = 0$, where the
abelian group is now the additive group of $L$.
(Hint: by the normal basis theorem
(see for example Lang, \emph{Algebra}),
there exists $\alpha \in L$ whose conjugates form a basis
of $L$ as a $K$-vector space.)
\item
Prove the following extension of Theorem 90 (also due to Speiser).
Let $L/K$ be a finite Galois extension with Galois group $G$.
Despite the fact that $H^1(G, \GL(n,L))$ does not make sense as a group (because $\GL(n,L)$ is not abelian), show nonetheless that ``$H^1(G, \GL(n,L))$ is trivial'' in the sense that every function $f: G \to \GL(n,L)$ for which $f(gh) = f(g)^h f(h)$ for all $g,h \in G$ can be written as $x (x^g)^{-1}$ for some $x \in \GL(n,L)$. (Hint: to imitate the proof in the case $n=1$,
one must find an $n \times n$ matrix $x$ over $L$ such that $t = \sum_{g \in G} x^g f(g)$
is not only nonzero but \emph{invertible}. To establish this, note that the set of possible values of $t$ on one hand is an $L$-vector space, and on the other hand satisfies no nontrivial $L$-linear relation.)
\end{enumerate}

%\end{document}




\chapter{The local Kronecker-Weber theorem}
%\documentclass[12pt]{article}
%\usepackage{amsfonts, amsthm, amsmath}
%
%\setlength{\textwidth}{6.5in}
%\setlength{\oddsidemargin}{0in}
%\setlength{\textheight}{8.5in}
%\setlength{\topmargin}{0in}
%\setlength{\headheight}{0in}
%\setlength{\headsep}{0in}
%\setlength{\parskip}{0pt}
%\setlength{\parindent}{20pt}
%
%\def\CC{\mathbb{C}}
%\def\FF{\mathbb{F}}
%\def\NN{\mathbb{N}}
%\def\PP{\mathbb{P}}
%\def\QQ{\mathbb{Q}}
%\def\RR{\mathbb{R}}
%\def\ZZ{\mathbb{Z}}
%\def\gotha{\mathfrak{a}}
%\def\gothb{\mathfrak{b}}
%\def\gothm{\mathfrak{m}}
%\def\gotho{\mathfrak{o}}
%\def\gothp{\mathfrak{p}}
%\def\gothq{\mathfrak{q}}
%\DeclareMathOperator{\disc}{Disc}
%\DeclareMathOperator{\Gal}{Gal}
%\DeclareMathOperator{\Norm}{Norm}
%\DeclareMathOperator{\Trace}{Trace}
%\DeclareMathOperator{\Cl}{Cl}
%
%\def\head#1{\medskip \noindent \textbf{#1}.}
%\def\fixme#1{\textbf{FIXME! #1}}
%
%\newtheorem{theorem}{Theorem}
%\newtheorem{lemma}[theorem]{Lemma}
%
%\begin{document}
%
%\begin{center}
%\bf
%Math 254B, UC Berkeley, Spring 2002 (Kedlaya) \\
%The Local Kronecker-Weber Theorem
%\end{center}

\head{Reference} Washington, \textit{Introduction to Cyclotomic Fields},
Chapter 14.

\medskip

We now prove the local Kronecker-Weber theorem
(Theorem~\ref{T:local Kronecker-Weber1}), modulo some steps
which will be left as exercises. As shown previously, this will imply
the original Kronecker-Weber theorem.
\begin{theorem}[Local Kronecker-Weber] \label{T:local Kronecker-Weber2}
If $K/\QQ_p$ is a finite abelian extension, then
$K \subseteq \QQ_p(\zeta_n)$ for some positive integer $n$.
\end{theorem}

Since $\Gal(K/\QQ_p)$ decomposes into a product of cyclic groups of 
prime-power order, by the structure theorem for finite abelian groups
we may write $K$ as the compositum of extensions of $\QQ_p$ whose Galois
groups are cyclic of prime-power order. In other words, it suffices to prove
local Kronecker-Weber under the assumption that $\Gal(K/\QQ_p) \cong
\ZZ/q^r \ZZ$ for some prime $q$ and some positive integer $r$.

We first recall the following facts from the theory of local fields
(e.g., see Neukirch II.7).
\begin{lemma} \label{lem:unram}
Let $L/K$ be an unramified extension of finite extensions of $\QQ_p$. 
Then $L = K(\zeta_{q-1})$, where $q$ is the cardinality of the residue
field of $L$.
\end{lemma}
\begin{lemma} \label{lem:tame}
Let $L/K$ be a totally and 
tamely ramified extension of finite extensions of $\QQ_p$ of degree $e$.
(Recall that tamely ramified means that $p$ does not divide $e$.) Then
there exists a generator $\pi$ of the maximal ideal of the valuation ring
of $K$ such that $L = K(\pi^{1/e})$.
\end{lemma}

We also recall one similarly easy but possible less familiar fact, whose proof we leave as an exercise.
\begin{lemma} \label{lem:zetap}
The fields $\QQ_p((-p)^{1/(p-1)})$ and $\QQ_p(\zeta_p)$ are equal.
\end{lemma}

We now proceed to the proof of the local Kronecker-Weber theorem.

\noindent
\textbf{Case 1: $q \neq p$.}

Let $L$ be the maximal unramified subextension of $K$. By
Lemma~\ref{lem:unram}, $L = \QQ_p(\zeta_n)$ for some $n$.
Let $e = [K:L]$. Since $e$ is a power of $q$, $e$ is not divisible by $p$,
so $K$ is totally and tamely ramified over $L$. Thus by Lemma~\ref{lem:tame},
there exists $\pi \in L$ generating the maximal ideal of $\gotho_L$ such 
that $K = L(\pi^{1/e})$.
Since $L/\QQ_p$ is unramified, $p$ also generates the maximal ideal of
$\gotho_L$, so we can write $\pi = -pu$ for some unit $u \in \gotho_L^*$.
Now $L(u^{1/e})/L$ is unramified since $e$ is prime to $p$ and $u$ is a unit.
In particular, $L(u^{1/e})/\QQ_p$ is unramified, hence abelian. Then
$K(u^{1/e})/\QQ_p$ is the compositum of the two abelian extensions
$K/\QQ_p$ and $L(u^{1/e})/\QQ_p$, so it's also abelian. Hence any subextension
is abelian, in particular $\QQ_p((-p)^{1/e})/\QQ_p$.

For $\QQ_p((-p)^{1/e})/\QQ_p$ to be Galois, it must contain the $e$-th roots
of unity (since it must contain all of the $e$-th roots of $-p$, and we
can divide one by another to get an $e$-th root of unity). But
$\QQ_p((-p)^{1/e})/\QQ_p$ is totally ramified, whereas $\QQ_p(\zeta_e)/\QQ_p$
is unramified. This is a contradiction unless $\QQ_p(\zeta_e)$ is actually
equal to $\QQ_p$, which only happens if $e|(p-1)$ (since the residue field
$\FF_p$ of $\QQ_p$ contains only $(p-1)$-st roots of unity).

Now $K \subseteq L((-p)^{1/e}, u^{1/e})$ as noted above. But on one hand, $L(u^{1/e})$
is unramified over $L$, so $L(u^{1/e}) = L(\zeta_m)$ for some $m$; on the
other hand, because $e|(p-1)$, we have
$\QQ_p((-p)^{1/e}) \subseteq \QQ_p((-p)^{1/(p-1)}) =
\QQ_p(\zeta_p)$ by Lemma~\ref{lem:zetap}.
Putting it all together,
\[
K \subseteq L((-p)^{1/e}, u^{1/e}) \subseteq \QQ_p(\zeta_n, \zeta_p, \zeta_m)
\subseteq \QQ_p(\zeta_{mnp}).
\]

\noindent
\textbf{Case 2: $q = p \neq 2$.}

Suppose $\Gal(K/\QQ_p) \cong \ZZ/p^r\ZZ$. We can use roots of unity to
construct two other extensions of $\QQ_p$ with this Galois group. Namely,
$\QQ_{p}(\zeta_{p^{p^r}-1})/\QQ_p$ is unramified of degree $p^r$, and 
automatically has cyclic Galois group; meanwhile, the index $p-1$ subfield
of $\QQ_p(\zeta_{p^{r+1}})$ is totally ramified with Galois group $\ZZ/p^r\ZZ$.
By assumption, $K$ is not contained in the compositum of these two fields,
so for some $s>0$,
\[
\Gal(K(\zeta_{p^{p^r}-1}, \zeta_{p^{r+1}})/\QQ_p) \cong
(\ZZ/p^r\ZZ)^2 \times \ZZ/p^s \ZZ \times \ZZ/(p-1)\ZZ.
\]
This group admits
$(\ZZ/p\ZZ)^3$ as a quotient, so we have an extension of $\QQ_p$ with
Galois group $(\ZZ/p\ZZ)^3$. It thus suffices to prove the following lemma.

\begin{lemma} \label{lem:three}
For $p \neq 2$, there is no extension of $\QQ_p$ with 
Galois group $(\ZZ/p\ZZ)^3$.
\end{lemma}
\begin{proof}
For convenience, put $\pi = \zeta_p - 1$. Then $\pi$ is a uniformizer of
$\QQ_p(\zeta_p)$.

If $\Gal(K/\QQ_p) \cong (\ZZ/p\ZZ)^3$, then
$\Gal(K(\zeta_p)/\QQ_p(\zeta_p)) \cong (\ZZ/p\ZZ)^3$ as well,
and $K(\zeta_p)$ is abelian over $\QQ_p$ with Galois group
$(\ZZ/p\ZZ)^* \times (\ZZ/p\ZZ)^3$. Applying
Kummer theory to $K(\zeta_p)/\QQ_p(\zeta_p)$ produces a subgroup $B
\subseteq \QQ_p(\zeta_p)^*/(\QQ_p(\zeta_p)^*)^p$ isomorphic to
$(\ZZ/p\ZZ)^3$ such that $K(\zeta_p) = \QQ_p(\zeta_p, B^{1/p})$.
Let $\omega: \Gal(\QQ_p(\zeta_p)/\QQ_p) \to (\ZZ/p\ZZ)^*$ be the
canonical map; since $\QQ_p(\zeta_p, b^{1/p}) \subseteq K(\zeta_p)$ is also
abelian over $\QQ_p$, by Lemma~\ref{L:Kummer Galois criterion},
\[
b^g/b^{\omega(g)} \in (\QQ_p(\zeta_p)^*)^p \qquad
(\forall b \in B, g \in \Gal(\QQ_p(\zeta_p)/\QQ_p)).
\]
Recall the structure of $\QQ_p(\zeta_p)^*$: the maximal ideal of
$\ZZ_p[\zeta_p]$ is generated by $\pi$, while
each unit of $\ZZ_p[\zeta_p]$ is congruent to a $(p-1)$-st root of unity modulo
$\pi$, and so
\[
\QQ_p(\zeta_p)^* = \pi^\ZZ \times (\zeta_{p-1})^\ZZ \times U_1,
\]
where $U_1$ denotes the set of units of $\ZZ_p[\zeta_p]$ congruent to
1 modulo $\pi$. Correspondingly,
\[
(\QQ_p(\zeta_p)^*)^p = \pi^{p\ZZ} \times (\zeta_{p-1})^{p\ZZ}
\times U_1^p.
\]
Now choose a representative $a \in L^*$ of some nonzero
element of $B$; without
loss of generality, we may assume $a = \pi^m u$ for some
$m \in \ZZ$ and $u \in U_1$. Then
\[
\frac{a^g}{a^{\omega(g)}}
= \frac{(\zeta_p^{\omega(g)}-1)^m}{\pi^{m\omega(g)}} \frac{u^g}{u^{\omega(g)}};
\]
but $v_\pi(\pi) = v_\pi(\zeta_p^{\omega(g)}-1) = 1$. Thus 
the valuation of the right hand side is $m(1-\omega(g))$, which can only
be a multiple of $p$ for all $g$ if $m \equiv 0 \pmod{p}$. (Notice we
just used that $p$ is odd!) That is,
we could have taken $m=0$ and $a = u \in U_1$.

As for $u^g/u^{\omega(g)}$, note that $U_1^p$ is precisely the set of units
congruent to 1 modulo $\pi^{p+1}$ (see exercises).
Since $\zeta_p = 1 + \pi + O(\pi^2)$, we can write
$u = \zeta_p^b(1 + c\pi^d + O(\pi^{d+1}))$, with $c \in \ZZ$
and $d \geq 2$. Since $\pi^g/\pi \equiv \omega(g) \pmod{\pi}$, we get
\[
u^g = \zeta_p^{b\omega(g)} (1 + c \omega(g)^d \pi^d + O(\pi^{d+1})),
\quad
u^{\omega(g)} = \zeta_p^{b\omega(g)} (1 + c \omega(g) \pi^d + O(\pi^{d+1})).
\]
But these two have to be congruent modulo $\pi^{p+1}$. Thus either
$d \geq p+1$ or $d \equiv 1 \pmod{p-1}$, the latter only occurring for
$d=p$.

What this means is that the set of possible $u$ is generated by
$\zeta_p$ and by $1 + \pi^p$. But these only generate a subgroup of
$U_1/U_1^p$ isomorphic to $(\ZZ/p\ZZ)^2$, whereas $B \cong (\ZZ/p\ZZ)^3$.
Contradiction.
\end{proof}

\noindent
\textbf{Case 3: $p=q=2$.}
This case is similar to the previous case, but a bit messier, because
$\QQ_2$ does admit an extension with Galois group $(\ZZ/2\ZZ)^3$.
We defer this case to the exercises.

\head{Exercises}

\begin{enumerate}
\item
Prove Lemma~\ref{lem:zetap}. (Hint: prove that $(\zeta_p-1)^{p-1}/p - 1$
belongs to the maximal ideal of $\ZZ_p[\zeta_p]$.)
\item
Prove that (in the notation of Lemma~\ref{lem:three})
$U_1^p$ is the set of units congruent to 1 modulo $\pi^{p+1}$.
(Hint: in one direction, write $u \in U_1$ as a power of $\zeta_p$ times
a unit congruent to 1 modulo $\pi^2$. In the other direction,
use the binomial series for $(1+x)^{1/p}$.)
\item
Prove that for any $r>0$, there is an extension of $\QQ_2$ with Galois group
$\ZZ/2\ZZ \times (\ZZ/2^r\ZZ)^2$ contained in $\QQ_2(\zeta_n)$ for some
$n>0$.
\item
Suppose that $K/\QQ_2$ is a $\ZZ/2^r\ZZ$-extension not contained in
$\QQ_2(\zeta_n)$ for any $n>0$. Prove that there exists
an extension of $\QQ_2$ with Galois group $(\ZZ/2\ZZ)^4$ or $(\ZZ/4\ZZ)^3$.
\item
Prove that there is no extension of $\QQ_2$ with Galois group $(\ZZ/2\ZZ)^4$.
(Hint: use Kummer theory.)
\item
Prove that there is no extension of $\QQ_2$ with Galois group $(\ZZ/4\ZZ)^3$.
(Hint: reduce to showing that there exists no extension of $\QQ_2$ containing
$\QQ_2(\sqrt{-1})$ with Galois group $\ZZ/4\ZZ$.)
\end{enumerate}

%\end{document}




\part{The statements of class field theory}

\chapter{The Hilbert class field}
%\documentclass[12pt]{article}
%\usepackage{amsfonts, amsthm, amsmath}
%
%\setlength{\textwidth}{6.5in}
%\setlength{\oddsidemargin}{0in}
%\setlength{\textheight}{8.5in}
%\setlength{\topmargin}{0in}
%\setlength{\headheight}{0in}
%\setlength{\headsep}{0in}
%\setlength{\parskip}{0pt}
%\setlength{\parindent}{20pt}
%
%\def\AA{\mathbb{A}}
%\def\CC{\mathbb{C}}
%\def\FF{\mathbb{F}}
%\def\PP{\mathbb{P}}
%\def\QQ{\mathbb{Q}}
%\def\RR{\mathbb{R}}
%\def\ZZ{\mathbb{Z}}
%\def\gotha{\mathfrak{a}}
%\def\gothb{\mathfrak{b}}
%\def\gothm{\mathfrak{m}}
%\def\gotho{\mathfrak{o}}
%\def\gothp{\mathfrak{p}}
%\def\gothq{\mathfrak{q}}
%\DeclareMathOperator{\disc}{Disc}
%\DeclareMathOperator{\Gal}{Gal}
%\DeclareMathOperator{\GL}{GL}
%\DeclareMathOperator{\Hom}{Hom}
%\DeclareMathOperator{\Norm}{Norm}
%\DeclareMathOperator{\Trace}{Trace}
%\DeclareMathOperator{\Cl}{Cl}
%
%\def\head#1{\medskip \noindent \textbf{#1}.}
%
%\newtheorem{theorem}{Theorem}
%\newtheorem{lemma}[theorem]{Lemma}
%
%\begin{document}
%
%\begin{center}
%\bf
%Math 254B, UC Berkeley, Spring 2002 (Kedlaya) \\
%The Hilbert Class Field
%\end{center}

\head{Reference} Milne, Introduction; Neukirch, VI.6.

\medskip

Recall that the field $\QQ$ has no extensions which are everywhere
unramified (Theorem~\ref{T:Minkowski}). This is quite definitely not true of other
number fields; we begin with an example illustrating this.

In the number field $K = \QQ(\sqrt{-5})$, the ring of integers is
$\ZZ[\sqrt{-5}]$ and the ideal $(2)$ factors as $\gothp^2$,
where the ideal $\gothp = (2, 1 + \sqrt{-5})$ is not principal.

Now let's see what happens when we adjoin a square root of $-1$,
obtaining $L = \QQ(\sqrt{-5}, \sqrt{-1})$. The
extension $\QQ(\sqrt{-1})/\QQ$ only ramifies over 2, so
$L/K$ can only be ramified over $\gothp$.
On the other hand, if we write $L = K(\alpha)$ where $\alpha =
(1 + \sqrt{5})/2$, then modulo $\gothp$ 
the minimal polynomial $x^2-x-1$ of $\alpha$ remains irreducible, so $\gothp$
is unramified (and not split) in $L$. 

We've now seen that $\QQ(\sqrt{-5})$ admits both a nonprincipal ideal and
an unramified abelian extension. It turns out these are not unrelated events.
Caution: until further notice, the phrase ``$L/K$ is unramified'' will mean
that $L/K$ is unramified over all finite places in the usual sense, \emph{and}
that every real embedding of $K$ extends to a real embedding of $L$. 
(Get used to this. The real and complex embeddings of a number field will
be treated like primes consistently throughout this text.)
\begin{theorem} \label{T:Hilbert class field}
Let $L$ be the maximal unramified
abelian extension of a number field $K$.
Then $L/K$ is finite, and its Galois group is isomorphic to
the ideal class group $\Cl(K)$ of $K$.
\end{theorem}
In fact, there is a canonical isomorphism, given by the Artin reciprocity
law. We'll see this a bit later. The field $L$ is called the \emph{Hilbert
class field} of $K$.

Warning: there can be infinite unramified \emph{nonabelian} extensions.
In fact, Golod and Shafarevich used unramified abelian extensions to 
construct these! Namely, starting from a number field $K = K_0$,
let $K_1$ be the Hilbert class field of $K_0$, let $K_2$ be the Hilbert
class field of $K_1$, and so on. Then $K_i$ is an unramified but not
necessarily abelian extension of $K_0$, and for a suitable choice of 
$K_0$, $[K_i:K_0]$ can be unbounded. (See Cassels-Frohlich for more discussion.)

\head{Exercises}

\begin{enumerate}
\item
Let $K$ be an imaginary
quadratic extension of $\QQ$ in which $t$ finite primes ramify.
Asuming Theorem~\ref{T:Hilbert class field},
prove that $\#(\Cl(K)/2\Cl(K)) = 2^{t-1}$;
this recovers a theorem of Gauss originally proved using binary quadratic forms. 
(Hint: if an odd prime $p$
ramifies in $K$,
show that $K(\sqrt{p^*})/K$ is unramified for $p^* = (-1)^{(p-1)/2} p$; if 2 ramifies in $K$, show that
$K(p^*)/K$ is unramified for one of $p^* = -1, 2, -2$.)
\item
Give an example, using a real quadratic field, to illustrate that:
\begin{enumerate}
\item[(a)] Theorem~\ref{T:Hilbert class field} fails if we don't require the extensions to be
unramified above the real place;
\item[(b)] the previous exercise fails for real quadratic fields.
\end{enumerate}
\item
Prove that Exercise~1 extends to real quadratic fields if one replaces the
class group by the \emph{narrow class group}, in which you
only mod out by principal ideals having a totally positive generator.
This gives an example of a \emph{ray class group}; more on those in the next chapter.
\item
The field $\QQ(\sqrt{-23})$ admits an ideal of order 3 in the class group and
an unramified
abelian extension of degree 3. Find both. (Hint: the extension contains a cubic field of discriminant -23.)
\end{enumerate}

%\end{document}




\chapter{Generalized ideal class groups and the Artin reciprocity law}
%\documentclass[12pt]{article}
%\usepackage{amsfonts, amsthm, amsmath}
%
%\setlength{\textwidth}{6.5in}
%\setlength{\oddsidemargin}{0in}
%\setlength{\textheight}{8.5in}
%\setlength{\topmargin}{0in}
%\setlength{\headheight}{0in}
%\setlength{\headsep}{0in}
%\setlength{\parskip}{0pt}
%\setlength{\parindent}{20pt}
%
%\def\AA{\mathbb{A}}
%\def\CC{\mathbb{C}}
%\def\FF{\mathbb{F}}
%\def\PP{\mathbb{P}}
%\def\QQ{\mathbb{Q}}
%\def\RR{\mathbb{R}}
%\def\ZZ{\mathbb{Z}}
%\def\gotha{\mathfrak{a}}
%\def\gothb{\mathfrak{b}}
%\def\gothm{\mathfrak{m}}
%\def\gotho{\mathfrak{o}}
%\def\gothp{\mathfrak{p}}
%\def\gothq{\mathfrak{q}}
%\DeclareMathOperator{\disc}{Disc}
%\DeclareMathOperator{\Frob}{Frob}
%\DeclareMathOperator{\Gal}{Gal}
%\DeclareMathOperator{\GL}{GL}
%\DeclareMathOperator{\Hom}{Hom}
%\DeclareMathOperator{\Norm}{Norm}
%\DeclareMathOperator{\Trace}{Trace}
%\DeclareMathOperator{\Cl}{Cl}
%
%\def\head#1{\medskip \noindent \textbf{#1}.}
%
%\newtheorem{theorem}{Theorem}
%\newtheorem{lemma}[theorem]{Lemma}
%
%\begin{document}
%
%\begin{center}
%\bf
%Math 254B, UC Berkeley, Spring 2002 (Kedlaya) \\
%Generalized Ideal Class Groups and the Artin Reciprocity Law
%\end{center}

\head{Reference} Milne V.1; Neukirch VI.6.

\head{An example (continued from the previous chapter)}

Before proceeding to generalized ideal class groups, we continue a bit
with the example from the previous chapter to illustrate what is about to happen.
Let $K = \QQ(\sqrt{-5})$ and let $L = \QQ(\sqrt{-5}, \sqrt{-1})$; recall that
$L/K$ is unramified everywhere.
\begin{theorem}
Let $\gothp$ be a prime of $\gotho_K$. Then $\gothp$ splits in $L$ if and only
if $\gothp$ is principal.
\end{theorem}
\begin{proof}
First suppose $\gothp = (p)$, where $p \neq 2, 5$
is a rational prime that remains inert (i.e., does not split and is not
ramified) in $K$. This happens if and only if $-5$ is not a square mod $p$.
In this case, one of $-1$ and $5$ is a square in $\FF_p$, so
$\gotho_K/\gothp$ contains a square root of one of them, hence of both
(since $-5$ already has a square root there). Thus the residue field does
not grow when we pass to $L$, that is, $\gothp$ is split.

Next suppose $p \neq 2,5$ is a rational prime that splits
as $\gothp \overline{\gothp}$.
If $\gothp = (\beta)$ is principal, then
the equation $x^2 + 5y^2 = p$ has a solution in $\ZZ$
(namely, for $x + y \sqrt{-5} = \beta$), but this is only
possible if $p \equiv 1 \pmod{4}$. Then $p$ splits in $\QQ(\sqrt{-1})$
as well, so $p$ is totally split in $L$, so
$\gothp$ splits in $L$.

Conversely, suppose $\gothp$ is not principal. Since there are only two
ideal classes in $\QQ(\sqrt{-5})$, we have $\gothp = \alpha
(2, 1 + \sqrt{-5})$ for some $\alpha \in K$. Thus
$\Norm(\gothp) = |\Norm(\alpha)| \Norm(2, 1 + \sqrt{-5})$. If
$\alpha = x + y \sqrt{-5}$ for $x,y \in \QQ$, we then have
$p = 2(x^2 + 5y^2)$. Considering things mod 4, we see that
$2x$ and $2y$ must be ratios of two odd integers, and $p \equiv 3 \pmod{4}$. Thus
$p$ does not split in $L$, so $\gothp$ cannot split in $L$.

The only cases left are $\gothp = (2, 1+\sqrt{-5})$, which does not split
(see above), and $\gothp = (\sqrt{-5})$, which does split (since $-1$
has a square root mod 5).
\end{proof}

Bonus aside: for any ideal $\gotha$ of $\gotho_K$, $\gotha \gotho_L$ is
principal. (It suffices to verify that $(2, 1+\sqrt{-5})\gotho_L =
(1+\sqrt{-1})\gotho_L$.)
This is a special case of the ``capitulation'' theorem; we'll
come back to this a bit later.

\head{Generalized ideal class groups}

In this section, we formulate (without proof) the Artin reciprocity law
for an arbitrary abelian extension $L/K$ of number fields. This map will
generalize the canonical isomorphism, in the case $K = \QQ$, of
$\Gal(L/\QQ)$ with a subgroup of $(\ZZ/m\ZZ)^*$ for some $m$,
as well as the splitting behavior we saw in the previous example. Before
proceeding, we need to define the appropriate generalization of
$(\ZZ/m\ZZ)^*$ to number fields.

Recall that the ideal class group of $K$ is defined as the group of
fractional ideals modulo the subgroup of principal
fractional ideals. Let $\gothm$ be a formal product of places of $K$;
you may regard such a beast as an ordinary integral ideal together
with a nonnegative coefficient for each infinite place. Let $I_K^\gothm$
be the group of fractional ideals of $K$ which are coprime to each finite
place of $K$ occurring in $\gothm$. Let $P_K^\gothm \subseteq I_K^\gothm$
be the group of principal fractional ideals generated by elements
$\alpha \in K$ such that:
\begin{itemize}
\item for $\gothp^e | \gothm$ finite, $\alpha \equiv 1 \pmod{\gothp^e}$;
\item for every real place $\tau$ in $\gothm$, $\tau(\alpha) > 0$.
\end{itemize}
(There is no condition for complex places.) Then the \emph{ray class group}
$\Cl^\gothm(K)$ is defined as the quotient $I_K^\gothm /
P_K^\gothm$. A quotient of
a ray class group is called a \emph{generalized ideal class group}.

\head{The Artin reciprocity law}

Now let $L/K$ be a (finite) abelian extension of number fields. We imitate
the ``reciprocity law'' construction we made for $\QQ(\zeta_m)/\QQ$, but
this time with no reason \emph{a priori} to expect it to give anything useful.
For each prime $\gothp$ of $K$ that does not ramify in $L$, let $\gothq$ be
a prime of $L$ above $K$, and put $\kappa = \gotho_K/\gothp$ and
$\lambda = \gotho_L/\gothq$.
Then the residue field extension $\lambda/\kappa$
is an extension of finite fields, so it has a canonical
generator $\sigma$, the Frobenius, which acts by raising to the $q$-th power.
(Here $q = \Norm(\gothp) = \#\kappa$ is the absolute norm of $\gothp$.)
Since $\gothp$ does not ramify, the decomposition group $G_\gothq$ is
isomorphic to $\Gal(\lambda/\kappa)$, so we get a canonical element of
$G_\gothq$, called the Frobenius of $\gothq$. In general, replacing
$\gothq$ by $\gothq^\tau$ for some $\tau \in \Gal(L/K)$ conjugates both the
decomposition group and the Frobenius by $\tau$; since $L/K$ is abelian in
our case, that conjugation has no effect. Thus we may speak of ``the
Frobenius of $\gothp$'' without ambiguity.

Now for $\gothm$ divisible by all primes of $K$ which ramify in $L$,
define a homomorphism (the Artin map)
\[
I_K^\gothm \to \Gal(L/K) \qquad \gothp \mapsto \Frob_\gothp.
\]
(Aside: the fact that we have to avoid the ramified primes will be a bit
of a nuisance later. Eventually
we'll get around this using the adelic formulation.)
Then the following miracle occurs.
\begin{theorem}[Artin reciprocity]
There exists a formal product $\gothm$ of places of $K$, including all
(finite and infinite) places over which $L$ ramifies, such that
$P_K^\gothm$ belongs to the kernel of the above homomorphism.
\end{theorem}
In particular, we get a map $I_K^\gothm/P_K^\gothm \to \Gal(L/K)$ which
turns out to be surjective (see exercises).
Only now, we don't have the Kronecker-Weber theorem to explain it.

Define the \emph{conductor} of $L/K$ to be the smallest formal product $\gothm$
for which the conclusion of the Artin reciprocity law holds.
We say $L/K$ is the \emph{ray class field} corresponding to the product
$\gothm$ if $L/K$ has conductor dividing $\gothm$ and the map
$I_K/I_K^\gothm \to \Gal(L/K)$ is an isomorphism.
\begin{theorem}[Existence of ray class fields]
Every formal product $\gothm$ has a ray class field.
\end{theorem}
For example, the ray class field of $\QQ$ of conductor $m\infty$ is
$\QQ(\zeta_m)$; the ray class field of $\QQ$ of conductor $m$ is
the maximal real subfield of $\QQ(\zeta_m)$. 

Unfortunately, for number fields other than $\QQ$, we do not have an
explicit description of the ray class fields as being generated by
particular algebraic numbers. (A salient exception is the imaginary
quadratic fields, for which the theory of elliptic curves
with complex multiplication provides such numbers. Also, if we were to
work with function fields instead of number fields, the theory of
Drinfeld modules would do something similar.) This gap in our
knowledge, also referred to as Hilbert's 12th Problem,
will make establishing class field theory somewhat more complicated than
it would be otherwise.

\head{Exercises}

\begin{enumerate}
\item
For $\gothp$ a prime ideal
of $K$ and $L/K$ an abelian extension in which $\gothp$ does not
ramify, let $\Frob_{L/K}(\gothp) \in \Gal(L/K)$ be the Frobenius of $\gothp$.
Prove that Frobenius obeys the following compatibilities:
\begin{enumerate}
\item[(a)] if $M/L$ is another extension with $M/K$ abelian,
$\gothq$ is a prime
of $L$ over $\gothp$, and $M/L$ is unramified over $\gothq$,
then $\Frob_{M/K}(\gothp)$ restricted to $L$ equals $\Frob_{L/K}(\gothp)$.
\item[(b)] with notation as in (a), $\Frob_{M/L}(\gothq)
= \Frob_{M/K}(\gothp)^{f(\gothq/\gothp)}$, where $f$ denotes the residue
field degree.
\end{enumerate}
\item
Find a formula for the order of $\Cl^\gothm(K)$ in terms of the order
of $\Cl(K)$ and other relevant stuff. (Hint: it's in Milne V.1. Make sure
you understand its proof!) Then use that formula to give a formula for the
order of $\Cl^{\gothm}(\QQ(\sqrt{D}))$ for $D$ odd and squarefree, in terms
of the prime factors of $\gothm$ and $D$ and the class number of
$\QQ(\sqrt{D})$.
\item
Show that the homomorphism $I_K^\gothm \to \Gal(L/K)$ is surjective. You 
may assume the following fact: if $L/K$ is an extension of number fields
(with $L \neq K$), there exists a prime of $K$ which does not split
completely in $L$.
\item
Find the ray class field of $\QQ(i)$ of conductor $(3)$, and verify
Artin reciprocity explicitly in this case.
\end{enumerate}

%\end{document}




\chapter{The principal ideal theorem}
\label{chap:principal}
%\documentclass[12pt]{article}
%\usepackage{amsfonts, amsthm, amsmath}
%\usepackage[all]{xy}
%
%\setlength{\textwidth}{6.5in}
%\setlength{\oddsidemargin}{0in}
%\setlength{\textheight}{8.5in}
%\setlength{\topmargin}{0in}
%\setlength{\headheight}{0in}
%\setlength{\headsep}{0in}
%\setlength{\parskip}{0pt}
%\setlength{\parindent}{20pt}
%
%\def\AA{\mathbb{A}}
%\def\CC{\mathbb{C}}
%\def\FF{\mathbb{F}}
%\def\PP{\mathbb{P}}
%\def\QQ{\mathbb{Q}}
%\def\RR{\mathbb{R}}
%\def\ZZ{\mathbb{Z}}
%\def\gotha{\mathfrak{a}}
%\def\gothb{\mathfrak{b}}
%\def\gothm{\mathfrak{m}}
%\def\gotho{\mathfrak{o}}
%\def\gothp{\mathfrak{p}}
%\def\gothq{\mathfrak{q}}
%\def\gothr{\mathfrak{r}}
%\DeclareMathOperator{\ab}{ab}
%\DeclareMathOperator{\disc}{Disc}
%\DeclareMathOperator{\Frob}{Frob}
%\DeclareMathOperator{\Gal}{Gal}
%\DeclareMathOperator{\GL}{GL}
%\DeclareMathOperator{\Hom}{Hom}
%\DeclareMathOperator{\Norm}{Norm}
%\DeclareMathOperator{\Trace}{Trace}
%\DeclareMathOperator{\Cl}{Cl}
%
%\def\head#1{\medskip \noindent \textbf{#1}.}
%
%\newtheorem{theorem}{Theorem}
%\newtheorem{lemma}[theorem]{Lemma}
%
%\begin{document}
%
%\begin{center}
%\bf
%Math 254B, UC Berkeley, Spring 2002 (Kedlaya) \\
%The Principal Ideal Theorem
%\end{center}

\head{Reference} 
Milne, section V.3 (but you won't find the proofs I've omitted there either);
Neukirch, section VI.7 (see also IV.5); 
Lang, \textit{Algebraic Number Theory}, section XI.5.

\medskip
For a change, we're going to prove something, if only assuming the
Artin reciprocity law which we haven't proved. Or rather, we're going to sketch
a proof that you will fill in by doing the exercises. (Why should I have all
the fun?)

The following theorem
is due to Furtw\"angler, a student of Hilbert. (It's also called the
``capitulation'' theorem, because in the old days the word ``capitulate''
meant ``to become principal''. Etymology left to the reader.)
\begin{theorem}[Principal ideal theorem] \label{T:principal ideal theorem}
Let $L$ be the Hilbert class field of the number field $K$. Then
every ideal of $K$ becomes principal in $L$.
\end{theorem}
(Warning: this does not mean that $L$ has class number 1!)
Example: if $K = \QQ(\sqrt{-5})$, then $L = \QQ(\sqrt{-5}, \sqrt{-1})$,
and the nonprincipal ideal class of $K$ is represented by $(2, 1+\sqrt{-5})$,
which is generated by $1+\sqrt{-1}$ in $L$.

The idea is that given Artin reciprocity, this reduces to a question
in group theory. Namely, let $M$ be the Hilbert class field of $L$; 
then an ideal of $L$ is principal if and only if its image under the Artin
map $I_L \to \Gal(M/L)$ is trivial. So what we need is to find a
map $V$ such that
\[
\xymatrix{
\Cl(K) \ar[r] \ar[d] & \Gal(L/K) \ar^{V}[d] \\
\Cl(L) \ar[r] & \Gal(M/L)
}
\]
commutes, then show that $V$ is the zero map. (The horizontal arrows are the
Artin maps.)

Regarding the relationship between $K$, $L$ and $M$:
\begin{enumerate}
\item[(a)] $M$ is Galois over $K$ (because its image under any element of
$\Gal(\overline{K}/K)$ is still an abelian extension of $L$ unramified
at all finite and infinite places,
and so is contained in $M$) and unramified everywhere
(since $M/L$ and $L/K$ are unramified);
\item[(b)] $L$ is the maximal subextension of $M/K$ which is abelian over $K$
(since any abelian subextension is unramified over $K$, and so is contained in $L$).
\end{enumerate}
Given a finite group $G$, let $G^{\ab}$ denote the maximal abelian quotient
of $G$; that is, $G^{\ab}$ is the quotient of $G$ by its commutator subgroup
$G'$. Then (b) implies that $\Gal(M/L)$ is the commutator subgroup
of $\Gal(M/K)$ and $\Gal(M/K)^{\ab} = \Gal(L/K)$.

Before returning to the principal ideal theorem, we need to do a bit of
group theory.
Let $G$ be a finite group and $H$ a subgroup (but not necessarily
normal!).
Let $g_1, \dots, g_n$ be
left
coset representatives of $H$ in $G$: that is, $G = g_1H \cup \cdots \cup g_nH$.
For $g \in G$, put $\phi(g) = g_i$ if $g \in g_iH$ (i.e.,
$g_i^{-1}g \in H$). Put
\[
V(g) = \prod_{i=1}^n \phi(gg_i)^{-1}(gg_i);
\]
then $V(g)$ always lands in $H$. In particular, it induces a map
$V: G \to H^{\ab}$.
\begin{theorem} \label{T:transfer homomorphism}
The map $V: G \to H^{\ab}$ is a homomorphism, does not depend on the choice
of the $g_i$, and induces a homomorphism $G^{\ab} \to H^{\ab}$ (i.e.,
kills commutators in $G$).
\end{theorem}
The map $G^{\ab} \to H^{\ab}$ is called the \emph{transfer map} (in German,
``Verlagerung'', hence the $V$).

Now let's return to that diagram:
\[
\xymatrix{
\Cl(K) \ar[r] \ar[d] & \Gal(L/K) = \Gal(M/K)^{\ab} \ar^{V}[d] \\
\Cl(L) \ar[r] & \Gal(M/L) = \Gal(M/L)^{\ab}
}
\]
and show that the transfer map $V$ does actually make this diagram commute;
it's enough to check this when we stick a prime $\gothp$ of $K$ in at the
top left. For consistency of notation, put $G = \Gal(M/K)$ and $H
= \Gal(M/L)$, so that $G/H = \Gal(L/K)$.
Choose a prime $\gothq$ of $L$ over $\gothp$ and a prime $\gothr$ of $M$
over $\gothq$, let $G_{\gothr} \subseteq G$
be the decomposition group of $\gothr$
over $K$ (i.e., the set of automorphisms mapping $\gothr$ to itself)
and let $g \in G_{\gothr}$ be the Frobenius of $\gothr$.
(Note: since $G$ is not abelian, $g$ depends on the choice of $\gothr$,
not just on $\gothq$. That is, there's no Artin map into $G$.)

Let $\gothq_1, \dots, \gothq_r$ be the primes
of $L$ above $\gothp$; then the image of $\gothp$ in $L$ is
$\prod_i \gothq_i$, and the image of that product under the Artin map
is $\prod_i \Frob_{M/L}(\gothq_i)$. To show that this equals $V(g)$, we make a careful choice of the coset representatives $g_i$ in the
definition of $V$. Namely, decompose $G$ as a union of double cosets
$G_{\gothr} \tau_i H$. Then the primes of $L$ above $\gothp$ correspond
to these double cosets, where the double coset $G_{\gothr} \tau_i H$
corresponds to $L \cap \gothr^{\tau_i}$.
Let $m$ be the order of $\Frob_{L/K}(\gothp)$
and write $G_{\gothr} \tau_i H = \tau_iH
\cup g\tau_i H \cup \cdots \cup g^{m-1}\tau_i H$
for each $i$; we then use the elements $g_{ij} = g^j \tau_i$ as the left
coset representatives to define $\phi$
and $V$.
Thus the equality $V(g) = \prod_i \Frob_{M/L}(\gothq_i)$ follows
from the following lemma.
\begin{lemma} \label{L:transfer Frobenius}
If $L \cap \gothr^{\tau_i} = \gothq_i$, then
$\Frob_{M/L}(\gothq_i) = \prod_{j=0}^{m-1} \phi(g g_{ij})^{-1} g g_{ij} $.
\end{lemma}

Thus the principal ideal theorem now follows from the following fact.
\begin{theorem} \label{T:transfer vanishes}
Let $G$ be a finite group and $H$ its commutator subgroup. Then the transfer
map $V: G^{\ab} \to H^{\ab}$ is zero.
\end{theorem}

\head{Exercises}

\begin{enumerate}
\item
Prove Theorem~\ref{T:transfer homomorphism}. (Hint: one approach to proving independence from choices
is to change one $g_i$ at a time. Also, notice that $\phi(gg_1), \dots,
\phi(gg_n)$ are a permutation of $g_1, \dots, g_n$.)
\item
Prove Lemma~\ref{L:transfer Frobenius}.
(Hint: see Neukirch, Proposition IV.5.9.)
\item
With notation as in Theorem~\ref{T:transfer homomorphism},
let $\ZZ[G]$ be the group algebra of $G$ (formal linear combinations
$\sum_{g \in G} n_g [g]$ with $n_g \in \ZZ$, multiplied by putting
$[g][h] = [gh]$) and let $I_G$ be the ideal of sums $\sum n_g[g]$ with
$\sum n_g = 0$ (called the \emph{augmentation ideal}; see
Chapter~\ref{chap:homology}). Let 
\[
\delta: H/H' \to (I_{H}+I_GI_{H})/I_GI_{H}
\]
be the homomorphism taking the class of $h$ to the class of $[h]-1$.
Prove that $\delta$ is an isomorphism. (Hint:
show that the elements 
\[
[g]([h]-1) \qquad \mbox{for $g \in \{g_1,\dots,g_n\}, h \in H$}
\]
form a basis of $I_H + I_G I_H$ as a $\ZZ$-module. For more clues, see Neukirch, Lemma VI.7.7.)
\item
With notation as in the previous exercise, prove that the following diagram commutes:
\[
\xymatrix{
G/G' \ar^{V}[r] \ar^{\delta}[d] & H/H' \ar^{\delta}[d] \\
I_G/I_G^2 \ar^(.3){S}[r] & (I_{H}+I_GI_{H})/I_GI_{H},
}
\]
where $S$
is given by $S(x) = x ([g_1] + \cdots + [g_n])$.
\item
Prove Theorem~\ref{T:transfer vanishes}.
(Hint: quotient by the commutator subgroup of $H$ to reduce
to the case where $H$ is abelian.
Apply the classification of finite abelian groups
to write $G/H$ as a product of cyclic groups $\ZZ/e_1 \ZZ \times \cdots \times \ZZ/e_m \ZZ$.
Let $f_i$ be an element of $G$ lifting a generator of $\ZZ/e_i \ZZ$
and put $h_i = f_i^{-e_i} \in H$; then $0 = \delta(f_i^{e_i} h_i)$,
which can be rewritten as $\delta(f_i) \mu_i$ for some $\mu_i \in \ZZ[G]$
congruent to $e_i$ modulo $I_G$.
Now check that 
\[n
\mu_1 \cdots \mu_m \equiv [g_1] + \cdots + [g_n] \pmod{I_H \ZZ[G]}.
\]
For more details, see
Neukirch, Theorem VI.7.6.)
\end{enumerate}

%\end{document}




\chapter{Zeta functions and the Chebotarev density theorem}
%\documentclass[12pt]{article}
%\usepackage{amsfonts, amsthm, amsmath}
%
%\setlength{\textwidth}{6.5in}
%\setlength{\oddsidemargin}{0in}
%\setlength{\textheight}{8.5in}
%\setlength{\topmargin}{0in}
%\setlength{\headheight}{0in}
%\setlength{\headsep}{0in}
%\setlength{\parskip}{0pt}
%\setlength{\parindent}{20pt}
%
%\def\AA{\mathbb{A}}
%\def\CC{\mathbb{C}}
%\def\FF{\mathbb{F}}
%\def\PP{\mathbb{P}}
%\def\QQ{\mathbb{Q}}
%\def\RR{\mathbb{R}}
%\def\ZZ{\mathbb{Z}}
%\def\gotha{\mathfrak{a}}
%\def\gothb{\mathfrak{b}}
%\def\gothm{\mathfrak{m}}
%\def\gotho{\mathfrak{o}}
%\def\gothp{\mathfrak{p}}
%\def\gothq{\mathfrak{q}}
%\DeclareMathOperator{\disc}{Disc}
%\DeclareMathOperator{\Gal}{Gal}
%\DeclareMathOperator{\GL}{GL}
%\DeclareMathOperator{\Hom}{Hom}
%\DeclareMathOperator{\Norm}{Norm}
%\DeclareMathOperator{\Real}{Re}
%\DeclareMathOperator{\Trace}{Trace}
%\DeclareMathOperator{\Cl}{Cl}
%
%\def\head#1{\medskip \noindent \textbf{#1}.}
%
%\newtheorem{theorem}{Theorem}
%\newtheorem{lemma}[theorem]{Lemma}
%
%\begin{document}
%
%\begin{center}
%\bf
%Math 254B, UC Berkeley, Spring 2002 (Kedlaya) \\
%Zeta Functions and the Chebotarev Density Theorem
%\end{center}

\head{Reference} Lang, \textit{Algebraic Number Theory}, Chapter VIII for starters; see also Milne, Chapter VI and
Neukirch, Chapter VII. For advanced reading, see
Tate's thesis (last chapter of Cassels-Frohlich), but wait until
we introduce the adeles.

\medskip
Although this is supposed to be a course on algebraic number theory, the
following analytic discussion is so fundamental that we must at least allude
to it here.

Let $K$ be a number field. The \emph{Dedekind zeta function} $\zeta_K(s)$
is a function on the complex plane given, for $\Real(s) > 1$,
by the absolutely convergent product and sum
\[
\zeta_K(s) = \prod_\gothp (1 - \Norm(\gothp)^{-s})^{-1}
=
\zeta_K(s) = \sum_{\gotha} \Norm(\gotha)^{-s},
\]
where in the sum $\gotha$ runs over the nonzero ideals of $\gotho_K$.

A fundamental fact about the zeta function is the following. We omit the proof.
\begin{theorem} \label{T:meromorphic continuation}
The function $\zeta_K(s)$ extends to a meromorphic function on $\CC$
whose only pole
is a simple pole at $s=1$ of residue $1$.
\end{theorem}
The case $K=\QQ$ is of course the famous Riemann zeta function.
There is also a functional equation relating the values of $\zeta_K$ at
$s$ and $1-s$, and an extended Riemann hypothesis: aside from ``trivial''
zeros along the negative real axis, the zeroes of $\zeta_K$ all have
real part $1/2$.

More generally, let $\gothm$ be a formal product of places of $K$,
and let $\chi_\gothm: \Cl^\gothm(K) \to \CC^*$ be a character of the
ray class group of conductor $\gothm$. Extend $\chi_\gothm$ to a function
on all ideals of $K$ by declaring its value to be 0 on ideals not coprime
to $\gothm$. Then we define the $L$-function
\[
L(s, \chi_\gothm) = \prod_{\gothp \not| \gothm} (1 - \chi(\gothp) \Norm(\gothp)^{-s})^{-1}
= \sum_{(\gotha, \gothm) = 1} \chi(\gotha) \Norm(\gotha)^{-s}.
\]
Then we have another basic fact whose proof we also omit.
\begin{theorem} \label{T:analytic continuation}
If $\chi_\gothm$ is not trivial, then
$L(s, \chi_\gothm)$ extends to an analytic function on $\CC$.
\end{theorem}
If $\chi_\gothm$ is trivial, then $L(s, \chi_\gothm)$ is just the Dedekind
zeta function with the Euler factors for primes dividing $\gothm$ removed,
so it still has a pole at $s=1$. 

\begin{theorem} \label{T:nonvanishing of L}
If $\chi_\gothm$ is not the trivial character, then
$L(1, \chi_\gothm) \neq 0$.
\end{theorem}
This is already a nontrivial, but important result over $\QQ$. It implies
Dirichlet's famous theorem that there are infinitely many primes in
arithmetic progression, by implying that for any nontrivial $\chi_\gothm$,
$\sum_{\gothp} \chi(\gothp) \Norm(\gothp)^{-s}$ remains bounded as
$s \to \infty$. In fact, we say that a set of primes $S$ in a number field
$K$ has Dirichlet density $d$ if
\[
\lim_{s \to 1^+} \frac{\sum_{\gothp \in S} \Norm(\gothp)^{-s}}{\log \frac{1}{s-1}} = d.
\]
Then the fact implies that the Dirichlet density of the
set of primes congruent to $a$ modulo $m$ (assuming $a$ is coprime to $m$)
is $1/\phi(m)$.

The fact also implies that for any number field $K$ and any formal
product of places $\gothm$,
there are infinitely many primes in each
class of the ray class group of conductor $\gothm$, the set of such 
primes having Dirichlet density
$1/\#\Cl^\gothm(K)$. (Proof: see exercises.)

Finally, we point out a result of class field theory that also applies
to nonabelian extensions. Recall that if $L/K$ is any Galois extension
of number fields with Galois group $G$,
$\gothp$ is a prime of $K$, and $\gothq$ is a prime
above $\gothp$ which is unramified, then there is a well-defined Frobenius
associated to $\gothq$ (it's the element $g$ of the decomposition group
$G_{\gothq}$ such that $x^g \equiv x^{\#(\gotho_K/\gothp)} \pmod{\gothq}$);
but as a function of $\gothp$, this Frobenius is only well-defined up
to conjugation in $G$.
\begin{theorem}[Chebotarev Density Theorem]
Let $L/K$ be an arbitrary Galois extension of number fields,
with Galois group $G$. Then for any $g \in G$, there exist infinitely many
primes $\gothp$ of $K$ such that there is a prime $\gothq$ of $L$ above
$\gothp$ with Frobenius $g$. In fact, the Dirichlet density of such 
primes $\gothp$ is the order of the conjugacy class of $G$ divided by
$\#G$.
\end{theorem}
\begin{proof}
This follows from everything we have said so far, plus Artin reciprocity,
in case $L/K$ is abelian. In the general case, let $f$ be the order of
$g$, and let $K'$ be the fixed
field of $g$; then we know that the set of primes of $K'$ with Frobenius
$g \in \Gal(L/K') \subset G$ has Dirichlet density $1/f$. The same is true
if we restrict to primes of absolute degree 1 (see exercises).

Let $Z$ be the centralizer of $g$ in $G$; that is, $Z = \{z \in G:
zg = gz\}$. Then for each prime of $K$ (of absolute degree 1)
with Frobenius in the conjugacy
class of $g$, there are $\#Z/f$ primes of $K'$ above it (also
of absolute degree 1) with Frobenius $g$.
(Say $\gothp$ is such a prime and $\gothq$ is a prime of $L$ above $\gothp$
with Frobenius $g$. Then for $h \in G$, the Frobenius of $\gothq^h$
is $hgh^{-1}$, so the number of primes $\gothq$ with Frobenius $g$ is
$\#Z$. But each prime of $L'$ below one of these is actually below
$f$ of them.)
Thus the density of primes of $K$ with Frobenius in the conjugacy class
of $g$ is $(1/f)(1/(\#Z/f)) = 1/\#Z$. To conclude, note that the order
of the conjugacy class of $G$ is $\#G/\#Z$.
\end{proof}

\head{Exercises}

\begin{enumerate}
\item
Show that the Dirichlet density of the set of all primes of a
number field is 1.
\item
Show that in any number field,
the Dirichlet density of the set of primes $\gothp$ of absolute
degree greater than 1 is zero. 
\item
Let $\gothm$ be a formal product of places of the number field $K$.
Using Theorems~\ref{T:meromorphic continuation}, \ref{T:analytic continuation},
 and~\ref{T:nonvanishing of L}, prove that the set of primes of $K$
lying in any specified class of the ray class group of conductor
$\gothm$ is $1/\#\Cl^\gothm(K)$. (Hint: combine the quantities 
$\sum_{\gothp} \chi(\gothp) \Norm(\gothp)^{-s}$ to cancel
out all but one class.)
\end{enumerate}

%\end{document}




\part{Cohomology of groups}

\chapter{Cohomology of finite groups I: abstract nonsense}
%\documentclass[12pt]{article}
%\usepackage{amsfonts, amsthm, amsmath}
%\usepackage[all]{xy}
%
%\setlength{\textwidth}{6.5in}
%\setlength{\oddsidemargin}{0in}
%\setlength{\textheight}{8.5in}
%\setlength{\topmargin}{0in}
%\setlength{\headheight}{0in}
%\setlength{\headsep}{0in}
%\setlength{\parskip}{0pt}
%\setlength{\parindent}{20pt}
%
%\def\AA{\mathbb{A}}
%\def\CC{\mathbb{C}}
%\def\FF{\mathbb{F}}
%\def\PP{\mathbb{P}}
%\def\QQ{\mathbb{Q}}
%\def\RR{\mathbb{R}}
%\def\ZZ{\mathbb{Z}}
%\def\gotha{\mathfrak{a}}
%\def\gothb{\mathfrak{b}}
%\def\gothm{\mathfrak{m}}
%\def\gotho{\mathfrak{o}}
%\def\gothp{\mathfrak{p}}
%\def\gothq{\mathfrak{q}}
%\def\gothr{\mathfrak{r}}
%\DeclareMathOperator{\ab}{ab}
%\DeclareMathOperator{\coker}{coker}
%\DeclareMathOperator{\disc}{Disc}
%\DeclareMathOperator{\Frob}{Frob}
%\DeclareMathOperator{\Gal}{Gal}
%\DeclareMathOperator{\GL}{GL}
%\DeclareMathOperator{\Hom}{Hom}
%\DeclareMathOperator{\im}{im}
%\DeclareMathOperator{\Ind}{Ind}
%\DeclareMathOperator{\Norm}{Norm}
%\DeclareMathOperator{\Trace}{Trace}
%\DeclareMathOperator{\Cl}{Cl}
%
%\def\head#1{\medskip \noindent \textbf{#1}.}
%
%\newtheorem{theorem}{Theorem}
%\newtheorem{lemma}[theorem]{Lemma}
%
%\begin{document}
%
%\begin{center}
%\bf
%Math 254B, UC Berkeley, Spring 2002 (Kedlaya) \\
%The Cohomology of Finite Groups I: Abstract Nonsense
%\end{center}

\head{Reference} Milne, II.1. See Serre, \textit{Galois Cohomology} for
a much more general presentation. (We will generalize ourselves from finite
to profinite groups a bit later on.) Warning: some authors (like Milne, and
Neukirch for the most part) put
group actions on the left and some (like Neukirch in chapter IV, and myself here)
put them on the right. Of course, the theory is the same either way!

\head{Caveat} This material may seem a bit dry. If
so, don't worry; only a small part of the theory will be relevant for
class field theory. However, it doesn't make sense to learn that small
part without knowing what it is a part of!

\medskip
Let $G$ be a finite group and $A$ an abelian group (itself not necessarily
finite) with a right $G$-action,
also known as a $G$-module. I'll write the $G$-action as a superscript,
i.e., the image of the action of $g$ on $m$ is $m^g$.
Alternatively, $A$ can be viewed as a right
module for the group algebra $\ZZ[G]$. A \emph{homomorphism} of $G$-modules
$\phi: M \to N$ is a homomorphism of abelian groups that is compatible
with the $G$-actions: i.e., $\phi(m^g) = \phi(m)^g$. (For those keeping
score, the category of $G$-modules is an abelian category.)

We would like to define some invariants of the pair $(G, A)$
that we can use to get information about $G$ and $A$. We will use
the general methodology of homological algebra to do this.
Before doing so, though,
we need a few lemmas about $G$-modules.

A $G$-module $M$ is \emph{injective} if for every inclusion $A \subset B$
of $G$-modules and every $G$-module homomorphism $\phi: A \to M$, there
is a homomorphism $\psi: B \to M$ that extends $\phi$.

\begin{lemma} \label{L:enough injectives}
Every $G$-module can be embedded into some injective $G$-module. (That is,
the category of $G$-modules has enough injectives.)
\end{lemma}
\begin{proof}
Exercise.
\end{proof}

In particular, any $G$-module $M$ admits an \emph{injective resolution}:
a complex
\[
0 \to M \to I_0 \stackrel{d_0}{\to} I_1 \stackrel{d_1}{\to} I_2
\stackrel{d_2}{\to} \dots
\]
(that is, $d_{i+1} \circ d_i = 0$ for all $i$)
in which each $I_i$ is injective and the complex itself is \emph{exact}:
$\im d_i = \ker d_{i+1}$.
(To wit, embed $M$ into $I_0$, embed $I_0/M$ into $I_1$, et cetera.)

Given a $G$-module $M$, let $M^G$ be the abelian group of $G$-invariant
elements of $M$:
\[
M^G = \{m \in M: m^g = m \quad \forall g \in G\}.
\]
The functor $M \to M^G$ from $G$-modules to abelian groups is left exact but 
not right exact: if $0 \to M' \to M \to M'' \to 0$ is an exact sequence,
then $0 \to (M')^G \to M^G \to (M'')^G$ is exact, but $M^G \to (M'')^G$ may
not be exact. (Example: take the sequence $0 \to \ZZ/p\ZZ \to \ZZ/p^2\ZZ
\to \ZZ/p\ZZ \to 0$ of $G$-modules for $G = \ZZ/p\ZZ$, which acts on the middle
factor by $a^g = a(1+pg)$. Then $M^G \to (M'')^G$ is the zero map but
$(M'')^G$ is nonzero.)

This is the general situation addressed by homological algebra: it provides
a canonical way to extend the truncated exact sequence
$0 \to (M')^G \to M^G \to (M'')^G$. (Or if you prefer, it helps measure the
failure of exactness of the $G$-invariants functor.) To do this, given
$M$ and an injective resolution as above, take $G$-invariants:
the result
\[
0 \to I_0^G \stackrel{d_0}{\to} I_1^G \stackrel{d_1}{\to} I_2^G
\stackrel{d_2}{\to} \dots
\]
is still a complex, but no longer exact. We turn this failure into success
by defining the $i$-th cohomology group as the quotient
\[
H^i(G, M) = \ker(d^i)/\im(d^{i-1}).
\]
By convention, we let $d_{-1}$ be the map $0 \to I_0^G$, so
$H^0(G,M)=M^G$.

Given a homomorphism $f: M \to N$ and a
injective resolution $0 \to N \to J_0 \to J_1 \to \cdots$, there exists a
commutative diagram
\[
\xymatrix{ 0 \ar[r] & M \ar[r] \ar_f[d] & I_0 \ar^{d_0}[r] \ar_{f_0}[d] &
 I_1  \ar_{f_1}[d] \ar^{d_1}[r] & I_2 \ar_{f_2}[d] \ar^{d_2}[r] &
\cdots \\
0 \ar[r] & N \ar[r] & J_0 \ar^{d_0}[r] & J_1 \ar^{d_1}[r] & 
J_2 \ar^{d_2}[r] &
\cdots
}
\]
and likewise after taking $G$-invariants, so we get maps
$H^i(f): H^i(G, M) \to H^i(G, N)$.
\begin{lemma} \label{L:injective resolutions to cohomology maps}
The map $H^i(f)$ does not depend
on the choice of the $f_i$ (given the choices
of injective resolutions).
\end{lemma}
\begin{proof}
This proof is a bit of ``abstract nonsense''. It suffices to check that if
$f=0$, then the $H^i(f)$ are all zero regardless of what the $f_i$ are.
In that case, it turns out one can
construct maps $g_i: I_{i+1} \to J_i$ (and by convention $g_{-1} = 0$)
such that
$f_i = g_i \circ d_i  + d_{i-1} \circ g_{i-1}$. (Such a set of maps is
called a \emph{homotopy}.) Details left as an exercise. (Warning: the diagonal
arrows in the diagram below don't commute!)
\[
\xymatrix{ 0 \ar[r] & M \ar[r] \ar_f[d] & I_0 \ar^{d_0}[r] \ar_{f_0}[d] &
 I_1  \ar_{g_0}[dl] \ar_{f_1}[d] \ar^{d_1}[r] & \ar_{g_1}[dl] I_2 
\ar_{f_2}[d] \ar^{d_2}[r] &
\cdots \\
0 \ar[r] & N \ar[r] & J_0 \ar^{d_0}[r] & J_1 \ar^{d_1}[r] & 
J_2 \ar^{d_2}[r] &
\cdots
}
\]
\end{proof}
In particular, if $M=N$ and $f$ is the identity, we get a canonical map
between $H^i(G,M)$ and $H^i(G,N)$ for each $i$. That is, the groups
$H^i(G,M)$ are well-defined independent of the choice of the injective
resolution. Likewise, the map $H^i(f)$ is also independent of the choice
of resolutions.

If you know any homological algebra, you'll recognize what comes next:
given a short exact sequence $0 \to M' \to M \to M'' \to 0$ of $G$-modules,
there is a canonical long exact sequence
\begin{gather*}
0 \to H^0(G, M') \to \cdots \to H^i(G, M'') \stackrel{\delta_i}{\to}
\\
\stackrel{\delta_i}{\to} H^{i+1}(G, M') \to H^{i+1}(G, M) \to H^{i+1}(G,M'') \to \cdots,
\end{gather*}
where the $\delta_i$ are certain ``connecting homomorphisms'' (or ``snake
maps''). I won't punish you with the proof of this; if you've never seen
it before, deduce it yourself from the Snake Lemma. (For the proof of the
latter, engage in ``diagram chasing'', or see the movie \emph{It's My Turn}.
To define $\delta$: given $x \in \ker(f_2) \subseteq M_2$, lift $x$ to
$M_1$, push it into $N_1$ by $f_1$, then check that the image has
a preimage in $N_0$. Then verify that the result is well-defined, et
cetera.)
\begin{lemma}[Snake Lemma] \label{L:snake lemma}
Given a commuting diagram
\[
\xymatrix{
0 \ar[r] & M_0 \ar[r] \ar^{f_0}[d] & M_1 \ar[r] \ar^{f_1}[d] & M_2 \ar[r] \ar^{f_2}[d]
& 0 \\
0 \ar[r] & N_0 \ar[r] & N_1 \ar[r] & N_2 \ar[r] & 0
}
\]
in which the rows are exact, there is a canonical map $\delta:
\ker(f_2) \to \coker(f_0)$ such that
the sequence
\[
0 \to \ker(f_0) \to \ker(f_1) \to \ker(f_2) \stackrel{\delta}{\to}
\coker(f_0) \to \coker(f_1) \to \coker(f_2) \to 0
\]
is exact.
\end{lemma}
One important consequence of the long exact sequence is that if
$0 \to M' \to M \to M'' \to 0$ is a short exact sequence of $G$-modules
and $H^1(G, M') = 0$, then $0 \to (M')^G \to M^G \to (M'')^G \to 0$ is
also exact.

More abstract nonsense:
\begin{itemize}
\item
If $0 \to M' \to M \to M'' \to 0$ is a short exact sequence of $G$-modules
and $H^i(G, M) = 0$ for all $i>0$, then the connecting homomorphisms
in the long exact sequence induce isomorphisms $H^i(G, M'') \to
H^{i+1}(G, M')$ for all $i > 0$ (and a surjection for $i=0$). This sometimes allows one to prove
general facts by proving them first for $H^0$, where they have a direct
interpretation, then ``dimension shifting''; however, getting from $H^0$ to $H^1$ typically requires some extra attention.
\item If $M$ is an injective $G$-module, then $H^i(G,M) = 0$ for all 
$i>0$. (Use $0 \to M \to M \to 0 \to \cdots$ as an injective resolution.)
This fact has a sort of converse: see next bullet.
\item We say $M$ is \emph{acyclic} if $H^i(G,M) =0$ for all $i>0$;
so in particular, injective $G$-modules are acyclic.
It turns out that we 
can replace the injective resolution in the definition by an acyclic resolution
for the purposes of doing a computation; see exercises.
\end{itemize}

Of course, the abstract nature of the proofs so far gives us almost no
insight into what the objects are that we've just constructed. We'll remedy
that next time by giving more concrete descriptions that one can actually
compute with.

\head{Exercises}

\begin{enumerate}
\item
Let $G$ be the one-element group. Show that a $G$-module 
(i.e., abelian group) is injective if and
only if it is divisible, i.e., the map $x \mapsto nx$ is surjective
for any nonzero integer $n$. (Hint: you'll need Zorn's lemma or equivalent
in one direction.)
\item
Let $A$ be an abelian group, regarded as a $G$-module for $G$ the trivial
group. Prove that $A$ can be embedded in an injective $G$-module.
\item
Prove Lemma~\ref{L:enough injectives}. (Hint: 
for $M$ a $G$-module, the previous exercises show that the underlying abelian group of $M$ embeds into a divisible group $N$. Now map
$M$ into $\Hom_{\ZZ}(\ZZ[G], N)$ and check that the latter is an injective $G$-module.)
\item
Prove Lemma~\ref{L:injective resolutions to cohomology maps}, following the sketch given. (Hint: construct $g_i$ given
$f_{i-1}$ and $g_{i-1}$, using that the $J$'s are injective $G$-modules.)
\item
Prove that if $0 \to M \to M_0 \to M_1 \to \cdots$ is an exact sequence of $G$-modules
and each $M_i$ is acyclic, then the cohomology groups of the complex
$0 \to M_0^G \to M_1^G \to \cdots$ coincide with $H^i(G, M)$. 
(Hint: construct the canonical long exact sequence from the
exact sequence 
\[
0 \to M \to M_0 \to M_0/M \to 0,
\]
then do dimension shifting using the fact that
\[
0 \to M_0/M \to M_1 \to M_2 \to \cdots
\]
is again exact. Don't forget to be careful about $H^1$!)
\end{enumerate}

%\end{document}




\chapter{Cohomology of finite groups II: concrete nonsense}
%\documentclass[12pt]{article}
%\usepackage{amsfonts, amsthm, amsmath}
%\usepackage[all]{xy}
%
%\setlength{\textwidth}{6.5in}
%\setlength{\oddsidemargin}{0in}
%\setlength{\textheight}{8.5in}
%\setlength{\topmargin}{0in}
%\setlength{\headheight}{0in}
%\setlength{\headsep}{0in}
%\setlength{\parskip}{0pt}
%\setlength{\parindent}{20pt}
%
%\def\AA{\mathbb{A}}
%\def\CC{\mathbb{C}}
%\def\FF{\mathbb{F}}
%\def\PP{\mathbb{P}}
%\def\QQ{\mathbb{Q}}
%\def\RR{\mathbb{R}}
%\def\ZZ{\mathbb{Z}}
%\def\gotha{\mathfrak{a}}
%\def\gothb{\mathfrak{b}}
%\def\gothm{\mathfrak{m}}
%\def\gotho{\mathfrak{o}}
%\def\gothp{\mathfrak{p}}
%\def\gothq{\mathfrak{q}}
%\def\gothr{\mathfrak{r}}
%\DeclareMathOperator{\ab}{ab}
%\DeclareMathOperator{\coker}{coker}
%\DeclareMathOperator{\disc}{Disc}
%\DeclareMathOperator{\Frob}{Frob}
%\DeclareMathOperator{\Gal}{Gal}
%\DeclareMathOperator{\GL}{GL}
%\DeclareMathOperator{\Hom}{Hom}
%\DeclareMathOperator{\id}{id}
%\DeclareMathOperator{\im}{im}
%\DeclareMathOperator{\Ind}{Ind}
%\DeclareMathOperator{\Norm}{Norm}
%\DeclareMathOperator{\Trace}{Trace}
%\DeclareMathOperator{\Cl}{Cl}
%
%\def\head#1{\medskip \noindent \textbf{#1}.}
%
%\newtheorem{theorem}{Theorem}
%\newtheorem{lemma}[theorem]{Lemma}
%\newtheorem{cor}[theorem]{Corollary}
%
%\begin{document}
%
%\begin{center}
%\bf
%Math 254B, UC Berkeley, Spring 2002 (Kedlaya) \\
%The Cohomology of Finite Groups II: Concrete Nonsense
%\end{center}

\head{Reference} Milne, II.1.

\medskip
In the previous chapter, we associated to a finite group $G$ and a (right) $G$-module
$M$ a sequence of abelian groups $H^i(G, M)$, called the cohomology
groups of $M$. (They're also called the \emph{Galois cohomology} groups because in
number theory, $G$ will invariably be the Galois group of some extension
of number fields, and $A$ will be some object manufactured from this
extension.) What we didn't do is make the construction at all usable in
practice! This time we will remedy this.

Recall the last point (and the last exercise) from the last chapter: if 
\[
0 \to M \to M_0 \to M_1 \to \cdots
\]
is an acyclic resolution of $M$ (i.e., the sequence is exact, and 
$H^i(G, M_j) = 0$ for $i > 0$ and all $j$), then
\[
H^i(G, M) = \ker(M_i^G \to M_{i+1}^G)/\im(M_{i-1}^G \to M_i^G).
\]
Thus to compute cohomology, we are going to need an ample supply of
acyclic $G$-modules. We will get these using a process known as \emph{induction}.
By way of motivation, we note first that if $G$ is the trivial group, 
\emph{every} $G$-module is acyclic: if $0 \to M \to I_0 \to I_1 \cdots$ is
an injective resolution, taking $G$-invariants has no effect, so
$0 \to I_0 \to I_1 \to \cdots$ is still exact except at $I_0$
(where we omitted $M$).

If $H$ is a subgroup of $G$ and $M$ is an $H$-module, 
we define the \emph{induced} $G$-module associated to $M$ to be 
$\Ind^G_H M = M \otimes_{\ZZ[H]} \ZZ[G]$. We may also identify $\Ind^G_H M$ with the set of functions $\phi: G \to M$ such that $\phi(gh)
= \phi(g)^h$ for $h \in H$, with the $G$-action on the latter being given by 
$\phi^g(g') = \phi(gg')$: namely, the element $m \otimes [g] \in M \otimes_{\ZZ[H]} \ZZ[G]$ corresponds to the function $\phi_{m,g}$ taking $g'$ to $m^{gg'}$ if $gg' \in H$ and to 0 otherwise.

\begin{lemma}[Shapiro's lemma] \label{L:Shapiro}
  If $H$ is a subgroup of $G$ and $N$ is an $H$-module, then
there is a canonical isomorphism $H^i(G, \Ind^G_H N) \to H^i(H, N)$.
In particular, $N$ is an acyclic $H$-module if and only if
$\Ind^G_H(N)$ is an acyclic $G$-module.
\end{lemma}
\begin{proof}
The key points are:
\begin{enumerate}
\item[(a)] $(\Ind^G_H N)^G = N^H$, so there is an isomorphism for $i=0$
(this is clearest from the description using functions);
\item[(b)] the functor $\Ind^G_H$ from $H$-modules to $G$-modules
is exact (that is, $\ZZ[G]$ is flat over $\ZZ[H]$, which is easy to see
because it in fact is free over $\ZZ[H]$);
\item[(c)] if $I$ is an injective $H$-module, then $\Ind^G_H(I)$ is an
injective $G$-module. This follows from the existence of a  canonical
isomorphism $\Hom_G(M, \Ind^G_H I) = \Hom_H(M, I)$, for which see Proposition~\ref{P:adjoint property} below.
\end{enumerate}
Now take an injective resolution of $N$, apply $\Ind^G_H$ to it, and the
result is an injective resolution of $\Ind^G_H N$.
\end{proof}
We say $M$ is an \emph{induced} $G$-module if it has the form $\Ind^G_{1}
N$
for some abelian group $N$, i.e.,
it can be written as $N \otimes_\ZZ \ZZ[G]$. (The subscript 1 stands for the trivial group, since $G$-modules for $G = 1$ are just abelian groups.)


\begin{cor} \label{C:induced acyclic}
If $M$ is an induced $G$-module, then $M$ is acyclic.
\end{cor}

To complete the previous argument, we need an important property of induced modules.
\begin{prop} \label{P:adjoint property}
Let $H$ be a subgroup of $G$, let $M$ be a $G$-module, and let $N$ be an $H$-module. Then there are natural isomorphisms
\begin{align*}
\Hom_G(M, \Ind^G_H N) &\cong \Hom_H(M, N) \\
\Hom_G(\Ind^G_H N, M) &\cong \Hom_H(N,M).
\end{align*}
\end{prop}
In other words, the restriction functor from $G$-modules to $H$-modules and the induction functor from $H$-modules to $G$-modules form a pair of \emph{adjoint functors} in both directions. This is rather unusual; it is far more common to have such a relationship in only one direction.
\begin{proof}
To begin with, note that if we take $N = M$ (or more precisely, $N$ is a copy of $M$ with only the action of $H$ retained), then the identity map between $M$ and $N$ is supposed to correspond both to a homomorphism $M \to \Ind^G_H M$ and to a homomorphism $\Ind^G_H M \to M$. Let us write these maps down first: the map $\Ind^G_H M \to M$ is
\[
\sum_{g \in G} m_g \otimes [g] \mapsto \sum_{g \in G} (m_g)^g,
\]
while the map $M \to \Ind^G_H M$ is
\[
m \mapsto \sum_i m^{g_i} \otimes [g_i^{-1}]
\]
where $g_i$ runs over a set of left coset representatives of $H$ in $G$. Note that this second map doesn't depend on the choice of the representatives; for $g \in G$, we can use the coset representatives $gg_i$ instead, so the equality
\[
m^{g} \mapsto \sum_{i} m^{gg_i} \otimes [g_i^{-1}]
= \left( \sum_{i} m^{gg_i} \otimes [(g g_i)^{-1}] \right)[g]
\]
means that we do in fact get a map compatible with the $G$-actions.
(Note that the composition of these two maps is not the identity! For more on this point, see the discussion of extended functoriality in Chapter~\ref{chap:homology}.)

Now let $N$ be general. Given a homomorphism $M \to N$ of $H$-modules, we get a corresponding homomorphism
$\Ind^G_H M \to \Ind^G_H N$ of $G$-modules, which we can then compose with the above map $M \to \Ind^G_H M$ to get a homomorphism $M \to \Ind^G_H N$ of $G$-modules. We thus get a map
\[
\Hom_H(M,N) \to \Hom_G(M, \Ind^G_H N);
\]
to get the map in the other direction, start with a homomorphism $M \to \Ind^G_H N$, identify the target with functions $\phi: G \to N$, then compose with the map $\Ind^G_H N \to N$ taking $\phi$ to $\phi(e)$.

In the other direction, given a homomorphism $N \to M$ of $H$-modules, we get a corresponding homomorphism
$\Ind^G_H N \to \Ind^G_H M$ of $G$-modules, which we can then compose with the above map $\Ind^G_H M \to M$ to get a homomorphism $\Ind^G_H N \to M$ of $G$-modules. 
We thus get a map
\[
\Hom_H(N,M) \to \Hom_G(\Ind^G_H N,M);
\]
to get the map in the other direction, start with a homomorphism $\Ind^G_H N \to M$ of $G$-modules and evaluate it on $n \otimes [e]$ to get a homomorphism $N \to M$ of $H$-modules.
\end{proof}

The point of all of this is that it is much easier to embed $M$ into an acyclic $G$-module than into an injective $G$-module: use the map $M \to \Ind^G_1 M$ constructed in
Proposition~\ref{P:adjoint property}!
Immediate consequence: if $M$ is finite, it can be embedded into a
finite acyclic $G$-module, and thus $H^i(G,M)$ is finite for all $i$. (But
contrary to what you might expect, for fixed $M$, the groups $H^i(G,M)$
do not necessarily become zero for $i$ large, even if $M$ is finite! 
We'll see explicit examples next time.)

Another consequence is the following result. (The case $i=1$ was an exercise earlier.)
\begin{theorem} \label{T:additive theorem 90}
Let $L/K$ be a finite Galois extension of fields. Then
\[
H^i(\Gal(L/K), L) = 0 \mbox{ for all $i>0$.}
\]
\end{theorem}
\begin{proof}
Put $G = \Gal(L/K)$.
The normal basis theorem (see Lang, \emph{Algebra} or Milne, Lemma
II.1.24)
states that there exists $\alpha \in L$ whose conjugates form a basis of
$L$ as a $K$-vector space. This implies that $L \cong \Ind^{G}_{1}
K$, so $L$ is an induced $G$-module and so is acyclic.
\end{proof}

Now let's see an explicit way to compute group cohomology.
Given a group $G$ and a $G$-module $M$, define the $G$-modules $N_i$
for $i \geq 0$ as the set of functions $\phi: G^{i+1} \to M$, with the $G$-action
\[
(\phi^g)(g_0, \dots, g_i) = \phi(g_0g^{-1}, \dots, g_ig^{-1})^g.
\]
Notice that this module is induced: we have $N_i = \Ind_{1}^G N_{i,0}$
where $N_{i,0}$ is the subset of $N_i$ consisting of functions for which
$\phi(g_0, \dots, g_i) = 0$ when $g_0 \neq e$.

Define the map $d_i: N_i \to N_{i+1}$ by
\[
(d_i \phi)(g_0, \dots, g_{i+1})
= \sum_{j=0}^{i+1} (-1)^j \phi(g_0, \dots, \widehat{g_j}, \dots, g_{i+1}),
\]
where the hat over $g_j$ means you omit it from the list.
Then one checks that the sequence
\[
0 \to M \to N_0 \to N_1 \to \dots
\]
is exact. Since the $N_i$ are induced, this is an acyclic resolution:
thus the cohomology of the complex
\[
0 \to N_0^G \to N_1^G \to \cdots
\]
coincides with the cohomology groups $H^i(G, M)$. And now we have
something we can actually compute! (Terminology: the elements of
$N_i^G$ in the kernel of $d_i$ are called \emph{(homogeneous) $i$-cochains};
the ones in the image of $d_{i-1}$ are called \emph{$i$-coboundaries}.)

\head{Fun with $H^1$}

For example, we can give a very simple description of $H^1(G,M)$.
Namely, a 1-cochain $\phi: G^2 \to M$ is determined by $\rho(g) = \phi(e, g)$,
which by $G$-invariance satisfies the relation
\begin{align*}
0 &= (d_1\phi)(e, h, gh) \\
&= \phi(h, gh) - \phi(e, gh) + \phi(e, h) \\
&= (\phi^h)(h,gh) - \rho(gh) + \rho(h) \\
&= \phi(e, g)^h - \rho(gh) + \rho(h) \\
&= \rho(g)^h + \rho(h) - \rho(gh).
\end{align*}
It is the coboundary of a 0-cochain $\psi: G \to M$ if and only if
\[
\rho(g) = \phi(e,g) = \psi(g) - \psi(e) = \psi(e)^g - \psi(e).
\]
That is, $H^1(G,M)$ consists of crossed homomorphisms modulo principal
crossed homomorphisms, consistent with the definition
we gave in Chapter~\ref{chap:Kummer theory}. 

We may also interpret $H^1(G,M)$ as the set of isomorphism classes of \emph{principal homogeneous spaces} of $M$.
Such objects are sets $A$ with both a $G$-action and an $M$-action, subject to the
following restrictions:
\begin{enumerate}
\item[(a)] for any $a \in A$, the map $M \to A$ given by $m \mapsto m(a)$ is a bijection;
\item[(b)] for $a \in A$, $g \in G$ and $m \in M$, $m(a)^g = m^g(a)$ 
(i.e., the $G$-action and $M$-action commute).
\end{enumerate}
To define the associated class in $H^1(G,M)$, pick any $a \in A$, 
take the map $\rho: G \to M$ given by $\rho(g) = a^g - a$, and
let $\phi$ be the 1-cocycle with $\phi(e, g) = \rho(g)$.
The verification that this defines a bijection is left to the reader.
(For example, the identity in $H^1(G,M)$ corresponds to the trivial 
principal homogeneous space $A=M$, on which $G$ acts as it does on $M$ while $M$ acts by 
translation: $m(a) = m+a$.)

This interpretation of $H^1$ appears prominently in the theory of elliptic curves:
For example, if $L$ is a finite extension of $K$ and $E$ is an elliptic curve over
$E$, then $H^1(\Gal(L/K), E(\overline{K}))$ is the set of $K$-isomorphism
classes of curves whose Jacobians are $K$-isomorphic to $E$
(but which might not themselves be isomorphic to $E$ by virtue of not
having a $K$-rational point). 
For another example,  $H^1(\Gal(L/K), \Aut(E))$ parametrizes
twists of $E$, elliptic curves defined over $K$ which are $L$-isomorphic
to $E$. (E.g., $y^2 = x^3+x+1$ versus $2y^2 = x^3 + x +1$, with $L = \QQ(\sqrt{2})$.) See Silverman,
\textit{The Arithmetic of Elliptic Curves}, especially Chapter X,
for all this and more fun with $H^1$, including the infamous \emph{Selmer group}
and \emph{Tate-Shafarevich group}.

\head{Fun with $H^2$}

We can also give an explicit interpretation of $H^2(G,M)$ (see
Milne, example II.1.18(b)). It classifies
short exact sequences
\[
1 \to M \to E \to G \to 1
\]
of (not necessarily abelian) groups on which $G$ has a fixed action on $M$.
(The action is given as follows: given $g \in G$ and $m \in M$, choose $h
\in E$ lifting $G$; then $h^{-1}mh$ maps to the identity in $G$, so comes
from $M$, and we call it $m^g$ since it depends only on $g$.)
Namely, given the sequence, choose a map $s: G \to E$ (not a homomorphism)
such that $s(g)$ maps to $g$ under the map $E \to G$. Then the map
$\phi: G^3 \to M$ given by
\[
\phi(a,b,c) = s(a)^{-1} s(ba^{-1})^{-1} s(cb^{-1})^{-1} s(ca^{-1}) s(a)
\]
is a homogeneous 2-cocycle, and any two choices of $s$ give maps that differ by
a 2-coboundary.

What ``classifies'' means here is that two sequences give the same
element of $H^2(G,M)$ if and only if one can find an arrow $E \to E'$ making
the following diagram commute:
\[
\xymatrix{
1 \ar[r] & M \ar^{\id}[d] \ar[r] & E \ar[d] \ar[r] & G \ar^{\id}[d] \ar[r] &
1 \\
1 \ar[r] & M \ar[r] & E' \ar[r] & G \ar[r] & 1
}
\]
Note that two sequences may not be isomorphic under this definition even
if $E$ and $E'$ are abstractly isomorphic as groups. For example, if
$G = M = \ZZ/p\ZZ$ and the action is trivial, then $H^2(G, M) = \ZZ/p\ZZ$
even though there are only two possible groups $E$, namely $\ZZ/p^2\ZZ$
and $\ZZ/p\ZZ \times \ZZ/p\ZZ$.

\head{Extended functoriality}

We already saw that if we have a homomorphism of $G$-modules, we get
induced homomorphisms on cohomology groups.
 But what if we want to relate $G$-modules for different groups $G$,
as will happen in our study of class field theory? It turns out that in
a suitable sense, the cohomology groups are also functorial with respect to changing $G$.

Let $M$ be a $G$-module and $M'$ a $G'$-module. Suppose we are given
a homomorphism $\alpha: G' \to G$ of groups and a homomorphism
$\beta: M \to M'$ of abelian groups (note that they go in opposite
directions!). We say these are \emph{compatible} if
$\beta(m^{\alpha(g)}) = \beta(m)^g$ for all $g \in G$ and $m \in M$.
In this case, one gets canonical homomorphisms $H^i(G, M) \to H^i(G', M')$
(construct them on pairs of injective
resolutions, then show that any two choices are homotopic).

The principal examples are as follows.
\begin{enumerate}
\item[(a)]
Note that cohomology groups don't
seem to carry a nontrivial $G$-action, because you compute them by taking
invariants. This can be reinterpreted in terms of
extended functoriality: let $\alpha: G \to G$ be the conjugation by some
fixed $h$: $g \mapsto h^{-1}gh$, and let $\beta: M \to M$ be the map
$m \mapsto m^h$. Then the induced homomorphisms $H^i(G,M) \to H^i(G,M)$ are
all the identity map.

\item[(b)] If $H$ is a subgroup of $G$, $M$ is a $G$-module, and
$M'$ is just $M$ with all but the $H$-action forgotten, we get
the \emph{restriction homomorphisms}
\[
\Res: H^i(G, M) \to H^i(H, M).
\]
Another way to get the same map: 
use the adjunction homomorphism
$M \to \Ind^G_H M$ from Proposition~\ref{P:adjoint property} sending $m$ to $\sum_i m^{g_i} \otimes [g_i^{-1}]$, where
$g_i$ runs over a set of right coset representatives of $H$ in $G$,
then apply Shapiro's Lemma to get
\[
H^i(G, M) \to H^i(G, \Ind^G_H M) \stackrel{\sim}{\to} H^i(H, M).
\]
\item[(c)]
Take notation as in (a), but this time 
consider the map $\Ind^G_H M \to M$ taking $m \otimes [g]$ to $m^g$.
We then get maps
$H^i(G, \Ind^G_H M) \to H^i(G,M)$ which, together with the isomorphisms
of Shapiro's lemma, give what are called the \emph{corestriction
homomorphisms}:
\[
\Cor: H^i(H, M) \stackrel{\sim}{\to} H^i(G, \Ind^G_H M) \to H^i(G, M).
\]
\item[(d)]
The composition $\Cor \circ \Res$ is induced by the homomorphism of
$G$-modules $M \to \Ind^G_H M \to M$ given by
\[
m \mapsto \sum_i m^{g_i} \otimes [g_i^{-1}] \to \sum_i m = [G:H]m.
\]
Thus $\Cor \circ \Res$ acts as multiplication by $[G:H]$ on each 
(co)homology group. Bonus consequence (hereafter excluding the case of $H^0$):
if we take $H$ to be the trivial group,
then the group in the middle is isomorphic to $H^i(H, M) = 0$.
So every cohomology group for $G$ is killed by $\#G$, and in particular
is a torsion group. In fact, if $M$ is finitely generated as an abelian
group, this means
$H^i(G, M)$ is always finite, because each of these will be finitely generated
and torsion. (Of course, this won't happen in many of
our favorite examples, e.g., $H^i(\Gal(L/K), L^*)$ for $L$ and $K$ fields.)
\item[(e)]
If $H$ is a \emph{normal} subgroup of $G$, let $\alpha$ be the surjection
$G \to G/H$, and let $\beta$ be the injection $M^H \hookrightarrow M$.
Note that $G/H$ acts on $M^H$; in this case, we get the \emph{inflation
homomorphisms}
\[
\Inf: H^i(G/H, M^H) \to H^i(G, M).
\]
The inflation and restriction maps will interact in an interesting way; see
Proposition~\ref{P:inflation restriction}.
\end{enumerate}


\head{Exercises}

\begin{enumerate}
\item
Complete the proof of the correspondence between $H^1(G,M)$ and principal homogeneous spaces.
\item
The set $H^2(G,M)$ has the structure of an abelian group. Describe the
corresponding structure on short exact sequences $0 \to M \to E \to G \to 0$.
\item
Let $G = S_3$ (the symmetric group on three letters), let $M = \ZZ^3$
with the natural $G$-action permuting the factors, and let $N = M^G$.
Compute $H^i(G, M/N)$ for $i=1,2$ however you want: you can explicitly
compute cochains, use the alternate interpretations given above,
or use the exact sequence $0 \to N \to M \to M/N \to 0$. Better yet, use
more than one method and make sure that you get the same answer.
\item
(Artin-Schreier)
Let $L/K$ be a $\ZZ/p\ZZ$-extension of fields of characteristic $p>0$.
Prove that $L = K(\alpha)$ for some $\alpha$ such that
$\alpha^p - \alpha \in K$. (Hint: let $K^{\sep}$ be a separable closure of $K$ containing $L$, and consider the short exact sequence
$0 \to \FF_p \to K^{\sep} \to K^{\sep} \to 0$ in which the map $K^{\sep} \to K^{\sep}$
is given by $x \mapsto x^p - x$.)
%\item
%Check the \emph{adjoint property} of induction: for $G$ a finite group, $H$ a subgroup, $M$ a $G$-module, and $N$ an $H$-module, there is a canonical bijection 
%\[
%\Hom_H(N, M) \cong \Hom_G(\Ind^G_H N, M),
%\]
%where a map $f: N \to M$ corresponds to the map $\Ind^G_H N \to M$ taking $n \otimes [g]$ to $f(n)^g$. In particular, by taking $N = M$ we get a map $\Ind^G_H M \to M$; by contrast, the proof of Corollary~\ref{C:induced acyclic} gives a map in the other direction.
%We'll use both directions later; see the discussion of extended functoriality in 
%Chapter~\ref{chap:homology}.
\end{enumerate}

%\end{document}




\chapter{Homology of finite groups}
\label{chap:homology}
%\documentclass[12pt]{article}
%\usepackage{amsfonts, amsthm, amsmath}
%\usepackage[all]{xy}
%
%\setlength{\textwidth}{6.5in}
%\setlength{\oddsidemargin}{0in}
%\setlength{\textheight}{8.5in}
%\setlength{\topmargin}{0in}
%\setlength{\headheight}{0in}
%\setlength{\headsep}{0in}
%\setlength{\parskip}{0pt}
%\setlength{\parindent}{20pt}
%
%\def\AA{\mathbb{A}}
%\def\CC{\mathbb{C}}
%\def\FF{\mathbb{F}}
%\def\PP{\mathbb{P}}
%\def\QQ{\mathbb{Q}}
%\def\RR{\mathbb{R}}
%\def\ZZ{\mathbb{Z}}
%\def\gotha{\mathfrak{a}}
%\def\gothb{\mathfrak{b}}
%\def\gothm{\mathfrak{m}}
%\def\gotho{\mathfrak{o}}
%\def\gothp{\mathfrak{p}}
%\def\gothq{\mathfrak{q}}
%\def\gothr{\mathfrak{r}}
%\DeclareMathOperator{\ab}{ab}
%\DeclareMathOperator{\Aut}{Aut}
%\DeclareMathOperator{\coker}{coker}
%\DeclareMathOperator{\Cor}{Cor}
%\DeclareMathOperator{\disc}{Disc}
%\DeclareMathOperator{\Frob}{Frob}
%\DeclareMathOperator{\Gal}{Gal}
%\DeclareMathOperator{\GL}{GL}
%\DeclareMathOperator{\Hom}{Hom}
%\DeclareMathOperator{\im}{im}
%\DeclareMathOperator{\Ind}{Ind}
%\DeclareMathOperator{\Inf}{Inf}
%\DeclareMathOperator{\Norm}{Norm}
%\DeclareMathOperator{\Res}{Res}
%\DeclareMathOperator{\Trace}{Trace}
%\DeclareMathOperator{\Cl}{Cl}
%
%\def\head#1{\medskip \noindent \textbf{#1}.}
%
%\newtheorem{theorem}{Theorem}
%\newtheorem{lemma}[theorem]{Lemma}
%\newtheorem{cor}[theorem]{Corollary}
%
%\begin{document}
%
%\begin{center}
%\bf
%Math 254B, UC Berkeley, Spring 2002 (Kedlaya) \\
%Homology of Finite Groups
%\end{center}

\head{Reference} Milne, II.2; for cyclic groups, also
Neukirch, IV.7 and Lang, \emph{Algebraic Number Theory}, IX.1.

\head{Caveat} The Galois cohomology groups used in Neukirch are
not the ones we defined earlier. They are the Tate cohomology groups we
are going to define below.

\head{Homology}

You may not be surprised to learn that there is a ``dual'' theory to the
theory of group cohomology, namely group homology. What you may be surprised
to learn is that one can actually fit the two together, so that in a sense
the homology groups become cohomology groups with negative indices. (Since the
arguments are similar to those for cohomology, I'm going to skip details.)

Let $M_G$ denote the maximal quotient of $M$ on which $G$ acts trivially.
In other words, $M_G$ is the quotient of $M$ by the submodule spanned by
$m^g-m$ for all $m \in M$ and $g \in G$. In yet other words,
$M_G = M/M I_G$, where $I_G$ is the \emph{augmentation ideal} of the
group algebra $\ZZ[G]$:
\[
I_G = \left\{\sum_{g \in G} z_g[g]: \sum_g z_g = 0\right\}.
\]
Or if you like, $M_G = M \otimes_{\ZZ[G]} \ZZ$. Since $M^G$ is the group
of $G$-invariants, we call $M_G$ the group of \emph{$G$-coinvariants}.

The functor $M \to M^G$ is right exact but not left exact: if 
$0 \to M' \to M \to M'' \to 0$, then $M'_G \to M_G \to M''_G \to 0$
is exact but the map on the left is not injective. Again, we can fill
in the exact sequence by defining homology groups.

A $G$-module $M$ is \emph{projective} if for any surjection $N \to N'$
of $G$-modules and any map $\phi: M \to N'$, there exists a map
$\psi: M \to N$ lifting $\phi$. This is the reverse notion to injective;
but it's much easier to find projectives than injectives. For example,
any $G$-module which is a free module over the ring $\ZZ[G]$ is projective,
e.g., $\ZZ[G]$ itself!

Given a projective resolution $\cdots \to P_1 \to P_0 \to M \to 0$ of
a $G$-module $M$ (an exact sequence in which the $P_i$ are projective), 
take coinvariants to get a no longer exact complex
\[
\cdots \stackrel{d_2}{\to} P_2 \stackrel{d_1}{\to} P_1 \stackrel{d_0}{\to}
P_0 \to 0,
\]
then put $H_i(G, M) = \ker(d_{i-1})/\im(d_i)$. Again, this is canonically
independent of the resolution and functorial, and there is a long
exact sequence which starts out
\[
\cdots \to H_1(G, M'') \to \stackrel{\delta}{\to} H_0(G, M') \to H_0(G, M) \to H_0(G, M'') \to 0.
\]
Also, you can replace the projective resolution by an acyclic resolution
(where here $M$ being acyclic means $H_i(G,M) =0$ for $i>0$) and get
the same homology groups.
For example, induced modules are again acyclic (and the analogue of Shapiro's lemma holds, in part because any free $\ZZ[H]$-module induces to a free $\ZZ[G]$-module).

One can give a concrete description of homology as well, but we won't need
it for our purposes.
Even without one, though, we can calculate $H_1(G, \ZZ)$, using the exact
sequence
\[
0 \to I_G \to \ZZ[G] \to \ZZ \to 0.
\]
By the long exact sequence in homology,
\[
0 = H_1(G, \ZZ[G]) \to H_1(G, \ZZ) \to H_0(G, I_G) \to H_0(G, \ZZ[G])
\]
is exact, i.e. $0 \to H_1(G, \ZZ) \to I_G/I_G^2 \to \ZZ[G]/I_G$ is
exact. The last map is induced by $I_G \hookrightarrow \ZZ[G]$ and so
is the zero map. Thus $H_1(G, \ZZ) \cong I_G/I_G^2$;
recall that in an earlier exercise
(see Chapter~\ref{chap:principal}),
it was shown that the map $g \mapsto [g] - 1$ defines an isomorphism $G^{\ab} \to I_G/I_G^2$.

\head{The Tate groups}

We now ``fit together'' the long exact sequences of cohomology and homology
to get a doubly infinite exact sequence. Define the map $\Norm_G: M \to M$
by
\[
\Norm_G(m) = \sum_{g \in G} m^g.
\]
(It looks like it should be called
``trace'', but in practice our modules $M$ will be groups which are most
naturally written multiplicatively, i.e., the nonzero elements of a field.)
Then $\Norm_G$ induces a homomorphism
\[
\Norm_G: H_0(G,M) = M_G \to M^G = H^0(G,M).
\]
Now define
\[
H_T^i = \begin{cases}
H^i(G, M) & i > 0 \\
M^G/\Norm_G M & i=0 \\
\ker(\Norm_G)/MI_G & i=-1 \\
H_{-i-1}(G,M) & i<-1.
\end{cases}
\]
then I claim that for any short exact sequence $0 \to M' \to M \to M'' \to 0$,
we get an exact sequence
\[
\cdots
\to H^{i-1}_T(G, M'') \to H^i_T(G, M') \to H^i_T(G, M) \to H^i_T(G, M'')
\to H^{i+1}_T(G, M') \to \cdots
\]
which extends infinitely in both directions. 
The only issue is exactness between $H^{-2}_T(G, M'')$ and $H^1_T(G, M')$ inclusive;
this follows by diagram-chasing (as in the snake lemma) on the commutative diagram
\[
\xymatrix{
H_1(G, M'') \ar[r] \ar@{-->}[d] & H_0(G, M') \ar[r] \ar^{\Norm_G}[d] & 
H_0(G, M) \ar[r] \ar^{\Norm_G}[d] & H_0(G, M'') \ar[r] \ar^{\Norm_G}[d] & 0 \ar@{-->}[d] \\
0 \ar[r] & H^0(G, M') \ar[r]  & H^0(G, M) \ar[r] & 
H^0(G, M') \ar[r] & H^1(G, M')
}
\]
with exact rows (noting that the diagram remains commutative with the dashed arrows added).

This construction is especially useful if $M$ is induced, in which case $H^i_T(G,M) = 0$ for all $i$.
(The $T$ stands for Tate, who among many other things was an early pioneer
in the use of Galois cohomology into algebraic number theory.)

\head{Finite cyclic groups}

In general, for any given $G$ and $M$, it is at worst a tedious exercise
to compute $H^i_T(G,M)$ for any single value of $i$, but try to compute
all of these at once and you discover that they exhibit very little obvious
structure. Thankfully, there is an exception to that dreary rule when
$G$ is cyclic.

\begin{theorem} \label{T:cyclic group periodicity}
Let $G$ be a finite cyclic group and $M$ a $G$-module. Then there is
a canonical (up to the choice of a generator of $G$), functorial isomorphism $H^i_T(G,M) \to H^{i+2}_T(G,M)$ for all
$i \in \ZZ$.
\end{theorem}
\begin{proof}
Choose a generator $g$ of $G$.
We start with the four-term exact sequence of $G$-modules
\[
0 \to \ZZ \to \ZZ[G] \to \ZZ[G] \to \ZZ \to 0
\]
in which the first map is $1 \mapsto \sum_{g \in G} [g]$, the second map is
$[h] \mapsto [hg] - [h]$, and the third map is $[h] \mapsto 1$.
Since everything in sight is a free abelian group, we can tensor over $\ZZ$
with $M$ and get another exact sequence:
\[
0 \to M \to M \otimes_\ZZ \ZZ[G] \to M \otimes_\ZZ \ZZ[G] \to M \to 0.
\]
The terms in the middle are just $\Ind^G_{1} M$, where we first
restrict $M$ to a module for the trivial group and then induce back up.
Thus their Tate groups are all zero. The desired result now
follows from the following general fact: if
\[
0 \to A \stackrel{f}{\to} B \stackrel{g}{\to} C \stackrel{h}{\to} D \to 0
\]
is exact and $B$ and $C$ have all Tate groups zero, then
there is a canonical isomorphism $H^{i+2}_T(G, A) \to H^i_T(G, D)$.
To see this, apply the long exact sequence to the short exact sequences
\begin{gather*}
0 \to A \to B \to B/\im(f) \to 0 \\
0 \to B/\ker(g) \to C \to D \to 0
\end{gather*}
to get
\[
H^{i+2}(G, A) \cong H^{i+1}(G, B/\im(f)) = H^{i+1}(G, B/\ker(g))
\cong H^i(G,D).
\]
\end{proof}
In particular, the long exact sequence of a short exact sequence
$0 \to M' \to M \to M'' \to 0$ of $G$-modules curls up into an exact hexagon:
\[
\xymatrix{
 & H^{-1}_T(G, M) \ar[r] & H^{-1}_T(G, M'') \ar[dr] & \\
H^{-1}_T(G, M') \ar[ru] & & & H^0_T(G, M') \ar[dl] \\
 & H^{0}_T(G, M'') \ar[lu] & H^{0}_T(G, M) \ar[l] &
}
\]
If the groups $H^i_T(G, M)$ are finite, we define the
\emph{Herbrand quotient} 
\[
h(M) = \#H^0_T(G,M) / \#H^{-1}_T(G, M).
\]
Then from the exactness of the hexagon, if $M', M, M''$ all have
Herbrand quotients, then
\[
h(M) = h(M') h(M'').
\]
Moreover, if two of $M', M, M''$ have Herbrand quotients, so does the third.
For example, if $M'$ and $M''$ have Herbrand quotients, i.e., their Tate
groups are finite, then we have an exact sequence
\[
H^{-1}_T(G, M') \to H^{-1}_T(G, M)
\to H^{-1}_T(G, M'')
\]
and the outer groups are all finite. In particular, the first map is out
of a finite group and so has finite image, and modulo that image,
$H^{-1}_T(G,M)$ injects into another finite group. So it's also finite,
and so on.

In practice, it will often be much easier to compute the Herbrand quotient
of a $G$-module than to compute either of its Tate groups directly. The
Herbrand quotient will then do half of the work for free: once one group
is computed directly, at least the order of the other will be automatically
known.

One special case is easy to work out: if $M$ is finite, then $h(M) = 1$.
To wit, the sequences
\begin{gather*}
0 \to M^G \to M \to M \to M_G \to 0 \\
0 \to H^{-1}_T(G,M) \to M_G \stackrel{\Norm_G}{\to} M^G \to H^0_T(G,M) \to 0
\end{gather*}
are exact, where $M \to M$ is the map $m \mapsto m^g - m$; thus
$M_G$ and $M^G$ have the same order, as do $H^{-1}$ and $H^0$.

\head{Extended functoriality revisited}

The extended functoriality for cohomology groups has an analogue for homology and Tate groups, but under more restrictive conditions. Again, let $M$ be a $G$-module and $M'$ a $G'$-module, and consider a homomorphism $\alpha: G' \to G$ of groups and a homomorphism
$\beta: M \to M'$ of abelian which are compatible.
We would like to obtain canonical homomorphisms 
$H_i(G, M) \to H_i(G', M')$ and $H^i_T(G, M) \to H^i_T(G',M')$,
but for this we need to add an additional condition to ensure that $M \to M'$
induces a well-defined map $M_G \to M'_{G'}$. For instance, this holds if $\alpha$ is surjective. (Note that for Tate groups, we don't need any extra condition to get functoriality for $i \geq 0$.)

\head{Exercises}

\begin{enumerate}
\item
The periodicity of the Tate groups for $G$ cyclic means that there is
a canonical (up to the choice of a generator of $G$) isomorphism between $H^{-1}_T(G, M)$ and $H^1_T(G,M)$,
i.e., between $\ker(\Norm_G)/MI_G$ and the set of equivalence classes
of 1-cocycles. What is this isomorphism explicitly? In other words, given an
element of $\ker(\Norm_G)/MI_G$, what is the corresponding 1-cocycle?
\item
Put $K = \QQ_p(\sqrt{p})$. Compute the Herbrand quotient of
$K^*$ as a $G$-module for $G = \Gal(\QQ_p(\sqrt{p})/\QQ_p)$. (Hint:
use the exact sequence $1 \to \gotho_K^* \to K^* \to \ZZ \to 1$.)
\item
Show that $\Res: H^{-2}_T(G, \ZZ) \to H^{-2}_T(H,\ZZ)$ corresponds to
the transfer (Verlagerung) map $G^{\ab} \to H^{\ab}$.
\end{enumerate}

%\end{document}




\chapter{Profinite groups and infinite Galois theory}
%\documentclass[12pt]{article}
%\usepackage{amsfonts, amsthm, amsmath}
%
%\setlength{\textwidth}{6.5in}
%\setlength{\oddsidemargin}{0in}
%\setlength{\textheight}{8.5in}
%\setlength{\topmargin}{0in}
%\setlength{\headheight}{0in}
%\setlength{\headsep}{0in}
%\setlength{\parskip}{0pt}
%\setlength{\parindent}{20pt}
%
%\def\AA{\mathbb{A}}
%\def\CC{\mathbb{C}}
%\def\FF{\mathbb{F}}
%\def\NN{\mathbb{N}}
%\def\PP{\mathbb{P}}
%\def\QQ{\mathbb{Q}}
%\def\RR{\mathbb{R}}
%\def\ZZ{\mathbb{Z}}
%\def\gotha{\mathfrak{a}}
%\def\gothb{\mathfrak{b}}
%\def\gothm{\mathfrak{m}}
%\def\gotho{\mathfrak{o}}
%\def\gothp{\mathfrak{p}}
%\def\gothq{\mathfrak{q}}
%\DeclareMathOperator{\disc}{Disc}
%\DeclareMathOperator{\Fix}{Fix}
%\DeclareMathOperator{\Gal}{Gal}
%\DeclareMathOperator{\GL}{GL}
%\DeclareMathOperator{\Hom}{Hom}
%\DeclareMathOperator{\Inf}{Inf}
%\DeclareMathOperator{\Norm}{Norm}
%\DeclareMathOperator{\Trace}{Trace}
%\DeclareMathOperator{\Cl}{Cl}
%
%\def\head#1{\medskip \noindent \textbf{#1}.}
%
%\newtheorem{theorem}{Theorem}
%\newtheorem{lemma}[theorem]{Lemma}
%\newtheorem{prop}[theorem]{Proposition}
%
%\begin{document}
%
%\begin{center}
%\bf
%Math 254B, UC Berkeley, Spring 2002 (Kedlaya) \\
%Profinite Groups and Infinite Galois Theory
%\end{center}

\head{Reference} 
Neukirch, Sections IV.1 and IV.2.

\medskip

We've mostly spoken so far about finite extensions of fields and
the corresponding finite Galois groups. However, Galois theory can be made
to work perfectly well for infinite extensions, and it's convenient to do so;
it will be more convenient at times to work with the absolute Galois group
of field instead of with the Galois groups of individual extensions.

Recall the Galois correspondence for a finite extension: if $L/K$ is Galois
and $G = \Gal(L/K)$, then the (normal) subgroups $H$ of $G$ correspond to the
(Galois) subextensions $M$ of $L$, the correspondence in each direction
being given by
\[
H \mapsto \Fix{H},
\qquad
M \mapsto \Gal(L/M).
\]
To see what we have to be careful about, here's one example. Let $\FF_q$
be a finite field; recall that $\FF_q$ has exactly one finite extension of
any degree. Moreover, for each $n$, $\Gal(\FF_{q^n}/\FF_q)$ is cyclic of
degree $n$, generated by the Frobenius map $\sigma$ which sends $x$ to
$x^q$. In particular, $\sigma$ generates a cyclic subgroup of
$\Gal(\overline{\FF_q}/\FF_q)$. But this Galois group is much bigger than
that! Namely,
let $\{s_n\}_{n=1}^\infty$ be a sequence with $s_n \in \ZZ/n\ZZ$, such that
if $m | n$, then $s_m \equiv s_n \pmod{m}$. 
The set of such sequences forms
a group $\widehat{\ZZ}$ by componentwise addition.
This group is much bigger
than $\ZZ$, and any element gives an automorphism of $\overline{\FF_q}$:
namely, the automorphism acts on $\FF_{q^n}$ as $\sigma^{s_n}$. In fact,
$\Gal(\overline{\FF_q}/\FF_q) \cong \widehat{\ZZ}$, and it is not true that
every subgroup of $\widehat{\ZZ}$ corresponds to a subfield of
$\overline{\FF_q}$: the subgroup generated by $\sigma$ has fixed field
$\FF_q$, and you don't recover the subgroup generated by $\sigma$ by taking
automorphisms over the fixed field.

In order to recover the Galois correspondence, we need to impose a little
extra structure on Galois groups; namely, we give them a topology.

A \emph{profinite group} is a topological group which is Hausdorff and
compact, and which admits a basis of neighborhoods of the identity consisting
of normal subgroups. More explicitly, a profinite group is a group $G$
plus a collection of subgroups of $G$ of finite index designated as
\emph{open subgroups}, such that the intersection of two open subgroups is
open, but the intersection of all of the open subgroups is trivial.
Profinite groups act a lot like finite groups; some of the ways in which
this is true are reflected in the exercises.

Examples of profinite groups include the group $\widehat{\ZZ}$ in which
the subgroups $n\widehat{\ZZ}$ are open, and the $p$-adic integers $\ZZ_p$
in which the subgroups $p^n \ZZ_p$ are open. More generally, for any local
field $K$, the additive group $\gotho_K$ and the multiplicative group
$\gotho_K^*$ are profinite. (The additive and multiplicative groups of $K$
are not profinite, because they're only locally compact, not compact.)
For a nonabelian example, see the exercises.

\head{Warning}
A profinite group may have subgroups of finite index that are
not open. For example, let $G = 1 + t \FF_p [[ t ]]$
(under multiplication). Then $G$ is profinite with the subgroups
$1 + t^n \FF_p [[ t ]]$ forming a basis of open
subgroups; in particular, it has countably many open subgroups.
But $G$ is isomorphic to a countable direct product of
copies of $\ZZ_p$, with generators $1 + t^{i}$ for $i$ not divisible by $p$.
Thus it has \emph{uncountably} many subgroups of finite index, most
of which are not open!

If $L/K$ is a Galois extension, but not necessarily finite, we make
$G = \Gal(L/K)$ into a profinite group by declaring that the open subgroups
of $G$ are precisely $\Gal(L/M)$ for all finite subextensions $M$ of $L$.

\begin{theorem}[The Galois correspondence]
  Let $L/K$ be a Galois extension (not necessarily finite). Then
there is a correspondence between (Galois) subextensions $M$ of $L$ and
(normal) \emph{closed} subgroups $H$ of $\Gal(L/K)$, given by
\[
H \mapsto \Fix H, \qquad M \mapsto \Gal(L/M).
\]
\end{theorem}
For example, the Galois correspondence works for $\overline{\FF_q}/\FF_q$
because the open subgroups of $\widehat{\ZZ}$ are precisely
$n\widehat{\ZZ}$ for all positive integers $n$.

Another way to construct profinite groups uses inverse limits. Suppose we
are given a partially ordered set $I$, a
family $\{G_i\}_{i \in I}$ of finite groups and a map $f_{ij}: G_i \to G_j$
for each pair $(i,j) \in I \times I$ such that $i > j$. For simplicity,
let's assume the $f_{ij}$ are all surjective (this is slightly more restrictive than absolutely necessary, but is always true for Galois groups). Then there is
a profinite group $G$ with open subgroups $H_i$ for $i \in I$ such that
$G/H_i \cong G_i$ and some other obvious compatibilities hold:
let $G$ be the set of families $\{g_i\}_{i \in I}$, where each $g_i$
is in $G_i$ and $f_{ij}(g_i) = g_j$.

For example, the group $\ZZ_p$ either as the completion of $\ZZ$ for the $p$-adic absolute value or as the inverse limit of the groups $\ZZ/p^n\ZZ$.
Similarly, the group $\widehat{\ZZ}$ can be viewed as the inverse limit
of the groups $\ZZ/n\ZZ$, with the usual surjections from $\ZZ/m\ZZ$
to $\ZZ/n\ZZ$ if $m$ is a multiple of $n$ (that is, the ones sending 1 to 1).
In fact, \emph{any} profinite group can be reconstructed as the inverse
limit of its quotients by open subgroups. (And it's enough to use just a
set of open subgroups which form a basis for the topology, i.e.,
for $\ZZ_p$, you can use $p^{2n}\ZZ_p$ as the subgroups.)

\head{Rule of thumb} If profinite groups make your head hurt, you can
always think instead of inverse systems of finite groups. But that might
make your head hurt more!

\head{Cohomology of profinite groups}

One can do group cohomology for groups which are profinite, not just finite,
but one has to be a bit careful: these groups only make sense when you
carry along the profinite topology. Thus if $G$ is profinite, by a \emph{$G$-module}
we mean a topological abelian group $M$ with a \emph{continuous} $G$-action
$M \times G \to M$. In particular, we say $M$ is \emph{discrete} if it
has the discrete topology; that implies that the stabilizer of any element of
$M$ is open, and that $M$ is the union of $M^H$ over all open subgroups
$H$ of $G$. Canonical example: $G = \Gal(L/K)$ acting on $L^*$, even if $L$
is not finite.

The category of discrete $G$-modules has enough injectives, so you can
define cohomology groups for any discrete $G$-module, and all the usual
abstract nonsense will still work. The main point is that you can compute them
from their finite quotients.
\begin{prop}
The group $H^i(G, M)$ is the direct limit of 
$H^i(G/H, M^H)$ using the inflation homomorphisms.
\end{prop}
That is, if $H_1 \subseteq H_2 \subseteq G$, you have the inflation
homomorphism
\[
\Inf: H^i(G/H_2, M^{H_2}) \to H^i(G/H_1, M^{H_1}),
\]
so the groups $H^i(G/H, M^H)$ form a direct system,
and $H^i(G,M)$ is the direct limit of these. (That is, you take the union
of all of the $H^i(G/H, M^H)$, then you identify pairs that become the
same somewhere down the line.)

Or if you prefer, you can compute these groups using continuous cochains:
use continuous maps $G^{i+1} \to M$ that satisfy the same algebraic
conditions as do the usual cochains. For example, $H^1(G,M)$ classifies
continuous crossed homomorphisms modulo principal ones, et cetera.

\head{Warning} The passage from finite to profinite groups is only
well-behaved for cohomology. In particular, we will not attempt to define
either homology or the Tate groups. (Remember that the formation of the
Tate groups involves the norm map, i.e., summing over elements of the group.)

\head{Exercises}

\begin{enumerate}
\item
Prove that every open subgroup of a profinite group contains an open
normal subgroup.
\item
For any ring $R$, we denote by $\GL_n(R)$ the group of $n \times n$ matrices
over $R$ which are invertible (equivalently, whose determinant is a unit).
Prove that $\GL_n(\widehat{\ZZ})$ is a profinite group, and say as much as
you can about its open subgroups.
\item
Let $A$ be an abelian torsion group. Show that $\Hom(A, \QQ/\ZZ)$ is
a profinite group, if we take the open subgroups to be all subgroups
of finite index. This group is called the \emph{Pontryagin dual} of $A$.
\item
Neukirch exercise IV.2.4:
a closed subgroup $H$ of a profinite group $G$ is called
a \emph{$p$-Sylow subgroup} of $G$ if, for every open normal subgroup $N$
of $G$, $HN/N$ is a $p$-Sylow subgroup of $G/N$. Prove that:
\begin{enumerate}
\item[(a)] For every prime $p$, there exists a $p$-Sylow subgroup of $G$.
\item[(b)] Every subgroup of $G$, the quotient of which by any open 
normal subgroup is a $p$-group, is contained in a $p$-Sylow subgroup.
\item[(b)] Every two $p$-Sylow subgroups of $G$ are conjugate.
\end{enumerate}
You may use Sylow's theorem (that (a)-(c) hold for finite groups)
without further comment. Warning: Sylow subgroups are usually not
open.
\item
Neukirch exercise IV.2.4:
Compute the $p$-Sylow subgroups of $\widehat{\ZZ}$, of $\ZZ_p^*$, and
of $\GL_2(\ZZ_p)$.
\end{enumerate}

%\end{document}




\part{Local class field theory}

\chapter{Overview of local class field theory}
%\documentclass[12pt]{article}
%\usepackage{amsfonts, amsthm, amsmath}
%\usepackage[all]{xy}
%
%\setlength{\textwidth}{6.5in}
%\setlength{\oddsidemargin}{0in}
%\setlength{\textheight}{8.5in}
%\setlength{\topmargin}{0in}
%\setlength{\headheight}{0in}
%\setlength{\headsep}{0in}
%\setlength{\parskip}{0pt}
%\setlength{\parindent}{20pt}
%
%\def\AA{\mathbb{A}}
%\def\CC{\mathbb{C}}
%\def\FF{\mathbb{F}}
%\def\PP{\mathbb{P}}
%\def\QQ{\mathbb{Q}}
%\def\RR{\mathbb{R}}
%\def\ZZ{\mathbb{Z}}
%\def\gotha{\mathfrak{a}}
%\def\gothb{\mathfrak{b}}
%\def\gothm{\mathfrak{m}}
%\def\gotho{\mathfrak{o}}
%\def\gothp{\mathfrak{p}}
%\def\gothq{\mathfrak{q}}
%\def\gothr{\mathfrak{r}}
%\DeclareMathOperator{\ab}{ab}
%\DeclareMathOperator{\coker}{coker}
%\DeclareMathOperator{\disc}{Disc}
%\DeclareMathOperator{\Frob}{Frob}
%\DeclareMathOperator{\Gal}{Gal}
%\DeclareMathOperator{\GL}{GL}
%\DeclareMathOperator{\Hom}{Hom}
%\DeclareMathOperator{\im}{im}
%\DeclareMathOperator{\Ind}{Ind}
%\DeclareMathOperator{\inv}{inv}
%\DeclareMathOperator{\Norm}{Norm}
%\DeclareMathOperator{\Res}{Res}
%\DeclareMathOperator{\Trace}{Trace}
%\DeclareMathOperator{\unr}{unr}
%\DeclareMathOperator{\Cl}{Cl}
%
%\def\head#1{\medskip \noindent \textbf{#1}.}
%
%\newtheorem{theorem}{Theorem}
%\newtheorem{lemma}[theorem]{Lemma}
%\newtheorem{cor}[theorem]{Corollary}
%
%\begin{document}
%
%\begin{center}
%\bf
%Math 254B, UC Berkeley, Spring 2002 (Kedlaya) \\
%Overview of local class field theory
%\end{center}

\head{Reference} Milne, I.1; Neukirch, V.1.

\medskip
We will spend the next few chapters establishing \emph{local class field
theory}, a classification of the abelian extensions of a local field. This
will serve two purposes. On one hand, the results of local class field theory
can be used to assist in the proofs of the global theorems, as we saw with
Kronecker-Weber. On the other hand, they also give us a model set of proofs
which we will attempt to emulate in the global case.

Recall that the term ``local field'' refers to a finite extension either
of the field of $p$-adic numbers $\QQ_p$ or of the field of power series
$\FF_q((t))$. I'm going to abuse language and ignore the second case, although
all but a few things I'll say go through in the second case, and I'll try
to flag those when they come up. (One big one: a lot of extensions have to
be assumed to be separable for things to work right.)

\head{The local reciprocity law}

The main theorem of local class field theory is the following. For $K$
a local field, let $K^{\ab}$ be the maximal abelian extension of $K$.
\begin{theorem}[Local Reciprocity Law] \label{T:local reciprocity}
  Let $K$ be a local field. Then there is a unique map
$\phi_K: K^* \to \Gal(K^{\ab}/K)$ satisfying the following conditions:
\begin{enumerate}
\item[(a)] for any generator $\pi$ of the maximal ideal of $\gotho_K$
and any finite unramified extension $L$ of $K$, $\phi_K(\pi)$ acts on
$L$ as the Frobenius automorphism;
\item[(b)] for any finite abelian extension $L$ of $K$, the group of
norms $\Norm_{L/K} L^*$ is in the kernel of $\phi_K$, and the induced
map $K^*/\Norm_{L/K} L^* \to \Gal(L/K)$ is an isomorphism.
\end{enumerate}
\end{theorem}
The map $\phi_K$ is variously called the \emph{local reciprocity map} or the
\emph{norm residue symbol}.
Using the local Kronecker-Weber theorem (Theorem~\ref{T:local Kronecker-Weber}), this can be explicitly verified
for $K=\QQ_p$ (see exercises).

The local reciprocity law is an analogue of the Artin reprocity law for
number fields. We also get an analogue of the existence of class fields.
\begin{theorem}[Local existence theorem] \label{T:local existence}
For every finite (not necessarily abelian) extension $L$ of $K$,
$\Norm_{L/K} L^*$ is an open subgroup of $K^*$ of finite index.
Conversely,
for every (open) subgroup $U$ of $K^*$ of finite index, there exists a
finite abelian extension $L$ of $K$ such that $U = \Norm_{L/K} L^*$.
\end{theorem}
The condition ``open'' is only needed in the function field case; for
$K$ a finite extension of $\QQ_p$, one can show that every subgroup
of $K^*$ of finite index is open.

The local existence theorem says that if we start with a nonabelian extension $L$,
then $\Norm_{L/K} L^*$ is also the group of norms of an abelian extension.
Which one?
\begin{theorem}[Norm limitation theorem] \label{T:norm limitation}
Let $M$ be the maximal abelian subextension of $L/K$. Then
$\Norm_{L/K} L^* = \Norm_{M/K} M^*$.
\end{theorem}

Aside: for each uniformizer (generator of the maximal ideal) $\pi$
of $K$, let $K_\pi$ be the composite of all finite abelian extensions
$L$ such that $\pi \in \Norm_{L/K} L^*$. Then the local reciprocity map
implies that $K^{\ab} = K_\pi \cdot K^{\unr}$. It turns out that $K_\pi$
can be explicitly constructed as the extension of $K$ by certain elements,
thus giving a generalization of local Kronecker-Weber to arbitrary local
fields!
These elements come from Lubin-Tate formal groups, which we will not
discuss further.

Note that for $L/K$ a finite extension of local fields,
the map 
\[
K^*/\Norm_{L/K} L^* \to \Gal(L/K) = G
\] 
obtained by combining the local
reciprocity law with the norm limitation theorem
is in fact an isomorphism of $G = G^{\ab} = H^{-2}_T(G, \ZZ)$
with $K^*/\Norm_{L/K} L^* = H^0_T(G, L^*)$. We will in fact show something
stronger, from which we will deduce both the local reciprocity law and the norm limitation theorem.

\begin{theorem} \label{T:cup product isomorphism}
For any finite Galois extension $L/K$ of local fields
with Galois group $G$, there is a canonical isomorphism
$H^i_T(G, \ZZ) \to H^{i+2}_T(G, L^*)$.
\end{theorem}
In fact, this map can be written in terms of the cup product in group
cohomology, which we have not defined (and will not).

\head{The local invariant map}

One way to deduce the local reciprocity law (the one we will carry out first)
is to first prove the following.
\begin{theorem} \label{T:Brauer group identification}
For any local field $K$, there exist canonical isomorphisms
\begin{gather*}
H^2(\Gal(K^{\unr}/K), (K^{\unr})^*) \to
H^2(\Gal(\overline{K}/K), \overline{K}^*)\\
\inv_K: H^2(\Gal(\overline{K}/K), \overline{K}^*) \to \QQ/\ZZ.
\end{gather*}
\end{theorem}
The first map is an inflation homomorphism; the second
map in this theorem is called the \emph{local invariant map}.
More precisely, for $L/K$ finite of degree $n$, we have
an isomorphism
\[
\inv_{L/K}: H^2(\Gal(L/K), L^*) \to \frac{1}{n}\ZZ/\ZZ,
\]
and these isomorphisms are compatible with inflation. (In particular, we
don't need to prove the first isomorphism separately. But that can be done,
by considerations involving the Brauer group; see below.)

To use this to prove Theorem~\ref{T:cup product isomorphism} and hence
the local reciprocity law and the norm limitation theorem, we employ
the following theorem of Tate, which we will prove a bit later
(see Theorem~\ref{T:tate thm2}).
\begin{theorem} \label{T:tate thm1}
Let $G$ be a finite group and $M$ a $G$-module. Suppose that for each
subgroup $H$ of $G$ (including $H=G$), $H^1(H,M) = 0$ and $H^2(H,M)$
is cyclic of order $\#H$. Then there exist isomorphisms
$H^i_T(G, \ZZ) \to H^{i+2}_T(G, M)$ for all $i$; these are canonical
once you fix a choice of a generator of $H^2(G,M)$.
\end{theorem}

In general, for any field $K$, the group $H^2(\Gal(\overline{K}/K),
\overline{K}^*)$ is called the \emph{Brauer group} of $K$. It is an
important invariant of $K$; it can be realized also in terms of certain
noncommutative algebras over $K$ (central simple algebras). I won't
pursue this connection further, nor study many of the interesting properties
and applications of Brauer groups.

\head{Abstract class field theory}

Having derived local class field theory once, we will do it again a slightly
different way. In the course of proving the above results, we will
have calculated that if $L/K$ is a cyclic extension of local fields, that
\[
\#H^0_T(\Gal(L/K), L^*) = [L:K], \qquad \#H^{-1}_T(\Gal(L/K), L^*)
=1.
\]
It turns out that this alone is enough number-theoretic input to prove
local class field theory! More precisely, given a field $K$ with
$G = \Gal(\overline{K}/K)$, a continuous $G$-module $A$,
a surjective continuous homomorphism $d: G
\to \widehat{\ZZ}$, and a homomorphism $v: A^G \to \widehat{\ZZ}$
satisfying suitable conditions, we will show
that for every finite extension $L$ of $K$
there is a canonical isomorphism $\Gal(L/K)^{\ab} \to
A_L \to \Norm_{L/K} A_K$, where $A_K$ and $A_L$ denote the
$\Gal(\overline{K}/K)$ and $\Gal(\overline{K}/L)$-invariants of $A$.
In particular, these conditions will hold for $K$ a local field,
$A = \overline{K}^*$, $d$ the map $\Gal(\overline{K}/K) \to \Gal(K^{\unr}/K)$,
and $v: \Gal(K^*) \to \ZZ \to \widehat{\ZZ}$ the valuation.

This is the precise sense in which we will use local class field theory
as a model for global class field theory. After we complete local
class field theory, our next goal will be to construct an analogous
module $A$ in the global case which is ``complete enough'' that its
$H^0_T$ and $H^{-1}_T$ will not be too big; the result will be the idele
class group. (One main difference is that
in the global case, the analogue of $v$ will really take values in
$\widehat{\ZZ}$, not just $\ZZ$.) 

\head{Exercises}

\begin{enumerate}
\item
For $K = \QQ_p$, the local reciprocity map plus the local Kronecker-Weber
theorem give a canonical map
$\QQ_p^* \to \Gal(\QQ_p^{\ab}/\QQ_p) \cong \widehat{\ZZ}$.
What is the map? From the answer, you should be able to turn things around and deduce
local Kronecker-Weber from local reciprocity.
\item
For $K = \QQ_p$, take $\pi = p$. Determine $K_\pi$, again using
local Kronecker-Weber.
\item
Prove that for any finite extension $L/K$ of finite extensions of $\QQ_p$,
$\Norm_{L/K} L^*$ is an open subgroup of $K^*$. (Hint: show that
already $\Norm_{L/K} K^*$ is open! The corresponding statement in positive characteristic is more subtle.)
\item
Prove that for any finite extension $L/K$ of finite separable extensions of $\FF_p((t))$,
$\Norm_{L/K} L^*$ is an open subgroup of $K^*$. (Hint: reduce to the case of a cyclic extension of prime degree. If the degree is prime to $p$, you may imitate the previous exercise; otherwise, that approach fails because $\Norm_{L/K} K^*$ lands inside the subfield $K^p$, but you can use this to your advantage to make an explicit calculation.)
\item
A \emph{quaternion algebra} over a field $K$ is a central simple algebra over $K$ of dimension 4. If $K$ is not of characteristic 2, any such algebra has the form
\[
K \oplus Ki \oplus Kj \oplus Kk, \qquad i^2 = a, j^2 = b, ij = -ji = k
\]
for some $a,b \in K^*$. (For example, the case $K = \RR$, $a=b=-1$ gives the standard Hamilton quaternions.) A quaternion algebra is \emph{split} if it is isomorphic to the ring of $2 \times 2$ matrices over $K$.
Give a direct proof of the following consequence of Theorem~\ref{T:Brauer group identification}: if $K$ is a local field, then any two quaternion algebras which are not split are isomorphic to each other. 

\end{enumerate}

%\end{document}


\chapter{Cohomology of local fields: some computations}
\label{chap:localcomp}
%\documentclass[12pt]{article}
%\usepackage{amsfonts, amsthm, amsmath}
%\usepackage[all]{xy}
%
%\setlength{\textwidth}{6.5in}
%\setlength{\oddsidemargin}{0in}
%\setlength{\textheight}{8.5in}
%\setlength{\topmargin}{0in}
%\setlength{\headheight}{0in}
%\setlength{\headsep}{0in}
%\setlength{\parskip}{0pt}
%\setlength{\parindent}{20pt}
%
%\def\kbar{\overline{k}}
%\def\AA{\mathbb{A}}
%\def\CC{\mathbb{C}}
%\def\FF{\mathbb{F}}
%\def\PP{\mathbb{P}}
%\def\QQ{\mathbb{Q}}
%\def\RR{\mathbb{R}}
%\def\ZZ{\mathbb{Z}}
%\def\gotha{\mathfrak{a}}
%\def\gothb{\mathfrak{b}}
%\def\gothm{\mathfrak{m}}
%\def\gotho{\mathfrak{o}}
%\def\gothp{\mathfrak{p}}
%\def\gothq{\mathfrak{q}}
%\def\gothr{\mathfrak{r}}
%\DeclareMathOperator{\ab}{ab}
%\DeclareMathOperator{\coker}{coker}
%\DeclareMathOperator{\disc}{Disc}
%\DeclareMathOperator{\Frob}{Frob}
%\DeclareMathOperator{\Gal}{Gal}
%\DeclareMathOperator{\GL}{GL}
%\DeclareMathOperator{\Hom}{Hom}
%\DeclareMathOperator{\im}{im}
%\DeclareMathOperator{\Ind}{Ind}
%\DeclareMathOperator{\Inf}{Inf}
%\DeclareMathOperator{\inv}{inv}
%\DeclareMathOperator{\Norm}{Norm}
%\DeclareMathOperator{\Res}{Res}
%\DeclareMathOperator{\Trace}{Trace}
%\DeclareMathOperator{\unr}{unr}
%\DeclareMathOperator{\Cl}{Cl}
%
%\def\head#1{\medskip \noindent \textbf{#1}.}
%
%\newtheorem{theorem}{Theorem}
%\newtheorem{lemma}[theorem]{Lemma}
%\newtheorem{cor}[theorem]{Corollary}
%\newtheorem{prop}[theorem]{Proposition}
%
%\begin{document}
%
%\begin{center}
%\bf
%Math 254B, UC Berkeley, Spring 2002 (Kedlaya) \\
%Cohomology of local fields: some computations
%\end{center}

\head{Reference} Milne, III.2 and III.3; Neukirch, V.1.

\head{Notation convention} If you catch me writing $H^i(L/K)$
for $L/K$ a Galois extension of fields,
that's shorthand for $H^i(\Gal(L/K), L^*)$. Likewise for $H_i$ or
$H^i_T$.

\medskip
We now make some computations of $H^i_T(L/K)$ for $L/K$
a finite Galois
extension of local fields.
To begin with, recall that by ``Theorem 90'' (Lemma~\ref{L:theorem 90}), $H^1(L/K) = 0$.
Our goal in this chapter will be to supplement this fact with a computation of $H^2(L/K)$.
\begin{prop} \label{P:local h2}
For any finite Galois extension $L/K$ of local fields, $H^2(L/K)$
is cyclic of order $[L:K]$. Moreover, this group can be canonically
identified with $\frac{1}{[L:K]}\ZZ/\ZZ$ in such a way that
if $M/L$ is another finite extension such that
$M/K$ is also Galois, the inflation homomorphism
$H^2(L/K) \to H^2(M/K)$ corresponds to the inclusion
$\frac{1}{[L:K]}\ZZ/\ZZ \subseteq \frac{1}{[M:K]}\ZZ/\ZZ$.
\end{prop}

Before continuing, it is worth keeping in a safe place the exact sequence
\[
1 \to \gotho_L^* \to L^* \to L^*/\gotho_L^* = \pi_L^\ZZ \to 1.
\]
In this exact sequence of $G
= \Gal(L/K)$-modules, the action on $\pi_L^\ZZ$ is always trivial
(since the valuation on $L$ is Galois-invariant). For convenience, we
write $U_L$ for the unit group $\gotho_L^*$.

\head{The unramified case}
Recall that unramified extensions are cyclic, since their Galois groups are
also the Galois groups of extensions of finite fields. 

\begin{prop}
For any finite extension $L/K$ of \emph{finite} fields, the map
$\Norm_{L/K}: L^* \to K^*$ is surjective.
\end{prop}
\begin{proof}
One can certainly give an elementary proof of this using the fact that
$L^*$ is cyclic (exercise).
But one can also see it using the machinery we have at hand.
Because $L^*$ is a finite module, its Herbrand quotient is 1. Also,
we know $H^1_T(L/K)$ is trivial by Lemma~\ref{L:theorem 90}.
Thus $H^0_T(L/K)$ is trivial too, that is,
$\Norm_{L/K}: L^* \to K^*$ is surjective.
\end{proof}

\begin{prop}
For any finite unramified extension $L/K$ of local fields, the map
$\Norm_{L/K}: U_L \to U_K$ is surjective.
\end{prop}
\begin{proof}
Say $u \in U_K$ is a unit. Pick $v_0 \in U_L$ such that in the residue fields,
the norm of $v_0$ coincides with $u$. Thus $u/\Norm(v_0) \equiv 1 \pmod{\pi}$,
where $\pi$ is a uniformizer of $K$. Now we construct units
$v_i \equiv 1 \pmod{\pi^i}$ such that $u_i = u/\Norm(v_0\cdots v_i) \equiv 1 
\pmod{\pi^{i+1}}$: simply take $v_i$ so that 
$\Trace((1-v_i)/\pi^i) \equiv (1-u_{i-1})/\pi^i \pmod{\pi}$. (That's possible
because the trace map on residue fields is surjective by the normal basis theorem.)
Then the product $v_0v_1\cdots$ converges to a unit $v$ with norm $u$.
\end{proof}
\begin{cor}
For any finite unramified extensions $L/K$ of local fields, then
$H^i_T(\Gal(L/K), U_L) = 1$ for all $i \in \ZZ$.
\end{cor}
\begin{proof}
Again, $\Gal(L/K)$ is cyclic, so by Theorem~\ref{T:cyclic group periodicity}
we need only check this for $i=0,1$. For $i=1$, the desired equality is Lemma~\ref{L:theorem 90}; for $i=0$, it is the previous
proposition.  
\end{proof}

Using the Herbrand quotient, we get
$h(L^*) = h(U_L) h(L^*/U_L)$. The previous corollary says that $h(U_L) = 1$,
and 
\begin{align*}
h(L^*/U_L) &= h(\ZZ) \\
&= \#H^0_T(\Gal(L/K), \ZZ)/\#H^1_T(\Gal(L/K), \ZZ) \\
&= \#\Gal(L/K)^{\ab} / \#\Hom(\Gal(L/K), \ZZ) \\
&= [L:K].
\end{align*}
Since $H^1_T(\Gal(L/K), L^*)$ is trivial, we conclude $H^0_T(\Gal(L/K), L^*)$
has order $[L:K]$. In fact, it is cyclic: the long exact sequence of Tate
groups gives
\[
1 \to
H^0_T(\Gal(L/K), L^*) \to H^0_T(\Gal(L/K), \ZZ) = \Gal(L/K) \to 1.
\]

Consider the short exact sequence
\[
0 \to \ZZ \to \QQ \to \QQ/\ZZ \to 0
\]
of modules with trivial Galois action. Since $\QQ$ is injective as an abelian
group, it is also injective as a $G$-module for any group $G$ (exercise).
Thus we get an isomorphism $H^0_T(\Gal(L/K), \ZZ) \to H^{-1}_T(\Gal(L/K),
\QQ/\ZZ)$. But the latter is
\[
H^1(\Gal(L/K), \QQ/\ZZ) = \Hom(\Gal(L/K),
\QQ/\ZZ);
\]
since $\Gal(L/K)$ has a canonical generator (Frobenius), we
can evaluate there and get a canonical map $\Hom(\Gal(L/K), \QQ/\ZZ)
\to \ZZ/[L:K]\ZZ \subset \QQ/\ZZ$. Putting it all together, we get a 
canonical map
\[
H^2(\Gal(L/K), L^*) \cong H^0_T(\Gal(L/K), L^*)
\cong H^1(\Gal(L/K), \QQ/\ZZ) \hookrightarrow \QQ/\ZZ.
\]
In this special
case, this is none other than the local invariant map! In fact, by
taking direct limits, we get a canonical isomorphism
\[
H^2(K^{\unr}/K) \cong \QQ/\ZZ.
\]

What's really going on here is that $H^0_T(\Gal(L/K), L^*)$ is a cyclic
group generated by a uniformizer $\pi$ (since every unit is a norm).
Under the map $H^0_T(\Gal(L/K), L^*) \to \QQ/\ZZ$, that uniformizer
is being mapped to $1/[L:K]$.

\head{The cyclic case}

Let $L/K$ be a cyclic but possibly ramified extension of local fields.
Again, $H^1_T(L/K)$ is trivial by Lemma~\ref{L:theorem 90}, so all there is
to compute is $H^0_T(L/K)$. We are going to show again that it
has order $[L:K]$. (It's actually cyclic again, but we won't prove this just
yet.)

\begin{lemma}
Let $L/K$ be a finite Galois extension of local fields. Then there
is an open, Galois-stable subgroup $V$
of $\gotho_L$ such that $H^i(\Gal(L/K), V) = 0$ for all $i>0$
(i.e., $V$ is acyclic for cohomology).
\end{lemma}
\begin{proof}
By the normal basis theorem, there exists $\alpha \in L$ such that
$\{\alpha^g: g \in \Gal(L/K)\}$ is a basis for $L$ over $K$. Without loss
of generality, we may rescale to get $\alpha \in \gotho_L$; then put
$V = \sum \gotho_K \alpha^g$. As in the proof of Theorem~\ref{T:additive theorem 90},
$V$ is induced: $V = \Ind^G_{1} \gotho_K$, so is acyclic.
\end{proof}

The following proof uses that we are in characteristic 0, but it can be modified to work also in the function field case.
\begin{lemma}
Let $L/K$ be a finite Galois extension of local fields. Then there
is an open, Galois-stable subgroup $W$
of $U_L = \gotho_L^*$ such that $H^i(\Gal(L/K), W) = 0$ for all $i>0$.
\end{lemma}
\begin{proof}
Take $V$ as in the previous lemma. If we choose $\alpha$ sufficiently divisible,
then $V$ lies in the radius of convergence of the exponential series
\[
\exp(x) = \sum_{i=0}^\infty \frac{x^i}{i!}
\]
(you need $v_p(x) > 1/(p-1)$, to be precise), and we may take $W = \exp(V)$.
\end{proof}

Since the quotient $U_L/W$ is finite, its Herbrand quotient is 1, so
$h(U_L) = h(V) = 1$. So again we may conclude that
$h(L^*) = h(U_L) h(\ZZ) = [L:K]$, and so $H^0_T(\Gal(L/K), L^*) = [L:K]$.
However, we cannot yet check that $H^0_T(\Gal(L/K), L^*)$ is cyclic because the groups
$H^1_T(\Gal(L/K), U_L)$ are not guaranteed to vanish; see the exercises.

\head{Note} This is all that we need for ``abstract''
local class field theory. We'll revisit this point later.

\head{The general case}

For those in the know, there is a spectral sequence underlying this next
result; see Milne, Remark II.1.35.
\begin{prop}[Inflation-Restriction Exact Sequence] \label{P:inflation restriction}
Let $G$ be a finite group, $H$ a normal subgroup, and $M$ a $G$-module.
If $H^i(H, M) = 0$ for $i=1, \dots, r-1$, then 
\[
0 \to H^r(G/H, M^H) \stackrel{\Inf}{\to} H^r(G,M) \stackrel{\Res}{\to}
H^r(H,M)
\]
is exact.
\end{prop}
\begin{proof}
For $r=1$, the condition on $H^i$ is empty. In this case, $H^1(G,M)$
classifies crossed homomorphisms $\phi:G \to M$. If one of these
factors through $G/H$, it becomes a constant map when restricted to $H$;
if that constant value itself belongs to $M^H$, then it must be zero
and so the restriction to $H$ is trivial.
Conversely, if there exists some $m \in M$ such that 
$\phi(h) = m^h - m$ for all $h \in H$, then
$\phi'(g) = \phi(g) - m^g + m$ is another crossed homomorphism representing the same class in $H^1(G,M)$, but taking the value 0 on each $h \in H$. For $g \in G, h \in H$, we have
\[
\phi'(hg) = \phi'(h)^g + \phi'(g) = \phi'(g),
\]
so $\phi'$ is constant on cosets of $H$ and so may be viewed as a crossed homomorphism from $G/H$ to $M$. On the other hand,
\[
\phi'(g) = \phi'(gh) = \phi'(g)^h + \phi(h) = \phi'(g)^h
\]
so $\phi'$ takes values in $M^H$.
 Thus the sequence is exact at $H^1(G,M)$; exactness at
$H^i(G/H,M^H)$ is similar but easier.

If $r>1$, we induct on $r$ by dimension shifting. 
Recall (from Proposition~\ref{P:adjoint property}) that there is an injective homomorphism $M \to \Ind^G_1 M$ of $G$-modules.
Let $N$ be the $G$-module
which makes the sequence
\[
0 \to M \to \Ind^G_{1} M \to N \to 0
\]
exact. We construct a commutative diagram
\[
\xymatrix{
0 \ar[r] & H^{r-1}(G/H, N^H) \ar^{\Inf}[r] \ar[d] & H^{r-1}(G, N) \ar^{\Res}[r] \ar[d] & 
H^{r-1}(H,N) \ar[d] \\
0 \ar[r] & H^{r}(G/H, M^H) \ar^{\Inf}[r] & H^{r}(G, M) \ar^{\Res}[r] & 
H^{r}(H,M).
}
\]
The second vertical arrow arises from the long exact sequence for $G$-cohomology;
since $\Ind^G_{1} M$ is an induced $G$-module, this arrow is an isomorphism.
Similarly, the third vertical arrow arises from the long exact sequence for $H$-cohomology,
and it is an isomorphism because $\Ind^G_1 M$ is also an induced $H$-module; moreover, $H^i(H, N) = 0$ for $i=1, \dots, r-2$. 
Finally, taking $H$-invariants yields another exact sequence
\[
0 \to M^H \to (\Ind^G_1 M)^H \to N^H \to H^1(H, M) = 0,
\]
so we may take the long exact sequence for $G/H$-cohomology to obtain the first vertical arrow; it is an isomorphism because $(\Ind^G_1 M)^H$ is an induced $G/H$-module. The induction hypothesis implies that the top row is exact, so the bottom row is also exact.
\end{proof}
By Lemma~\ref{L:theorem 90}, we have the following.
\begin{cor} \label{C:inflation restriction h2}
If $M/L/K$ is a tower of fields with $M/K$ and $L/K$ finite and Galois,
the sequence
\[
0 \to H^2(L/K) \stackrel{\Inf}{\to} H^2(M/K) \stackrel{\Res}{\to} H^2(M/L)
\]
is exact.
\end{cor}

We now prove the following.
\begin{prop}
For any finite Galois extension $L/K$ of local fields, the group $H^2(\Gal(L/K), L^*)$
has order at most $[L:K]$.
\end{prop}
A key fact we need to recall is that any finite Galois extension of local fields
is \emph{solvable}: the maximal unramified extension is cyclic, the
maximal tamely ramified extension is cyclic over that, and the rest is
an extension of order a power of $p$, so its Galois group is automatically
solvable. This lets us induct on $[L:K]$.
\begin{proof}
We've checked the case of $L/K$ cyclic, so we may use it as the basis for an
induction. If $L/K$ is not cyclic, since it is solvable, we can find a
Galois subextension $M/K$. Now the exact sequence
\[
0 \to H^2(M/K) \to H^2(L/K) \to H^2(L/M)
\]
implies that $\#H^2(L/K) \leq \#H^2(M/K)
\#H^2(L/M) = [M:K][L:M] = [L:K]$.
\end{proof}

To complete the proof that $H^2(L/K)$ is cyclic of order
$[L:K]$, it now suffices to produce a cyclic subgroup of order $[L:K]$.
Let $M/K$ be an unramified extension of degree $[L:K]$. Then we have a diagram
\[
\xymatrix{
& & H^2(M/K) \ar[d]^{\Inf} \ar[rd]
&  \\
0 \ar[r] & H^2(L/K) \ar^{\Inf}[r] & H^2(ML/K) \ar^{\Res}[r]
& H^2(ML/L)
}
\]
in which the bottom row is exact and the vertical arrows are injective,
both by Corollary~\ref{C:inflation restriction h2}. It suffices to show that
the diagonal arrow
$H^2(M/K) \to H^2(ML/L)$ is the zero map; then we can push a generator
of $H^2(M/K)$ down to $H^2(ML/K)$, then pull it back to $H^2(L/K)$ by exactness
to get an element of order $[L:K]$.

Let $e = e(L/K)$ and $f = f(L/K)$ be the ramification index and residue field degree, so that
$[ML:L] = e$. Let $U$ be the maximal unramified subextension of $L/K$;
then we have a canonical isomorphism $\Gal(ML/L) \cong \Gal(M/U)$ of cyclic groups.
By using the same generators in both groups, we can make a commutative diagram
\[
\xymatrix{
H^0_T(M/K) \ar^{\Res}[r] \ar[d] & H^0_T(M/U) \ar[r]  \ar[d]
& H^0_T(ML/L) \ar[d]  \\
H^2(M/K) \ar^{\Res}[r] & H^2(M/U) \ar[r]
 & H^2(ML/L)
}
\]
in which the vertical arrows are isomorphisms. 
(Remember that extended functoriality for Tate groups starts in degree 0, yielding 
the first horizontal arrow.)
The composition in the bottom row is the map $H^2(M/K) \to H^2(ML/L)$ which we want to be zero; it thus suffices to check that the top row composes to zero. This composition is none other than the canonical map
$K^*/\Norm_{M/K} M^* \to L^*/\Norm_{ML/L} (ML)^*$.
Now $K^*/\Norm_{M/K} M^*$ is a cyclic group of order $ef$
generated by $\pi_K$, a uniformizer
of $K$, and $L^*/\Norm_{ML/L} (ML)^*$ is a cyclic group of order
$e$ generated by $\pi_L$, a uniformizer of $L$. But
$\pi_K$ is a unit of $\gotho_L$ times $\pi_L^e$, so the map in question is indeed zero.

\head{Note}
If $L/K$ is a finite extension of degree $n$, then the map
$\Res: H^2(K^{\unr}/K) \to H^2(L^{\unr}/L)$ translates, via the local
reciprocity map, into a map from $\QQ/\ZZ$ to itself. This map turns out
to be multiplication by $n$ (see Milne, Proposition II.2.7).

\head{The local invariant map}

By staring again at the above argument, we can in fact prove that
$H^2(\overline{K}/K) \cong \QQ/\ZZ$.
First of all, we have
an injection $H^2(K^{\unr}/K) \to H^2(\overline{K}/K)$ by
Corollary~\ref{C:inflation restriction h2}, and the former is canonically
isomorphic to $\QQ/\ZZ$; so we have to prove that this injection is
actually also surjective. Remember that $H^2(\overline{K}/K)$ is the
direct limit of $H^2(M/K)$ running over all finite extensions $M$ of $K$.
What we just showed above is that if $[M:K] = n$ and $L$ is the unramified
extension of $K$ of degree $n$, then the images of $H^2(M/K)$ and
$H^2(L/K)$ in $H^2(ML/K)$ are the same. In particular, that means that
$H^2(M/K)$ is in the image of the map $H^2(K^{\unr}/K) \to 
H^2(\overline{K}/K)$. Since that's true for any $M$, we get that the
map is indeed surjective, hence an isomorphism.

Next time, we'll use this map to obtain the local reciprocity map.

\head{Exercises}

\begin{enumerate}
\item
Give an elementary proof (without cohomology)
that the norm map from one finite field to another
is always surjective.
\item
Give an example of a cyclic ramified extension $L/K$ of local fields
in which the groups $H^i_T(\Gal(L/K), U_L)$ are nontrivial.
\end{enumerate}

%\end{document}


\chapter{Local class field theory via Tate's theorem}
%\documentclass[12pt]{article}
%\usepackage{amsfonts, amsthm, amsmath}
%\usepackage[all]{xy}
%
%\setlength{\textwidth}{6.5in}
%\setlength{\oddsidemargin}{0in}
%\setlength{\textheight}{8.5in}
%\setlength{\topmargin}{0in}
%\setlength{\headheight}{0in}
%\setlength{\headsep}{0in}
%\setlength{\parskip}{0pt}
%\setlength{\parindent}{20pt}
%
%\def\kbar{\overline{k}}
%\def\AA{\mathbb{A}}
%\def\CC{\mathbb{C}}
%\def\FF{\mathbb{F}}
%\def\PP{\mathbb{P}}
%\def\QQ{\mathbb{Q}}
%\def\RR{\mathbb{R}}
%\def\ZZ{\mathbb{Z}}
%\def\gotha{\mathfrak{a}}
%\def\gothb{\mathfrak{b}}
%\def\gothm{\mathfrak{m}}
%\def\gotho{\mathfrak{o}}
%\def\gothp{\mathfrak{p}}
%\def\gothq{\mathfrak{q}}
%\def\gothr{\mathfrak{r}}
%\DeclareMathOperator{\ab}{ab}
%\DeclareMathOperator{\coker}{coker}
%\DeclareMathOperator{\Cor}{Cor}
%\DeclareMathOperator{\disc}{Disc}
%\DeclareMathOperator{\Frob}{Frob}
%\DeclareMathOperator{\Gal}{Gal}
%\DeclareMathOperator{\GL}{GL}
%\DeclareMathOperator{\Hom}{Hom}
%\DeclareMathOperator{\im}{im}
%\DeclareMathOperator{\Ind}{Ind}
%\DeclareMathOperator{\Inf}{Inf}
%\DeclareMathOperator{\inv}{inv}
%\DeclareMathOperator{\Norm}{Norm}
%\DeclareMathOperator{\Res}{Res}
%\DeclareMathOperator{\Trace}{Trace}
%\DeclareMathOperator{\unr}{unr}
%\DeclareMathOperator{\Cl}{Cl}
%
%\def\head#1{\medskip \noindent \textbf{#1}.}
%
%\newtheorem{theorem}{Theorem}
%\newtheorem{lemma}[theorem]{Lemma}
%\newtheorem{cor}[theorem]{Corollary}
%\newtheorem{prop}[theorem]{Proposition}
%
%\begin{document}
%
%\begin{center}
%\bf
%Math 254B, UC Berkeley, Spring 2002 (Kedlaya) \\
%Local Class Field Theory, via Tate's Theorem
%\end{center}

\head{Reference} Milne II.3, III.5.

\medskip

For $L/K$ a finite extension of local fields, we have now computed that
$H^1(L/K) = 0$ (Lemma~\ref{L:theorem 90})
and $H^2(L/K)$ is cyclic of order $[L:K]$ (Proposition~\ref{P:local h2}).
In this chapter, we use these ingredients to establish all of the statements of local class field theory.

\head{Tate's theorem}

We first prove the theorem of Tate stated earlier (Theorem~\ref{T:tate thm1}).
\begin{theorem}[Tate] \label{T:tate thm2}
Let $G$ be a finite (solvable) group and let $M$ be a $G$-module. Suppose
that for all subgroups $H$ of $G$ (including $G$ itself), $H^1(H,M)=0$
and $H^2(H,M)$ is cyclic of order $\#H$.
Then there are isomorphisms
$H^i_T(G, \ZZ) \to H^{i+2}_T(G, M)$ which are canonical up to a choice of
generator of $H^2(G, M)$.
\end{theorem}
\begin{proof}
Let $\gamma$ be a generator of $H^2(G, M)$. Since $\Cor \circ \Res
= [G:H]$, $\Res(\gamma)$ generates $H^2(H,M)$ for any $H$. We start
out by explicitly constructing a $G$-module containing $M$ in which
$\gamma$ becomes a coboundary.

Choose a 2-cocycle $\phi: G^3 \to M$ representing $\gamma$; by the definition
of a cocycle, 
\begin{gather*}
\phi(g_0 g, g_1 g, g_2 g) = \phi(g_0, g_1, g_2)^g, \\
\phi(g_1, g_2, g_3) - \phi(g_0, g_2, g_3) + \phi(g_0, g_1, g_3)
- \phi(g_0, g_1, g_2) = 0.
\end{gather*}
Moreover, $\phi$ is a coboundary if and only if it's of the form
$d(\rho)$, that is, $\phi(g_0, g_1, g_2) = \rho(g_1, g_2) -
\rho(g_0, g_2) + \rho(g_0, g_1)$. This $\rho$ must itself be $G$-invariant:
$\rho(g_0, g_1)^g = \rho(g_0g, g_1g)$. Thus $\phi$ is a coboundary if
$\phi(e, g, hg) = \rho(e,h)^g - \rho(e,hg) + \rho(e,g)$.

Let $M[\phi]$ be the direct sum of $M$ with the free abelian group
with one generator $x_g$ for each element $g$
of $G - \{e\}$, with the $G$-action
\[
x_h^g = x_{hg} - x_g + \phi(e, g, hg).
\]
(The symbol $x_e$ should be interpreted as $\phi(e,e,e)$.)
Using the cocycle property of $\phi$,
one may verify that this is indeed a $G$-action; by construction,
the cocycle $\phi$ becomes zero in $H^2(G, M[\phi])$ by setting
$\rho(e,g) = x_g$. (Milne calls $M[\phi]$ the \emph{splitting module} of
$\phi$.)

The map $\alpha: M[\phi] \to \ZZ[G]$ sending $M$ to zero and $x_g$
to $[g]-1$ is a homomorphism of $G$-modules. Actually it maps into the
augmentation ideal $I_G$, and the sequence
\[
0 \to M \to M[\phi] \to I_G \to 0
\]
is exact. (Note that this is backwards from the usual exact sequence featuring $I_G$ as a submodule, which will appear again momentarily.)
For any subgroup $H$ of $G$, we can restrict to $H$-modules, then
take the long exact sequence:
\[
0 = H^1(H,M) \to H^1(H, M[\phi]) \to H^1(H, I_G)
\to H^2(H, M) \to H^2(H, M[\phi]) \to H^2(H, I_G).
\]
To make headway with this, view $0 \to I_G \to \ZZ[G] \to \ZZ \to 0$
as an exact sequence of $H$-modules. 
 Since $\ZZ[G]$ is induced, its Tate
groups all vanish. So we get a dimension shift:
\[
H^1(H, I_G) \cong H^0_T(H, \ZZ) = \ZZ/(\#H)\ZZ.
\]
Similarly, $H^2(H, I_G) \cong H^1(H, \ZZ) = 0$.
Also, the map $H^2(H, M) \to H^2(H, M[\phi])$ is zero because
we constructed this map so as to kill off the generator $\phi$. 
Thus $H^2(H, M[\phi]) = 0$ and $H^1(H, I_G) \to H^2(H,M)$ is surjective. But these groups are both finite
of the same order! So the map is also injective, and $H^1(H, M[\phi])$
is also zero.

Now apply Lemma~\ref{L:tate thm lemma} below to conclude that $H^i_T(G, M) = 0$ for all $i$.
This allows us to use the four-term exact sequence
\[
0 \to M \to M[\phi] \to \ZZ[G] \to \ZZ \to 0
\]
(as in the proof of Theorem~\ref{T:cyclic group periodicity})
to conclude that $H^i_T(G, \ZZ) \cong H^{i+2}_T(G, M)$.
\end{proof}

Note: we only need the results of this section for $G$ solvable, because
in our desired application $G$ is the Galois group of a finite extension
of local fields. But one can remove this restriction: see the note after
this lemma.
\begin{lemma} \label{L:tate thm lemma}
Let $G$ be a finite (solvable) group and $M$ a $G$-module. Suppose that
$H^i(H,M) =0$ for $i=1,2$ and $H$ any subgroup of $G$ (including $G$
itself). Then $H^i_T(G,M) = 0$ for all $i \in \ZZ$.
\end{lemma}
\begin{proof}
For $G$ cyclic, this follows by periodicity. We prove the general result
by induction on $\#G$. Since $G$ is solvable, it has a proper subgroup $H$ for which
$G/H$ is cyclic. By the induction hypothesis, $H^i_T(H,M) = 0$ for all
$i$. Thus by the inflation-restriction exact sequence (Proposition~\ref{P:inflation restriction}),
\[
0 \to H^i(G/H, M^H) \to H^i(G, M) \to H^i(H, M)
\]
is exact for all $i>0$. The term on the end being zero, we have
$H^i(G/H, M^H) \cong H^i(G,M) = 0$ for $i=1, 2$. By periodicity (Theorem~\ref{T:cyclic group periodicity}),
$H^i_T(G/H, M^H) = 0$ for all $i$, so $H^i(G/H, M^H) = 0$ for all
$i>0$, and $H^i(G,M) = 0$ for $i>0$. As for $H^0_T(G,M)$, we have that
$H^0_T(G/H, M^H) = 0$, so for any $x \in M^G$, there exists $y \in M^H$
such that $\Norm_{G/H}(y) = x$. Since $H^0_T(H,M) = 0$, there exists
$z \in M$ such that $\Norm_{H}(z) = x$. Now
$\Norm_G(z) = \Norm_{G/H} \circ \Norm_H(z) = x$. Thus
$H^0_T(G,M) = 0$, as desired.

So far so good, but we want to kill off the Tate groups with negative indices
too, so we do a dimension shift. 
Make the exact sequence
\[
0 \to N \to \Ind^G_1 M \to M \to 0
\]
in which $m \otimes [g]$ maps to $m^g$. The term in the middle is acyclic, so
$H^{i+1}_T(H', N) \cong H^{i}_T(H', M)$ for any subgroup $H'$
of $G$. In particular, $H^1(H', N) = 
H^2(H', N) = 0$, so the above argument gives $H^i_T(G, N) = 0$ for $i\geq 0$.
Now from $H^0_T(G, N) = 0$
we get $H^{-1}_T(G, M) = 0$; since the same argument applies to $N$,
we also get $H^{-2}_T(G, M) = 0$ and so on.
\end{proof}
To go from the solvable case to the general case, one shows that
the $p$-primary component of $H^i(G,M)$ injects into $H^i(G_p, M)$,
where $G_p$ is the $p$-Sylow subgroup. (Apply $\Cor \circ \Res$ from
$G$ to $G_p$; the result is multiplication by $[G:H]$ which is prime to
$p$.)

\head{The results of local class field theory}

Let $L/K$ be a finite Galois extension of local fields.
For any intermediate extension $M/K$, we know that
$H^1(L/M) = 0$ and $H^2(L/M)$ is cyclic of order $[L:M]$. We may thus apply Theorem~\ref{T:tate thm2} with for $G = \Gal(L/K)$, $M = L^*$
to obtain isomorphisms $H^i_T(G, \ZZ) \to H^{i+2}_T(G,M)$,
thus proving Theorem~\ref{T:cup product isomorphism}. This yields
 a canonical isomorphism 
\[
K^*/\Norm_{L/K} L^* = H^0_T(L/K)
\to  H^{-2}_T(\Gal(L/K), \ZZ) = \Gal(L/K)^{\ab}.
\]
This establishes the existence of the local reciprocity map (Theorem~\ref{T:local reciprocity};
note that part (a) follows from the explicit computations in Chapter~\ref{chap:localcomp})
and the norm limitation theorem (Theorem~\ref{T:norm limitation}), modulo one subtlety:
if $M/K$ is another finite Galois extension containing $L$, we need to know that the diagram
\[
\xymatrix{
K^*/\Norm_{M/K} M^* \ar[r] \ar[d]  & \Gal(M/K)^{\ab} \ar[d] \\
K^*/\Norm_{L/K} L^* \ar[r] & \Gal(L/K)^{\ab}
}
\]
commutes,
so the maps $K^* \to \Gal(L/K)^{\ab}$ fit together to give a map $K^* \to \Gal(K^{\sep}/K)^{\ab}$. 
In other words, we need a commuting diagram
\[
\xymatrix{
H^0_T(\Gal(M/K), M^*)  \ar[r] \ar[d] & H^{-1}_T(\Gal(M/K), I_{\Gal(M/K)}) \ar[d] \\
 H^0_T(\Gal(L/K), L^*) \ar[r]& H^{-1}_T(\Gal(L/K), I_{\Gal(L/K)})
}
\]
This appears to be a gap in Milne's presentation.
To fix it, choose a 2-cocycle $\phi_M: \Gal(M/K)^3 \to M^*$ representing the preferred generator of $H^2(M/K)$; then the upper horizontal arrow is a connecting homomorphism
for the exact sequence
\[
1 \to M^* \to M^*[\phi_M] \to I_{\Gal(M/K)} \to 1.
\]
The lower horizontal arrow arises similarly from the exact sequence
\[
1 \to L^* \to L^*[\phi_L] \to I_{\Gal(L/K)} \to 1,
\]
where $\phi_L$ represents a class whose inflation is $[G:H]$ times the class represented by $\phi_M$.
Further details omitted.

In any case, it remains to prove the local existence theorem (Theorem~\ref{T:local existence}). We begin with a lemma, in which we take advantage of Kummer theory to establish an easy case of the existence theorem.
\begin{lemma} \label{L:hilbert symbol}
Let $\ell$ be a prime number. Let $K$ be a local field containing a primitive $\ell$-th root of unity. Then $x \in K^*$ is an $\ell$-th power in $K$ if and only if belongs to $\Norm_{L/K} L^*$ for every cyclic extension $L$ of $K$ of degree $\ell$.
\end{lemma}
The same statement holds even if $\ell$ is not prime (exercise)
and can be interpreted in terms of the \emph{Hilbert symbol}, whose properties generalize quadratic reciprocity to higher powers; see Milne, III.4.
\begin{proof}
Let $M$ be the compositum of all cyclic $\ell$-extensions of $K$.
The group $K^*/(K^*)^{\ell}$ is finite (exercise),
and hence is isomorphic to $(\ZZ/\ell \ZZ)^n$ for some positive integer $n$.
By Kummer theory (Theorem~\ref{T:Kummer reformulated}),
we also have $\Gal(M/K) \cong (\ZZ/\ell \ZZ)^n$. By the local reciprocity law,
$K^*/\Norm_{M/K}M^* \cong (\ZZ/\ell \ZZ)^n$; consequently, on one hand
$(K^*)^{\ell} \subseteq \Norm_{M/K}M^*$, and on other hand these subgroups of $K^*$ have the same index $\ell^n$. They are thus equal, proving the claim.
\end{proof}

This allows to deduce a corollary of the existence theorem which is needed in its proof.
\begin{cor} \label{C:universal norms divisible}
Let $K$ be a local field.
Then the intersection of the groups $\Norm_{L/K} L^*$ for all finite extensions $L$ of $K$ is the trivial group.
\end{cor}
\begin{proof}
Let $D_K$ be the intersection in question; note that $D_K \subseteq U_K$ by considering unramified extensions of $K$, so $D_K$ is in particular a compact topological group.
By Lemma~\ref{L:hilbert symbol}, every element of $D_K$ is an $\ell$-th power in $K$ for every prime $\ell$; it remains to check that one can find an $\ell$-th root which is also in $D_K$. This would then imply that $D_K$ is a divisible subgroup of $U_K$, and hence the trivial group (see exercises).

For $L/K$ a finite extension, it is true but  not immediately clear that 
\[
\Norm_{L/K} D_L = D_K;
\]
that is, for $x \in D_K$, for each finite extension $M$ of $K$, $x = \Norm_{M/K}(z)$ for some $z \in M$, but may not be apparent that the elements $y = \Norm_{M/L}(z)$ can be chosen to be equal. However, for a given $M$, the set of such $y$ is a nonempty compact subset of $U_L$, and any finite intersection of these sets is nonempty (since it contains the set corresponding to the compositum of the corresponding fields), so the whole intersection is nonempty.

Let $\ell$ be a prime number and choose $x \in D_K$. For each finite extension $L$ of $K$ containing a primitive $\ell$-th root of unity, let $E(L)$ be the set of $\ell$-th roots of $x$ in $K$ which belong to $\Norm_{L/K} L^*$.
This set is finite (it can contain at most $\ell$ elements) and nonempty: we have $x = \Norm_{L/K}(y)$ for some $y \in D_L$, so $y$ has an $\ell$-th root $z$ in $L$ 
and $\Norm_{L/K}(z) \in E(L)$. Again by the finite intersection property,
we find an $\ell$-th root of $x$ in $K$ belonging to $D_K$, completing the proof.
\end{proof}

Returning to the local existence theorem,
let $U$ be an open subgroup of $K^*$ of finite index; we wish to find a finite abelian extension $L$ of $K$ such that $U = \Norm_{L/K}L^*$. 
We note first that by the local reciprocity law, it is enough to construct $L$ so that
$U$ contains $\Norm_{L/K}L^*$: in this case, we will have $\Gal(L/K) \cong K^*/\Norm_{L/K}L^*$, and then $U/\Norm_{L/K}L^*$ will corresponding to $\Gal(L/M)$ for some intermediate extension $M/K$ having the desired effect. We note next that by the norm limitation theorem, it suffices to produce \emph{any} finite extension $L/K$, not necessarily abelian, such that $U$ contains $\Norm_{L/K}L^*$.

Let $m\ZZ \subseteq \ZZ$ be the image of $U$ in $K^*/U_K \cong \ZZ$;
by choosing $L$ to contain the unramified extension of $K$ of degree $m$,
we may ensure that the image of $\Norm_{L/K} L^*$ in $K^*/U_K$ is also contained in $m\ZZ$.
It thus remains to further ensure that 
\[
(\Norm_{L/K} L^*) \cap U_K \subseteq U \cap U_K.
\]
Since $U_K$ is compact, each open subgroup $(\Norm_{L/K}L^*) \cap U_K$ is also closed and hence compact. By Corollary~\ref{C:universal norms divisible}, as $L/K$ runs over all finite extensions of $K$,
the intersection of the groups $(\Norm_{L/K} L^*) \cap U_K$ is trivial; in particular, the intersection of the compact subsets 
\[
((\Norm_{L/K} L^*) \cap U_K) \cap (U_K \setminus U)
\]
of $U_K$ is empty. By the finite intersection property (and taking a compositum), there exists a single $L/K$ for which $(\Norm_{L/K} L^*) \cap U_K \subseteq U \cap U_K$;
this completes the proof of Theorem~\ref{T:local existence}.


\head{Making things explicit}

It is natural to ask whether the local reciprocity map can be described more explicitly. In fact, given an explicit cocycle $\phi$ generating $H^2(L/K)$, we can trace 
through the arguments to get the local reciprocity map. However, the argument is somewhat messy, so I won't torture you with all of the details; the point is
simply to observe that everything we've done can be used for explicit
computations. (This observation is apparently due to Dwork.) If you find
this indigestible, you may hold out until we hit abstract class field theory; that
point of view will give a different (though of course related) mechanism
for computing the reciprocity map.

Put $G = \Gal(L/K)$.
First recall
that $G^{\ab} = H^{-2}_T(G, \ZZ)$ is isomorphic to $H^{-1}_T(G, I_G)
= I_G/I_G^2$, with $g \mapsto [g]-1$. Next, use the exact sequence
\[
0 \to M \to M[\phi] \to I_G \to 0
\]
and apply the ``snaking'' construction: pull $[g]-1$ back to
$x_g \in M[\phi]$, take the norm to get $\prod_h x_g^h = \prod_h
(x_{gh} x_h^{-1} \phi(e,h,gh))$ (switching to multiplicative notation).
The $x_{gh}$ and $x_h$ term cancel out when you take the product, so we
get $\prod_h \phi(e, h, gh) \in L^*$ as the inverse image of $g \in \Gal(L/K)$.

As noted above, one needs $\phi$ to make this truly explicit; one can get
$\phi$ using explicit generators of $L/K$ if you have them. For $K = \QQ_p$,
one can use roots of unity; for general $K$, one can use the Lubin-Tate
construction. In general, one can at least do the following, imitating
our proof that $H^2(L/K)$ is cyclic of order $n$. Let $M/K$
be unramified of degree $n$; then $H^2(M/K) \to H^2(ML/K)$ is injective,
and its image lies in the image of $H^2(L/K) \to H^2(ML/K)$.

Now $H^2(M/K)$ is isomorphic to $H^0_T(M/K) = K^*/\Norm_{M/K}M^*$, which
is generated by a uniformizer $\pi \in K$. To explicate that isomorphism,
we recall generally how to construct the isomorphism
$H^0_T(G,M) \to H^2_T(G,M)$ for $G$ cyclic with a distinguished generator $g$.
Recall the exact sequence we used to produce the isomorphism
in Theorem~\ref{T:cyclic group periodicity}:
\[
0 \to M \to M \otimes_{\ZZ} \ZZ[G] \to M \otimes_{\ZZ} \ZZ[G] \to M \to 0.
\]
(Remember, $G$ acts on both factors in $M \otimes_{\ZZ}
\ZZ[G]$. The first map is $m \mapsto \sum_{h \in G} m \otimes [h]$,
the second is $m \otimes [h] \mapsto m \otimes ([gh] - [h])$, and
the third is $[h] \mapsto 1$.)
Let $A = M \otimes_{\ZZ} I_G$ be the kernel of the third arrow, so
$0 \to M \to M \otimes_{\ZZ} \ZZ[G] \to A \to 0$
and $0 \to A \to M \otimes_{\ZZ} \ZZ[G] \to M \to 0$ are exact.

Given $x \in H^0_T(M/K) = M^G/\Norm_G(M)$, lift it to $x \otimes [1]$.
Now view this as a 0-cochain $\phi_0: G \to M \otimes_{\ZZ} \ZZ[G]$ given by
$\phi_0(h) = x \otimes [h]$. Apply $d$ to get a 1-cocycle:
\[
\phi_1(h_0, h_1) = \phi_0(h_1) - \phi_0(h_0) = x \otimes ([h_1]- [h_0])
\]
which actually takes values in $A$. Now snake again: pull this back to
a 1-cochain $\psi_1: G^2 \to M \otimes_{\ZZ} \ZZ[G]$ given by
\[
\psi_1(g^i, g^{i+j}) = x \otimes ([g^i] + [g^{i+1}] + \cdots + [g^{j-1}])
\]
for $i,j=0, \dots, \#G-1$.
Apply $d$ again: now we have a 2-cocycle $\psi_2: G^3 \to M \otimes_{\ZZ}
\ZZ[G]$ given by (again for $i,j=0, \dots, \#G-1$)
\begin{align*}
\psi_2(e, g^i, g^{i+j}) &= \psi_1(g^i, g^{i+j}) - \psi_1(e, g^{i+j})
+ \psi_1(e, g^i) \\
&= x \otimes ([e] + \cdots + [g^{i-1}] + [g^i] + \cdots + [g^{i+j-1}]
- [e] - \cdots - [g^{i+j-1}]) \\
&= \begin{cases} 0 & i+j < \#G \\
-x \otimes ([e] + \cdots + [g^{\#G-1}]) & i+j \geq \#G.
\end{cases}
\end{align*}
This pulls back to a 2-cocycle $\phi_2: G^3 \to M$ given by
\[
\phi_2(e, g^i, g^{i+j}) = \begin{cases} 0 & i+j < \#G \\
-x & i+j \geq \#G.
\end{cases}
\]
If you prefer, you can shift by a coboundary to get $x$ if $i+j < \#G$
and 0 if $i+j \geq \#G$.

Back to the desired computation. Applying this to $\Gal(M/K)$ acting on
$M^*$, with the canonical generator $g$ equal to the Frobenius,
we get that $H^2(M/K)$ is generated by a cocycle $\phi$ with
$\phi(e, g^i, g^{i+j}) = \pi$ if $i+j < \#G$ and 1 otherwise. Now push
this into $H^2(ML/K)$; the general theory says the image comes from 
$H^2(L/K)$. That is, for $h \in \Gal(ML/K)$, let $f(h)$ be the integer $i$
such that $h$ restricted to $\Gal(M/K)$ equals $g^i$. Then
there exists a 1-cochain $\rho: \Gal(ML/K)^2 \to (ML)^*$ such that
$\phi(e, h_1, h_2h_1) /(\rho(h_1, h_2h_1) \rho(e, h_2h_1)^{-1} \rho(e, h_1))$
belongs to $L^*$ and depends only on the images of $h_1, h_2$ in $\Gal(M/K)$.
Putting $\sigma(h) = \rho(e, h)$, we thus have
\[
\frac{\phi(e, h_1, h_2h_1) \sigma(h_2h_1)}{\sigma(h_2)^{h_1} \sigma(h_1)}
\]
depends only on $h_1, h_2$ modulo $\Gal(ML/L)$.

The upshot: once you compute such a $\sigma$ (which I won't describe how to
do, since it requires an explicit description of $L/K$), to
find the inverse image of $g \in \Gal(L/K)$ under the Artin map, 
choose a lift $g_1$ of $g$ into $\Gal(ML/K)$, then compute
\[
\prod_h \frac{\phi(e, h, gh) \sigma(gh)}{\sigma(g)^h \sigma(h)}
\]
for $h$ running over a set of lifts of the elements of $\Gal(L/K)$ into
$\Gal(ML/K)$.

\head{Exercises}

\begin{enumerate}
\item
Prove that for any local field $K$ and any positive integer $n$ not divisible by the characteristic of $K$, the group
$K^*/(K^*)^{n}$ is finite.
\item
Prove that for any local field $K$ of characteristic $0$, the intersection of the groups
$(K^*)^n$ over all positive integers $n$ is the trivial group. (Hint: first get the intersection into $\gotho_K^*$, then use prime-to-$p$ exponents to get it into the 1-units, then use powers of $p$ to finish. The last step is the only one which fails in characteristic $p$.)
\item
Extend Lemma~\ref{L:hilbert symbol} to the case where $\ell$ is an arbitrary positive integer, not necessarily prime. (Hint: it may help to use the structure theorem for finite abelian groups.)
\end{enumerate}

%\end{document}


\chapter{Abstract class field theory}
\label{chap:abstractcft}
%\documentclass[12pt]{article}
%\usepackage{amsfonts, amsthm, amsmath}
%\usepackage[all]{xy}
%
%\setlength{\textwidth}{6.5in}
%\setlength{\oddsidemargin}{0in}
%\setlength{\textheight}{8.5in}
%\setlength{\topmargin}{0in}
%\setlength{\headheight}{0in}
%\setlength{\headsep}{0in}
%\setlength{\parskip}{0pt}
%\setlength{\parindent}{20pt}
%
%\def\kbar{\overline{k}}
%\def\AA{\mathbb{A}}
%\def\CC{\mathbb{C}}
%\def\FF{\mathbb{F}}
%\def\NN{\mathbb{N}}
%\def\PP{\mathbb{P}}
%\def\QQ{\mathbb{Q}}
%\def\RR{\mathbb{R}}
%\def\ZZ{\mathbb{Z}}
%\def\gotha{\mathfrak{a}}
%\def\gothb{\mathfrak{b}}
%\def\gothm{\mathfrak{m}}
%\def\gotho{\mathfrak{o}}
%\def\gothp{\mathfrak{p}}
%\def\gothq{\mathfrak{q}}
%\def\gothr{\mathfrak{r}}
%\DeclareMathOperator{\ab}{ab}
%\DeclareMathOperator{\coker}{coker}
%\DeclareMathOperator{\cyc}{cyc}
%\DeclareMathOperator{\disc}{Disc}
%\DeclareMathOperator{\Frob}{Frob}
%\DeclareMathOperator{\Gal}{Gal}
%\DeclareMathOperator{\GL}{GL}
%\DeclareMathOperator{\Hom}{Hom}
%\DeclareMathOperator{\im}{im}
%\DeclareMathOperator{\Ind}{Ind}
%\DeclareMathOperator{\Inf}{Inf}
%\DeclareMathOperator{\inv}{inv}
%\DeclareMathOperator{\Norm}{Norm}
%\DeclareMathOperator{\Res}{Res}
%\DeclareMathOperator{\Trace}{Trace}
%\DeclareMathOperator{\unr}{unr}
%\DeclareMathOperator{\Ver}{Ver}
%\DeclareMathOperator{\Cl}{Cl}
%
%\def\head#1{\medskip \noindent \textbf{#1}.}
%
%\newtheorem{theorem}{Theorem}
%\newtheorem{lemma}[theorem]{Lemma}
%\newtheorem{prop}[theorem]{Proposition}
%\newtheorem{cor}[theorem]{Corollary}
%
%\begin{document}
%
%\begin{center}
%\bf
%Math 254B, UC Berkeley, Spring 2002 (Kedlaya) \\
%Abstract Class Field Theory
%\end{center}

\head{Reference} Neukirch, IV.4-IV.6. Remember that Neukirch's cohomology
groups are all Tate groups, so he doesn't put the subscript ``T'' on them.

\medskip
We now turn to an alternate method for deriving the main result of local
class field theory, the local reciprocity law. This method, based on a
presentation of Artin and Tate, makes it clear what the main cohomological
inputs are in the local case, and gives an outline of how to proceed to
global class field theory. (Warning: this method does not give information
about the local invariant map.)

\head{Caveat} We are going to work with the absolute Galois group of a
field $K$, i.e., the Galois group of its algebraic closure. One could work
with a smaller overfield as well. In fact, one can go further: one really
is working with the Galois group and not the fields, so one can replace the
Galois group by an arbitrary profinite group! This is what Neukirch does, but
fortunately he softens the blow by ``pretending'' that his profinite group
corresponds to a field and its extensions via the Galois correspondence.
This means you can simply assume that his group $G$ is the absolute Galois
group of a field without getting confused.

\head{Caveat} Certain words you thought you knew what they meant, such as
``unramified'', are going to be reassigned more abstract meanings. But
these meanings will coincide with the correct definitions over a local field.

\head{Abstract ramification theory}

Let $k$ be a field, $\kbar$ a separable closure of $k$, and
$G = \Gal(\kbar/k)$. Let $d: G \to \widehat{\ZZ}$ be a continuous
surjective homomorphism. The example we have in mind is when
$k$ is a local field
and $d$ is the surjection of $G$ onto
$\Gal(k^{\unr}/k) \cong \widehat{\ZZ}$.

We now make some constructions that, in our example, recover information
about ramification of extensions of $k$. For starters, define the
\emph{inertia group} $I_k$ as the kernel of $d$, and define
the \emph{maximal unramified extension} $k^{\unr}$ of $k$ as the fixed
field of $I_k$. More generally, for any field $L$ between $k$ and $\kbar$,
put $G_L = \Gal(\kbar/L)$,
put $I_L = G_L \cap I_k$ and let $L^{\unr}$ be the fixed field of $I_L$.
We say an extension $L/K$ is \emph{unramified} if $L \subseteq K^{\unr}$.
Note that this implies that $G_L$ contains $I_K$, necessarily as a normal
subgroup, and $G_L/I_K \subseteq G_K/I_K$ injects via $d$ into
$\widehat{\ZZ}$; thus $G_L/I_K$ is abelian and any finite quotient of it
is cyclic. In particular, $G_K$ is Galois in $G_L$ and $\Gal(L/K) =
G_L/G_K$ is cyclic. (Note also that $K^{\unr}$ is the compositum of
$K$ and $k^{\unr}$.) 
If $K \neq k$, then $d$ doesn't map $G_K$ onto $\widehat{\ZZ}$, so it
will be convenient to renormalize things. Put
\[
d_K = \frac{1}{[\widehat{\ZZ}: d(G_K)]}d: G_K \to \widehat{\ZZ};
\]
then $d_K$ is surjective, and induces an isomorphism $d_K: \Gal(K^{\unr}/K)
\to \widehat{\ZZ}$.

Given a finite extension $L/K$ of fields between $k$ and $\kbar$,
define the \emph{inertia degree} (or \emph{residue field degree})
$f_{L/K} = [d(G_K):d(G_L)]$
and the \emph{ramification degree} $e_{L/K} = [I_K:I_L]$. By design
we have multiplicativity: $e_{M/K} = e_{M/L}e_{L/K}$ and
$f_{M/K} = f_{M/L}f_{L/K}$. Moreover, if $L/K$ is Galois, we have
an exact sequence
\[
1 \to I_K/I_L \to \Gal(L/K) \to d(G_K)/d(G_L) \to 1
\]
so the ``fundamental identity'' holds:
\[
e_{L/K}f_{L/K} = [L:K].
\]
The same is true if $L/K$ is not Galois: let $M$ be a Galois extension
of $K$ containing $L$, then apply the fundamental identity to $M/L$
and $M/K$ and use multiplicativity.

\head{Abstract valuation theory}

Now suppose that, in addition to the field $k$ and the map $d: G \to \widehat{\ZZ}$,
we have a $G$-module $A$ (written multiplicatively) and a homomorphism
$v: A^G \to \widehat{\ZZ}$. We wish to write down conditions that will be
satisfied in case $k$ is a local field (with $d$ as before, $A = \kbar^*$
and $v: k^* \to \ZZ$ the valuation of the local field), but which will
in general give a notion of ``valuation'' on all of $A$.

Given $k$, $d: G \to \widehat{\ZZ}$, and the $G$-module $A$, write
$A_K = A^{G_K} = A^{\Gal(K/k)}$ for any field $K$ between $k$ and $\kbar$.
Also, recall that the norm map $\Norm_{L/K}: A_L \to A_K$
is given by $\Norm_{L/K}(a) = \prod_g a^g$, where $g$ runs over a set
of right coset representatives of $G_K$ in $G_L$, at least when $L$ is
finite. (The norm doesn't make sense for an infinite extension, but it
still makes sense to write $\Norm_{L/K} A_L$ to mean the intersection
of $\Norm_{M/K} A_M$ over all finite subextensions $M/K$ of $L$.)

A \emph{henselian valuation} of $A_k$ with respect to $d$ is a homomorphism
$v: A_k \to \widehat{\ZZ}$ such that:
\begin{enumerate}
\item[(a)] if $Z = \im(v)$, then $Z$ contains $\ZZ$ and $Z/nZ \cong
\ZZ/n\ZZ$ for all positive integers $n$;
\item[(b)] $v(\Norm_{K/k} A_K) = f_{K/k} Z$ for all finite extensions $K$
of $k$.
\end{enumerate}
This valuation immediately extends to a valuation $v_K: A_K \to Z$ for all
fields $K$ between $k$ and $\kbar$, by setting
\[
v_K = \frac{1}{f_{K/k}} \circ \Norm_{K/k}.
\]
Then $v_{K}(a) = v_{K^g}(a^g)$ for any $a \in A$ and $g \in G$,
and for $L/K$ a finite extension,
$v_K(\Norm_{L/K}(a)) = f_{L/K} v_L(a)$ for any $a \in A_L$.

For any field $K$ between $k$ and $\kbar$,
define the \emph{unit subgroup} $U_K$ as the set of $u \in A_K$ with
$v_k(u) = 0$. If $K/k$ is finite, we say $\pi \in A_K$ is a \emph{uniformizer} for $K$
if $v_K(\pi) = 1$.

\head{The reciprocity map: definition}

\head{Warning} The multiplicativity of the reciprocity map is proven 
in Neukirch (Proposition IV.5.5), but I find this proof unreadable.

\medskip
Now we bring in the key cohomological input. Suppose that for every
\emph{cyclic} extension $L/K$ of finite extensions of $k$,
\[
\#H^i_T(\Gal(L/K), A_L) = \begin{cases} [L:K] & i=0 \\
1 & i = -1. \end{cases}
\]
In Neukirch, this assumption is called the \emph{class field axiom}.
(Note that it's not enough just to check cyclic extensions of $k$ itself.)
Then we will prove the following theorem.
\begin{theorem}[Reciprocity law]
For each finite Galois extension $L/K$ of finite extensions of $k$,
there is a canonical isomorphism $r_{L/K}: \Gal(L/K)^{\ab}
\to A_K / \Norm_{L/K} A_L$.
\end{theorem}
Since we've already checked the class field axiom in the example where
$k$ is a local field and $A = \kbar^*$, this immediately recovers the
local reciprocity law.

Before defining the reciprocity map, we verify a consequence of the class
field axiom. (Notice the similarities between this argument and what we
have done; essentially we are running the computation of the cohomology
of an unramified extension of local fields in reverse!)
\begin{prop}
For $L/K$ an unramified extension of finite extensions of $k$
(i.e., $e_{L/K} = 1$), the class field axiom implies that
$H^i_T(\Gal(L/K), U_L) = 1$ for $i=0, -1$. Moreover,
$H^1_T(\Gal(L/K), A_L)$ is cyclic and is generated by any
uniformizer $\pi_L$ for $L$.
\end{prop}
\begin{proof}
We'll drop $\Gal(L/K)$ from the notation, because it's the same group
throughout the proof.
Note that an unramified extension is always Galois and cyclic.
Consider the short exact sequence $0 \to U_L \to A_L \to A_L/U_L \to 0$.
Applying Herbrand quotients, we have $h(A_L) = h(U_L) h(A_L/U_L)$,
where $h(A_L) = \#H^0_T(A_L)/\#H^{-1}_T(A_L)$ and so on. By
the class field axiom, $h(A_K) = [L:K]$. Also, $A_L/U_L$ is isomorphic
to $Z = \im(v)$ with trivial group action, so $H^0_T(Z)$ is cyclic
of order $[L:K]$ and $H^{-1}_T(Z)$ is trivial. (Recall
that $H^0_T(Z) = Z/\Norm(Z)$ and $H^{-1}_T(Z) = \ker(\Norm)$,
since the action is trivial.)
Otherwise put, the long exact sequence in Tate groups gives
\[
1 = H^{-1}_T(A_L/U_L) \to
H^{0}_T(U_L) \to H^{0}_T(A_L) \to H^{0}_T(A_L/U_L)
\to H^{1}_T(U_L) \to H^1_T(A_L) = 1
\]
and the two groups in the middle have the same order, so we just have
to show that one of the outer groups is trivial, and then the middle
map will be an isomorphism.

Thus it suffices to check that $H^{1}_T(U_L) = 1$,
or equivalently $H^{-1}_T(U_L) = 1$.
Here is where we use that $L/K$ is unramified, not just cyclic.
Recall that $H^{-1}_T(U_L)$ consists of elements
$u$ of $U_L$ of norm 1, modulo those of the form $v^\sigma/v$ for some
$v \in U_L$, where $\sigma$ is a generator of $\Gal(L/K)$.
By hypothesis, $H^{-1}_T(A_L)$ is trivial,
so any $u \in U_L$ of norm 1 can be written as $w^\sigma/w$ for some
$w \in A_L$. Now because $L/K$ is unramified, there exists
$x \in A_K$ such that $w/x \in U_L$. Now $u = v^\sigma/v$ for
$v = w/x$, so $u$ defines the trivial class in $H^{-1}_T(U_L)$, proving the claim.
\end{proof}
\begin{cor} \label{C:abstract unramified}
  If $L/K$ is unramified, then $U_K = \Norm_{L/K} U_L$.
(Remember, this makes sense even if $L/K$ is not finite!)
\end{cor}

We now define the reciprocity map $r: \Gal(L/K) \to A_K/\Norm_{L/K} A_L$;
as a bonus, this definition will actually give an explicit recipe for
computing the reciprocity map in local class field theory.
For starters, let $H$ be the semigroup of $g \in \Gal(L^{\unr}/K)$ such
that $d_K(g)$ is a positive integer. Define the map $r': H \to A_K/\Norm_{L/K}
A_{L}$ as follows. For $g \in \Gal(L^{\unr}/K)$, let $M$ be
the fixed field of $g$ (so that $e(M/K) = e((M \cap L)/K)$ and
$f(M/K) = d_K(g)$),
and set $r'(g) = \Norm_{M/K} (\pi_M)$ for some uniformizer $\pi_M$.
This doesn't depend on the choice of uniformizer: if $\pi'_M$
is another one, then $\pi_M/\pi'_M \in U_L$ belongs
to $\Norm_{L^{\unr}/L} U_{L^{\unr}}$ by Corollary~\ref{C:abstract unramified}, so
$\Norm_{M/K} (\pi_M/\pi'_M)$ belongs to
$\Norm_{L^{\unr}/K} U_{L^{\unr}} \subseteq \Norm_{L/K} U_L$.
So at least $r'$ is now a well-defined map, if not yet a semigroup
homomorphism. 

Let's make some other easy observations about this definition before
doing the hard stuff. Note that $r'$ is invariant under conjugation:
if we replace $g$ by $h^{-1}gh$, then its fixed field $M$ is replaced
by $M^h$ and we can take the uniformizer $\pi_M^h$.
Also, if $g \in H$ is actually in $\Gal(L^{\unr}/L)$,
then $r'(g) \in \Norm_{L/K} A_L$. In that case, $M$ contains $L$,
so $r'(g) = \Norm_{M/K}(\pi_M)$ can be rewritten as
$\Norm_{L/K} \Norm_{M/L} (\pi_M)$, so is clearly a norm. That is, if
$r'$ were known to be multiplicative, it would induce a group
homomorphism from $\Gal(L/K)$ to $A_K/\Norm_{L/K} A_L$.

Now for the hard part: we have to check
that $r'$ is multiplicative. Let $g_1, g_2 \in H$ be arbitrary,
and put $g_3 = g_1g_2$.
Let $M_i$ be the fixed field of $g_i$,
let $\pi_i$ be a uniformizer of $M_i$, and put
$\rho_i = r(g_i) = \Norm_{M_i/K}(\pi_i)$. Again, we want
$\rho_1\rho_2/\rho_3$ to be in $\Norm_{L^{\unr}/K} A_{L^{\unr}}$;
what makes this hard is that the $\rho_i$ all lie in different fields
over $K$. 
At least one thing is clear: $v_K(\rho_i) = f(M_i/K) v_{M_i}(\pi_i)
= f(M_i/K) = d_K(g_i)$, so $v_K(\rho_1 \rho_2 /\rho_3) = 0$.

To make progress, we have to push our problem into a single field.
Choose $\phi \in \Gal(L^{\unr}/K)$ such that $d_K(\phi) = 1$,
and put $d_i = d_K(g_i)$; then we can write
$g_i = \phi^{d_i} h_i$ for some $h_i$ with $d_K(h_i) = 0$, that is,
$h_i \in \Gal(L^{\unr}/K^{\unr})$.
Put
\[
\sigma_i = \pi_i \pi_i^\phi \cdots \pi_i^{\phi^{d_i-1}};
\]
then $\rho_i = \Norm_{L^{\unr}/K^{\unr}}(\sigma_i)$.
\begin{prop} \label{prop:norm}
Let $M$ be the fixed field of some $h \in \Gal(L^{\unr}/K)$
with $d_K(h) = n$ a positive integer,
and suppose $\phi \in H$ satisfies $d_K(\phi) = 1$. Then
for any $x \in A_M$,
\[
\Norm_{M/K}(x) = \Norm_{L^{\unr}/K^{\unr}}(x x^\phi \cdots x^{\phi^{n-1}}).
\]
\end{prop}

Now put $u = \sigma_1 \sigma_2/\sigma_3$; then $u \in U_{L^{\unr}}$
and $\Norm_{L^{\unr}/K^{\unr}}(u) = \rho_1 \rho_2/\rho_3$ is the thing
we need to be in $\Norm_{L/K} U_L$. Let $N$ be a finite unramified
extension of $L$ such that $u \in U_N$. Then
$\Norm_{L^{\unr}/K^{\unr}}(u) = \Norm_{N/N \cap K^{\unr}}(u)$,
and by the lemma below, that implies that $u \in \Norm_{N/K} U_N$
and so $u \in \Norm_{L/K}(U_L)$.

\begin{lemma}
If $M/L$ and $L/K$ are finite extensions with $M/K$ Galois and
$L/K$ unramified, and
$u \in U_M$ is such that $\Norm_{M/L}(u) \in U_K$, then
$\Norm_{M/L}(u) \in \Norm_{M/K} U_L$.
\end{lemma}
\begin{proof}
There is a noncohomological proof in Neukirch (Lemma~IV.5.4), but
I couldn't follow it, so here's a cohomological argument instead.
If $v = \Norm_{M/L}(u) \in U_K$, then
$v$ represents an element of $H^0_T(\Gal(M/K), U_M) = U_K/\Norm_{M/K}(U_M)$
which maps
to zero under the map $\Res: H^0_T(\Gal(M/K), U_M) \to H^0_T(\Gal(M/L),
U_M)$. By the following lemma, $v$ is then in the image of
$\Inf: H^0_T(\Gal(L/K), U_L) \to H^0_T(\Gal(M/K), U_M)$; but the
former space is zero by Corollary~\ref{C:abstract unramified}! Thus $v$ is cohomologous to zero in
$H^0_T(\Gal(M/K), U_M)$; that is, $v = \Norm_{M/K}(w)$ for
some $w \in U_M$.
\end{proof}
This lemma is of course a variant of the inflation-restriction
exact sequence; we get it from there by dimension shifting.
\begin{lemma}
Let $H$ be a normal subgroup of a finite group $G$ and $M$ a
$G$-module. Then
the sequence
\[
0 \to H^0_T(G/H, M^H) \stackrel{\Inf}{\to}
H^0_T(G,M) \stackrel{\Res}{\to}
H^0_T(H,M)
\]
is exact.
\end{lemma}
\begin{proof}
Choose $N$ so that
$0 \to N \to \Ind^G_{1} M \to M \to 0$ is exact
(where again $\Ind^G_1 M \to M$ is the map $m \otimes [g] \mapsto m^g$);
then by the usual inflation-restriction
exact sequence (Proposition~\ref{P:inflation restriction}),
\[
0 \to H^1_T(G/H, N^H) \stackrel{\Inf}{\to} H^1_T(G,N)
\stackrel{\Res}{\to} H^1_T(H,N)
\]
is exact.
Now $\Ind^G_{1} M$ is acyclic
for $G$ and for $H$, and $(\Ind^G_1 M)^H$ is acyclic for $G/H$.
Moreover, if we take $H$-invariants, we have
an exact sequence
\[
0 \to N^H \to (\Ind^G_1 M)^H \to M^H \to 0;
\]
namely, exactness on the right holds because any $m \in M^H$
lifts to $m \otimes [1] \in (\Ind^G_1 M)^H$.
Using long exact sequences, we may thus shift dimensions to deduce the desired result.
\end{proof}

Putting everything together, we have a semigroup homomorphism $r': H
\to A_K/\Norm_{L/K}A_L$ which kills $\Gal(L^{\unr}/L)$.
Thus $r'$ induces a homomorphism $r = r_{L/K}: \Gal(L/K) \to
A_K/\Norm_{L/K}A_L$. We call this the \emph{reciprocity map}.
Some straightforward functorialities are left to the reader, including the following.
\begin{prop} \label{P:abstract functorialities}
  If $L/K$ and $L'/K'$ are finite Galois extensions such that
$K \subseteq K'$ and $L \subseteq L'$, then the natural map
$\Gal(L'/K')^{\ab} \to \Gal(L/K)^{\ab}$ is compatible via the reciprocity
map with $\Norm_{K'/K}: A_{K'} \to A_K$. If moreover $K' \subseteq L$,
then the natural map $A_K \to A_{K'}$ is compatible with the transfer
map $\Ver: \Gal(L/K)^{\ab} \to \Gal(L'/K')^{\ab}$.
\end{prop}

\head{Proof of the reciprocity law}

We continue to assume the class field axiom.
Recall that we want the following result.
\begin{theorem}[Reciprocity law] \label{T:abstract reciprocity law}
For each finite Galois extension $L/K$ of finite extensions of $k$,
there is a canonical isomorphism $r_{L/K}: \Gal(L/K)^{\ab}
\to A_K / \Norm_{L/K} A_L$.
\end{theorem}
From the definition of $r$, it's easy enough to check this for
$L/K$ unramified.
\begin{prop}
  If $L/K$ is finite unramified, the reciprocity map
$r_{L/K}$ sends the Frobenius of $\Gal(L/K)$ to a uniformizer of $K$, and is
an isomorphism.
\end{prop}
\begin{proof}
The groups $\Gal(L/K)$ and $A_K/\Norm_{L/K}(A_L) = H^0_T(\Gal(L/K), A_L)$
are both cyclic of the same order $[L:K]$, the latter by the class field
axiom. If $g \in \Gal(L/K)$ is the Frobenius, and $h \in \Gal(L^{\unr}/K)$
lifts $h$, then the fixed field of $h$ is just $K$ itself, and 
from the definition of $r'$, $r(g) = r'(h)$ is just a uniformizer
of $K$. Since that uniformizer generates $H^0(\Gal(L/K), A_L)$, we conclude
$r_{L/K}$ is an isomorphism.
\end{proof}
\begin{prop}
  If $L/K$ is finite, cyclic and totally ramified (i.e., $f_{L/K} = 1$), then
$r_{L/K}$ is an isomorphism.
\end{prop}  
\begin{proof}
Since $r_{L/K}$ maps between two groups of the same order by the
$H^0_T$ clause of the class field axiom, it suffices
to show that it is injective.

The extension $L^{\unr}/K$ is the compositum of two linearly disjoint
extensions $L/K$ and $K^{\unr}/K$, so its Galois group is canonically a
product $\Gal(L/K) \times \Gal(K^{\unr}/K)$. Let $g$ be a generator of
the first factor and $\phi$ a generator of the second factor. Put
$\tau = g\phi$, so that $d_K(\tau)=1$, and let $M$ be the fixed field
of $\tau$.
Pick uniformizers $pi_L$ and $\pi_M$ of $L$ and
$M$, so that
$r(g) = r'(\tau) = \Norm_{M/K}(\pi_M)$. Let $N$ be the compositum
of $L$ and $M$.

Put $n = [L:K]$, and suppose $r(g^j) = \Norm_{M/K}(\pi_M^j)$ is 
the identity in $A_K/\Norm_{L/K} A_L$.
Since $d_K(\tau) = 1$, we have
$r(g) = \Norm_{L^{\unr}/K^{\unr}}(\pi_M)$. On the other hand
(by Proposition~\ref{prop:norm} with $n=0$!),
$\Norm_{L^{\unr}/K^{\unr}}(\pi_L)$ is the identity in
$A_K/\Norm_{L/K} A_L$. Thus we also have
$r(g) = \Norm_{L^{\unr}/K^{\unr}}(\pi_M/\pi_L)$.

Put $u = \pi_L^j/\pi_M^j \in U_N$. If $r(g^j)$ is in $\Norm_{L/K} A_L$, 
then there exists $v \in U_L$ such that $\Norm_{L^{\unr}/K^{\unr}}(v)
= \Norm_{L^{\unr}/K^{\unr}}(u)$. By the $H^{-1}_T$ clause of the
class field axiom, we can write $u/v$ as $a^g/a$ for some
$a \in A_N$. Now
\[
(\pi_L^j/v)^{g-1} = (\pi_L^j/v)^{\tau-1} = (\pi_M^j u/v)^{\tau - 1}
= (u/v)^{\tau-1} = (a^\tau/a)^{g-1}.
\]
If we put $x = (\pi_L^j/v)(a/a^\tau)$, that means $x$ is $g$-invariant,
so it belongs to $A_{N_0}$, where $N_0 = N \cap K^{\unr}$. On one hand,
that means $v_{N_0}(x) \in \widehat{\ZZ}$. On the other hand,
we have $nv_{N_0}(x) = v_N(x) = j$. Thus $j$ is a multiple of $n$,
and $r$ must be injective.
\end{proof}

Now we proceed to the proof of the reciprocity law. Any resemblance
with the method used to calculate the local invariant map is not
coincidental!
\begin{proof}[Proof of Theorem~\ref{T:abstract reciprocity law}]
For reference, we record the following commutative diagram, for $L/K$
a finite extension and $M$ an intermediate field:
\[
\xymatrix{
1 \ar[r] & \Gal(L/M) \ar[r] \ar^{r_{L/M}}[d] & \Gal(L/K)
\ar[r] \ar^{r_{L/K}}[d] & \Gal(M/K) \ar[r] \ar[d]^{r_{M/K}} & 1 \\
& A_M/\Norm_{L/M}A_L \ar^{\Norm_{M/K}}[r] & A_K/\Norm_{L/K} A_L \ar[r] &
A_K / \Norm_{M/K} A_M \ar[r] & 1
}
\]
in which the rows are exact. We're going to do a lot of diagram-chasing
on this picture. 

First suppose $L/K$ is abelian; we induct on $[L:K]$.
If $L/K$ is cyclic of prime order, 
then either it is unramified or totally ramified, and we already know
$r_{L/K}$ is an isomorphism in those cases. Otherwise, let $M$ be
a subextension of $L/K$. Then chasing the above diagram gives that
$r_{L/K}$ is surjective. Now the diagram shows that the kernel of $r_{L/K}$
lies in the kernel of $\Gal(L/K) \to \Gal(N/K)$ for \emph{every}
proper subextension $N$ of $L/K$. Since $L/K$ is abelian, the intersection
of these kernels is trivial. Thus $r_{L/K}$ is also injective, so is 
an isomorphism.

Next, suppose $L/K$ is solvable; we again induct on $[L:K]$.
If $L$ is abelian, we are done. If not, let $M$ be the maximal abelian
subextension of $L/K$; by the same diagram chase as in the
previous paragraph, $r_{L/K}$ is surjective.
Also, we have a diagram
\[
\xymatrix{
\Gal(L/K)^{\ab} \ar^{r_{L/K}}[r] \ar[d] & A_K / \Norm_{L/K} A_L \ar[d] \\
\Gal(M/K) \ar^{r_{M/K}}[r] & A_K / \Norm_{M/K} A_M
}
\]
in which the left vertical and bottom horizontal arrows are isomorphisms.
Thus the composite $\Gal(L/K)^{\ab} \to A_K / \Norm_{M/K} A_M$ is
an isomorphism, so $r_{L/K}$ must be injective. Again, we conclude
$r_{L/K}$ is an isomorphism.

Finally, let $L/K$ be not solvable. The same argument as in the previous
paragraph shows that $r_{L/K}$ is injective. To show $r_{L/K}$ is surjective,
let $M$ be the fixed field of a $p$-Sylow
subgroup of $\Gal(L/K)$. Then $M/K$ need not be Galois, so the original diagram
doesn't actually make sense. But the square on the left still commutes,
and $r_{L/M}$ is an isomorphism by what we already know. If we can show
the bottom arrow $\Norm_{M/K}$ surjects onto the $p$-Sylow subgroup
$S_p$ of $A_K/\Norm_{L/K}A_L$, then the same will be true of $r_{L/K}$. In fact,
the inclusion $A_K \subseteq A_M$ induces a homomorphism
$i: A_K/\Norm_{L/K}A_L \to A_M /\Norm_{L/M} A_L$ such that
$\Norm_{M/K} \circ i$ is multiplication by $[M:K]$, which is not divisible
by $p$, and so is an isomorphism on $S_p$. Thus $\Norm_{M/K}$ surjects onto
$S_p$, as does $r_{L/K}$; since $r_{L/K}$ surjects onto each $p$-Sylow
subgroup of $A_K/\Norm_{L/K}A_L$, it is in fact surjective.
\end{proof}
As a bonus byproduct of the proof, we get the following.
\begin{cor}[Norm limitation theorem]
If $M$ is the maximal abelian subextension of the finite Galois extension
$L/K$, then $\Norm_{L/K} A_L = \Norm_{M/K} A_M$.
\end{cor}

\head{A look ahead}

What does this tell us about the global Artin reciprocity law? If $L/K$
is a finite abelian extension of number fields, we are trying to
prove that $\Gal(L/K)$ is canonically isomorphic to a generalized ideal
class group of $K$. So we need to use for $A$ something related to ideal
classes. You might try taking the group of fractional ideals in $L$, then
taking the direct limit over all finite extensions $L$ of $K$.
In this case,
we would have to find $H^i(\Gal(L/K), A_L)$ for $A_L$ the group of fractional
ideals in $L$, where $L/K$ is cyclic and $i=0, -1$. Unfortunately, these groups
are not so well-behaved as that!

The cohomology groups would behave better if $A_L$ were
``complete'' in some sense, the way that $K^*$ is complete when $K$ is
a local field. But there is no good reason to distinguish one place over
another in the global case. So we're going to make the target group $A$
by ``completing $K^*$ at all places simultaneously''.

Even without $A$, I can at least tell you what $d$ is going to be over
$\QQ$. To begin with, note that there is a surjective map $\Gal(\overline{\QQ}/\QQ) \to \Gal(\QQ^{\cyc}/\QQ)$ that turns an automorphism into its action on roots of unity.
The latter group is unfortunately isomorphic to the multiplicative group $\widehat{\ZZ}^*$
rather than the additive group $\widehat{\ZZ}$, but this is a start. To make more progress, write $\widehat{\ZZ}$ as the product $\prod_p \ZZ_p$, so that $\widehat{\ZZ}^* \cong \prod_p \ZZ_p^*$. Then recall that there exist isomorphisms
\[
\ZZ_p^* \cong \begin{cases} \ZZ/(p-1)\ZZ \times \ZZ_p & p > 2 \\ \ZZ/2\ZZ \times \ZZ_p & p = 2. \end{cases}
\]
In particular, $\ZZ_p^*$ modulo its torsion subgroup is isomorphic to $\ZZ_p$, but not in a canonical way. But never mind about this; let us choose an isomorphism for each $p$ and then obtain a surjective map $\widehat{\ZZ}^* \to \widehat{\ZZ}$. Composing, we get a surjective map $\Gal(\overline{\QQ}/\QQ) \to \widehat{\ZZ}$ which in principle depends on some choices, but the ultimate statements of the theory will be independent of these choices.
(Note that in this setup, every ``unramified'' extensions of a number field is a subfield of a cyclotomic extension, but not conversely.)

\head{Exercises}
\begin{enumerate}
\item
Prove Proposition~\ref{P:abstract functorialities}.
\end{enumerate}

%\end{document}


\part{The adelic formulation}

\chapter{Ad\`eles and id\`eles}
%\documentclass[12pt]{article}
%\usepackage{amsfonts, amsthm, amsmath}
%
%\setlength{\textwidth}{6.5in}
%\setlength{\oddsidemargin}{0in}
%\setlength{\textheight}{8.5in}
%\setlength{\topmargin}{0in}
%\setlength{\headheight}{0in}
%\setlength{\headsep}{0in}
%\setlength{\parskip}{0pt}
%\setlength{\parindent}{20pt}
%
%\def\AA{\mathbb{A}}
%\def\CC{\mathbb{C}}
%\def\FF{\mathbb{F}}
%\def\PP{\mathbb{P}}
%\def\QQ{\mathbb{Q}}
%\def\RR{\mathbb{R}}
%\def\ZZ{\mathbb{Z}}
%\def\gotha{\mathfrak{a}}
%\def\gothb{\mathfrak{b}}
%\def\gothm{\mathfrak{m}}
%\def\gotho{\mathfrak{o}}
%\def\gothp{\mathfrak{p}}
%\def\gothq{\mathfrak{q}}
%\DeclareMathOperator{\disc}{Disc}
%\DeclareMathOperator{\fin}{fin}
%\DeclareMathOperator{\Gal}{Gal}
%\DeclareMathOperator{\GL}{GL}
%\DeclareMathOperator{\Hom}{Hom}
%\DeclareMathOperator{\Norm}{Norm}
%\DeclareMathOperator{\Trace}{Trace}
%\DeclareMathOperator{\Cl}{Cl}
%
%\def\head#1{\medskip \noindent \textbf{#1}.}
%
%\newtheorem{theorem}{Theorem}
%\newtheorem{lemma}[theorem]{Lemma}
%\newtheorem{prop}[theorem]{Proposition}
%\newtheorem{cor}[theorem]{Corollary}
%
%\begin{document}
%
%\begin{center}
%\bf
%Math 254B, UC Berkeley, Spring 2002 (Kedlaya) \\
%Adeles and Ideles
%\end{center}

\head{Reference} 
Milne, Section V.4; Neukirch, Section VI.1 and VI.2; Lang, \textit{Algebraic Number Theory}, Chapter VII.

\medskip
The $p$-adic numbers, and more general local
fields, were introduced into number theory as a way to translate local facts about number fields (i.e., facts concerning a single prime ideal) into
statements of a topological flavor. To prove the statements of class field theory, we need an analogous global construction.
To this end, we construct a topological object that includes all of the
completions of a number field, including both the archimedean and nonarchimedean
ones. This object will be the ring of ad\`eles, and it will lead us to
the right target group for use in the abstract class field theory we
have just set up.

\head{Spelling note} 
There is a lack of consensus regarding the presence or absence of accents in  the words \emph{ad\`ele} and \emph{id\`ele}. The term \emph{id\`ele} is thought to be a contraction of ``ideal element''; it makes its first appearance, with the accent, in Chevalley's 1940 paper ``La th\'eorie du corps de classes.'' The term \emph{ad\`ele} appeared in the 1950s, possibly as a contraction of ``additive id\`ele''; it appears to have been suggested by Weil as a replacement for Tate's term ``valuation vector'' and Chevalley's term ``repartition''.
Based on this history, we have opted for the accented spellings here.

\head{Jargon watch}
By a \emph{place} of a number field $K$, we mean either an archimedean
completion $K \hookrightarrow \RR$ or $K \hookrightarrow \CC$
(an \emph{infinite place}), or a
$\gothp$-adic completion $K \hookrightarrow K_\gothp$ for some nonzero
prime ideal $\gothp$ of $\gotho_K$ (a \emph{finite place}). (Note:
there is only one place for each pair of complex embeddings of $K$.)
Each place corresponds to an equivalence class of absolute values on
$K$; if $v$ is a place, we write $K_v$ for the corresponding completion,
which is either $\RR$, $\CC$, or $K_\gothp$ for some prime $\gothp$.

\head{The ad\`eles}

The basic idea is that we want some sort of ``global completion'' of a
number field $K$. In fact, we already know one way to complete $\ZZ$,
namely its profinite completion $\widehat{\ZZ} = \prod_p \ZZ_p$. But we
really want something containing $\QQ$. We define the \emph{ring of
finite ad\`eles} $\AA^{\fin}_\QQ$ as any of the following isomorphic objects:
\begin{itemize}
\item the tensor product $\widehat{\ZZ} \otimes_{\ZZ} \QQ$;
\item the direct limit of $\frac{1}{n} \widehat{\ZZ}$ over all nonzero
integers $n$;
\item the \emph{restricted direct product} $\sideset{}{'_p}\prod \QQ_p$,
where we only allow tuples $(\alpha_p)$ for which $\alpha_p \in \ZZ_p$
for almost all $p$.
\end{itemize}
For symmetry, we really should allow \emph{all} places, not just the finite
places. So we also define the \emph{ring of ad\`eles} over $\QQ$ as
$\AA_{\QQ} = \RR \times \AA^{\fin}_{\QQ}$. Then $\AA_{\QQ}$ is a locally
compact topological ring with a canonical embedding $\QQ \hookrightarrow
\AA_{\QQ}$.

Now for a general number field $K$.
The profinite completion
$\widehat{\gotho_K}$ is canonically isomorphic to $\prod_{\gothp}
\gotho_{K_{\gothp}}$, so we define the \emph{ring of finite ad\`eles}
$\AA^{\fin}_{K}$ as any of the following isomorphic objects:
\begin{itemize}
\item the tensor product $\widehat{\gotho_K} \otimes_{\gotho_K} K$;
\item the direct limit of $\frac{1}{\alpha} \widehat{\gotho_K}$ over all
nonzero $\alpha \in \gotho_K$;
\item the \emph{restricted direct product} $\sideset{}{'_p}\prod K_{\gothp}$,
where we only allow tuples $(\alpha_p)$ for which $\alpha_p \in \gotho_{K_{\gothp}}$
for almost all $\gothp$.
\end{itemize}
The ring of ad\`eles $\AA_K$ is the product of $\AA^{\fin}_K$ with each
archimedean completion. (That's one copy of $\RR$ for each real embedding and
one copy of $\CC$ for each conjugate pair of complex embeddings.)

One has a natural norm on the ring of ad\`eles, because one has a natural norm
on each completion:
\[
|(\alpha_v)_v| = \prod_v |\alpha_v|_v.
\]
One should normalize these in the following way: for $v$ real, take
$|\cdot|_v$ to be the usual absolute value. For $v$ complex, take
$|\cdot|_v$ to be the \emph{square} of the usual absolute value. (That means
the result is not an absolute value, in that it doesn't satisfy the triangle
inequality. Sorry.) For $v$ nonarchimedean corresponding to a prime above $p$,
normalize so that $|p|_v = p^{-1}$.

Again, there is a natural embedding of $K$ into $\AA_K$ because there
is such an embedding for each completion.
With the normalization as above, one has the product formula:
\begin{prop}
If $\alpha \in K$, then $|\alpha| = 1$.
\end{prop}
In particular, $K$ is \emph{discrete} in $\AA_K$ (the difference between
two elements of $K$ cannot be simultaneously small in all embeddings).
This is a generalization/analogue of the fact that $\gotho_K$ is
discrete in Minkowski space (the product of the archimedean completions).

For any finite set $S$ of places, let $\AA_S$ (resp. $\AA^{\fin}_S$)
be the subring of $\AA_K$ (resp. $\AA^{\fin}_K$) consisting of
those ad\`eles which are integral at all finite places not contained in $S$.
Then we have the following result, which is essentially the Chinese remainder
theorem.
\begin{prop} \label{P:adelic CRT}
For any finite set $S$ of places, $K + \AA_S^{\fin} = \AA_K^{\fin}$ and
$K + \AA_S = \AA_K$.
\end{prop}
\begin{cor}
The quotient group $\AA_K/K$ is compact.
\end{cor}
\begin{proof}
Choose a compact subset $T$ of the Minkowski space $M$ containing a fundamental domain for the lattice $\gotho_K$.
Then every element of $M \times \AA^{\fin}_K$ is congruent modulo $\gotho_K$
to an element of $T \times \AA^{\fin}_K$. By the proposition,
the compact set $T \times \AA^{\fin}_K$ surjects onto $\AA_K/K$, so the
latter is also compact.
\end{proof}

\head{Alternate description: restricted products of topological groups}

Let $G_1, G_2, \dots$ be a sequence of locally compact topological groups
and let $H_i$ be a compact subgroup of $G_i$. The \emph{restricted product} $G$
of the pairs $(G_i, H_i)$ is the set of tuples $(g_i)_{i=1}^\infty$ such
that $g_i \in H_i$ for all but finitely many indices $i$. For each set $S$,
this product contains the subgroup $G_S$ of tuples $(g_i)$ such that $g_i
\in H_i$ for $i \notin S$, and indeed $G$ is the direct limit of the $G_S$.
We make $G$ into a topological group by giving each $G_S$ the product topology
and saying that $U \subset G$ is open if its intersection with each $G_S$
is open there.

In this language, the additive group of ad\`eles over $\QQ$
is simply the restricted
product of the pairs $(\RR, \RR)$ and $(\QQ_p, \ZZ_p)$ for each $p$,
and likewise over a number field.

\head{Id\`eles and the id\`ele class group}

An \emph{id\`ele} is a unit in the ring $\AA_K$. In other words, it is a
tuple $(\alpha_v)$, one element of $K_v^*$ for each place $v$ of $K$, such that
$\alpha_v \in \gotho_{K_v}^*$ for all but finitely many finite places $v$.
Let $I_K$ denote the group of id\`eles of $K$ (sometimes thought
of as $\GL_1(\AA_K)$). This group is the restricted product of the pairs
$(\RR^*, \RR^*)$, $(\CC^*, \CC^*)$, and $(K_{\gothp}^*, \gotho_{\gothp}^*)$.

For example, for each element $\beta \in K$, we get an ad\`ele
in which $\alpha_v = \beta$ for all $v$; this ad\`ele is an id\`ele if $\beta \neq 0$. We
call these the \emph{principal ad\`eles} and \emph{principal id\`eles},
and define the \emph{id\`ele class group} of $K$ as the quotient
$C_K = I_K/K^*$ of the id\`eles by the principal id\`eles.

\head{Warning}  While the embedding of the id\'eles into the ad\'eles is continuous,
the restricted product topology on id\'eles does not coincide with the subspace topology for the embedding!
For
example, the set of id\`eles whose component at each finite prime $\gothp$ is in $\gotho_{\gothp}^*$ 
is open, but not an
intersection of the id\`ele group with an open subset of the ad\`eles. See the exercises for one way to fix this, and the last part of this section for another.

\medskip
There is a homomorphism from $I_K$ to the group of fractional ideals
of $K$:
\[
(\alpha_\nu)_\nu \mapsto \prod_{\gothp} \gothp^{v_{\gothp}(\alpha_\gothp)},
\]
which is continuous for the discrete topology on the group of fractional
ideals.
The principal id\`ele corresponding to $\alpha \in K$ maps to the principal
ideal generated by $\alpha$. Thus we have a surjection $C_K \to \Cl(K)$.

Since the norm is trivial on $K^*$, we get a well-defined norm map
$|\cdot|: C_K \to \RR^*_+$. Let $C_K^0$ be the kernel of the norm map;
then $C_K^0$ also surjects onto $\Cl(K)$. (The surjection onto $\Cl(K)$ ignores
the infinite places, so you can adjust there to force norm 1.)
\begin{prop} \label{P:idele group compact}
The group $C_K^0$ is compact.
\end{prop}
This innocuous-looking fact actually implies two key theorems of algebraic number theory:
\begin{enumerate}
\item[(a)]
The class group of $K$ is finite.
\item[(b)]
The group of units of $K$ has rank $r+s-1$, where $r$ and $s$ are the number
of real and complex places, respectively. More generally, if $S$ is a finite
set of places containing the archimedean places,
the group of $S$-units of $K$ (elements of $K$ which
have valuation 0 at each finite place not contained in $S$) has rank
$\#(S)-1$.
\end{enumerate}
In fact, (a) is immediate: $C_K^0$ is compact and it surjects onto 
$\Cl(K)$, so the latter must also be compact for the discrete topology,
i.e., it must be finite. (In fact, $\Cl(K)$ is isomorphic to the group
of connected components of $C_K^0$.)
To see (b), let $I_S$ be the group of id\`eles which are units outside $S$,
and define the map $\log: I_S \to \RR^{\#(S)}$ by taking log of the absolute
value of the norm of each component in $S$. By the product formula, this maps
into the sum-of-coefficients-zero hyperplane $H$ in $\RR^{\#(S)}$, and the
image of the group $K_S^*$ of $S$-units is discrete therein. (Restricting 
an element of $K_S^*$ to a bounded subset of $H$ bounds all of its
absolute values, so this follows from the discreteness of $K$ in $\AA_K$.)
Let $W$ be the span in $H$ of the image of $K_S^*$; then
we get a continuous homomorphism $C_K^0 \to H/W$ whose image generates
$H/W$. But its image is compact; this is a contradiction unless $H/W$
is the zero vector space. Thus $K_S^*$ must be a lattice in $H$,
so it has rank $\dim H = \#S - 1$.

\begin{proof}[Proof of Proposition~\ref{P:idele group compact}]
The inverse images of any two positive real numbers under the norm map are
homeomorphic. So rather than prove that the inverse image of 0 is compact,
we'll prove that the inverse image of some $\rho > 0$ is compact.
Namely, we choose $\rho>c$, where $c$ has the property that any id\`ele of
norm $\rho > c$ is congruent modulo $K^*$ to an id\`ele whose components
all have norms in $[1, \rho]$. (The existence of $c$ is left as an exercise.)

The set
of id\`eles with each component having norm in $[1, \rho]$ is the product
of ``annuli'' in the archimedean places and finitely many of the
nonarchimedean places, and the group of units in the rest. (Most of the
nonarchimedean places don't have any valuations between 1 and $\rho$.)
This is a compact set, the set of id\`eles therein of norm $\rho$ is a closed
subset and so is also compact, and the latter set surjects onto
$C_K^0$, so that's compact too.
\end{proof}

One more comment worth making: what are the open subgroups of $I_K$?
In fact, for each formal product $\gothm$ of places, one gets an open
subgroup of id\`eles $(\alpha_v)_v$ such that:
\begin{enumerate}
\item[(a)] if $v$ is a real place occurring in $\gothm$, then
$\alpha_v > 0$;
\item[(b)] if $v$ is a finite place corresponding to the prime $\gothp$,
occurring to the power $e$, then $\alpha_v \equiv 1 \pmod{\gothp^e}$.
\end{enumerate}
Moreover, every open subgroup contains one of these. Thus
using the surjection $C_K \mapsto \Cl(K)$, we get a bijection between open
subgroups of $C_K$ and generalized ideal class groups!

\head{A presentation of $\AA_\QQ$}
In the special case $K = \QQ$, the id\`ele class group has a nice presentation.
Namely, given an arbitrary id\`ele in $I_\QQ$, there is a unique positive
rational with the same norms at the finite places. Thus
\[
I_\QQ \cong \RR^* \times \prod_p \ZZ_p^*.
\]
This definitely does not generalize: as noted above, the id\`ele class group
has multiple connected components when the class number is bigger than 1.

\head{Aside: beyond class field theory}
You can think of the id\`ele group as $\GL_1(\AA_K)$. In that case, class field
theory will become a correspondence between one-dimensional representations
of $\Gal(\overline{K}/K)$ and certain representations of $\GL_1(\AA_K)$. This
is the form in which class field theory generalizes to the nonabelian case:
the Langlands program predicts a correspondence between $n$-dimensional
representations of $\Gal(\overline{K}/K)$ and certain representations
of $\GL_n(\AA_K)$. In fact, with $K$ replaced by the function field of a
curve over a finite field, this prediction is a deep theorem of L. Lafforgue
(based on work of Drinfeld).

\head{Exercises}

\begin{enumerate}
\item
Prove Proposition~\ref{P:adelic CRT}.
\item
Show that the restricted direct product topology on $I_K$ is the subspace topology for the embedding into
$\AA_K \times \AA_K$ given by the map $x \mapsto (x,x^{-1})$.
\item
Complete the proof of Proposition~\ref{P:idele group compact} by establishing the existence of the constant $c$. (Hint: see Lang, Section V.1, Theorem 0.)
\end{enumerate}

%\end{document}


\chapter{Ad\`eles and id\`eles in field extensions}
%\documentclass[12pt]{article}
%\usepackage{amsfonts, amsthm, amsmath}
%
%\setlength{\textwidth}{6.5in}
%\setlength{\oddsidemargin}{0in}
%\setlength{\textheight}{8.5in}
%\setlength{\topmargin}{0in}
%\setlength{\headheight}{0in}
%\setlength{\headsep}{0in}
%\setlength{\parskip}{0pt}
%\setlength{\parindent}{20pt}
%
%\def\AA{\mathbb{A}}
%\def\CC{\mathbb{C}}
%\def\FF{\mathbb{F}}
%\def\PP{\mathbb{P}}
%\def\QQ{\mathbb{Q}}
%\def\RR{\mathbb{R}}
%\def\ZZ{\mathbb{Z}}
%\def\gotha{\mathfrak{a}}
%\def\gothb{\mathfrak{b}}
%\def\gothm{\mathfrak{m}}
%\def\gotho{\mathfrak{o}}
%\def\gothp{\mathfrak{p}}
%\def\gothq{\mathfrak{q}}
%\DeclareMathOperator{\disc}{Disc}
%\DeclareMathOperator{\Gal}{Gal}
%\DeclareMathOperator{\GL}{GL}
%\DeclareMathOperator{\Hom}{Hom}
%\DeclareMathOperator{\Norm}{Norm}
%\DeclareMathOperator{\Trace}{Trace}
%\DeclareMathOperator{\Cl}{Cl}
%
%\def\head#1{\medskip \noindent \textbf{#1}.}
%
%\newtheorem{theorem}{Theorem}
%\newtheorem{lemma}[theorem]{Lemma}
%\newtheorem{prop}[theorem]{Proposition}
%
%\begin{document}
%
%\begin{center}
%\bf
%Math 254B, UC Berkeley, Spring 2002 (Kedlaya) \\
%ad\`eles and id\`eles in Field Extensions
%\end{center}

\head{Reference} 
Neukirch, Section VI.1 and VI.2.

\head{Ad\`eles in Field Extensions}

If $L/K$ is an extension of number fields, we get an embedding
$\AA_K \hookrightarrow \AA_L$ as follows: given $\alpha \in \AA_K$,
each place $w$ of $L$ restricts to a place $v$ of $K$, so set the $w$-component
of the image of $\alpha$ to $\alpha_v$.
This embedding induces an inclusion $I_K \hookrightarrow I_L$ of id\`ele groups
as well.

If $L/K$ is Galois with Galois group $G$, then $G$ acts naturally on
$\AA_L$ and $I_L$; more generally, if $g \in \Gal(\overline{K}/K)$, then
$g$ maps $L$ to some other extension $L^g$ of $K$, and $G$ induces
isomorphisms of $\AA_L$ with $\AA_{L^g}$ and
of $I_L$ with $I_{L^g}$. Namely, if $(\alpha_v)_v$ is an
id\`ele over $L$ and $g \in G$, then $g$ maps the completion $L_v$
of $L$ to a completion $L_{v^g}$ of $L^g$. (Remember, a place $v$
corresponds to an absolute value $|\cdot|_v$ on $L$; the absolute
value $|\cdot|_{v^g}$ on $L^g$ is given by $|a^g|_{v^g}| = |a|_v$.)
As you might expect, this action is compatible with the embeddings
of $L$ in $I_L$ and $L^g$ in $I_{L^g}$, so it induces an isomorphism
$C_L \to C_{L^g}$ of id\`ele class groups.

\head{Aside}
Neukirch points out that you can regard $\AA_L$ as the tensor
product $\AA_K \otimes_K L$; in particular, this is a good way to see
the Galois action on $\AA_L$. Details are left to the reader.

\medskip
We can define trace and norm maps as well:
\[
\Trace_{\AA_L/\AA_K}(x) = \sum_g x^g, \qquad
\Norm_{I_L/I_K}(x) = \prod_g x^g
\]
where $g$ runs over coset representatives of $\Gal(\overline{K}/L)$
in $\Gal(\overline{K}/K)$, the sum and product taking places in the ad\`ele
and id\`ele rings of the Galois closure of $L$ over $K$.
In particular, if $L/K$ is Galois, $g$ simply runs over $\Gal(L/K)$.

In terms of components, these definitions translate as
\begin{align*}
  (\Trace_{\AA_L/\AA_K}(\alpha))_{v} &= \sum_{w | v}
\Trace_{L_w/K_v}(\alpha_w) \\
  (\Norm_{I_L/I_K}(\alpha))_{v} &= \prod_{w | v}
\Norm_{L_w/K_v}(\alpha_w).
\end{align*}
The trace and norm do what you expect on principal ad\`eles/id\`eles. In
particular, the norm descends to a map $\Norm_{L/K}: C_L \to C_K$.

\head{Aside}
You can also define the trace of an ad\`ele $\alpha \in \AA_L$ as the trace of
addition by $\alpha$ as an endomorphism of the $\AA_K$-module $\AA_L$,
and the norm of an id\`ele $\alpha \in I_L$ as the determinant of 
multiplication by $\alpha$ as an automorphism of the $\AA_K$-module
$\AA_L$. (Yes, the action is on the \emph{ad\`eles} in both cases.
Remember, id\`eles should be thought of as automorphisms of the ad\`eles,
not as elements of the ad\`ele ring, in order to topologize them correctly.)

If $L/K$ is a Galois extension, then $\Gal(L/K)$ acts on $\AA_L$ and
$I_L$ fixing $\AA_K$ and $I_K$, respectively, and we have the following.
\begin{prop}
If $L/K$ is a Galois extension with Galois group $G$, then
$\AA_L^G = \AA_K$ and $I_L^G = I_K$.
\end{prop}
\begin{proof}
If $v$ is a place of $K$, then for each place $w$ of $K$ above
$v$, the decomposition group $G_w$ of $w$ is isomorphic to
$\Gal(L_w/K_v)$. So if $(\alpha)$ is an ad\`ele or id\`ele which is
$G$-invariant, then $\alpha_w$ is $\Gal(L_w/K_v)$-invariant for each
$w$, so belongs to $K_v$. Moreover, $G$ acts transitively on the
places $w$ above $v$, so $\alpha_w = \alpha_{w'}$ for any two places
$w, w'$ above $v$. Thus $(\alpha)$ is an ad\`ele or id\`ele over $K$.
\end{proof}

This has the following nice consequence for the id\`ele class group,
a fact which is quite definitely not true for the ideal class group:
the map $\Cl_K \to \Cl_L^G$ is neither injective nor surjective in general.
This is our first hint of why the id\`ele class group will be a more
convenient target for a reciprocity map than the ideal class group.
\begin{prop}[Galois descent]
  If $L/K$ is a Galois extension with Galois group $G$, then
$G$ acts on $C_L$, and the $G$-invariant elements are precisely $C_K$.
\end{prop}
\begin{proof}
We start with an exact sequence
\[
1 \to L^* \to I_L \to C_L \to 1
\]
of $G$-modules. Taking $G$-invariants, we get a long exact sequence
\[
1 \to (L^*)^G = K^* \to (I_L)^G = I_K \to C_L^G \to H^1(G, L^*),
\]
and the last term is 1 by Theorem 90 (Lemma~\ref{L:theorem 90}). So 
we again have a short exact sequence, and $C_L^G \cong I_K/K^* = C_K$.
\end{proof}

%\end{document}




\chapter{The adelic reciprocity law and Artin reciprocity}
%\documentclass[12pt]{article}
%\usepackage{amsfonts, amsthm, amsmath}
%
%\setlength{\textwidth}{6.5in}
%\setlength{\oddsidemargin}{0in}
%\setlength{\textheight}{8.5in}
%\setlength{\topmargin}{0in}
%\setlength{\headheight}{0in}
%\setlength{\headsep}{0in}
%\setlength{\parskip}{0pt}
%\setlength{\parindent}{20pt}
%
%\def\AA{\mathbb{A}}
%\def\CC{\mathbb{C}}
%\def\FF{\mathbb{F}}
%\def\NN{\mathbb{N}}
%\def\PP{\mathbb{P}}
%\def\QQ{\mathbb{Q}}
%\def\RR{\mathbb{R}}
%\def\ZZ{\mathbb{Z}}
%\def\gotha{\mathfrak{a}}
%\def\gothb{\mathfrak{b}}
%\def\gothm{\mathfrak{m}}
%\def\gotho{\mathfrak{o}}
%\def\gothp{\mathfrak{p}}
%\def\gothq{\mathfrak{q}}
%\DeclareMathOperator{\ab}{ab}
%\DeclareMathOperator{\disc}{Disc}
%\DeclareMathOperator{\Gal}{Gal}
%\DeclareMathOperator{\GL}{GL}
%\DeclareMathOperator{\Hom}{Hom}
%\DeclareMathOperator{\Norm}{Norm}
%\DeclareMathOperator{\Trace}{Trace}
%\DeclareMathOperator{\Cl}{Cl}
%
%\def\head#1{\medskip \noindent \textbf{#1}.}
%
%\newtheorem{theorem}{Theorem}
%\newtheorem{lemma}[theorem]{Lemma}
%\newtheorem{prop}[theorem]{Proposition}
%
%\begin{document}
%
%\begin{center}
%\bf
%Math 254B, UC Berkeley, Spring 2002 (Kedlaya) \\
%The Adelic Reciprocity Law and Artin Reciprocity
%\end{center}

We now describe the setup by which we create a reciprocity law in
global class field theory, imitating the ``abstract'' setup from local
class field theory but using the id\`ele class group in place of the
multiplicative group of the field. We then work out why the reciprocity
law and existence theorem in the adelic setup imply Artin reciprocity
and the existence theorem (and a bit more) in the classical language.

\head{Convention note} We are going to fix an algebraic closure
$\overline{\QQ}$ of $\QQ$, and regard ``number fields'' as finite
subextensions of $\overline{\QQ}/\QQ$. That is, we are fixing the
embeddings of number fields into $\overline{\QQ}$. We'll use these embeddings
to decide how to embed one number field in another.

\head{The adelic reciprocity law and existence theorem}

Here are the adelic reciprocity law and existence theorem; notice that
they look just like the local case except that the multiplicative group
has been replaced by the id\`ele class group.
\begin{theorem}[Adelic reciprocity law]
There is a canonical map $r_K: C_K \to \Gal(K^{\ab}/K)$ which
induces, for each finite extension $L/K$ of number fields, an
isomorphism $r_{L/K}: C_K/\Norm_{L/K} C_L \to \Gal(L/K)^{\ab}$.
\end{theorem}
\begin{theorem}[Adelic existence theorem] \label{T:adelic existence theorem1}
For every number field $K$ and 
every open subgroup $H$ of $C_K$ of finite index, there exists
a finite (abelian) extension $L$ of $K$ such that
$H = \Norm_{L/K} C_L$.
\end{theorem}


In fact, using local class field theory, we can construct the map that
will end up being $r_K$. For starters, let $L/K$ be a finite abelian
extension and $v$ a place of $K$. Put $G = \Gal(L/K)$, and let $G_v$ be
the decomposition group of $v$, that is, the set of $g \in G$
such that $v^g = v$. Then for any place $w$ above $v$,
$G_v \cong \Gal(L_w/K_v)$, so we can view the local reciprocity map
$K_v^* \to \Gal(L_w/K_v)$ as a map $r_{K,v}: K_v^* \to G$. That is, if $v$
is a finite place. If $v = \CC$, then $\Gal(L_w/K_v)$ is trivial, so
we just take $K_v^* \to G$ to be the identity map. If $v = \RR$, then we
take $K_v^* = \RR^* \to \Gal(L_w/K_v) = \Gal(\CC/\RR)$ to be the map
sending everything positive to the identity, and everything negative
to complex conjugation.

Now note that
\[
(\alpha_v) \mapsto \prod_v r_{K,v}(\alpha_v) 
\]
is well-defined on id\`eles: for $(\alpha_v)$ an id\`ele, $\alpha_v$ is
a unit for almost all $v$ and $L_w/K_v$ is unramified for almost all
$v$. For the (almost all) $v$ in both categories, $r_{K,v}$ maps
$\alpha_v$ to the identity.

The subtle part is the following. As noted below, before proving reciprocity,
we'll only be able to check this for the map obtained from $r_{K,v}$ by
projecting from $\Gal(K^{\ab}/K)$ to the torsion-free quotient of $\Gal(K(\zeta_\infty)/K)$, the Galois
group of the maximal cyclotomic extension; in that case, we can reduce to
$K=\QQ$ and do an explicit computation. The general case will actually only
follow after the fact from the construction of global reciprocity!
\begin{prop}
The map $r_{K,v}$ is trivial on $K^*$.    
\end{prop}
Thus it induces a map $r_K: C_K \to \Gal(L/K)$ for each $L/K$ abelian,
and in fact to $r_K: C_K \to \Gal(K^{\ab}/K)$ using the analogous
compatibility for local reciprocity.

Since each of the local reciprocity maps is continuous, so is the map $r_K$.
That means the kernel of $r_K: C_K \to \Gal(L/K)$, for $L/K$ abelian,
is an open subgroup of $C_K$. Now recall that the
quotient of $C_K$ by any open subgroup of finite index is a generalized
ideal class group. Thus $r_K$ is giving us a canonical isomorphism between
$\Gal(L/K)$ and a generalized ideal class group; could this be anything
but Artin reciprocity itself? 

Indeed, let $U$ be the kernel of $r_K$,  let $\gothm$ be a conductor
for the generalized ideal class group $C_K/U$, and let $\gothp$ be a
prime of $K$ not dividing $\gothm$ and unramified in $L$. Then
the id\`ele $\alpha = (1,1, \dots, \pi, \dots)$ with a uniformizer $\pi$
of $\gotho_{K_\gothp}$ in the $\gothp$ component and ones elsewhere
maps onto the class of $\gothp$ in $C_K/U$. On the other hand,
$r_K(\alpha) = r_{K, \gothp}(\pi)$ is (because $L$ is unramified over $K$)
precisely the Frobenius of $\gothp$. So indeed, $\gothp$ is being mapped
to its Frobenius, so the map $C_K/U \to \Gal(L/K)$ is indeed Artin reciprocity.

In fact, we discover from this a little bit more than we knew already about
the Artin map. All we said before about the Artin map is that it factors
through a generalized ideal class group, and that the conductor $\gothm$
of that group is divisible precisely by the ramified primes (which
we see from local reciprocity). In fact, we can now say \emph{exactly}
what is in the kernel of the classical Artin map: it is generated by
\begin{itemize}
\item all principal ideals congruent to 1 modulo $\gothm$;
\item norms of ideals of $L$ not divisible by any ramified primes.
\end{itemize}


\head{What needs to be done}

Many of these steps will be analogous to the steps in local class field theory.
\begin{itemize}
\item 
It would be natural to start by verifying
that the map $r_K$ given above does indeed kill principal id\`eles,
but this is too hard to do all at once (except for cyclotomic extensions, for which the explicit calculation is easy and an important input into the machine). Instead, we postpone this step all the way until the end; see below.
\item Verify that for $L/K$ cyclic, the Herbrand quotient of
$C_L$ as a $\Gal(L/K)$-module is $[L:K]$. In particular,
that forces $\#H^0(\Gal(L/K), C_L) \geq [L:K]$ (the ``First Inequality'').
\item
For $L/K$ cyclic, determine that
\[
\#H^0(\Gal(L/K), C_L) = [L:K], \qquad \#H^1(\Gal(L/K), C_L) = 1
\]
(the ``Second Inequality''). This step is trivial in local CFT by
Theorem 90, but is actually pretty subtle in the global case. We'll
do it by reducing to the case where $K$ contains enough roots of unity,
so that $L/K$ becomes a Kummer extension and we can compute everything
explicitly. There is also an analytic proof given in Milne which I'll
very briefly allude to.
\item
Check the conditions for abstract class field theory, using the setup described at the end of Chapter~\ref{chap:abstractcft}. In particular, the role of the unramified extensions in local class field theory will be played by certain cyclotomic extensions.
This gives an ``abstract'' reciprocity map, not yet known to be related to Artin reciprocity.
\item
Prove the existence theorem, by showing that every open subgroup
of $C_K$ contains a norm group. Again, we can enlarge $K$ in order
to do this, so we can get into the realm of Kummer theory.
\item
Use the compatibility between the proofs of local and global class field theory to see that the ``abstract'' global reciprocity map restricts to the usual reciprocity map from local class field theory. This will finally imply that the abstract map coincides with the adelic Artin reciprocity map, and therefore yield the adelic reciprocity map. It is only at this point that we will deduce that the reciprocity map $r_K$ that we tried to define at the outset
actually does kill principal id\`eles!

\item
We will also briefly sketch the approach taken in Milne, in which one uses Galois cohomology in place of abstract class field theory.
 Specifically, one first checks that
 $H^2(\Gal(L/K), C_L)$ is cyclic of order $[L:K]$ in certain ``unramified'' (i.e., cyclotomic) cases; as in the local case, one can then deduce this result in general by induction on degree. Using Tate's theorem (Theorem~\ref{T:tate thm1}),
one gets a reciprocity map from $H^{-2}_T(\Gal(L/K), \ZZ) = \Gal(L/K)^{\ab}$
to $H^0_T(\Gal(L/K), C_K/\Norm_{L/K} C_L)$, which again can be reconciled with local reciprocity to get the Artin reciprocity map. 
\end{itemize}


%\end{document}


\part{The main results}

\chapter{Cohomology of the id\`eles I: the ``First Inequality''}
%\documentclass[12pt]{article}
%\usepackage{amsfonts, amsthm, amsmath}
%\usepackage[all]{xy}
%
%\setlength{\textwidth}{6.5in}
%\setlength{\oddsidemargin}{0in}
%\setlength{\textheight}{8.5in}
%\setlength{\topmargin}{0in}
%\setlength{\headheight}{0in}
%\setlength{\headsep}{0in}
%\setlength{\parskip}{0pt}
%\setlength{\parindent}{20pt}
%
%\def\AA{\mathbb{A}}
%\def\CC{\mathbb{C}}
%\def\FF{\mathbb{F}}
%\def\PP{\mathbb{P}}
%\def\QQ{\mathbb{Q}}
%\def\RR{\mathbb{R}}
%\def\ZZ{\mathbb{Z}}
%\def\gotha{\mathfrak{a}}
%\def\gothb{\mathfrak{b}}
%\def\gothm{\mathfrak{m}}
%\def\gotho{\mathfrak{o}}
%\def\gothp{\mathfrak{p}}
%\def\gothq{\mathfrak{q}}
%\DeclareMathOperator{\disc}{Disc}
%\DeclareMathOperator{\Gal}{Gal}
%\DeclareMathOperator{\GL}{GL}
%\DeclareMathOperator{\Hom}{Hom}
%\DeclareMathOperator{\Ind}{Ind}
%\DeclareMathOperator{\Norm}{Norm}
%\DeclareMathOperator{\Trace}{Trace}
%\DeclareMathOperator{\Cl}{Cl}
%
%\def\head#1{\medskip \noindent \textbf{#1}.}
%
%\newtheorem{theorem}{Theorem}
%\newtheorem{lemma}[theorem]{Lemma}
%\newtheorem{prop}[theorem]{Proposition}
%
%\begin{document}
%
%\begin{center}
%\bf
%Math 254B, UC Berkeley, Spring 2002 (Kedlaya) \\
%Cohomology of the id\`eles 1: The ``First Inequality''
%\end{center}

\head{Reference} 
Milne VII.2-VII.4; Neukirch VI.3; but see below about Neukirch.

\medskip

By analogy with local class field theory, we want to prove that for $K,L$
number fields and $C_K, C_L$ their id\`ele class groups,
\[
H^1(\Gal(L/K), C_L) = 1, \qquad H^2(\Gal(L/K), C_L) = \ZZ/[L:K]\ZZ.
\]
In this chapter, we'll look at the special case $L/K$ cyclic, and prove
that
\[
\#H^0_T(\Gal(L/K), C_L)/\#H^1_T(\Gal(L/K), C_L) = [L:K].
\]
That is,
the Herbrand quotient of $C_L$ is $[L:K]$. As we'll see, this
will end up reducing to looking at units in a real vector space, much
as in the proof of Dirichlet's units theorem.

This will imply the ``First Inequality''.
\begin{theorem} \label{T:first inequality}
For $L/K$ a cyclic extension of number fields,
\[
\#H^0_T(\Gal(L/K), C_L) \geq [L:K].
\]
\end{theorem}
The ``Second Inequality'' will be the reverse, which will be a bit
more subtle (see Theorem~\ref{T:first and second inequality}).

\head{Some basic observations}
But first, some general observations. Put $G = \Gal(L/K)$.
\begin{prop}
  For each $i>0$, $H^i(G, I_L) = \oplus_v H^i(G_v, L_v^*)$. For each
$i$, $H^i_T(G, I_L) = \oplus_v H^i_T(G_v, L_v^*)$.
\end{prop}
\begin{proof}
For any finite set $S$ of places of $K$ containing all infinite places
and all ramified primes, let $I_{L,S}$ be the set of id\`eles
with a unit at each component other than at the places dividing 
any places in $S$. Note that $I_{L,S}$ is stable under $G$ (because we
defined it in terms of places of $K$, not $L$). By definition,
$I_L$ is the direct limit of the $I_{L,S}$ over all $S$, so $H^i(G,I_L)$
is the direct limit of the $H^i(G, I_{L,S})$. The latter is the product of
$H^i(G, \prod_{w|v} L_w^*)$ over all $v \in S$ and 
$H^i(G, \prod_{w|v} \gotho_{L_w}^*)$ over all $v \notin S$, but the latter
is trivial because $v \notin S$ cannot ramify. By Shapiro's lemma (Lemma~\ref{L:Shapiro}),
$H^i(G, \prod_{w|v} L_w^*) = H^i(G_v, L_w^*)$, so we have what we want.
The argument for Tate groups is analogous.
\end{proof}
Notice what this says for $i=0$ on the Tate groups: an id\`ele is a norm
if and only if each component is a norm. Obvious, perhaps, but useful.

In particular,
\[
H^1(G, I_L) = 0, \qquad H^2(G, I_L) = \bigoplus_v \frac{1}{[L_w:K_v]} \ZZ/\ZZ.
\]

One other observation: if $S$ contains all 
infinite places and all ramified places, then
\[
\Norm_{L/K} I_{L,S} = \prod_{v \in S} U_v \times \prod_{v \notin S} \gotho_{K_v}^*
\]
where $U_v$ is open in $K_v^*$. The group on the right is open in $I_K$, so
$\Norm_{L/K} I_K$ is open.

By quotienting down to $C_K$, we see that $\Norm_{L/K} C_K$ is open. In fact,
the snake lemma on the diagram
\[
\xymatrix{
0 \ar[r] & L^* \ar[r] \ar^{\Norm_{L/K}}[d] & I_L \ar[r] \ar^{\Norm_{L/K}}[d]
& C_L \ar[r] \ar^{\Norm_{L/K}}[d] & 0 \\
0 \ar[r] & K^* \ar[r] & I_K \ar[r] & C_K \ar[r] & 0
}
\]
implies that the quotient $I_K/(K^* \times \Norm_{L/K} I_L)$ is isomorphic
to $C_K$.

\head{Cohomology of the units}

Remember, we're going to be assuming $G = \Gal(L/K)$ is cyclic until further notice,
so that we may use periodicity of the Tate groups, and the Herbrand quotient.

First of all, working with $I_L$ all at once is a bit unwieldy; we'd rather
work with $I_{L,S}$ for some finite set $S$. In fact, we can choose $S$ to
make our lives easier: we choose $S$ containing all infinite places,
and all ramified primes, and perhaps some extra primes so that
\[
I_L = I_{L,S} L^*.
\]
This is possible because the ideal class group of $L$ is finite, so 
it is generated by some finite set of primes, which we introduce into $S$; then I can
move a generator of any other prime to some stuff in $S$ times units.
(This argument can also be used to prove that $C^0_L$ is compact, but then one doesn't recover the finiteness of the ideal class group as a corollary.)

%The reason I can do this is the same as my ``cheating''
%proof that $C^0_L$ is compact: it's enough to make sure $S$ contains 
%a set of primes that generate the ideal class group of $L$, so then I can
%move a generator of any other prime to some stuff in $S$ times units.
%
Put $L_S = L^* \cap I_{L,S}$; that is, $L_S$ is the group of $S$-units
in $L$. From the exact sequence
\[
1 \to L_S \to I_{L,S} \to I_{L,S}/L_S = C_L \to 1
\]
we have, in case $L/K$ is cyclic, an equality of Herbrand quotients
\[
h(C_L) = h(I_{L,S})/h(L_S).
\]
From the computation of $H^i(G, I_{L,S})$, it's easy to read off its
Herbrand quotient:
\[
h(I_{L,S}) = \prod_{v \in S} \#H^0_T(G_v, L_w^*)= \prod_{v \in S}
[L_w:K_v].
\]
So to get $h(C_L) = [L:K]$, we need
\[
h(L_S) = \frac{1}{[L:K]} \prod_{v \in S} [L_w:K_v].
\]
This will in fact be true even if we only assume $S$ contains all infinite
places, as we now check.

Let $T$ be the set of places of $L$ dividing the places of $S$.
Let $V$ be the real vector space consisting of one copy of $\RR$ for
each place in $T$. Define the map $L_S \to V$ by sending
\[
\alpha \to \prod_w \log |\alpha|_w,
\]
with the caveat that the norm at a complex place is the \emph{square} of
the usual absolute value; the kernel of this map consists solely of roots
of unity (by Kronecker's theorem: any algebraic integer whose conjugates in $\CC$ all have norm 1 is a root of unity). Let $M$ be the quotient of $L_S$ by the group of roots of unity;
since the latter is finite, $h(M) = h(L_S)$.
Let $H \subset V$ be the hyperplane of vectors
with sum of coordinates 0; by the product formula, $M$ maps into $H$.
As noted earlier, in fact $M$ is a discrete subgroup of $H$ of rank equal
to the dimension of $H$; that is, $M$ is a lattice in $H$. Moreover,
we have an action of $G$ on $V$ compatible with the embedding of $M$;
namely, $G$ acts on the places in $T$, so acts on $V$ by permuting the
coordinates.

\head{Caveat} There seems to be an error in Neukirch's derivation at this
point. Namely, his Lemma VI.3.4 is only proved assuming that $G$ acts
transitively on the coordinates of $V$; but in the above situation,
this is not the case: $G$ permutes the places above any given place $v$
of $K$ but those are separate orbits. So we'll follow Milne instead.

We can write down two natural lattices in $V$. One of them is the lattice
generated by $M$ together with the all-ones vector, on which $G$ acts
trivially. As a $G$-module, the Herbrand quotient of that lattice is
$h(M) h(\ZZ) = [L:K] h(M)$. The other is the lattice $M'$ in which, in the
given coordinate system, each element has integral coordinates. To
compute its Herbrand quotient, notice that the projection of this 
lattice onto the coordinates corresponding to the places $w$ above some
$v$ form a copy of $\Ind^G_{G_v} \ZZ$. Thus
\[
h(G, M') = \prod_v h(G, \Ind^G_{G_v} \ZZ) = \prod_v h(G_v, \ZZ)
= \prod_v \#G_v = \prod_v [L_w:K_v].
\]

So all that remains is to prove the following.
\begin{lemma}
Let $V$ be a real vector space on which a finite group $G$ acts linearly,
and let $L_1$ and $L_2$ be $G$-stable lattices in $V$ for which
$h(L_1)$ and $h(L_2)$ are both defined. Then $h(L_1) = h(L_2)$.
\end{lemma}
In fact, one can show that if one of the Herbrand quotients is defined, so
is the other.
\begin{proof}
We first show that $L_1 \otimes_{\ZZ} \QQ$ and $L_2 \otimes_{\ZZ} \QQ$
are isomorphic as $\QQ[G]$-modules. We are given that $L_1 \otimes_{\ZZ} \RR$
and $L_2 \otimes_{\ZZ} \RR$ are isomorphic as $\RR[G]$-modules.
That is, the real vector space $W = \Hom_{\RR}(L_1 \otimes_{\ZZ} \RR,
L_2 \otimes_{\ZZ}
\RR)$, on which $G$ acts by the formula $T^g(x) = T(x^{g^{-1}})^g$, contains an invariant vector which, as a linear
transformation, is invertible. Now $W$ can also be written as
\[
\Hom_{\ZZ}(L_1, L_2) \otimes_{\ZZ} \RR;
\]
that is, $\Hom_{\ZZ}(L_1, L_2)$ sits inside as a sublattice. The fact that
$W$ has an invariant vector says that a certain set of linear equations
has a nonzero solution over $\RR$, namely the equations that express
the fact that the action of $G$ leaves the vector invariant. But those
equations have coefficients in $\QQ$, so there must already be invariant
vectors over $\QQ$. Moreover, if we fix an isomorphism (not $G$-equivariant)
between $L_2 \otimes_{\ZZ} \RR$ and $L_1 \otimes_{\ZZ} \RR$, we can
compose this with any element of $W$ to get a map from $L_1$ to itself,
which has a determinant; and by hypothesis, there is some invariant
vector of $W$ whose determinant is nonzero. Thus the determinant
doesn't vanish identically on the set of invariant vectors in $W$,
so it also doesn't vanish identically on the set of invariant vectors in
$\Hom_{\ZZ}(L_1, L_2) \otimes_{\ZZ} \QQ$.

Thus there is a $G$-equivariant isomorphism between $L_1 \otimes_{\ZZ} \QQ$
and $L_2 \otimes_{\ZZ} \QQ$; that is, $L_1$ is isomorphic to a sublattice of
$L_2$. Since a lattice has the same Herbrand quotient as any sublattice
(the quotient is finite, so its Herbrand quotient is 1), that means
$h(L_1) = h(L_2)$.
\end{proof}

\head{Aside: splitting of primes}

As a consequence of the First Inequality, we record the following fact which is \emph{a posteriori} an immediate consequence of the adelic reciprocity law, but which will be needed in the course of the proofs.
(See Neukirch, Corollary VI.3.8 for more details).
\begin{cor} \label{C:split completely}
For any nontrivial extension $L/K$ of number fields, there are infinitely many primes of $K$ which do not split completely in $L$.
\end{cor}
\begin{proof}
Suppose first that $L/K$ is of prime order. Then if all but finitely many
primes split completely, we can put the remaining primes into $S$ and
deduce that $C_K = \Norm_{L/K} C_L$, whereas the above calculation
forces $H^0_T(\Gal(L/K), C_L) \geq [L:K]$, contradiction.

In the general case, let $M$ be the Galois closure of $L/K$; then a prime of $K$ splits completely in $L$ if and only if it splits completely in $M$. Since $\Gal(M/K)$ is a nontrivial finite group, it contains a cyclic subgroup of prime order; let $N$ be the fixed field of this subgroup. By the previous paragraph, there are infinitely many prime ideals of $N$ which do not split completely in $M$, proving the original result.
\end{proof}

%\end{document}


\chapter{Cohomology of the id\`eles II: the ``Second Inequality''}
%\documentclass[12pt]{article}
%\usepackage{amsfonts, amsthm, amsmath}
%\usepackage[all]{xy}
%
%\setlength{\textwidth}{6.5in}
%\setlength{\oddsidemargin}{0in}
%\setlength{\textheight}{8.5in}
%\setlength{\topmargin}{0in}
%\setlength{\headheight}{0in}
%\setlength{\headsep}{0in}
%\setlength{\parskip}{0pt}
%\setlength{\parindent}{20pt}
%
%\def\AA{\mathbb{A}}
%\def\CC{\mathbb{C}}
%\def\FF{\mathbb{F}}
%\def\PP{\mathbb{P}}
%\def\QQ{\mathbb{Q}}
%\def\RR{\mathbb{R}}
%\def\ZZ{\mathbb{Z}}
%\def\gotha{\mathfrak{a}}
%\def\gothb{\mathfrak{b}}
%\def\gothm{\mathfrak{m}}
%\def\gotho{\mathfrak{o}}
%\def\gothp{\mathfrak{p}}
%\def\gothq{\mathfrak{q}}
%\DeclareMathOperator{\Cor}{Cor}
%\DeclareMathOperator{\disc}{Disc}
%\DeclareMathOperator{\Gal}{Gal}
%\DeclareMathOperator{\GL}{GL}
%\DeclareMathOperator{\Hom}{Hom}
%\DeclareMathOperator{\Ind}{Ind}
%\DeclareMathOperator{\Norm}{Norm}
%\DeclareMathOperator{\Res}{Res}
%\DeclareMathOperator{\Trace}{Trace}
%\DeclareMathOperator{\Cl}{Cl}
%
%\def\head#1{\medskip \noindent \textbf{#1}.}
%
%\newtheorem{theorem}{Theorem}
%\newtheorem{lemma}[theorem]{Lemma}
%\newtheorem{prop}[theorem]{Proposition}
%
%\begin{document}
%
%\begin{center}
%\bf
%Math 254B, UC Berkeley, Spring 2002 (Kedlaya) \\
%Cohomology of the id\`eles 2: The ``Second Inequality''
%\end{center}

\head{Reference} 
Milne VII.5; Neukirch VI.4.

\medskip
In the previous chapter, we proved that for $L/K$ a cyclic extension of number fields,
the Herbrand quotient $h(C_L)$ of the id\`ele class group of $L$ is equal
to $[L:K]$. This time we'll prove the following.
\begin{theorem} \label{T:first and second inequality}
Let $L/K$ be a Galois extension of number fields, with Galois group $G$.
Then:
\begin{enumerate}
\item[(a)] the group $I_K/(K^* \Norm_{L/K} I_L)$ is finite of order at most
$[L:K]$;
\item[(b)] the group $H^1(G, C_L)$ is trivial;
\item[(c)] the group $H^2(G, C_L)$ is finite of order at most $[L:K]$.
\end{enumerate}
\end{theorem}
By the first inequality, for $L/K$ cyclic, these three are equivalent and
all imply that $H^2(G, C_L)$ has order exactly $[L:K]$. That would suffice
to prove the class field axiom in Neukirch's abstract class field theory.

There are two basic ways to prove this result: an analytic proof and
an algebraic proof. Although the analytic proof is somewhat afield of
what we have been doing (it requires some properties of zeta functions
that we haven't discussed previously), it's somewhat simpler overall than the
algebraic proof. So we'll sketch it first before proceeding to the algebraic
proof.

\head{The analytic proof}

For the analytic proof, we need to recast the Second Inequality back into
classical, ideal-theoretic language. Let $L/K$ be a finite Galois extension
and $\gothm$ a formal product of places of $K$. Back when we defined
generalized ideal class groups, we defined the group $I_{\gothm}$ of
fractional ideals of $K$ coprime to $\gothm$ and $P_{\gothm}$ the group
of principal ideals admitting a generator $\alpha$ such that 
$\alpha \equiv 1 \pmod{\gothp^e}$ if the prime power $\gothp^e$ occurs
in $\gothm$ for a finite prime $\gothp$, and $\tau(\alpha) > 0$ if 
$\tau$ is the real embedding corresponding to a real place in $\gothm$.
Also, let $J_{\gothm}$ be the group of fractional ideals of $L$
coprime to $\gothm$. Then the Second Inequality states that
\[
\# I_{\gothm} / P_{\gothm} \Norm_{L/K} J_{\gothm} \leq [L:K].
\]
Note that we don't have to assume $\gothm$ is divisible by the ramified
primes of $L/K$.

We'll need the following special case of the Chebotarev density theorem,
which fortunately we can prove without already having all of class field
theory.
\begin{prop}
Let $L$ be a finite extension of $K$ and let $M/K$ be its Galois closure.
Then the set $S$ of prime ideals of $K$ that split completely in $L$ has
Dirichlet density $1/[M:K]$.
\end{prop}
\begin{proof}
A prime of $K$ splits completely in $L$ if and only if it splits completely
in $M$, so we may assume $L=M$ is Galois. Recall that the set $T$ of unramfied
primes
$\gothq$ of $L$ of absolute degree 1 has Dirichlet density 1;
each such prime lies over an unramified prime $\gothp$ of $K$ of absolute
degree 1 
which splits completely in $L$.

Now recall how the Dirichlet density works: the set $T$ having
Dirichlet density 1 means that
\[
\sum_{\gothq \in T} \frac{1}{\Norm(\gothq)^s} \sim \frac{1}{s-1}
\qquad s \searrow  1
\]
($s$ approaching 1 from above, that is). If we group the primes in $T$ by
which prime of $S$ they lie over, then we get
\[
[L:K] \sum_{\gothp \in T} \frac{1}{\Norm(\gothp)^s} \sim \frac{1}{s-1}.
\]
That is, the Dirichlet density of $S$ is $1/[L:K]$.
\end{proof}

Now for the inequality. For $\chi: I_{\gothm}/P_{\gothm} \to \CC^*$ a
character, we defined the L-function
\[
L(s, \chi) = \prod_{\gothp \not| \gothm} \frac{1}{1 - \chi(\gothp) 
\Norm(\gothp)^{-s}}.
\]
We'll use some basic properties of this function which can be found in any standard algebraic number theory text.
For starters,
\[
\log L(s, 1) \sim \log \zeta_K(s) \sim \log \frac{1}{s-1} \qquad
s \searrow 1,
\]
while if $\chi$ is not the trivial
character, $L(s, \chi)$ is holomorphic at $s=1$. If $L(s, \chi) =
(s-1)^{m(\chi)} g(s)$ where $g$ is holomorphic and nonvanishing at
$s=1$, then $m(\chi) \geq 0$, and
\[
\log L(s, \chi) \sim m(\chi) \log(s-1) = - m(\chi) \log \frac{1}{s-1}.
\]
Let $H$ be a subgroup of $I_{\gothm}$ containing $P_{\gothm}$.
By finite Fourier analysis, or orthogonality of
characters,
\[
\sum_{\chi: I_{\gothm}/H \to \CC^*} \log L(s, \chi) \sim
\#(I_{\gothm}/H) \sum_{\gothp \in H} \frac{1}{\Norm(\gothp)^{-s}}.
\]
We conclude that the set of primes in $H$ has Dirichlet density
\[
\frac{1 - \sum_{\chi \neq 1} m(\chi)}{\#(I_{\gothm}/H)};
\]
this is $1/\#(I_{\gothm}/H)$ if the $m(\chi)$ are all zero, and 0 otherwise.

We apply this with $H = P_{\gothm} \Norm_{L/K} J_{\gothm}$. This 
in particular includes
every prime of $K$ that splits completely, since such a prime is the norm
of any prime of $L$ lying over it. Thus the set of primes in $H$ has
Dirichlet density, on one hand, is at least $1/[L:K]$. On the other hand,
this set has density either zero or $1/\#(I_{\gothm}/H)$. We conclude
$\#I_{\gothm}/H \leq [L:K]$, as desired.

\head{The algebraic proof}

We now proceed to the algebraic proof of the Second Inequality.
To prove the theorem in general, we can very quickly reduce to the
case of $L/K$ not just cyclic, but cyclic of prime order. The reductions
are similar to those we used to compute $H^2(L^*)$ in the local case.
First of all, if we have the theorem for all solvable groups, then if
$G$ is general and $H$ is a Sylow $p$-subgroup of $G$, then for any 
$G$-module $M$,
\[
\Res: H^i_T(G, M) \to H^i_T(H, M)
\]
is injective on $p$-primary components (because $\Cor \circ \Res$
is multiplication by $[G:H]$), so we can deduce the desired result.
Thus it suffices to consider $L/K$ solvable. In that case, starting
from the cyclic-of-prime-order case we can induct using the 
inflation-restriction exact sequence (Corollary~\ref{C:inflation restriction h2}): if $K'/K$ is a subextension
and $H = \Gal(L/K')$, then for $i=1,2$, 
\[
0 \to H^i(G/H, C_{K'}) \to H^i(G, C_L) \to H^i(H, C_L)
\]
(using the fact that $H^1(H, C_L) = 0$ by the induction hypothesis).
Upshot: we need only consider $L/K$ cyclic of prime order $p$.

One more reduction to make things simpler: we reduce to the case where
$K$ contains a $p$-th root of unity. Let $K' = K(\zeta_p)$ and $L' =
L(\zeta_p)$; then $K'$ and $L$ are linearly disjoint over $K$ (since
their degrees are coprime), so $[L':K'] = [L:K] = p$ and the Galois groups
of $L/K$ and $L'/K'$ are canonically isomorphic. To complete the reduction, it suffices to check
that the homomorphism
\[
H^0_T(\Gal(L/K), C_L) \to H^0_T(\Gal(L'/K'), C_{L'})
\]
induced by the inclusion $C_L \to C_{L'}$ is injective. These groups
are both killed by multiplication by $p$, since for $x \in C_K$,
$\Norm_{L/K} (x) = x^p$. Thus multiplication by $d = [K':K]$, which divides
$p-1$, is an isomorphism on these groups. If $x \in C_K$ maps to the
identity in $H^0_T(\Gal(L'/K'), C_{L'})$, we can choose a representative
of the same class as $x$ in $H^0_T(\Gal(L/K), C_L)$ of the form $y^d$;
then $y$ also maps to the identity in $H^0_T(\Gal(L'/K'), C_{L'})$.
That is, $y = \Norm_{L'/K'}(z')$ for some $z' \in C_{L'}$, and
\[
y^d = \Norm_{K'/K}(y) = \Norm_{L'/K}(z') \in \Norm_{L/K} C_L.
\]
Thus $x \in \Norm_{L/K} C_L$, so the homomorphism is injective.

\head{The key case}

To sum up: it suffices to prove the theorem for $K$ containing a $p$-th
root of unity $\zeta_p$ and $L/K$ cyclic of order $p$. We now address
this case.

As in the proof of the First Inequality, we will use a set $S$ of places
of $K$ containing the infinite places, the primes that ramify in $L$,
and enough additional primes so that $I_K = I_{K,S} K^*$; we also include all places above $(p)$. Again, we
put $K_S = I_{K,S} \cap K^*$. Also we write $s = \# S$.

The plan is to explicitly produce a subgroup of $C_K$ of index $[L:K]$
consisting of norms from $C_L$. We do this by using an auxiliary set of 
places $T$ disjoint from $S$. For such $T$, we define
\[
J = \prod_{v \in S} (K_v^*)^p \times \prod_{v \in T} K_v^*
\times \prod_{v \notin S \cup T} \gotho_{K_v}^*.
\]
Let $\Delta = (L^*)^p \cap K_S$. We will show that:
\begin{enumerate}
\item[(a)] $L = K(\Delta^{1/p})$;
\item[(b)] we can choose a set $T$ of $s-1$ primes such that $\Delta$
is the kernel of the map $K_S \to \prod_{v \in T} K_v^*/(K_v^*)^p$;
\item[(c)] for such a set $T$, if we put $C_{K,S,T} = J K^*/K^*$,
then
\[
\#C_K/C_{K,S,T} = [L:K] = p;
\]
\item[(d)]
with the same notation, $C_{K,S,T} \subseteq \Norm_{L/K} C_L$.
\end{enumerate}
That will imply $\#C_K/\Norm_{L/K} C_L \leq p$, as desired.

We first concentrate on (a).
By Kummer theory, since $K$ contains a
primitive $p$-th root of unity, we can write $L = K(D^{1/p})$ for
$D = (L^*)^p \cap K^*$. Thus $K(\Delta^{1/p}) \subseteq L$ and
since there is no room between $K$ and $L$ for an intermediate extension ($[L:K]$
being prime),
all that we have to check is that 
$K(\Delta^{1/p}) \neq K$.
Choose a single $x \in D$ such that $L = K(x^{1/p})$.
For each $v \notin S$, the extension $K_v(x^{1/p})/K_v$ is unramified,
so we can write $x$ as a unit times a $p$-th power, say
$x = u_v y_v^p$. If we put $y_v = 1$ for $v \in S$, we can assemble the
$y_v$ into an id\`ele $y$, which by $I_K = K^* I_{K,S}$ we can rewrite as
$zw$ for $z \in K^*$ and $w \in I_{K,S}$. Now for $v \notin S$,
$(x/z^p)_v = u_v/w_v^p \in \gotho_{K_v}^*$. Thus $x/z^p \in (L^*)^p \cap
K_S \in \Delta$ but $x \notin (K^*)^p$. We conclude $L
= K(\Delta^{1/p})$.

Now we move to (b). Put $N = K(K_S^{1/p})$. By Kummer theory,
\[
\Gal(N/K) \cong \Hom(K_S/K_S^p, \ZZ/p\ZZ).
\]
By the generalization of Dirichlet's units theorem to $S$-units,
$K_S$ modulo torsion is a free abelian group of rank $s-1$,
and the torsion subgroup consists of roots of unity, so is cyclic
of order divisible by $p$. Thus $K_S/K_S^p \cong (\ZZ/p\ZZ)^s$.
Choose generators $g_1, \dots, g_{s-1}$ of 
$\Gal(N/L)$; these correspond in $\Hom(K_S/K_S^p, \ZZ/p\ZZ)$
to a set of homomorphisms whose common kernel is precisely
$\Delta/K_S^p$. 

So to establish (b), we need to find for each $g_i$
a place $v_i$ such that the kernel of $g_i$ is the same as the kernel of
$K_S \to K_{v_i}^*/(K_{v_i}^*)^p$. Let $N_i$ be the fixed field of $g_i$; 
by the First Inequality (see Corollary~\ref{C:split completely}), there are infinitely
many primes of $N_i$ that do not split in $N$. So we can choose
a place $w_i$ of each $N_i$ such that their restrictions $v_i$ to $K$ are
distinct, not contained in $S$, and don't divide $p$. 

We claim $N_i$ is
the maximal subextension of $N/K$ in which $v_i$ splits completely (a/k/a
the \emph{decomposition field} of $v_i$). On one hand, $v_i$ does not split
completely in $N$, so the decomposition field is no larger than $N_i$.
On the other hand, the decomposition field is the fixed field of
the decomposition group, which has exponent $p$ and is cyclic (since
$v_i$ does not ramify in $N$). Thus it must have index $p$ in $N$,
so must be $N_i$ itself.

Thus $L = \cap N_i$ is the maximal subextension of $N$ in which all of
the $v_i$ split completely. We conclude that for $x \in K_S$,
$x$ belongs to $\Delta$ iff $K_{v_i}(x^{1/p}) = K_{v_i}$ for all $i$,
which occurs iff $x \in K_{v_i}^p$. That is, $\Delta$ is precisely the
kernel of the map $K_S \to \prod_i K_{v_i}^*/(K_{v_i}^*)^p$.
In fact, under this map, $K_S$ actually maps to the units in
$K_{v_i}^*$ for each $i$. This proves (b).

Next, we verify (c), using the following lemma.
\begin{lemma} \label{L:adelic intersection}
$J \cap K^* = (K_{S \cup T})^p$.
\end{lemma}
\begin{proof}
Clearly $K_{S \cup T}^p \subseteq J \cap K^*$; we have to work
to show the other inclusion. Take $y \in J \cap K^*$ and
$M = K(y^{1/p})$. We'll show that $\Norm_{M/K} C_M = C_K$; by the First
Inequality, this will imply $M = K$, so $y \in (K^*)^p \cap J
= (K_{S \cup T})^p$.

Since $I_K = I_{K,S}K^*$, it is enough to choose $\alpha \in I_{K,S}$
and show that $\alpha/x \in \Norm_{M/K} I_M$ for some $x \in K^*$.
As noted above, the map
\[
K_S \to \prod_{v \in T} \gotho_{K_v}^*/(\gotho_{K_v}^*)^p
\]
is surjective, and $\#K_S/\Delta = p^{s-1}$. That's also the order of
the product, so the map is actually an isomorphism. Thus we can find
$x \in K_S$ so that $\alpha/x$ has component the $p$-th power of a unit
of $K_v$
at each $v \in T$. In particular, such a component is the norm of its
$p$-th root, so $\alpha/x$ is a norm at each $v \in T$. For $v \in S$,
we don't have anything to check: because $y$ is a $p$-th power at $v$,
$M_w = K_v$. Finally, for $v \notin S \cup T$, $M_w/K_v$ is unramified,
so any unit is a norm. Thus $\alpha/x$ is indeed a norm. We conclude
$\Norm_{M/K} C_M = C_K$, so $M=K$ and $y \in K_{S\cup T}^p$,
as desired.
\end{proof}

Given the lemma, we now have an exact sequence
\[
1 \to (I_{K,S\cup T} \cap K^*)/(J \cap K^*) 
\to I_{K,S\cup T}/J \to
I_{K,S \cup T}K^* / JK^* \to 1.
\]
We can rewrite $I_{K,S \cup T}K^*$ as simply $I_K$, so the group on the
right is precisely $C_K/C_{K,S,T}$. By the lemma, the group on the left is
$K_{S\cup T}^*/(K_{S \cup T}^*)^n$, which has order $p^{2s-1}$ because
$K_{S\cup T}$ is free of rank $2s-2$ plus a cyclic group of order a multiple
of $p$. The group in the middle is the product of $K_v^*/(K_v^*)^p$
over all $v \in S$, and each of those has order $p^2$ (generated by
$\zeta_p$ and a uniformizer of $K_v$). Adding it all up,
we get $\#C_K/C_{K,S,T} = p$, proving (c).

Finally, to check (d), it suffices to check that $J
\subseteq \Norm_{L/K} I_L$, which we may check component by component.
It's automatic for the places $v \notin S \cup T$, since those places
are unramified, so every unit is a norm. For places $v \in S$, any element
of $(K_v^*)^p$ is a norm from $K_v(K_v^{1/p})$ by local reciprocity, so also from $L_w$.
Finally, for places $v \in T$, from the construction of $T$, we see
that $\Delta \subseteq (K_v^*)^p$, so $L_w = K_v$, and so
$K_v^*$ consists entirely of norms.

\head{Aside} We get from this calculation that $H^{-1}_T(G, C_L) = 1$, so
$H^0_T(G, L^*) \to H^0_T(G, C_L)$ is injective. That is,
\[
K^*/\Norm_{L/K} L^* \to \bigoplus_v K_v^*/\Norm_{L_w/K_v} L_w^*
\]
is injective. In other words, we have an interesting ``local-to-global''
statement, namely Hasse's Norm Theorem: if $L/K$
is cyclic, $x \in K^*$ is a norm if and only if it is locally a norm.

%\end{document}


\chapter{An ``abstract'' reciprocity map}
%\documentclass[12pt]{article}
%\usepackage{amsfonts, amsthm, amsmath}
%\usepackage[all]{xy}
%
%\setlength{\textwidth}{6.5in}
%\setlength{\oddsidemargin}{0in}
%\setlength{\textheight}{8.5in}
%\setlength{\topmargin}{0in}
%\setlength{\headheight}{0in}
%\setlength{\headsep}{0in}
%\setlength{\parskip}{0pt}
%\setlength{\parindent}{20pt}
%
%\def\AA{\mathbb{A}}
%\def\CC{\mathbb{C}}
%\def\FF{\mathbb{F}}
%\def\NN{\mathbb{N}}
%\def\PP{\mathbb{P}}
%\def\QQ{\mathbb{Q}}
%\def\RR{\mathbb{R}}
%\def\ZZ{\mathbb{Z}}
%\def\gotha{\mathfrak{a}}
%\def\gothb{\mathfrak{b}}
%\def\gothm{\mathfrak{m}}
%\def\gotho{\mathfrak{o}}
%\def\gothp{\mathfrak{p}}
%\def\gothq{\mathfrak{q}}
%\DeclareMathOperator{\ab}{ab}
%\DeclareMathOperator{\Cor}{Cor}
%\DeclareMathOperator{\cyc}{cyc}
%\DeclareMathOperator{\disc}{Disc}
%\DeclareMathOperator{\Gal}{Gal}
%\DeclareMathOperator{\GL}{GL}
%\DeclareMathOperator{\Hom}{Hom}
%\DeclareMathOperator{\Ind}{Ind}
%\DeclareMathOperator{\Norm}{Norm}
%\DeclareMathOperator{\Res}{Res}
%\DeclareMathOperator{\smcy}{smcy}
%\DeclareMathOperator{\Trace}{Trace}
%\DeclareMathOperator{\Cl}{Cl}
%
%\def\head#1{\medskip \noindent \textbf{#1}.}
%
%\newtheorem{theorem}{Theorem}
%\newtheorem{lemma}[theorem]{Lemma}
%\newtheorem{prop}[theorem]{Proposition}
%
%\begin{document}
%
%\begin{center}
%\bf
%Math 254B, UC Berkeley, Spring 2002 (Kedlaya) \\
%An ``Abstract'' Reciprocity Map
%\end{center}

\head{Reference} 
Milne VII.5; Neukirch VI.4, but only loosely.

\medskip

In this chapter, we'll manufacture a canonical isomorphism
$\Gal(L/K)^{\ab} \to C_K/\Norm_{L/K} C_L$ for any finite extension
$L/K$ of number fields, where $C_K$ and $C_L$ are the corresponding
id\`ele class groups. However, we won't yet know it agrees with our proposed
reciprocity map, which is the product of the local reciprocity maps.
We'll check that in the next chapter.

\head{Cyclotomic extensions}

The cyclotomic extensions (extensions by roots of unity)
of a number field play a role in class field
theory analogous to the role played by the unramified extensions in local
class field theory. This makes it essential to make an explicit study
of them for use in proving the main results.

First of all, we should further articulate a distinction that has come up already.
The extension $\cup_n \QQ(\zeta_n)$ of $\QQ$ obtained by adjoining all roots
of unity has Galois group $\widehat{\ZZ}^* =
\prod_p \ZZ_p^*$. That group has a lot
of torsion, since each $\ZZ_p^*$ contains a torsion subgroup of order $p-1$
(or 2, if $p=2$). If we take the fixed field for the torsion subgroup of
$\ZZ^*$, we get a slightly smaller extension, which I'll call the
\emph{small cyclotomic} extension of $\QQ$ and denote $\QQ^{\smcy}$.
Its Galois group is
$\prod_p \ZZ_p = \widehat{\ZZ}$. For $K$ a number field, define
$K^{\smcy} = K \QQ^{\smcy}$; then $\Gal(K^{\smcy}/K) \cong \widehat{\ZZ}$
as well, even if $K$ contains some extra roots of unity.

\head{The reciprocity map via abstract CFT}

First of all, we choose an isomorphism of $\Gal(\QQ^{\smcy}/\QQ)$
with $\widehat{\ZZ}$; our results are not going to depend on the choice.
That gives a continuous surjection
\[
d: \Gal(\overline{\QQ}/\QQ) \to \Gal(\QQ^{\smcy}/\QQ) \cong \widehat{\ZZ};
\]
if we regard $\QQ^{\smcy}/\QQ$ as the ``maximal unramified extension''
of $\QQ$, we can define the ramification index $e_{L/K}$ and inertia
degree $f_{L/K}$ for any extension of number fields, by the rules
\[
f_{L/K} = [L \cap \QQ^{\smcy}:K \cap \QQ^{\smcy}], \qquad
e_{L/K} = \frac{[L:K]}{f_{L/K}}.
\]
To use abstract class field theory to exhibit the reciprocity map, we
need a ``henselian valuation'' $v: C_{\QQ} \to \widehat{\ZZ}$,
i.e., a homomorphism satisfying:
\begin{enumerate}
\item[(i)]
$v(C_{\QQ})$ is a subgroup $Z$ of $\widehat{\ZZ}$ containing $\ZZ$
with $Z/nZ \cong \ZZ/n\ZZ$ for all positive integers $n$;
\item[(ii)]
$v(\Norm_{K/\QQ} C_K) = f_{K/\QQ} Z$ for all finite extensions $K/\QQ$.
\end{enumerate}
Once we have that, our calculations from the preceding chapters
(Theorem~\ref{T:first inequality}, Theorem~\ref{T:first and second inequality}) imply
that the class field axiom is satisfied: for $L/K$ cyclic,
\[
\#H^0_T(\Gal(L/K), C_L) = [L:K], \qquad
\#H^1_T(\Gal(L/K), C_L) = 1.
\]
So then abstract class field theory will kick in.

We can make that valuation using Artin reciprocity for $\QQ(\zeta_n)/\QQ$.
Recall that there is a canonical surjection
\[
I_n \to (\ZZ/n\ZZ)^* \cong \Gal(\QQ(\zeta_n)/\QQ):
\]
for $p$ not dividing $n$, the ideal $(p)$ goes to $p \in (\ZZ/n\ZZ)^*$
and then to the automorphism $\zeta_n \mapsto \zeta_n^p$, which indeed
does act as the $p$-th power map modulo any prime of $\QQ(\zeta_n)$
above $p$.

That induces a homomorphism $I_{\QQ} \to (\ZZ/n\ZZ)^*$ as follows:
given an id\`ele $\alpha$, pick $x \in \QQ^*$ so that
$\alpha_{\RR}/x > 0$ and, for each prime
$p$ with $p^e | n$, $\alpha/x$ has $p$-component congruent to 1
modulo $p^e$. Then map $\alpha/x$ to $(\ZZ/n\ZZ)^*$ as follows:
\[
\alpha/x \mapsto \prod_{\ell \not| n} \ell^{v_{\ell}(\alpha_{\ell}/x)}.
\]
This gives a well-defined map:
if $y$ is an alternate choice for $x$, then $x/y \equiv 1 \pmod{n}$
and $x/y > 0$, so the product on the right side is precisely $x/y$ itself,
and so is congruent to 1 in $(\ZZ/n\ZZ)^*$.

We now have maps $I_{\QQ} \to (\ZZ/n\ZZ)^*$ which are easily seen to
be compatible, so by taking inverse limits we get $I_{\QQ} \to
\widehat{\ZZ}^* \cong \Gal(\QQ^{\cyc}/\QQ)$. We define $v$ by channeling this
map through the projection $\Gal(\QQ^{\cyc}/\QQ) \to \Gal(\QQ^{\smcy}/\QQ)$
and then using our chosen isomorphism $\Gal(\QQ^{\smcy}/\QQ) \cong
\widehat{\ZZ}$.

Another way to say this:
$I_{\QQ}$ can be written as $\QQ^* \times \RR^* \times
\widehat{\ZZ}^*$, and the map to $\widehat{\ZZ}^*$ is just projection
onto the third factor! In particular, the map factors through
$C_\QQ$, and property (i) above is straightforward.

To check (ii), we need to do the same thing that we just did a bit more
generally. For $K$ now a number field, define the map
\[
I_n \to (\ZZ/n\ZZ)^* \supseteq \Gal(K(\zeta_n)/K),
\]
where now $I_n$ is the group of fractional ideals of $K$ coprime to $(n)$,
by sending a prime $\gothp$ first to its absolute norm. We then have to check
that the result is always in the image of $\Gal(K(\zeta_n)/K)$, but in fact
it must be: whatever the Frobenius of $\gothp$ is, it sends $\zeta_n$
to a power of $\zeta_n$ congruent to $\zeta_n^{\Norm(\gothp)}$
modulo $\gothp$. Since $\gothp$ is prime to $n$, it's prime to the difference
between any two powers of $\zeta_n$, so the Frobenius of $\gothp$ must
in fact send $\zeta_n$ to $\zeta_n^{\Norm(\gothp)}$. This tells us first
that the map
above sends $I_n$ to $\Gal(K(\zeta_n)/K)$ and second that it coincides
with the Artin map.

From the First Inequality, we can deduce the following handy fact.
\begin{prop}
For $L/K$ a finite abelian extension of number fields, the Artin map
always surjects onto $\Gal(L/K)$.
\end{prop}
\begin{proof}
If the Artin map only hit the subgroup $H$ of $\Gal(L/K)$, the fixed
field $M$ of $H$ would have the property that all but finitely many primes
of $M$ split completely in $L$. We've already seen that this contradicts
the First Inequality (Corollary~\ref{C:split completely}).
\end{proof}
In particular, the Artin map $I_n \to \Gal(K(\zeta_n)/K)$ we wrote down
above is surjective.
Using that, we can verify (ii): given a prime ideal $\gothp$ of
$K$, the Artin map of $K(\zeta_n)/K$ applied to it gives the same
element of $(\ZZ/n\ZZ)^*$ as the Artin map of $\QQ$ applied to
$\Norm_{K/\QQ}(\gothp)$. Meanwhile, the Artin map of $K(\zeta_n)/K$
surjects onto $\Gal(K(\zeta_n)/K)$, which has index
$[K \cap \QQ(\zeta_n):\QQ]$ in $(\ZZ/n\ZZ)^*$. This verifies (ii).

Thus lo and behold, we get from abstract class field theory a
reciprocity isomorphism for any finite extension of number fields:
\[
r'_{L/K}: C_K/\Norm_{L/K} C_L \stackrel{\sim}{\to} \Gal(L/K)^{\ab}.
\]
These are compatible in the usual way, so we get a map
$r'_K: C_K \to \Gal(K^{\ab}/K)$. Of course, we don't know what this map is,
so we can't yet use it to recover Artin reciprocity. (That depended on
the reciprocity map being the product of the local maps.) But at least
we deduce the norm limitation theorem.
\begin{theorem} \label{T:adelic norm limitation}
If $L/K$ is a finite extension of number fields and
$M = L \cap K^{\ab}$, then $\Norm_{L/K} C_L = \Norm_{M/K} C_M$.
\end{theorem}

We do know one thing about the map $r'_{L/K}$: for ``unramified'' extensions
$L/K$ (i.e., $L \subseteq K^{\smcy}$),
the ``Frobenius'' in $\Gal(L/K)$ maps to a ``uniformizer'' in
$C_K$. That is, the element of $\Gal(L/K)$ coming from the element
of $\Gal(K^{\smcy}/K)$ which maps to 1 under $d_K$ is the element of
$C_K$ which maps to 1 under $v_K$. But we made $v_K$ simply by
mapping $C_K$ to $\Gal(K^{\smcy}/K)$ via the Artin map and then 
identifying the latter with $\widehat{\ZZ}$ by the same identification
we used to make $d_K$. Upshot: the choice of that identification drops
out, and the reciprocity map coincides with the Artin map that we
wrote down earlier.

A bit later (see Chapter~\ref{chap:connection}), 
we will check that $r'_{L/K}$ agrees with
the map that I called $r_{L/K}$, namely
the product of the local reciprocity maps. Remember, I need this in order
to recover Artin reciprocity in general.

%\end{document}


\chapter{The existence theorem}
%\documentclass[12pt]{article}
%\usepackage{amsfonts, amsthm, amsmath}
%\usepackage[all]{xy}
%
%\setlength{\textwidth}{6.5in}
%\setlength{\oddsidemargin}{0in}
%\setlength{\textheight}{8.5in}
%\setlength{\topmargin}{0in}
%\setlength{\headheight}{0in}
%\setlength{\headsep}{0in}
%\setlength{\parskip}{0pt}
%\setlength{\parindent}{20pt}
%
%\def\AA{\mathbb{A}}
%\def\CC{\mathbb{C}}
%\def\FF{\mathbb{F}}
%\def\PP{\mathbb{P}}
%\def\QQ{\mathbb{Q}}
%\def\RR{\mathbb{R}}
%\def\ZZ{\mathbb{Z}}
%\def\gotha{\mathfrak{a}}
%\def\gothb{\mathfrak{b}}
%\def\gothm{\mathfrak{m}}
%\def\gotho{\mathfrak{o}}
%\def\gothp{\mathfrak{p}}
%\def\gothq{\mathfrak{q}}
%\DeclareMathOperator{\Cor}{Cor}
%\DeclareMathOperator{\disc}{Disc}
%\DeclareMathOperator{\Gal}{Gal}
%\DeclareMathOperator{\GL}{GL}
%\DeclareMathOperator{\Hom}{Hom}
%\DeclareMathOperator{\Ind}{Ind}
%\DeclareMathOperator{\Norm}{Norm}
%\DeclareMathOperator{\Res}{Res}
%\DeclareMathOperator{\smcy}{smcy}
%\DeclareMathOperator{\Trace}{Trace}
%\DeclareMathOperator{\Cl}{Cl}
%
%\def\head#1{\medskip \noindent \textbf{#1}.}
%
%\newtheorem{theorem}{Theorem}
%\newtheorem{lemma}[theorem]{Lemma}
%\newtheorem{prop}[theorem]{Proposition}
%
%\begin{document}
%
%\begin{center}
%\bf
%Math 254B, UC Berkeley, Spring 2002 (Kedlaya) \\
%The Existence Theorem \\
%\end{center}

\head{Reference} 
Milne VII.9, Neukirch VI.6.

\medskip
With the ``abstract'' reciprocity theorem in hand,
we now prove the Existence Theorem, that every generalized
ideal class group of a number field
is identified by Artin reciprocity with the Galois group of a 
suitable abelian extension. Recall that the idelic formulation of this statement is as follows
(see Theorem~\ref{T:adelic existence theorem1}).
\begin{theorem} \label{T:adelic existence theorem2}
For $K$ a number field, the finite abelian extensions $L/K$ correspond
one-to-one with the open subgroups of $C_K$ of finite index, via the
map $L \mapsto \Norm_{L/K} C_L$.
\end{theorem}

The proof of this result is very similar to the proof we gave in 
the local case. For example, the reciprocity law immediately lets us reduce
to the following proposition.
\begin{prop}
Every open subgroup $U$ of $C_K$ of finite index contains 
$\Norm_{L/K} C_L$ for some finite extension $L$ of $K$.
\end{prop}
The proof of this proposition again uses Kummer theory, but more in the spirit of the algebraic proof of the Second Inequality.
\begin{proof}
We first prove this proposition in case $U$ has prime index $p$.
Let $J$ be the preimage of $U$ under the projection
$I_K \to C_K$, so that $J$ is open in $I_K$ of finite index.
Then $J$ contains a subgroup of the form
\[
V = \prod_{v \in S} \{1\} \times \prod_{v \notin S} \gotho_{K_v}^*
\]
for some set $S$ of places of $K$ containing
the infinite places and all places dividing $(p)$, which we may choose
large enough so that $I_{K,S} K^* = I_K$. Let $K_S = K^* \cap I_{K,S}$
be the group of $S$-units of $K$.

The group $J$ must also contain
$I_K^p$, so in particular contains
\[
W_S = \prod_{v \in S} (K_v^*)^p \times \prod_{v \notin S} U_v.
\]
Put $C_S = W_S K^*/K^*$; then $C_S \subseteq U$, so it suffices to show that
$C_S$ contains a norm subgroup.

If $K$ contains a primitive $p$-th root of unity, then an argument
as in the algebraic proof of the Second Inequality gives
$C_{S} = \Norm_{L/K} C_L$ for $L = K(K_S^{1/p})$.
Namely, one first computes that $\#C_K/C_S = p^{\#S} = [L:K]$
as in that proof, by reading orders off of the short exact sequence
\[
1 \to K_S / (W_S \cap K_S) \to I_{K,S}/{W_S} \to C_K /C_S \to 1:
\]
on one hand, we have $W_S \cap K_S = K_S^p$ (as in the proof of Lemma~\ref{L:adelic intersection}),
which gives $\#K_S/(W_S \cap K_S) = p^{\#S}$; on the other hand,
$I_{K,S}/W_S$ is the product of $\#S$ quotients of the form
$K_v^*/(K_v^*)^p$, each of which has order $p^2$ (generated by a uniformizer
and a $p$-th root of unity, since $p$ is prime to the residue characteristic).

One then checks that $W_S \subseteq \Norm_{L/K} I_L$ by checking this
place by place; the places not in $S$ are straightforward (they don't
ramify in $L$, so local units are local norms), and the
ones in $S$ follow from the fact that for any local field $M$
containing a $p$-th root of unity,
if $N = M((M^*)^{1/p})$, then
\[
\Norm_{N/M} N^* = (M^*)^p,
\]
which we proved in the course of proving the local existence theorem
(see Lemma~\ref{L:hilbert symbol}).
Putting this all together, we have
$C' \subseteq \Norm_{L/K} C_L$ and these two groups have the same index
$[L:K]$ in $C_K$ by the First and Second Inequalities
(Theorem~\ref{T:first inequality}, Theorem~\ref{T:first and second inequality}).

We next drop the restriction that $K$ contains a $p$-th root of unity
by reducing to the previous case. Namely, put $K' = K(\zeta_p)$.
For a choice of $S$ as above, let $S'$ be the set of places of $K'$ above $S$;
we can make $S$ large enough so that $I_{K',S'} (K')^* = I_{K'}$.
Then as above, $C_{S'} = \Norm_{L'/K'} C_{L'}$ if $L'$ is the extension of
$K'$ obtained by adjoining all $p$-th roots. Also as above,
$\Norm_{K'/K} W_{S'} \subseteq W_S$, so
\[
\Norm_{L'/K} C_{L'} = \Norm_{K'/K} (\Norm_{L'/K'} C_{L'})
= \Norm_{K'/K} C_{S'} \subseteq C_S \subseteq U.
\]

Finally, we handle the case where $U$ has arbitrary index, by induction
on that index using the above result as the base case. If $\#C_K/U$
is not prime, choose an intermediate subgroup $V$ between $U$ and $C_K$.
By the induction hypothesis, $V$ contains $N = \Norm_{L/K} C_L$ for some 
finite extension $L$ of $K$. Then 
\[
\#N / (U \cap N) =
\#UN / U \leq \#V/U.
\]
Let $W$ be the subgroup of $C_L$ consisting of those
$x$ whose norms lie in $U$. Then
\[
\#C_L/W \leq \# N/(U \cap N) \leq \#V/U,
\]
so by the induction hypothesis, $W$ contains $\Norm_{M/L} C_M$ for some
finite extension $M/L$. Thus $U$ contains $\Norm_{M/K} C_M$, as desired.  
\end{proof}

%\end{document}


\chapter{The connection with local reciprocity}
\label{chap:connection}
%\documentclass[12pt]{article}
%\usepackage{amsfonts, amsthm, amsmath}
%\usepackage[all]{xy}
%
%\setlength{\textwidth}{6.5in}
%\setlength{\oddsidemargin}{0in}
%\setlength{\textheight}{8.5in}
%\setlength{\topmargin}{0in}
%\setlength{\headheight}{0in}
%\setlength{\headsep}{0in}
%\setlength{\parskip}{0pt}
%\setlength{\parindent}{20pt}
%
%\def\AA{\mathbb{A}}
%\def\CC{\mathbb{C}}
%\def\FF{\mathbb{F}}
%\def\NN{\mathbb{N}}
%\def\PP{\mathbb{P}}
%\def\QQ{\mathbb{Q}}
%\def\RR{\mathbb{R}}
%\def\ZZ{\mathbb{Z}}
%\def\gotha{\mathfrak{a}}
%\def\gothb{\mathfrak{b}}
%\def\gothm{\mathfrak{m}}
%\def\gotho{\mathfrak{o}}
%\def\gothp{\mathfrak{p}}
%\def\gothq{\mathfrak{q}}
%\DeclareMathOperator{\ab}{ab}
%\DeclareMathOperator{\Cor}{Cor}
%\DeclareMathOperator{\disc}{Disc}
%\DeclareMathOperator{\Gal}{Gal}
%\DeclareMathOperator{\GL}{GL}
%\DeclareMathOperator{\Hom}{Hom}
%\DeclareMathOperator{\Ind}{Ind}
%\DeclareMathOperator{\Norm}{Norm}
%\DeclareMathOperator{\Res}{Res}
%\DeclareMathOperator{\sign}{sign}
%\DeclareMathOperator{\smcy}{smcy}
%\DeclareMathOperator{\Trace}{Trace}
%\DeclareMathOperator{\unr}{unr}
%\DeclareMathOperator{\Cl}{Cl}
%
%\def\head#1{\medskip \noindent \textbf{#1}.}
%
%\newtheorem{theorem}{Theorem}
%\newtheorem{lemma}[theorem]{Lemma}
%\newtheorem{prop}[theorem]{Proposition}
%
%\begin{document}
%
%\begin{center}
%\bf
%Math 254B, UC Berkeley, Spring 2002 (Kedlaya) \\
%The Connection With Local Reciprocity
%\end{center}

\head{Reference} 
Milne VII.5; Neukirch VI.4.

\medskip
So far, we've used abstract class field theory to construct
reciprocity isomorphisms
\[
r'_{L/K}: C_K/\Norm_{L/K} C_L \to \Gal(L/K)^{\ab}
\]
and to establish the adelic form of the existence theorem. We also
know that if $L/K$ is a small cyclotomic extension, then this
map induces the usual Artin map.

This time, we'll verify that this map coincides with the product of the
local reciprocity maps. As noted earlier, this is enough to recover the
classical Artin reciprocity law and existence theorem.

I've also included a sketch of a Galois-cohomological approach to the reciprocity isomorphism (as found in Milne),
using $H^2$ and an explicit computation in local class field theory.
One of the sketchy points is that this computation 
requires a little of the Lubin-Tate construction, which makes the local existence theorem rather explicit but will not be discussed herein.

\head{The relationship with local reciprocity}

\head{Caveat} This still does not follow Milne or Neukirch.

\medskip
For any extension $L/K$ of number fields,
we currently have the map $r_{L/K}: I_K \to \Gal(L/K)^{\ab}$
formed as the product of the local reciprocity maps,
and the abstract reciprocity map $r'_{L/K}: I_K \to \Gal(L/K)^{\ab}$,
which actually factors through $C_K$ and even through $C_K/\Norm_{L/K} C_L$.
We want to show that these agree. Before doing so, let's observe some
consequences of that which we'll then use in the proof that they agree.

If $L/K$ is abelian, $v$ is a place of $K$ and $w$ is a place of $L$
above $v$, then we have an injection $K_v^* \to I_K$, which we then
funnel through $r'_{L/K}$ to get a map into $\Gal(L/K)$.
The following properties would follow from knowing that $r = r'$, but must be checked independently as part of the proof.
\begin{lemma} \label{L:reciprocity corollaries}
The following statements hold:
\begin{enumerate}
\item[(i)] the composite map $K_v^* \to \Gal(L/K)$ actually maps into
the decomposition group of $w$;
\item[(ii)] the subgroup $\Norm_{L_w/K_v} L_w^*$ is contained in the kernel
of $K_v^* \to \Gal(L/K)$.
\end{enumerate}
\end{lemma}
In (ii), we would also know that ``contained in'' can be replaced by 
``equal to'', but we won't try to check that independently.
\begin{proof}
For (i), let $M$ be the
fixed field of the decomposition group of $w$; then we have the compatibility
\[
\xymatrix{
\Gal(L/K) \ar[r] \ar[d] & C_K/\Norm_{L/K} C_L \ar[d] \\
\Gal(M/K) \ar[r] & C_K/\Norm_{M/K} C_M
}
\]
and the image of $K_v^* \to I_K$ lands in $\Norm_{M/K} I_M$ because
$v$ splits completely in $M$. So this image lies in the kernel of
$\Gal(L/K) \to \Gal(M/K)$, which is to say $\Gal(L/M)$, the decomposition
group of $w$.

For (ii), we need only check that 
$\Norm_{L_w/K_v} L_w^*$ is contained in the kernel
of $K_v^* \to C_K/\Norm_{L/K} C_L$.
But $\Norm_{L_w/K_v} L_w^*$ is already in the kernel of
$K_v^* \to I_K/\Norm_{L/K} I_L$, so we're all set.
\end{proof}

Our plan now is to attempt to recover the local reciprocity map
from the maps $r'_{L/K}$. To do this, we need some auxiliary global extensions,
provided by the Existence Theorem.
\begin{lemma}
Let $K$ be a number field, $v$ a place of $K$ and $M$ a finite
abelian extension
of $K_v$. Then there exists a finite abelian extension $L$ of $K$ such that
for any place $w$ of $L$ above $v$, $L_w$ contains $M$.
\end{lemma}
\begin{proof}
This is easy if $v$ is infinite: if $v$ is complex there is nothing to
prove, and if $v$ is real then we may take $L = K(\sqrt{-1})$. So
assume hereafter that $v$ is finite.

By the Existence Theorem (Theorem~\ref{T:adelic existence theorem2}) and Lemma~\ref{L:reciprocity corollaries}(ii),
it suffices to produce an open subgroup $U$ of $C_K$ of finite index
such that the preimage of $U$ under $K_v^* \to C_K$ is contained in
$N = \Norm_{M/K_v} M^*$. Let $S$ be the set of infinite places
and $T = S \cup \{v\}$,
and let $G = K_T \cap N$. Then one can choose an additional place
$u$ (finite and distinct from $v$) and an open subgroup $V$ of $\gotho_{K_u}^*$
such that $V \cap K_T \subseteq G$. Now put
\[
W = N \times V \times \prod_{w \in S} K_w^* \times \prod_{w \notin S \cup \{u,v\}} \gotho_K^*
\]
and $U = W K^*/K^*$. If $\alpha_v \in K_v^*$ maps into $U$, then there
exists $\beta \in K^*$ such that $\alpha_v \beta \in W$. That means
first of all that $\beta \in K_T$ and then that $\beta \in V$, so that
$\beta \in G$ and so also $\beta \in N$. It also means that $\alpha_v \beta
\in N$. Thus $\alpha_v \in N$, as desired.
\end{proof}

For each place $v$ of $K$ and each abelian extension
$M$ of $K_v$, we can now write down
a map $r'_{K,v}: K_v^* \to \Gal(M/K_v)$ 
by choosing an abelian extension $L$ such that $M \subseteq L_w$ for
any place $w$ of $L$ above $v$, letting $N$ be the fixed field of the
decomposition group of $w$,
and setting $r'_{K,v}$ equal to the composition
\[
K_v^* \stackrel{r'_{L/K}}{\to} \Gal(L/N) = \Gal(L_w/K_v) \to \Gal(M/K_v).
\]
By the same compatibility as above, this doesn't change if we enlarge $L$.
Thus it doesn't depend on the choice of $L$ at all! (Any two choices of $L$
sit inside an abelian extension of $K$; compare both with that bigger field.)

Again by the usual compatibilities,
these maps fit together to give a single map
$r'_{K,v}: K_v^* \to \Gal(K_v^{\ab}/K_v)$. This map has the following
properties:
\begin{enumerate}
\item[(a)]
For $M/K$ unramified,
the induced map $K_v^* \to \Gal(M/K_v^*)$ kills units
and maps a uniformizer of $K_v$ to the Frobenius automorphism. Since that
extension is generated by roots of unity, we can check this using a 
suitable small cyclotomic extension of $K$, on which $r'$ may be computed explicitly.
We leave further details to the reader.

\item[(b)]
For any finite extension $M/K_v^*$, $r'_{K,v}$ induces an isomorphism
\[
K_v^*/\Norm_{M/K_v} M^* \to \Gal(M/K_v^*).
\]
Note that \emph{a priori} we only know that this map is injective, but by the local reciprocity law the
two groups have the same order, so it's actually an isomorphism. (For this and other reasons, we do not get an independent proof of
local class field theory by this process.)
\end{enumerate}
But these properties \emph{uniquely} characterize the local reciprocity map!
We conclude that $r'_{K,v}$ is the local reciprocity map for $K_v$,
and so $r_{L/K} = r'_{L/K}$ and at long last Artin reciprocity (and the
classical existence theorem, and the whole lot) follows.
Hooray!

It's worth repeating that only now do we know that the product
$r_{L/K}$ of the local reciprocity maps kills principal id\`eles. That fact,
which relates local behavior for different primes in a highly global
fashion, is the basis of various \emph{higher reciprocity laws}. See Milne,
Chapter VIII for details.

\head{An explicit computation in local CFT}

We sketch an alternate approach for comparing the ``abstract'' reciprocity
map $r'_{L/K}$ with the product $r_{L/K}$ of the local reciprocity maps,
following Milne (and Neukirch V.2).

We first verify that $r = r'$ for cyclotomic extensions of $\QQ$,
using an explicit computation in local class field theory. Namely,
we compute that if we identify $\Gal(\QQ(\zeta_{p^m})/\QQ)$ with 
$(\ZZ/p^m\ZZ)^*$, then
the local reciprocity maps are given by
\[
r_{\QQ_{\ell}(\zeta_{p^m})/\QQ_{\ell}}(a)
= \begin{cases} \sign(a) & \ell = \infty \\
\ell^{v_{\ell}}(a) & \ell \neq \infty, p \\
u^{-1} & \ell = p.
\end{cases}
\]
This is straightforward for $\ell = \infty$. For $\ell \neq \infty, p$,
we have an unramified extension of local fields, where we know the 
local reciprocity map takes a uniformizer to a Frobenius. In this case
the latter is simply $\ell$.

The hard work is in the case $\ell=p$. For that computation one uses what
amounts to a very special case of the Lubin-Tate construction of
explicit class field theory for local fields, using formal groups.
Put $K = \QQ_p$, $\zeta = \zeta_{p^m}$ and $L = \QQ_p(\zeta)$.

Suppose without loss of generality that $u$ is a positive integer, and let $\sigma
\in \Gal(L/K)$ be the automorphism corresponding to $u^{-1}$.
Since $L/K$ is totally ramified at $p$, we have
$\Gal(L/K) \cong \Gal(L^{\unr}/K^{\unr})$, and we can view $\sigma$
as an element of $\Gal(L^{\unr}/K)$. Let $\phi_L \in \Gal(L^{\unr}/L)$
denote the Frobenius, and put $\tau = \sigma \phi_L$. Then
$\tau$ restricts to the Frobenius in $\Gal(K^{\unr}/K)$
and to $\sigma$ in $\Gal(L/K)$. By Neukirch's definition of the reciprocity
map, we may compute $r^{-1}_{L/K}(\sigma)$ as
$\Norm_{M/K} \pi_M$, where $M$ is the fixed field of
$\tau$ and $\pi_M$ is a uniformizer. We want that norm to be $u$
times a norm from $L$ to $K$, i.e.,
\[
r^{-1}_{L/K}(\sigma) \in u \Norm_{L/K} L^*.
\]

Define the polynomial
\[
e(x) = x^p + upx
\]
and put
\[
P(x) = e^{(n-1)}(x)^{p-1} + pu,
\]
where $e^{(k+1)}(x) = e(e^{(k)}(u))$. Then $P(x)$ satisfies Eisenstein's
criterion, so its splitting field over $\QQ_p$ is totally ramified,
any root of $P$ is a uniformizer, and the norm of said uniformizer is
$(-1)^{[L:K]} pu \in \Norm_{L/K} L^*$, since
$\Norm_{L/K}(\zeta-1) = (-1)^{[L:K]}(p)$.

The punch line is that the splitting field of $P(x)$ is precisely
$M$! Here is where the Lubin-Tate construction comes to the rescue...
and where I will stop this sketch. See Neukirch V.2 and V.4 and/or Milne
I.3.

\head{A bit about Brauer groups}

For background about Brauer groups, see Milne, IV. We'll be following Milne VII.8 for now,
and omitting many details.

\begin{prop}
Put $L = K(\zeta_n)$. Then
$r_{L/K}: I_K \to \Gal(L/K)$ maps all principal id\`eles to the
identity.
\end{prop}
\begin{proof}
For $K = \QQ$, this follows from the previous section (factor $n$ into
prime powers and apply the previous argument to each factor).
In general, we have a compatibility
\[
\xymatrix{
I_L \ar[r] \ar^{\Norm_{L_w/\QQ_p}}[d] & \Gal(L_w(\zeta_n)/L_w) \ar[d]\\
I_{\QQ} \ar[r] & \Gal(\QQ_p(\zeta_n)/\QQ_p)
}
\]
and we know the bottom row kills principal id\`eles and the
right column is injective. Thus the top row kills principal id\`eles too.
\end{proof}

To make more progress, we need to bring in $H^2$, as we did in local
reciprocity. (Unfortunately, trying to compute $H^2$ of the id\`ele class
group is a headache, so we can't imitate the argument perfectly.)
Recall there that we saw that every element of $H^2(L/K)$
could be ``brought in'' from a suitable unramified extension of $K$.
We have a similar situation here with ``unramified'' replaced by
``cyclotomic''.
\begin{prop}
Let $L/K$ be any finite Galois extension of number fields. Then for any
element $x$ of $H^2(\Gal(L/K), L^*)$, there exists a cyclic, cyclotomic
extension $M$ of $K$ and an element $y$ of $H^2(\Gal(M/K), M^*)$ such that
$x$ and $y$ map to the same element of $H^2(\Gal(ML/K), ML^*)$.
\end{prop}
\begin{proof}
Omitted. See above references.
\end{proof}

Hereafter $L/K$ is abelian.
From the exact sequence
\[
0 \to L^* \to I_L \to C_L \to 0
\]
we get a fragment
\[
1 = H^1(\Gal(L/K), C_L)
\to H^2(\Gal(L/K), L^*) \to H^2(\Gal(L/K), I_L)
\]
so the map $H^2(\Gal(L/K), L^*) \to H^2(\Gal(L/K), I_L)
= \oplus H^2(\Gal(L/K), I_L)$ is injective. Each factor in the direct
sum is canonically a subgroup of $\QQ/\ZZ$, so we get a sum map
$H^2(\Gal(L/K), I_L) \to \QQ/\ZZ$.

It turns out (see Milne, Lemma VII.8.5)
that for any map $\Gal(L/K) \to \QQ/\ZZ$,
there is a commuting diagram
\[
\xymatrix{
K^* \ar[r] \ar[d] & I_K \ar^{r_{L/K}}[r] \ar[d] &\Gal(L/K) \ar[d] \\
H^2(L^*) \ar[r] & H^2(I_L) \ar[r] & \QQ/\ZZ
}
\]
If $L/K$ is cyclic, we may choose the map $\Gal(L/K) \to \QQ/\ZZ$
to be injective, and then the first vertical arrow will be surjective.
(In fact, it's $K^* \to K^*/\Norm_{L/K} L^* = H^0_T(L^*)$ plus the
periodicity isomorphism $H^0_T(L^*) \to H^2_T(L^*)$.)
Then the fact that $r_{L/K}$ kills principal id\`eles implies that
the composite $H^2(L^*) \to \QQ/\ZZ$ is the zero map.

Now if we know $H^2(\Gal(L/K), L^*) \to \QQ/\ZZ$ vanishes for all
cyclic extensions, we know it in particular for cyclic cyclotomic extensions.
But then the previous proposition tells us that it also vanishes for
any finite Galois extension!
Now we can use the diagram in reverse: it tells us that 
for $a \in K^*$, $r_{L/K}(a)$ is killed by any homomorphism
$\Gal(L/K) \to \QQ/\ZZ$. Since $\Gal(L/K)$ is an abelian group, that 
implies $r_{L/K}(a)$ is trivial.

To conclude, we now have that $r_{L/K}$ kills principal id\`eles in general.
By construction, it also kills norms (since it does so locally),
so it induces a surjection $C_K/\Norm_{L/K} C_L \to \Gal(L/K)$.
(Remember, the fact that it's surjective follows from the First
Inequality.)
But the order of the first group is less than or equal to the order of the
second by the Second Inequality. So it's an isomorphism, and 
the reciprocity law is established. Hooray again!

%\end{document}


\part{Coda}

\chapter{Parting thoughts}
%\documentclass[12pt]{article}
%\usepackage{amsfonts, amsthm, amsmath}
%\usepackage[all]{xy}
%
%\setlength{\textwidth}{6.5in}
%\setlength{\oddsidemargin}{0in}
%\setlength{\textheight}{8.5in}
%\setlength{\topmargin}{0in}
%\setlength{\headheight}{0in}
%\setlength{\headsep}{0in}
%\setlength{\parskip}{0pt}
%\setlength{\parindent}{20pt}
%
%\def\AA{\mathbb{A}}
%\def\CC{\mathbb{C}}
%\def\FF{\mathbb{F}}
%\def\PP{\mathbb{P}}
%\def\QQ{\mathbb{Q}}
%\def\RR{\mathbb{R}}
%\def\ZZ{\mathbb{Z}}
%\def\gotha{\mathfrak{a}}
%\def\gothb{\mathfrak{b}}
%\def\gothm{\mathfrak{m}}
%\def\gotho{\mathfrak{o}}
%\def\gothp{\mathfrak{p}}
%\def\gothq{\mathfrak{q}}
%\DeclareMathOperator{\Cor}{Cor}
%\DeclareMathOperator{\disc}{Disc}
%\DeclareMathOperator{\Gal}{Gal}
%\DeclareMathOperator{\GL}{GL}
%\DeclareMathOperator{\Hom}{Hom}
%\DeclareMathOperator{\Ind}{Ind}
%\DeclareMathOperator{\Norm}{Norm}
%\DeclareMathOperator{\Res}{Res}
%\DeclareMathOperator{\smcy}{smcy}
%\DeclareMathOperator{\Trace}{Trace}
%\DeclareMathOperator{\Cl}{Cl}
%
%\def\head#1{\medskip \noindent \textbf{#1}.}
%
%\newtheorem{theorem}{Theorem}
%\newtheorem{lemma}[theorem]{Lemma}
%\newtheorem{prop}[theorem]{Proposition}
%
%\begin{document}
%
%\begin{center}
%\bf
%Math 254B, UC Berkeley, Spring 2002 (Kedlaya) \\
%Parting Thoughts
%\end{center}

Class field theory is a vast expanse of mathematics, so it's worth
concluding by taking stock of what we've seen and what we haven't.
First, a reminder of the main topics we have covered.
\begin{itemize}
\item
The Kronecker-Weber theorem: the maximal abelian extension of $\QQ$
is generated by roots of unity.
\item
The Artin reciprocity law for an abelian extension of a number field.
\item
The existence theorem classifying abelian extensions of number fields
in terms of generalized ideal class groups.
\item
The Chebotarev density theorem, describing the distribution over primes
of a number field of various splitting behaviors in an extension field.
\item
Some group cohomology ``nuts and bolts'', including some key results
of Tate.
\item
The local reciprocity law and existence theorem.
\item
Ad\`eles, id\`eles, and the idelic formulations of reciprocity and the
existence theorem.
\item
Computations of group cohomology in the local (multiplicative group)
and global (id\`ele class group) cases.
\end{itemize}

Now, some things that we haven't covered. When this course was first taught, these topics were assigned as final projects to individual students in the course.
\begin{itemize}
\item
The Lubin-Tate construction of explicit class field theory for local
fields.
\item
The Brauer group of a field (i.e., $H^2(\Gal(\overline{K}/K), K^*)$),
its relationship with central simple algebras, and the Fundamental Exact
Sequence.
\item
More details about zeta functions and L-functions, including
the class number formula and the distribution of norms in ideal classes.
\item
Another application of group cohomology: to computing
ranks of elliptic curves.
\item
Orders in number fields, and the notion of a ``ring class field.''
\item
An analogue of the Kronecker-Weber theorem over the function field
$\FF_q(t)$, and even over its extensions.
\item
Explicit class field theory for imaginary quadratic fields, via elliptic
curves with complex multiplication.
\item
Quadratic forms over number fields and the Hasse-Minkowski theorem.
\item
Artin (nonabelian) L-series, the basis of ``nonabelian class field theory.''
\end{itemize}

Some additional topics for further reading would include the following.
\begin{itemize}
\item The Golod-Shafarevich inequality and the class field tower problem
(see Cassels-Fr\"ohlich).
\item Class field theory for function fields used to produce curves
over finite fields with unusually many points (see the web site \url{http://manypoints.org} for references).
\item
Application of Artin reciprocity to cubic, quartic, and higher reciprocity
(see Milne).
\item
Algorithmic class field theory (see the books of Henri Cohen).
\end{itemize}

And finally, some ruminations about where number theory has gone in the fifty
or so years since the results of class field theory were established in the
form that we saw them. 
In its cleanest form, class field theory describes a
correspondence between one-dimensional representations of
$\Gal(\overline{K}/K)$, for $K$ a number field, and certain representations
of $\GL_1(\AA_K)$, otherwise known as the group of id\`eles. But what about
the nonabelian extensions of $K$, or what is about the same, the
higher-dimensional representations of $\Gal(\overline{K}/K)$?

In fact, building on work of many authors, Langlands has proposed that for
every $n$, there should be a correspondence between
$n$-dimensional representations of $\Gal(\overline{K}/K)$ and representations
of $\GL_n(\AA_K)$. This correspondence is the heart of the
so-called ``Langlands Program'', an unbelievably deep web of statements
which has driven much of the mathematical establishment for the last few 
decades. For example, for $n=2$, this correspondence includes on one hand
the $2$-dimensional Galois representations coming from elliptic curves,
and on the other hand representations of $\GL_2(\AA_K)$ corresponding to
modular forms. In particular, it includes the ``modularity of elliptic
curves'', proved by Breuil, Conrad, Diamond, and Taylor following on the
celebrated work of Wiles on Fermat's Last Theorem.

Various analogues of the Langlands correspondence have been worked out
very recently: for local fields by Taylor and Harris (and again, more
simply, by Henniart and Scholze), and for function fields by Lafforgue, based on the 
work of Drinfeld. The work of Laumon and Ngo on the Langlands fundamental lemma is also part of this story.

Okay, enough rambling for now; I hope that helps provide a bit of perspective.
Thanks for reading!

%\end{document}




\end{document}