%\documentclass[12pt]{article}
%\usepackage{amsfonts, amsthm, amsmath}
%
%\setlength{\textwidth}{6.5in}
%\setlength{\oddsidemargin}{0in}
%\setlength{\textheight}{8.5in}
%\setlength{\topmargin}{0in}
%\setlength{\headheight}{0in}
%\setlength{\headsep}{0in}
%\setlength{\parskip}{0pt}
%\setlength{\parindent}{20pt}
%
%\def\AA{\mathbb{A}}
%\def\CC{\mathbb{C}}
%\def\FF{\mathbb{F}}
%\def\PP{\mathbb{P}}
%\def\QQ{\mathbb{Q}}
%\def\RR{\mathbb{R}}
%\def\ZZ{\mathbb{Z}}
%\def\gotha{\mathfrak{a}}
%\def\gothb{\mathfrak{b}}
%\def\gothm{\mathfrak{m}}
%\def\gotho{\mathfrak{o}}
%\def\gothp{\mathfrak{p}}
%\def\gothq{\mathfrak{q}}
%\DeclareMathOperator{\disc}{Disc}
%\DeclareMathOperator{\Gal}{Gal}
%\DeclareMathOperator{\GL}{GL}
%\DeclareMathOperator{\Hom}{Hom}
%\DeclareMathOperator{\Norm}{Norm}
%\DeclareMathOperator{\Trace}{Trace}
%\DeclareMathOperator{\Cl}{Cl}
%
%\def\head#1{\medskip \noindent \textbf{#1}.}
%
%\newtheorem{theorem}{Theorem}
%\newtheorem{lemma}[theorem]{Lemma}
%\newtheorem{prop}[theorem]{Proposition}
%
%\begin{document}
%
%\begin{center}
%\bf
%Math 254B, UC Berkeley, Spring 2002 (Kedlaya) \\
%ad\`eles and id\`eles in Field Extensions
%\end{center}

\head{Reference} 
Neukirch, Section VI.1 and VI.2.

\head{Ad\`eles in Field Extensions}

If $L/K$ is an extension of number fields, we get an embedding
$\AA_K \hookrightarrow \AA_L$ as follows: given $\alpha \in \AA_K$,
each place $w$ of $L$ restricts to a place $v$ of $K$, so set the $w$-component
of the image of $\alpha$ to $\alpha_v$.
This embedding induces an inclusion $I_K \hookrightarrow I_L$ of id\`ele groups
as well.

If $L/K$ is Galois with Galois group $G$, then $G$ acts naturally on
$\AA_L$ and $I_L$; more generally, if $g \in \Gal(\overline{K}/K)$, then
$g$ maps $L$ to some other extension $L^g$ of $K$, and $G$ induces
isomorphisms of $\AA_L$ with $\AA_{L^g}$ and
of $I_L$ with $I_{L^g}$. Namely, if $(\alpha_v)_v$ is an
id\`ele over $L$ and $g \in G$, then $g$ maps the completion $L_v$
of $L$ to a completion $L_{v^g}$ of $L^g$. (Remember, a place $v$
corresponds to an absolute value $|\cdot|_v$ on $L$; the absolute
value $|\cdot|_{v^g}$ on $L^g$ is given by $|a^g|_{v^g}| = |a|_v$.)
As you might expect, this action is compatible with the embeddings
of $L$ in $I_L$ and $L^g$ in $I_{L^g}$, so it induces an isomorphism
$C_L \to C_{L^g}$ of id\`ele class groups.

\head{Aside}
Neukirch points out that you can regard $\AA_L$ as the tensor
product $\AA_K \otimes_K L$; in particular, this is a good way to see
the Galois action on $\AA_L$. Details are left to the reader.

\medskip
We can define trace and norm maps as well:
\[
\Trace_{\AA_L/\AA_K}(x) = \sum_g x^g, \qquad
\Norm_{I_L/I_K}(x) = \prod_g x^g
\]
where $g$ runs over coset representatives of $\Gal(\overline{K}/L)$
in $\Gal(\overline{K}/K)$, the sum and product taking places in the ad\`ele
and id\`ele rings of the Galois closure of $L$ over $K$.
In particular, if $L/K$ is Galois, $g$ simply runs over $\Gal(L/K)$.

In terms of components, these definitions translate as
\begin{align*}
  (\Trace_{\AA_L/\AA_K}(\alpha))_{v} &= \sum_{w | v}
\Trace_{L_w/K_v}(\alpha_w) \\
  (\Norm_{I_L/I_K}(\alpha))_{v} &= \prod_{w | v}
\Norm_{L_w/K_v}(\alpha_w).
\end{align*}
The trace and norm do what you expect on principal ad\`eles/id\`eles. In
particular, the norm descends to a map $\Norm_{L/K}: C_L \to C_K$.

\head{Aside}
You can also define the trace of an ad\`ele $\alpha \in \AA_L$ as the trace of
addition by $\alpha$ as an endomorphism of the $\AA_K$-module $\AA_L$,
and the norm of an id\`ele $\alpha \in I_L$ as the determinant of 
multiplication by $\alpha$ as an automorphism of the $\AA_K$-module
$\AA_L$. (Yes, the action is on the \emph{ad\`eles} in both cases.
Remember, id\`eles should be thought of as automorphisms of the ad\`eles,
not as elements of the ad\`ele ring, in order to topologize them correctly.)

If $L/K$ is a Galois extension, then $\Gal(L/K)$ acts on $\AA_L$ and
$I_L$ fixing $\AA_K$ and $I_K$, respectively, and we have the following.
\begin{prop}
If $L/K$ is a Galois extension with Galois group $G$, then
$\AA_L^G = \AA_K$ and $I_L^G = I_K$.
\end{prop}
\begin{proof}
If $v$ is a place of $K$, then for each place $w$ of $K$ above
$v$, the decomposition group $G_w$ of $w$ is isomorphic to
$\Gal(L_w/K_v)$. So if $(\alpha)$ is an ad\`ele or id\`ele which is
$G$-invariant, then $\alpha_w$ is $\Gal(L_w/K_v)$-invariant for each
$w$, so belongs to $K_v$. Moreover, $G$ acts transitively on the
places $w$ above $v$, so $\alpha_w = \alpha_{w'}$ for any two places
$w, w'$ above $v$. Thus $(\alpha)$ is an ad\`ele or id\`ele over $K$.
\end{proof}

This has the following nice consequence for the id\`ele class group,
a fact which is quite definitely not true for the ideal class group:
the map $\Cl_K \to \Cl_L^G$ is neither injective nor surjective in general.
This is our first hint of why the id\`ele class group will be a more
convenient target for a reciprocity map than the ideal class group.
\begin{prop}[Galois descent]
  If $L/K$ is a Galois extension with Galois group $G$, then
$G$ acts on $C_L$, and the $G$-invariant elements are precisely $C_K$.
\end{prop}
\begin{proof}
We start with an exact sequence
\[
1 \to L^* \to I_L \to C_L \to 1
\]
of $G$-modules. Taking $G$-invariants, we get a long exact sequence
\[
1 \to (L^*)^G = K^* \to (I_L)^G = I_K \to C_L^G \to H^1(G, L^*),
\]
and the last term is 1 by Theorem 90 (Lemma~\ref{L:theorem 90}). So 
we again have a short exact sequence, and $C_L^G \cong I_K/K^* = C_K$.
\end{proof}

%\end{document}


