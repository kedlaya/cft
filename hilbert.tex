%\documentclass[12pt]{article}
%\usepackage{amsfonts, amsthm, amsmath}
%
%\setlength{\textwidth}{6.5in}
%\setlength{\oddsidemargin}{0in}
%\setlength{\textheight}{8.5in}
%\setlength{\topmargin}{0in}
%\setlength{\headheight}{0in}
%\setlength{\headsep}{0in}
%\setlength{\parskip}{0pt}
%\setlength{\parindent}{20pt}
%
%\def\AA{\mathbb{A}}
%\def\CC{\mathbb{C}}
%\def\FF{\mathbb{F}}
%\def\PP{\mathbb{P}}
%\def\QQ{\mathbb{Q}}
%\def\RR{\mathbb{R}}
%\def\ZZ{\mathbb{Z}}
%\def\gotha{\mathfrak{a}}
%\def\gothb{\mathfrak{b}}
%\def\gothm{\mathfrak{m}}
%\def\gotho{\mathfrak{o}}
%\def\gothp{\mathfrak{p}}
%\def\gothq{\mathfrak{q}}
%\DeclareMathOperator{\disc}{Disc}
%\DeclareMathOperator{\Gal}{Gal}
%\DeclareMathOperator{\GL}{GL}
%\DeclareMathOperator{\Hom}{Hom}
%\DeclareMathOperator{\Norm}{Norm}
%\DeclareMathOperator{\Trace}{Trace}
%\DeclareMathOperator{\Cl}{Cl}
%
%\def\head#1{\medskip \noindent \textbf{#1}.}
%
%\newtheorem{theorem}{Theorem}
%\newtheorem{lemma}[theorem]{Lemma}
%
%\begin{document}
%
%\begin{center}
%\bf
%Math 254B, UC Berkeley, Spring 2002 (Kedlaya) \\
%The Hilbert Class Field
%\end{center}

\head{Reference} Milne, Introduction; Neukirch, VI.6.

\medskip

Recall that the field $\QQ$ has no extensions which are everywhere
unramified (Theorem~\ref{T:Minkowski}). This is quite definitely not true of other
number fields; we begin with an example illustrating this.

In the number field $K = \QQ(\sqrt{-5})$, the ring of integers is
$\ZZ[\sqrt{-5}]$ and the ideal $(2)$ factors as $\gothp^2$,
where the ideal $\gothp = (2, 1 + \sqrt{-5})$ is not principal.

Now let's see what happens when we adjoin a square root of $-1$,
obtaining $L = \QQ(\sqrt{-5}, \sqrt{-1})$. The
extension $\QQ(\sqrt{-1})/\QQ$ only ramifies over 2, so
$L/K$ can only be ramified over $\gothp$.
On the other hand, if we write $L = K(\alpha)$ where $\alpha =
(1 + \sqrt{5})/2$, then modulo $\gothp$ 
the minimal polynomial $x^2-x-1$ of $\alpha$ remains irreducible, so $\gothp$
is unramified (and not split) in $L$. 

We've now seen that $\QQ(\sqrt{-5})$ admits both a nonprincipal ideal and
an unramified abelian extension. It turns out these are not unrelated events.
Caution: until further notice, the phrase ``$L/K$ is unramified'' will mean
that $L/K$ is unramified over all finite places in the usual sense, \emph{and}
that every real embedding of $K$ extends to a real embedding of $L$. 
(Get used to this. The real and complex embeddings of a number field will
be treated like primes consistently throughout this text.)
\begin{theorem} \label{T:Hilbert class field}
Let $L$ be the maximal unramified
abelian extension of a number field $K$.
Then $L/K$ is finite, and its Galois group is isomorphic to
the ideal class group $\Cl(K)$ of $K$.
\end{theorem}
In fact, there is a canonical isomorphism, given by the Artin reciprocity
law. We'll see this a bit later. The field $L$ is called the \emph{Hilbert
class field} of $K$.

Warning: there can be infinite unramified \emph{nonabelian} extensions.
In fact, Golod and Shafarevich used unramified abelian extensions to 
construct these! Namely, starting from a number field $K = K_0$,
let $K_1$ be the Hilbert class field of $K_0$, let $K_2$ be the Hilbert
class field of $K_1$, and so on. Then $K_i$ is an unramified but not
necessarily abelian extension of $K_0$, and for a suitable choice of 
$K_0$, $[K_i:K_0]$ can be unbounded. (See Cassels-Frohlich for more discussion.)

\head{Exercises}

\begin{enumerate}
\item
Let $K$ be an imaginary
quadratic extension of $\QQ$ in which $t$ finite primes ramify.
Asuming Theorem~\ref{T:Hilbert class field},
prove that $\#(\Cl(K)/2\Cl(K)) = 2^{t-1}$;
this recovers a theorem of Gauss originally proved using binary quadratic forms. 
(Hint: if an odd prime $p$
ramifies in $K$,
show that $K(\sqrt{p^*})/K$ is unramified for $p^* = (-1)^{(p-1)/2} p$; if 2 ramifies in $K$, show that
$K(p^*)/K$ is unramified for one of $p^* = -1, 2, -2$.)
\item
Give an example, using a real quadratic field, to illustrate that:
\begin{enumerate}
\item[(a)] Theorem~\ref{T:Hilbert class field} fails if we don't require the extensions to be
unramified above the real place;
\item[(b)] the previous exercise fails for real quadratic fields.
\end{enumerate}
\item
Prove that Exercise~1 extends to real quadratic fields if one replaces the
class group by the \emph{narrow class group}, in which you
only mod out by principal ideals having a totally positive generator.
This gives an example of a \emph{ray class group}; more on those in the next chapter.
\item
The field $\QQ(\sqrt{-23})$ admits an ideal of order 3 in the class group and
an unramified
abelian extension of degree 3. Find both. (Hint: the extension contains a cubic field of discriminant -23.)
\end{enumerate}

%\end{document}


