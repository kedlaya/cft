%\documentclass[12pt]{article}
%\usepackage{amsfonts, amsthm, amsmath}
%\usepackage[all]{xy}
%
%\setlength{\textwidth}{6.5in}
%\setlength{\oddsidemargin}{0in}
%\setlength{\textheight}{8.5in}
%\setlength{\topmargin}{0in}
%\setlength{\headheight}{0in}
%\setlength{\headsep}{0in}
%\setlength{\parskip}{0pt}
%\setlength{\parindent}{20pt}
%
%\def\AA{\mathbb{A}}
%\def\CC{\mathbb{C}}
%\def\FF{\mathbb{F}}
%\def\PP{\mathbb{P}}
%\def\QQ{\mathbb{Q}}
%\def\RR{\mathbb{R}}
%\def\ZZ{\mathbb{Z}}
%\def\gotha{\mathfrak{a}}
%\def\gothb{\mathfrak{b}}
%\def\gothm{\mathfrak{m}}
%\def\gotho{\mathfrak{o}}
%\def\gothp{\mathfrak{p}}
%\def\gothq{\mathfrak{q}}
%\DeclareMathOperator{\Cor}{Cor}
%\DeclareMathOperator{\disc}{Disc}
%\DeclareMathOperator{\Gal}{Gal}
%\DeclareMathOperator{\GL}{GL}
%\DeclareMathOperator{\Hom}{Hom}
%\DeclareMathOperator{\Ind}{Ind}
%\DeclareMathOperator{\Norm}{Norm}
%\DeclareMathOperator{\Res}{Res}
%\DeclareMathOperator{\smcy}{smcy}
%\DeclareMathOperator{\Trace}{Trace}
%\DeclareMathOperator{\Cl}{Cl}
%
%\def\head#1{\medskip \noindent \textbf{#1}.}
%
%\newtheorem{theorem}{Theorem}
%\newtheorem{lemma}[theorem]{Lemma}
%\newtheorem{prop}[theorem]{Proposition}
%
%\begin{document}
%
%\begin{center}
%\bf
%Math 254B, UC Berkeley, Spring 2002 (Kedlaya) \\
%Parting Thoughts
%\end{center}

Class field theory is a vast expanse of mathematics, so it's worth
concluding by taking stock of what we've seen and what we haven't.
First, a reminder of the main topics we have covered.
\begin{itemize}
\item
The Kronecker-Weber theorem: the maximal abelian extension of $\QQ$
is generated by roots of unity.
\item
The Artin reciprocity law for an abelian extension of a number field.
\item
The existence theorem classifying abelian extensions of number fields
in terms of generalized ideal class groups.
\item
The Chebotarev density theorem, describing the distribution over primes
of a number field of various splitting behaviors in an extension field.
\item
Some group cohomology ``nuts and bolts'', including some key results
of Tate.
\item
The local reciprocity law and existence theorem.
\item
Ad\`eles, id\`eles, and the idelic formulations of reciprocity and the
existence theorem.
\item
Computations of group cohomology in the local (multiplicative group)
and global (id\`ele class group) cases.
\end{itemize}

Now, some things that we haven't covered. When this course was first taught, these topics were assigned as final projects to individual students in the course.
\begin{itemize}
\item
The Lubin-Tate construction of explicit class field theory for local
fields.
\item
The Brauer group of a field (i.e., $H^2(\Gal(\overline{K}/K), K^*)$),
its relationship with central simple algebras, and the Fundamental Exact
Sequence.
\item
More details about zeta functions and L-functions, including
the class number formula and the distribution of norms in ideal classes.
\item
Another application of group cohomology: to computing
ranks of elliptic curves.
\item
Orders in number fields, and the notion of a ``ring class field.''
\item
An analogue of the Kronecker-Weber theorem over the function field
$\FF_q(t)$, and even over its extensions.
\item
Explicit class field theory for imaginary quadratic fields, via elliptic
curves with complex multiplication.
\item
Quadratic forms over number fields and the Hasse-Minkowski theorem.
\item
Artin (nonabelian) L-series, the basis of ``nonabelian class field theory.''
\end{itemize}

Some additional topics for further reading would include the following.
\begin{itemize}
\item The Golod-Shafarevich inequality and the class field tower problem
(see Cassels-Fr\"ohlich).
\item Class field theory for function fields used to produce curves
over finite fields with unusually many points (see the web site \url{http://manypoints.org} for references).
\item
Application of Artin reciprocity to cubic, quartic, and higher reciprocity
(see Milne).
\item
Algorithmic class field theory (see the books of Henri Cohen).
\end{itemize}

And finally, some ruminations about where number theory has gone in the fifty
or so years since the results of class field theory were established in the
form that we saw them. 
In its cleanest form, class field theory describes a
correspondence between one-dimensional representations of
$\Gal(\overline{K}/K)$, for $K$ a number field, and certain representations
of $\GL_1(\AA_K)$, otherwise known as the group of id\`eles. But what about
the nonabelian extensions of $K$, or what is about the same, the
higher-dimensional representations of $\Gal(\overline{K}/K)$?

In fact, building on work of many authors, Langlands has proposed that for
every $n$, there should be a correspondence between
$n$-dimensional representations of $\Gal(\overline{K}/K)$ and representations
of $\GL_n(\AA_K)$. This correspondence is the heart of the
so-called ``Langlands Program'', an unbelievably deep web of statements
which has driven much of the mathematical establishment for the last few 
decades. For example, for $n=2$, this correspondence includes on one hand
the $2$-dimensional Galois representations coming from elliptic curves,
and on the other hand representations of $\GL_2(\AA_K)$ corresponding to
modular forms. In particular, it includes the ``modularity of elliptic
curves'', proved by Breuil, Conrad, Diamond, and Taylor following on the
celebrated work of Wiles on Fermat's Last Theorem.

Various analogues of the Langlands correspondence have been worked out
very recently: for local fields by Taylor and Harris (and again, more
simply, by Henniart and Scholze), and for function fields by Lafforgue, based on the 
work of Drinfeld. The work of Laumon and Ngo on the Langlands fundamental lemma is also part of this story.

Okay, enough rambling for now; I hope that helps provide a bit of perspective.
Thanks for reading!

%\end{document}


