%\documentclass[12pt]{article}
%\usepackage{amsfonts, amsthm, amsmath}
%\usepackage[all]{xy}
%
%\setlength{\textwidth}{6.5in}
%\setlength{\oddsidemargin}{0in}
%\setlength{\textheight}{8.5in}
%\setlength{\topmargin}{0in}
%\setlength{\headheight}{0in}
%\setlength{\headsep}{0in}
%\setlength{\parskip}{0pt}
%\setlength{\parindent}{20pt}
%
%\def\AA{\mathbb{A}}
%\def\CC{\mathbb{C}}
%\def\FF{\mathbb{F}}
%\def\NN{\mathbb{N}}
%\def\PP{\mathbb{P}}
%\def\QQ{\mathbb{Q}}
%\def\RR{\mathbb{R}}
%\def\ZZ{\mathbb{Z}}
%\def\gotha{\mathfrak{a}}
%\def\gothb{\mathfrak{b}}
%\def\gothm{\mathfrak{m}}
%\def\gotho{\mathfrak{o}}
%\def\gothp{\mathfrak{p}}
%\def\gothq{\mathfrak{q}}
%\DeclareMathOperator{\ab}{ab}
%\DeclareMathOperator{\Cor}{Cor}
%\DeclareMathOperator{\cyc}{cyc}
%\DeclareMathOperator{\disc}{Disc}
%\DeclareMathOperator{\Gal}{Gal}
%\DeclareMathOperator{\GL}{GL}
%\DeclareMathOperator{\Hom}{Hom}
%\DeclareMathOperator{\Ind}{Ind}
%\DeclareMathOperator{\Norm}{Norm}
%\DeclareMathOperator{\Res}{Res}
%\DeclareMathOperator{\smcy}{smcy}
%\DeclareMathOperator{\Trace}{Trace}
%\DeclareMathOperator{\Cl}{Cl}
%
%\def\head#1{\medskip \noindent \textbf{#1}.}
%
%\newtheorem{theorem}{Theorem}
%\newtheorem{lemma}[theorem]{Lemma}
%\newtheorem{prop}[theorem]{Proposition}
%
%\begin{document}
%
%\begin{center}
%\bf
%Math 254B, UC Berkeley, Spring 2002 (Kedlaya) \\
%An ``Abstract'' Reciprocity Map
%\end{center}

\head{Reference} 
Milne VII.5; Neukirch VI.4, but only loosely.

\medskip

In this chapter, we'll manufacture a canonical isomorphism
$\Gal(L/K)^{\ab} \to C_K/\Norm_{L/K} C_L$ for any finite extension
$L/K$ of number fields, where $C_K$ and $C_L$ are the corresponding
id\`ele class groups. However, we won't yet know it agrees with our proposed
reciprocity map, which is the product of the local reciprocity maps.
We'll check that in the next chapter.

\head{Cyclotomic extensions}

The cyclotomic extensions (extensions by roots of unity)
of a number field play a role in class field
theory analogous to the role played by the unramified extensions in local
class field theory. This makes it essential to make an explicit study
of them for use in proving the main results.

First of all, we should further articulate a distinction that has come up already.
The extension $\cup_n \QQ(\zeta_n)$ of $\QQ$ obtained by adjoining all roots
of unity has Galois group $\widehat{\ZZ}^* =
\prod_p \ZZ_p^*$. That group has a lot
of torsion, since each $\ZZ_p^*$ contains a torsion subgroup of order $p-1$
(or 2, if $p=2$). If we take the fixed field for the torsion subgroup of
$\ZZ^*$, we get a slightly smaller extension, which I'll call the
\emph{small cyclotomic} extension of $\QQ$ and denote $\QQ^{\smcy}$.
Its Galois group is
$\prod_p \ZZ_p = \widehat{\ZZ}$. For $K$ a number field, define
$K^{\smcy} = K \QQ^{\smcy}$; then $\Gal(K^{\smcy}/K) \cong \widehat{\ZZ}$
as well, even if $K$ contains some extra roots of unity.

\head{The reciprocity map via abstract CFT}

First of all, we choose an isomorphism of $\Gal(\QQ^{\smcy}/\QQ)$
with $\widehat{\ZZ}$; our results are not going to depend on the choice.
That gives a continuous surjection
\[
d: \Gal(\overline{\QQ}/\QQ) \to \Gal(\QQ^{\smcy}/\QQ) \cong \widehat{\ZZ};
\]
if we regard $\QQ^{\smcy}/\QQ$ as the ``maximal unramified extension''
of $\QQ$, we can define the ramification index $e_{L/K}$ and inertia
degree $f_{L/K}$ for any extension of number fields, by the rules
\[
f_{L/K} = [L \cap \QQ^{\smcy}:K \cap \QQ^{\smcy}], \qquad
e_{L/K} = \frac{[L:K]}{f_{L/K}}.
\]
To use abstract class field theory to exhibit the reciprocity map, we
need a ``henselian valuation'' $v: C_{\QQ} \to \widehat{\ZZ}$,
i.e., a homomorphism satisfying:
\begin{enumerate}
\item[(i)]
$v(C_{\QQ})$ is a subgroup $Z$ of $\widehat{\ZZ}$ containing $\ZZ$
with $Z/nZ \cong \ZZ/n\ZZ$ for all positive integers $n$;
\item[(ii)]
$v(\Norm_{K/\QQ} C_K) = f_{K/\QQ} Z$ for all finite extensions $K/\QQ$.
\end{enumerate}
Once we have that, our calculations from the preceding chapters
(Theorem~\ref{T:first inequality}, Theorem~\ref{T:first and second inequality}) imply
that the class field axiom is satisfied: for $L/K$ cyclic,
\[
\#H^0_T(\Gal(L/K), C_L) = [L:K], \qquad
\#H^1_T(\Gal(L/K), C_L) = 1.
\]
So then abstract class field theory will kick in.

We can make that valuation using Artin reciprocity for $\QQ(\zeta_n)/\QQ$.
Recall that there is a canonical surjection
\[
I_n \to (\ZZ/n\ZZ)^* \cong \Gal(\QQ(\zeta_n)/\QQ):
\]
for $p$ not dividing $n$, the ideal $(p)$ goes to $p \in (\ZZ/n\ZZ)^*$
and then to the automorphism $\zeta_n \mapsto \zeta_n^p$, which indeed
does act as the $p$-th power map modulo any prime of $\QQ(\zeta_n)$
above $p$.

That induces a homomorphism $I_{\QQ} \to (\ZZ/n\ZZ)^*$ as follows:
given an id\`ele $\alpha$, pick $x \in \QQ^*$ so that
$\alpha_{\RR}/x > 0$ and, for each prime
$p$ with $p^e | n$, $\alpha/x$ has $p$-component congruent to 1
modulo $p^e$. Then map $\alpha/x$ to $(\ZZ/n\ZZ)^*$ as follows:
\[
\alpha/x \mapsto \prod_{\ell \not| n} \ell^{v_{\ell}(\alpha_{\ell}/x)}.
\]
This gives a well-defined map:
if $y$ is an alternate choice for $x$, then $x/y \equiv 1 \pmod{n}$
and $x/y > 0$, so the product on the right side is precisely $x/y$ itself,
and so is congruent to 1 in $(\ZZ/n\ZZ)^*$.

We now have maps $I_{\QQ} \to (\ZZ/n\ZZ)^*$ which are easily seen to
be compatible, so by taking inverse limits we get $I_{\QQ} \to
\widehat{\ZZ}^* \cong \Gal(\QQ^{\cyc}/\QQ)$. We define $v$ by channeling this
map through the projection $\Gal(\QQ^{\cyc}/\QQ) \to \Gal(\QQ^{\smcy}/\QQ)$
and then using our chosen isomorphism $\Gal(\QQ^{\smcy}/\QQ) \cong
\widehat{\ZZ}$.

Another way to say this:
$I_{\QQ}$ can be written as $\QQ^* \times \RR^* \times
\widehat{\ZZ}^*$, and the map to $\widehat{\ZZ}^*$ is just projection
onto the third factor! In particular, the map factors through
$C_\QQ$, and property (i) above is straightforward.

To check (ii), we need to do the same thing that we just did a bit more
generally. For $K$ now a number field, define the map
\[
I_n \to (\ZZ/n\ZZ)^* \supseteq \Gal(K(\zeta_n)/K),
\]
where now $I_n$ is the group of fractional ideals of $K$ coprime to $(n)$,
by sending a prime $\gothp$ first to its absolute norm. We then have to check
that the result is always in the image of $\Gal(K(\zeta_n)/K)$, but in fact
it must be: whatever the Frobenius of $\gothp$ is, it sends $\zeta_n$
to a power of $\zeta_n$ congruent to $\zeta_n^{\Norm(\gothp)}$
modulo $\gothp$. Since $\gothp$ is prime to $n$, it's prime to the difference
between any two powers of $\zeta_n$, so the Frobenius of $\gothp$ must
in fact send $\zeta_n$ to $\zeta_n^{\Norm(\gothp)}$. This tells us first
that the map
above sends $I_n$ to $\Gal(K(\zeta_n)/K)$ and second that it coincides
with the Artin map.

From the First Inequality, we can deduce the following handy fact.
\begin{prop}
For $L/K$ a finite abelian extension of number fields, the Artin map
always surjects onto $\Gal(L/K)$.
\end{prop}
\begin{proof}
If the Artin map only hit the subgroup $H$ of $\Gal(L/K)$, the fixed
field $M$ of $H$ would have the property that all but finitely many primes
of $M$ split completely in $L$. We've already seen that this contradicts
the First Inequality (Corollary~\ref{C:split completely}).
\end{proof}
In particular, the Artin map $I_n \to \Gal(K(\zeta_n)/K)$ we wrote down
above is surjective.
Using that, we can verify (ii): given a prime ideal $\gothp$ of
$K$, the Artin map of $K(\zeta_n)/K$ applied to it gives the same
element of $(\ZZ/n\ZZ)^*$ as the Artin map of $\QQ$ applied to
$\Norm_{K/\QQ}(\gothp)$. Meanwhile, the Artin map of $K(\zeta_n)/K$
surjects onto $\Gal(K(\zeta_n)/K)$, which has index
$[K \cap \QQ(\zeta_n):\QQ]$ in $(\ZZ/n\ZZ)^*$. This verifies (ii).

Thus lo and behold, we get from abstract class field theory a
reciprocity isomorphism for any finite extension of number fields:
\[
r'_{L/K}: C_K/\Norm_{L/K} C_L \stackrel{\sim}{\to} \Gal(L/K)^{\ab}.
\]
These are compatible in the usual way, so we get a map
$r'_K: C_K \to \Gal(K^{\ab}/K)$. Of course, we don't know what this map is,
so we can't yet use it to recover Artin reciprocity. (That depended on
the reciprocity map being the product of the local maps.) But at least
we deduce the norm limitation theorem.
\begin{theorem} \label{T:adelic norm limitation}
If $L/K$ is a finite extension of number fields and
$M = L \cap K^{\ab}$, then $\Norm_{L/K} C_L = \Norm_{M/K} C_M$.
\end{theorem}

We do know one thing about the map $r'_{L/K}$: for ``unramified'' extensions
$L/K$ (i.e., $L \subseteq K^{\smcy}$),
the ``Frobenius'' in $\Gal(L/K)$ maps to a ``uniformizer'' in
$C_K$. That is, the element of $\Gal(L/K)$ coming from the element
of $\Gal(K^{\smcy}/K)$ which maps to 1 under $d_K$ is the element of
$C_K$ which maps to 1 under $v_K$. But we made $v_K$ simply by
mapping $C_K$ to $\Gal(K^{\smcy}/K)$ via the Artin map and then 
identifying the latter with $\widehat{\ZZ}$ by the same identification
we used to make $d_K$. Upshot: the choice of that identification drops
out, and the reciprocity map coincides with the Artin map that we
wrote down earlier.

A bit later (see Chapter~\ref{chap:connection}), 
we will check that $r'_{L/K}$ agrees with
the map that I called $r_{L/K}$, namely
the product of the local reciprocity maps. Remember, I need this in order
to recover Artin reciprocity in general.

%\end{document}
