%\documentclass[12pt]{article}
%\usepackage{amsfonts, amsthm, amsmath}
%\usepackage[all]{xy}
%
%\setlength{\textwidth}{6.5in}
%\setlength{\oddsidemargin}{0in}
%\setlength{\textheight}{8.5in}
%\setlength{\topmargin}{0in}
%\setlength{\headheight}{0in}
%\setlength{\headsep}{0in}
%\setlength{\parskip}{0pt}
%\setlength{\parindent}{20pt}
%
%\def\AA{\mathbb{A}}
%\def\CC{\mathbb{C}}
%\def\FF{\mathbb{F}}
%\def\PP{\mathbb{P}}
%\def\QQ{\mathbb{Q}}
%\def\RR{\mathbb{R}}
%\def\ZZ{\mathbb{Z}}
%\def\gotha{\mathfrak{a}}
%\def\gothb{\mathfrak{b}}
%\def\gothm{\mathfrak{m}}
%\def\gotho{\mathfrak{o}}
%\def\gothp{\mathfrak{p}}
%\def\gothq{\mathfrak{q}}
%\def\gothr{\mathfrak{r}}
%\DeclareMathOperator{\ab}{ab}
%\DeclareMathOperator{\Aut}{Aut}
%\DeclareMathOperator{\coker}{coker}
%\DeclareMathOperator{\Cor}{Cor}
%\DeclareMathOperator{\disc}{Disc}
%\DeclareMathOperator{\Frob}{Frob}
%\DeclareMathOperator{\Gal}{Gal}
%\DeclareMathOperator{\GL}{GL}
%\DeclareMathOperator{\Hom}{Hom}
%\DeclareMathOperator{\im}{im}
%\DeclareMathOperator{\Ind}{Ind}
%\DeclareMathOperator{\Inf}{Inf}
%\DeclareMathOperator{\Norm}{Norm}
%\DeclareMathOperator{\Res}{Res}
%\DeclareMathOperator{\Trace}{Trace}
%\DeclareMathOperator{\Cl}{Cl}
%
%\def\head#1{\medskip \noindent \textbf{#1}.}
%
%\newtheorem{theorem}{Theorem}
%\newtheorem{lemma}[theorem]{Lemma}
%\newtheorem{cor}[theorem]{Corollary}
%
%\begin{document}
%
%\begin{center}
%\bf
%Math 254B, UC Berkeley, Spring 2002 (Kedlaya) \\
%Homology of Finite Groups
%\end{center}

\head{Reference} Milne, II.2; for cyclic groups, also
Neukirch, IV.7 and Lang, \emph{Algebraic Number Theory}, IX.1.

\head{Caveat} The Galois cohomology groups used in Neukirch are
not the ones we defined earlier. They are the Tate cohomology groups we
are going to define below.

\head{Homology}

You may not be surprised to learn that there is a ``dual'' theory to the
theory of group cohomology, namely group homology. What you may be surprised
to learn is that one can actually fit the two together, so that in a sense
the homology groups become cohomology groups with negative indices. (Since the
arguments are similar to those for cohomology, I'm going to skip details.)

Let $M_G$ denote the maximal quotient of $M$ on which $G$ acts trivially.
In other words, $M_G$ is the quotient of $M$ by the submodule spanned by
$m^g-m$ for all $m \in M$ and $g \in G$. In yet other words,
$M_G = M/M I_G$, where $I_G$ is the \emph{augmentation ideal} of the
group algebra $\ZZ[G]$:
\[
I_G = \left\{\sum_{g \in G} z_g[g]: \sum_g z_g = 0\right\}.
\]
Or if you like, $M_G = M \otimes_{\ZZ[G]} \ZZ$. Since $M^G$ is the group
of $G$-invariants, we call $M_G$ the group of \emph{$G$-coinvariants}.

The functor $M \to M^G$ is right exact but not left exact: if 
$0 \to M' \to M \to M'' \to 0$, then $M'_G \to M_G \to M''_G \to 0$
is exact but the map on the left is not injective. Again, we can fill
in the exact sequence by defining homology groups.

A $G$-module $M$ is \emph{projective} if for any surjection $N \to N'$
of $G$-modules and any map $\phi: M \to N'$, there exists a map
$\psi: M \to N$ lifting $\phi$. This is the reverse notion to injective;
but it's much easier to find projectives than injectives. For example,
any $G$-module which is a free module over the ring $\ZZ[G]$ is projective,
e.g., $\ZZ[G]$ itself!

Given a projective resolution $\cdots \to P_1 \to P_0 \to M \to 0$ of
a $G$-module $M$ (an exact sequence in which the $P_i$ are projective), 
take coinvariants to get a no longer exact complex
\[
\cdots \stackrel{d_2}{\to} P_2 \stackrel{d_1}{\to} P_1 \stackrel{d_0}{\to}
P_0 \to 0,
\]
then put $H_i(G, M) = \ker(d_{i-1})/\im(d_i)$. Again, this is canonically
independent of the resolution and functorial, and there is a long
exact sequence which starts out
\[
\cdots \to H_1(G, M'') \to \stackrel{\delta}{\to} H_0(G, M') \to H_0(G, M) \to H_0(G, M'') \to 0.
\]
Also, you can replace the projective resolution by an acyclic resolution
(where here $M$ being acyclic means $H_i(G,M) =0$ for $i>0$) and get
the same homology groups.
For example, induced modules are again acyclic (and the analogue of Shapiro's lemma holds, in part because any free $\ZZ[H]$-module induces to a free $\ZZ[G]$-module).

One can give a concrete description of homology as well, but we won't need
it for our purposes.
Even without one, though, we can calculate $H_1(G, \ZZ)$, using the exact
sequence
\[
0 \to I_G \to \ZZ[G] \to \ZZ \to 0.
\]
By the long exact sequence in homology,
\[
0 = H_1(G, \ZZ[G]) \to H_1(G, \ZZ) \to H_0(G, I_G) \to H_0(G, \ZZ[G])
\]
is exact, i.e. $0 \to H_1(G, \ZZ) \to I_G/I_G^2 \to \ZZ[G]/I_G$ is
exact. The last map is induced by $I_G \hookrightarrow \ZZ[G]$ and so
is the zero map. Thus $H_1(G, \ZZ) \cong I_G/I_G^2$;
recall that in an earlier exercise
(see Chapter~\ref{chap:principal}),
it was shown that the map $g \mapsto [g] - 1$ defines an isomorphism $G^{\ab} \to I_G/I_G^2$.

\head{The Tate groups}

We now ``fit together'' the long exact sequences of cohomology and homology
to get a doubly infinite exact sequence. Define the map $\Norm_G: M \to M$
by
\[
\Norm_G(m) = \sum_{g \in G} m^g.
\]
(It looks like it should be called
``trace'', but in practice our modules $M$ will be groups which are most
naturally written multiplicatively, i.e., the nonzero elements of a field.)
Then $\Norm_G$ induces a homomorphism
\[
\Norm_G: H_0(G,M) = M_G \to M^G = H^0(G,M).
\]
Now define
\[
H_T^i = \begin{cases}
H^i(G, M) & i > 0 \\
M^G/\Norm_G M & i=0 \\
\ker(\Norm_G)/MI_G & i=-1 \\
H_{-i-1}(G,M) & i<-1.
\end{cases}
\]
then I claim that for any short exact sequence $0 \to M' \to M \to M'' \to 0$,
we get an exact sequence
\[
\cdots
\to H^{i-1}_T(G, M'') \to H^i_T(G, M') \to H^i_T(G, M) \to H^i_T(G, M'')
\to H^{i+1}_T(G, M') \to \cdots
\]
which extends infinitely in both directions. 
The only issue is exactness between $H^{-2}_T(G, M'')$ and $H^1_T(G, M')$ inclusive;
this follows by diagram-chasing (as in the snake lemma) on the commutative diagram
\[
\xymatrix{
H_1(G, M'') \ar[r] \ar@{-->}[d] & H_0(G, M') \ar[r] \ar^{\Norm_G}[d] & 
H_0(G, M) \ar[r] \ar^{\Norm_G}[d] & H_0(G, M'') \ar[r] \ar^{\Norm_G}[d] & 0 \ar@{-->}[d] \\
0 \ar[r] & H^0(G, M') \ar[r]  & H^0(G, M) \ar[r] & 
H^0(G, M') \ar[r] & H^1(G, M')
}
\]
with exact rows (noting that the diagram remains commutative with the dashed arrows added).

This construction is especially useful if $M$ is induced, in which case $H^i_T(G,M) = 0$ for all $i$.
(The $T$ stands for Tate, who among many other things was an early pioneer
in the use of Galois cohomology into algebraic number theory.)

\head{Finite cyclic groups}

In general, for any given $G$ and $M$, it is at worst a tedious exercise
to compute $H^i_T(G,M)$ for any single value of $i$, but try to compute
all of these at once and you discover that they exhibit very little obvious
structure. Thankfully, there is an exception to that dreary rule when
$G$ is cyclic.

\begin{theorem} \label{T:cyclic group periodicity}
Let $G$ be a finite cyclic group and $M$ a $G$-module. Then there is
a canonical (up to the choice of a generator of $G$), functorial isomorphism $H^i_T(G,M) \to H^{i+2}_T(G,M)$ for all
$i \in \ZZ$.
\end{theorem}
\begin{proof}
Choose a generator $g$ of $G$.
We start with the four-term exact sequence of $G$-modules
\[
0 \to \ZZ \to \ZZ[G] \to \ZZ[G] \to \ZZ \to 0
\]
in which the first map is $1 \mapsto \sum_{g \in G} [g]$, the second map is
$[h] \mapsto [hg] - [h]$, and the third map is $[h] \mapsto 1$.
Since everything in sight is a free abelian group, we can tensor over $\ZZ$
with $M$ and get another exact sequence:
\[
0 \to M \to M \otimes_\ZZ \ZZ[G] \to M \otimes_\ZZ \ZZ[G] \to M \to 0.
\]
The terms in the middle are just $\Ind^G_{1} M$, where we first
restrict $M$ to a module for the trivial group and then induce back up.
Thus their Tate groups are all zero. The desired result now
follows from the following general fact: if
\[
0 \to A \stackrel{f}{\to} B \stackrel{g}{\to} C \stackrel{h}{\to} D \to 0
\]
is exact and $B$ and $C$ have all Tate groups zero, then
there is a canonical isomorphism $H^{i+2}_T(G, A) \to H^i_T(G, D)$.
To see this, apply the long exact sequence to the short exact sequences
\begin{gather*}
0 \to A \to B \to B/\im(f) \to 0 \\
0 \to B/\ker(g) \to C \to D \to 0
\end{gather*}
to get
\[
H^{i+2}(G, A) \cong H^{i+1}(G, B/\im(f)) = H^{i+1}(G, B/\ker(g))
\cong H^i(G,D).
\]
\end{proof}
In particular, the long exact sequence of a short exact sequence
$0 \to M' \to M \to M'' \to 0$ of $G$-modules curls up into an exact hexagon:
\[
\xymatrix{
 & H^{-1}_T(G, M) \ar[r] & H^{-1}_T(G, M'') \ar[dr] & \\
H^{-1}_T(G, M') \ar[ru] & & & H^0_T(G, M') \ar[dl] \\
 & H^{0}_T(G, M'') \ar[lu] & H^{0}_T(G, M) \ar[l] &
}
\]
If the groups $H^i_T(G, M)$ are finite, we define the
\emph{Herbrand quotient} 
\[
h(M) = \#H^0_T(G,M) / \#H^{-1}_T(G, M).
\]
Then from the exactness of the hexagon, if $M', M, M''$ all have
Herbrand quotients, then
\[
h(M) = h(M') h(M'').
\]
Moreover, if two of $M', M, M''$ have Herbrand quotients, so does the third.
For example, if $M'$ and $M''$ have Herbrand quotients, i.e., their Tate
groups are finite, then we have an exact sequence
\[
H^{-1}_T(G, M') \to H^{-1}_T(G, M)
\to H^{-1}_T(G, M'')
\]
and the outer groups are all finite. In particular, the first map is out
of a finite group and so has finite image, and modulo that image,
$H^{-1}_T(G,M)$ injects into another finite group. So it's also finite,
and so on.

In practice, it will often be much easier to compute the Herbrand quotient
of a $G$-module than to compute either of its Tate groups directly. The
Herbrand quotient will then do half of the work for free: once one group
is computed directly, at least the order of the other will be automatically
known.

One special case is easy to work out: if $M$ is finite, then $h(M) = 1$.
To wit, the sequences
\begin{gather*}
0 \to M^G \to M \to M \to M_G \to 0 \\
0 \to H^{-1}_T(G,M) \to M_G \stackrel{\Norm_G}{\to} M^G \to H^0_T(G,M) \to 0
\end{gather*}
are exact, where $M \to M$ is the map $m \mapsto m^g - m$; thus
$M_G$ and $M^G$ have the same order, as do $H^{-1}$ and $H^0$.

\head{Extended functoriality revisited}

The extended functoriality for cohomology groups has an analogue for homology and Tate groups, but under more restrictive conditions. Again, let $M$ be a $G$-module and $M'$ a $G'$-module, and consider a homomorphism $\alpha: G' \to G$ of groups and a homomorphism
$\beta: M \to M'$ of abelian which are compatible.
We would like to obtain canonical homomorphisms 
$H_i(G, M) \to H_i(G', M')$ and $H^i_T(G, M) \to H^i_T(G',M')$,
but for this we need to add an additional condition to ensure that $M \to M'$
induces a well-defined map $M_G \to M'_{G'}$. For instance, this holds if $\alpha$ is surjective. (Note that for Tate groups, we don't need any extra condition to get functoriality for $i \geq 0$.)

\head{Exercises}

\begin{enumerate}
\item
The periodicity of the Tate groups for $G$ cyclic means that there is
a canonical (up to the choice of a generator of $G$) isomorphism between $H^{-1}_T(G, M)$ and $H^1_T(G,M)$,
i.e., between $\ker(\Norm_G)/MI_G$ and the set of equivalence classes
of 1-cocycles. What is this isomorphism explicitly? In other words, given an
element of $\ker(\Norm_G)/MI_G$, what is the corresponding 1-cocycle?
\item
Put $K = \QQ_p(\sqrt{p})$. Compute the Herbrand quotient of
$K^*$ as a $G$-module for $G = \Gal(\QQ_p(\sqrt{p})/\QQ_p)$. (Hint:
use the exact sequence $1 \to \gotho_K^* \to K^* \to \ZZ \to 1$.)
\item
Show that $\Res: H^{-2}_T(G, \ZZ) \to H^{-2}_T(H,\ZZ)$ corresponds to
the transfer (Verlagerung) map $G^{\ab} \to H^{\ab}$.
\end{enumerate}

%\end{document}


