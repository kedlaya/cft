%\documentclass[12pt]{article}
%\usepackage{amsfonts, amsthm, amsmath}
%\usepackage[all]{xy}
%
%\setlength{\textwidth}{6.5in}
%\setlength{\oddsidemargin}{0in}
%\setlength{\textheight}{8.5in}
%\setlength{\topmargin}{0in}
%\setlength{\headheight}{0in}
%\setlength{\headsep}{0in}
%\setlength{\parskip}{0pt}
%\setlength{\parindent}{20pt}
%
%\def\AA{\mathbb{A}}
%\def\CC{\mathbb{C}}
%\def\FF{\mathbb{F}}
%\def\PP{\mathbb{P}}
%\def\QQ{\mathbb{Q}}
%\def\RR{\mathbb{R}}
%\def\ZZ{\mathbb{Z}}
%\def\gotha{\mathfrak{a}}
%\def\gothb{\mathfrak{b}}
%\def\gothm{\mathfrak{m}}
%\def\gotho{\mathfrak{o}}
%\def\gothp{\mathfrak{p}}
%\def\gothq{\mathfrak{q}}
%\def\gothr{\mathfrak{r}}
%\DeclareMathOperator{\ab}{ab}
%\DeclareMathOperator{\coker}{coker}
%\DeclareMathOperator{\disc}{Disc}
%\DeclareMathOperator{\Frob}{Frob}
%\DeclareMathOperator{\Gal}{Gal}
%\DeclareMathOperator{\GL}{GL}
%\DeclareMathOperator{\Hom}{Hom}
%\DeclareMathOperator{\im}{im}
%\DeclareMathOperator{\Ind}{Ind}
%\DeclareMathOperator{\inv}{inv}
%\DeclareMathOperator{\Norm}{Norm}
%\DeclareMathOperator{\Res}{Res}
%\DeclareMathOperator{\Trace}{Trace}
%\DeclareMathOperator{\unr}{unr}
%\DeclareMathOperator{\Cl}{Cl}
%
%\def\head#1{\medskip \noindent \textbf{#1}.}
%
%\newtheorem{theorem}{Theorem}
%\newtheorem{lemma}[theorem]{Lemma}
%\newtheorem{cor}[theorem]{Corollary}
%
%\begin{document}
%
%\begin{center}
%\bf
%Math 254B, UC Berkeley, Spring 2002 (Kedlaya) \\
%Overview of local class field theory
%\end{center}

\head{Reference} Milne, I.1; Neukirch, V.1.

\medskip
We will spend the next few chapters establishing \emph{local class field
theory}, a classification of the abelian extensions of a local field. This
will serve two purposes. On one hand, the results of local class field theory
can be used to assist in the proofs of the global theorems, as we saw with
Kronecker-Weber. On the other hand, they also give us a model set of proofs
which we will attempt to emulate in the global case.

Recall that the term ``local field'' refers to a finite extension either
of the field of $p$-adic numbers $\QQ_p$ or of the field of power series
$\FF_q((t))$. I'm going to abuse language and ignore the second case, although
all but a few things I'll say go through in the second case, and I'll try
to flag those when they come up. (One big one: a lot of extensions have to
be assumed to be separable for things to work right.)

\head{The local reciprocity law}

The main theorem of local class field theory is the following. For $K$
a local field, let $K^{\ab}$ be the maximal abelian extension of $K$.
\begin{theorem}[Local Reciprocity Law] \label{T:local reciprocity}
  Let $K$ be a local field. Then there is a unique map
$\phi_K: K^* \to \Gal(K^{\ab}/K)$ satisfying the following conditions:
\begin{enumerate}
\item[(a)] for any generator $\pi$ of the maximal ideal of $\gotho_K$
and any finite unramified extension $L$ of $K$, $\phi_K(\pi)$ acts on
$L$ as the Frobenius automorphism;
\item[(b)] for any finite abelian extension $L$ of $K$, the group of
norms $\Norm_{L/K} L^*$ is in the kernel of $\phi_K$, and the induced
map $K^*/\Norm_{L/K} L^* \to \Gal(L/K)$ is an isomorphism.
\end{enumerate}
\end{theorem}
The map $\phi_K$ is variously called the \emph{local reciprocity map} or the
\emph{norm residue symbol}.
Using the local Kronecker-Weber theorem (Theorem~\ref{T:local Kronecker-Weber}), this can be explicitly verified
for $K=\QQ_p$ (see exercises).

The local reciprocity law is an analogue of the Artin reprocity law for
number fields. We also get an analogue of the existence of class fields.
\begin{theorem}[Local existence theorem] \label{T:local existence}
For every finite (not necessarily abelian) extension $L$ of $K$,
$\Norm_{L/K} L^*$ is an open subgroup of $K^*$ of finite index.
Conversely,
for every (open) subgroup $U$ of $K^*$ of finite index, there exists a
finite abelian extension $L$ of $K$ such that $U = \Norm_{L/K} L^*$.
\end{theorem}
The condition ``open'' is only needed in the function field case; for
$K$ a finite extension of $\QQ_p$, one can show that every subgroup
of $K^*$ of finite index is open.

The local existence theorem says that if we start with a nonabelian extension $L$,
then $\Norm_{L/K} L^*$ is also the group of norms of an abelian extension.
Which one?
\begin{theorem}[Norm limitation theorem] \label{T:norm limitation}
Let $M$ be the maximal abelian subextension of $L/K$. Then
$\Norm_{L/K} L^* = \Norm_{M/K} M^*$.
\end{theorem}

Aside: for each uniformizer (generator of the maximal ideal) $\pi$
of $K$, let $K_\pi$ be the composite of all finite abelian extensions
$L$ such that $\pi \in \Norm_{L/K} L^*$. Then the local reciprocity map
implies that $K^{\ab} = K_\pi \cdot K^{\unr}$. It turns out that $K_\pi$
can be explicitly constructed as the extension of $K$ by certain elements,
thus giving a generalization of local Kronecker-Weber to arbitrary local
fields!
These elements come from Lubin-Tate formal groups, which we will not
discuss further.

Note that for $L/K$ a finite extension of local fields,
the map 
\[
K^*/\Norm_{L/K} L^* \to \Gal(L/K) = G
\] 
obtained by combining the local
reciprocity law with the norm limitation theorem
is in fact an isomorphism of $G = G^{\ab} = H^{-2}_T(G, \ZZ)$
with $K^*/\Norm_{L/K} L^* = H^0_T(G, L^*)$. We will in fact show something
stronger, from which we will deduce both the local reciprocity law and the norm limitation theorem.

\begin{theorem} \label{T:cup product isomorphism}
For any finite Galois extension $L/K$ of local fields
with Galois group $G$, there is a canonical isomorphism
$H^i_T(G, \ZZ) \to H^{i+2}_T(G, L^*)$.
\end{theorem}
In fact, this map can be written in terms of the cup product in group
cohomology, which we have not defined (and will not).

\head{The local invariant map}

One way to deduce the local reciprocity law (the one we will carry out first)
is to first prove the following.
\begin{theorem} \label{T:Brauer group identification}
For any local field $K$, there exist canonical isomorphisms
\begin{gather*}
H^2(\Gal(K^{\unr}/K), (K^{\unr})^*) \to
H^2(\Gal(\overline{K}/K), \overline{K}^*)\\
\inv_K: H^2(\Gal(\overline{K}/K), \overline{K}^*) \to \QQ/\ZZ.
\end{gather*}
\end{theorem}
The first map is an inflation homomorphism; the second
map in this theorem is called the \emph{local invariant map}.
More precisely, for $L/K$ finite of degree $n$, we have
an isomorphism
\[
\inv_{L/K}: H^2(\Gal(L/K), L^*) \to \frac{1}{n}\ZZ/\ZZ,
\]
and these isomorphisms are compatible with inflation. (In particular, we
don't need to prove the first isomorphism separately. But that can be done,
by considerations involving the Brauer group; see below.)

To use this to prove Theorem~\ref{T:cup product isomorphism} and hence
the local reciprocity law and the norm limitation theorem, we employ
the following theorem of Tate, which we will prove a bit later
(see Theorem~\ref{T:tate thm2}).
\begin{theorem} \label{T:tate thm1}
Let $G$ be a finite group and $M$ a $G$-module. Suppose that for each
subgroup $H$ of $G$ (including $H=G$), $H^1(H,M) = 0$ and $H^2(H,M)$
is cyclic of order $\#H$. Then there exist isomorphisms
$H^i_T(G, \ZZ) \to H^{i+2}_T(G, M)$ for all $i$; these are canonical
once you fix a choice of a generator of $H^2(G,M)$.
\end{theorem}

In general, for any field $K$, the group $H^2(\Gal(\overline{K}/K),
\overline{K}^*)$ is called the \emph{Brauer group} of $K$. It is an
important invariant of $K$; it can be realized also in terms of certain
noncommutative algebras over $K$ (central simple algebras). I won't
pursue this connection further, nor study many of the interesting properties
and applications of Brauer groups.

\head{Abstract class field theory}

Having derived local class field theory once, we will do it again a slightly
different way. In the course of proving the above results, we will
have calculated that if $L/K$ is a cyclic extension of local fields, that
\[
\#H^0_T(\Gal(L/K), L^*) = [L:K], \qquad \#H^{-1}_T(\Gal(L/K), L^*)
=1.
\]
It turns out that this alone is enough number-theoretic input to prove
local class field theory! More precisely, given a field $K$ with
$G = \Gal(\overline{K}/K)$, a continuous $G$-module $A$,
a surjective continuous homomorphism $d: G
\to \widehat{\ZZ}$, and a homomorphism $v: A^G \to \widehat{\ZZ}$
satisfying suitable conditions, we will show
that for every finite extension $L$ of $K$
there is a canonical isomorphism $\Gal(L/K)^{\ab} \to
A_L \to \Norm_{L/K} A_K$, where $A_K$ and $A_L$ denote the
$\Gal(\overline{K}/K)$ and $\Gal(\overline{K}/L)$-invariants of $A$.
In particular, these conditions will hold for $K$ a local field,
$A = \overline{K}^*$, $d$ the map $\Gal(\overline{K}/K) \to \Gal(K^{\unr}/K)$,
and $v: \Gal(K^*) \to \ZZ \to \widehat{\ZZ}$ the valuation.

This is the precise sense in which we will use local class field theory
as a model for global class field theory. After we complete local
class field theory, our next goal will be to construct an analogous
module $A$ in the global case which is ``complete enough'' that its
$H^0_T$ and $H^{-1}_T$ will not be too big; the result will be the idele
class group. (One main difference is that
in the global case, the analogue of $v$ will really take values in
$\widehat{\ZZ}$, not just $\ZZ$.) 

\head{Exercises}

\begin{enumerate}
\item
For $K = \QQ_p$, the local reciprocity map plus the local Kronecker-Weber
theorem give a canonical map
$\QQ_p^* \to \Gal(\QQ_p^{\ab}/\QQ_p) \cong \widehat{\ZZ}$.
What is the map? From the answer, you should be able to turn things around and deduce
local Kronecker-Weber from local reciprocity.
\item
For $K = \QQ_p$, take $\pi = p$. Determine $K_\pi$, again using
local Kronecker-Weber.
\item
Prove that for any finite extension $L/K$ of finite extensions of $\QQ_p$,
$\Norm_{L/K} L^*$ is an open subgroup of $K^*$. (Hint: show that
already $\Norm_{L/K} K^*$ is open! The corresponding statement in positive characteristic is more subtle.)
\item
Prove that for any finite extension $L/K$ of finite separable extensions of $\FF_p((t))$,
$\Norm_{L/K} L^*$ is an open subgroup of $K^*$. (Hint: reduce to the case of a cyclic extension of prime degree. If the degree is prime to $p$, you may imitate the previous exercise; otherwise, that approach fails because $\Norm_{L/K} K^*$ lands inside the subfield $K^p$, but you can use this to your advantage to make an explicit calculation.)
\item
A \emph{quaternion algebra} over a field $K$ is a central simple algebra over $K$ of dimension 4. If $K$ is not of characteristic 2, any such algebra has the form
\[
K \oplus Ki \oplus Kj \oplus Kk, \qquad i^2 = a, j^2 = b, ij = -ji = k
\]
for some $a,b \in K^*$. (For example, the case $K = \RR$, $a=b=-1$ gives the standard Hamilton quaternions.) A quaternion algebra is \emph{split} if it is isomorphic to the ring of $2 \times 2$ matrices over $K$.
Give a direct proof of the following consequence of Theorem~\ref{T:Brauer group identification}: if $K$ is a local field, then any two quaternion algebras which are not split are isomorphic to each other. 

\end{enumerate}

%\end{document}
